\documentclass[12pt]{ociamthesis}

\usepackage{amssymb}
\usepackage{titlesec}
\usepackage{amsmath}
\usepackage{float}
\usepackage{graphicx}
\usepackage{caption}
\usepackage{subfig}
\usepackage{graphicx}
\usepackage{xcolor}
\usepackage[round]{natbib}
\usepackage[section]{placeins}
\usepackage{mathrsfs}
\usepackage{bm}
\usepackage{stmaryrd}
\usepackage[utf8]{inputenc}
\usepackage{bibentry}
\usepackage{wasysym}
\usepackage{xfrac}
\usepackage{enumitem}

\usepackage{geometry}
 \geometry{
 a4paper,
 left=40mm,
 right=30mm,
 top=30mm,
 bottom=30mm
 }

\definecolor{theblue}{HTML}{0000CD}

% disable this package for printed version
\usepackage[colorlinks=true, linktocpage=true, allcolors=theblue]{hyperref}

\titleformat{\chapter}[display]
  {\bfseries\Large}
  {\filright\MakeUppercase{\chaptertitlename} \Large\thechapter}
  {1ex}
  {}
  [\vspace{1ex} \hrule \vspace{1pt} \hrule]

\newcommand{\adv}{    {\it Adv. Space Res.}} 
\newcommand{\araa}{    {\it Annual Review of Astron and Astrophys.}} 
\newcommand{\annG}{   {\it Ann. Geophys.}} 
\newcommand{\aap}{    {\it Astron. Astrophys.}}
\newcommand{\aaps}{   {\it Astron. Astrophys. Suppl.}}
\newcommand{\aapr}{   {\it Astron. Astrophys. Rev.}}
\newcommand{\ag}{     {\it Ann. Geophys.}}
\newcommand{\aj}{     {\it Astron. J.}} 
\newcommand{\apj}{    {\it Astrophys. J.}}
\newcommand{\apjl}{   {\it Astrophys. J. Lett.}}
\newcommand{\apss}{   {\it Astrophys. Space Sci.}} 
\newcommand{\bain}{   {\it Bulletin of the Astronomical Institutes of the Netherlands.}} 
\newcommand{\cjaa}{   {\it Chin. J. Astron. Astrophys.}} 
\newcommand{\gafd}{   {\it Geophys. Astrophys. Fluid Dyn.}}
\newcommand{\grl}{    {\it Geophys. Res. Lett.}}
\newcommand{\ijga}{   {\it Int. J. Geomagn. Aeron.}}
\newcommand{\jastp}{  {\it J. Atmos. Solar-Terr. Phys.}} 
\newcommand{\jgr}{    {\it J. Geophys. Res.}}
\newcommand{\mnras}{  {\it Mon. Not. Roy. Astron. Soc.}}
\newcommand{\na}{     {\it New Astronomy}}
\newcommand{\nat}{    {\it Nature}}
\newcommand{\pasp}{   {\it Pub. Astron. Soc. Pac.}}
\newcommand{\pasj}{   {\it Pub. Astron. Soc. Japan}}
\newcommand{\pre}{    {\it Phys. Rev. E}}
\newcommand{\solphys}{{\it Solar Phys.}}
\newcommand{\sovast}{ {\it Soviet  Astron.}} 
\newcommand{\ssr}{    {\it Space Sci. Rev.}}
\newcommand{\caa}{    {\it Chinese Astron. Astrohpys.}} 
\newcommand{\apjs}{   {\it Astrophys. J. Suppl.}}
\newcommand{\zap}{   {\it Zeitschrift fuer Astrophysik}}

\def\UrlFont{\sf}

\newcommand{\bs}[1]{\boldsymbol{#1}}
\newcommand{\bn}{\boldsymbol{\nabla}}
\newcommand{\rgas}{\mathcal{R}}
\newcommand{\eref}[1]{Eq. \eqref{#1}}
\newcommand{\fref}[1]{fig. \ref{#1}}
\newcommand\encircle[1]{%
  \tikz[baseline=(X.base)] 
    \node (X) [draw, shape=circle, inner sep=0] {\strut #1};}
\newcommand{\Alfven}{Alfv\'{e}n } 
\newcommand{\Alfvenic}{Alfv\'{e}nic }
\newcommand{\size}{0.75}
\newcommand\measureISpecification{4ex}% not defined in mwe
\newcommand{\ctab}[1]{\raisebox{\dimexpr \measureISpecification/2 -.748ex}{#1}}% vertically centers numbers
\newcommand{\si}[1]{\;\rm{#1}}
\newcommand{\mfig}[4]{
  \begin{figure}
  \begin{center}
  \includegraphics[width=#1\linewidth]{#2}
  \caption{#3}
  \label{#4}
  \end{center}
  \end{figure}}
\newcommand{\kms}{~\rm{km ~s^{-1}}}
\newcommand{\kgm}{~\rm{kg ~m^{-3}}}
\newcommand{\np}{\\ \\}
\begin{document}

\baselineskip=18pt

\setcounter{secnumdepth}{3}
\setcounter{tocdepth}{3}

\setcounter{chapter}{1}

%%%%%%%%%%%%%%%%%%%%%%%%%%%%%%%%%%%%%%%%%%%%%%%%%%%%%%%
% START COPYING HERE
%%%%%%%%%%%%%%%%%%%%%%%%%%%%%%%%%%%%%%%%%%%%%%%%%%%%%%%

\chapter{The Dynamics of Field-aligned Jets in the Solar Atmosphere}
\label{chap:sj}
%-------------------------------------------------------------------------------
\let\thefootnote\relax\footnotetext{
%
This chapter is based on the following refereed journal article:
\begin{itemize}[label={}]
\item Mackenzie Dover, F., Sharma, R., Erd\'elyi, R.; Magnetohydrodynamic Simulations of Spicular Jet Propagation Applied to Lower Solar Atmosphere Model, \apj, in press.
\end{itemize}
}
%------------------------------------------------------------------------------
\section{Introduction}
\label{sec:c2intro}
%------------------------------------------------------------------------------
% NEED TO KILL THIS FOR FINAL COMPILATION!!!!!!!!!!
%\pagenumbering{gobble}
% ALLOWS TO REMOVE PAGE NUMBERS, WHICH IS BETTER FOR CONVERTING TO WORD DOC
The aim of this chapter is to investigate the kinematics and morphology of solar spicular jets under a variety of physical conditions. A peculiar property of the solar atmosphere is that the corona is significantly hotter (approx. $1~\rm{MK}$) than the Sun's surface (approx. $6,500~\rm{K}$), apparently breaking the 2nd law of thermodynamics. This ``coronal heating problem" was discovered by \cite{Grotrian1939} and \cite{Edl1943} when they realized that a wrongly acclimated new element, coronium, was in fact high levels of ionization of iron, which would only be possible under extreme temperatures. More recently, the appropriateness of the term ``coronal heating problem" has been called into question by \cite{Aschwanden2007ApJ}, who state that it is a ``paradoxical misnomer", as there is no evidence of local heating in the corona itself, but rather heating occurs in the TR and upper chromosphere. Essentially, the focus for solving the heating of the solar atmosphere should be shifted to instead solve the chromospheric heating problem, and the way in which heat is transported from these regions into the upper solar atmosphere. Spicular jets are a strong candidate to resolve this problem \citep{Kudoh1999ApJ514493K, Pontieu2007PASJ, Martinez-Sykora2017,Moore2011ApJ731L18M, Pontieu2017ApJ, Samanta2019Sci, Zuo2019AcASn, Bale2019Natur}. The energy flux needed to heat the corona is approximately $3\times10^{5}~\rm{erg~cm^{-2}~s^{-1}}$ \citep{Withbroe1977ARAA15363W}. The esitmated energy flux hypthesised for spicules ranges from $\sim 1\times10^5-10^9~\rm{erg~cm^{-2}~s^{-1}}$  \citep{Athay1982ApJ255743A,Zaqarashvili_2009SSRv,dePontieu2011Sci33155D} and with an estimated $2 \times 10^{7}$ \citep{Judge_2010ApJ} Ca II spicules on the Sun's surface at any time, only a small fraction of this energy flux needs to be extracted from a single spicule to maintain the energy budget of the solar atmosphere. \np
%
The Sun's magnetic field is omnipresent throughout the whole solar atmosphere, which dictates the flow and dynamics of plasma. This magnetic field is manifested in numerous thin magnetic flux tubes (MFT), which link the photosphere with the corona and facilitate the transport of mass, energy, and momentum across the TR. Moreover, these MFT structures have the potential to act as conduits for magnetohydrodynamic (MHD) wave modes to propagate from lower solar atmosphere to the corona, and to transfer energy between these regions \citep{Pontieu2004Natur, Kukhianidze2006AA449L35K, Zaqarashvili2007AA474627Z, He2009AA497525H}. As outlined in section~\ref{Solar Jets}, due to multiple observed factors; observed wavelengths, physical characteristics (e.g., length, lifetime, velocity), regions (quiet Sun, active region or coronal hole), and/or locations (at-limb or on-disk), MFT structures are categorized as different features (e.g., spicules, mottles, dynamic fibrils, macrospicules, x-ray jets, EUV jets, coronal jets) that permeate the solar atmosphere \citep[see reviews by:][]{Beckers1968, Beckers1972ARA&A, Tsiropoula2012}.  In recent years, field-aligned/longitudinal propagation of dominant MHD wave and pulse in MFT structures become has a key focus of research to understand chromospheric heating \citep{Narain1990, Zaqarashvili_2009SSRv, Jess2015}. Understanding the propagation of MHD waves and pulses in a dynamic, gravitationally stratified, inhomogeneous environment can provide vital clues about dissipation mechanisms and the formation of shock-like behaviour associated with thin MFT structures. Observations suggest upward flow velocities in the range $15-110 ~\kms$ for typical lifetimes of between $50-400~\rm{s}$ for off-limb spicular structures (see Table~\ref{solar_jet_table}). The visible apex of these observed features attains a maximum height of approximately $7~\rm{Mm}$ and descends back to the surface following a parabolic trajectory \citep{Pereira2012,Pereira2016ApJ82465P}. \np
%
As highlighted in Section~\ref{sec:models}, many theoretical and numerical approaches have been undertaken to gain an insight into the morphology and dynamical properties of jets in the solar atmosphere. The plethora of models stems from the difficultly constraining parameters with observations and unanswered problems surrounding solar jets: What is their driver? At what height do they originate? What are their typical density, pressure and temperature profiles? Is the same driver responsible for multiple observed jets, or are there multiple ways of driving the same feature? When it comes to the parameterization of solar jets, the picture becomes muddier when considering the relationship between limb and disk features such as TI spicules, mottles, and dynamic fibrils (see Table~\ref{solar_jet_table}). This makes it challenging, from a mathematical modelling perspective, to determine what are the ``correct" parameters on which to base a model. Despite numerous valuable studies, an extensive and qualitative investigation on parameters that influence the observed dynamical characteristics of these spicular features remains missing. In this chapter, we aim to study the propagation of a momentum pulse originating near the photosphere and propagating through the chromosphere and the TR to lower corona in an idealized/stratified solar atmosphere using a simple numerical model. The parameter space comprises driver times, amplitudes, and magnetic field strength, which are examined to assess their role in determining the height, width, and beam structures.
%------------------------------------------------------------------------------     
\section{Parameter Space}
\label{subsec:paramater_space}
%------------------------------------------------------------------------------
Using the numerical setup as described in chapter \ref{section:numerical_recipe}, we have conducted numerous simulations to quantify the effects of three key parameters; magnetic field strength ($B$), driver time ($P$), and initial amplitude ($A$) on spicular jet's journey through the solar atmosphere. The parameter space covers the values from $P=50,~200,~300~\rm{s}$, $B_y=20,~40,~60,~80,~100~\rm{G}$, and $A=20,~40,~60,~80~\kms$ to study the impact on jet morphology (maximum height, width and structure) and kinematics (trajectory, transverse displacement). All ranges are based around typical values for classical spicules (see Section \ref{subsec:Spicules} and Table \ref{solar_jet_table}). The driver time is chosen within range of the $5$-minute oscillations of the p-mode \citep{Leighton1962ApJ135474L}, as they are a potential driver for multiple solar features (spicules, mottles  and dynamic fibrils) \citep{Pontieu2004Natur}. In addition, the driver time is based around the measured lifetime of spicular jets (see Table~\ref{solar_jet_table}). The magnetic field strengths are in the range of observed values for spicules, which is consistent with spectropolarimetric estimates for limb spicules \citep{centeno2010, suarez2015}. The range of velocities is chosen based on projectile motion (ignoring drag) to reach spicules' heights. Based on projectile motion, the maximum height for a launching angle of $90^{\circ}$ is,
\begin{equation}
h_{max} = \frac{A^2}{2g_{\odot}}
\end{equation}
where $A$ is initial amplitude, $g_{\odot}=0.274~\rm{km~s^{-2}}$ is solar gravity. For $A=20-80~\rm{km s^{-1}}$, corresponds to $h_{max}=0.73-11.6~\rm{Mm}$. These values fall in the range of observed speeds (heights) seen in spicules which have a range of $10-150\kms$ ($0.4-12~\rm{Mm}$) (see table \ref{solar_jet_table}). The magnetic field estimates are consistent with reported magnitudes from spectropolarimetric studies for off-limb spicules \citep{centeno2010, suarez2015}.
% may include a summarry table in the intro that is used in presenations, then we could refere to it in text.
% to lift near photospheric mass ($2.34\times10^{-4}~\rm{kg~m^{-3}}$)

%-------------------------
\subsection{Jet Tracking}
\label{subsec:jet_tracking}
%--------------------
To understand the jet dynamics, speed, lifetime, trajectory and jet boundary deformation are measured and compared with observational studies. This is achieved by temporally tracking the jet's width (solid blue dots) and apex (yellow triangle), using jet tracking software (see \fref{jet_tracker}). The jet tracking software developed for this thesis uses the tracer quantity first introduced in MPI-AMRVAC by \cite{Porth_2014}. In essence, the tracer is a non-physical quantity, akin to dye injections, which tracks the advection of a selected area of a fluid (in our case flow entering the computational domain). The injected jet material is given a high arbitrary value of $100$ (jet value), while the ambient environment is set to $0$, and using a threshold of $15\%$, any snapshot can be split into cells belonging to the jet and ambient medium. This is converted into a binary array where $1$ belongs to jet cells and $0$ is the ambient medium, which facilitates easy edge detection. The cross-sectional widths are estimated as a function of the distance between the opposite jet cells at edges, taken from a horizontal slice across the simulated jet at specified heights (see blue dots in fig.~\ref{jet_tracker}). By having a high resolution with temporal tracking of edges it is possible to obtain useful information of the jet boundary, such as the evolution of boundary deformation at multiple heights. Furthermore, the visible apex of the jet is selected as the highest index of a jet cell in the simulation, which allows us to track the jet's trajectory, speed, acceleration and deceleration (see a yellow triangle in Fig.~\ref{jet_tracker}).
\begin{figure}
\centering
{\includegraphics[width=0.8\linewidth]{figures/jet_P300_B60A_60T_0103.png}}
\caption{Example of jet tracking software that accurately estimates height and cross-sectional width parameters for the simulated jet structure. The jet-apex is marked with a yellow triangle with blue dots on jet edges providing an estimate of widths for each height during different evolutionary phases (rise/fall) of the jet.}
\label{jet_tracker}
\end{figure}

%---------------------------
\subsection{Jet Trajectories}
\label{subsec:jet_traj}
%------------------------
% !possibley add figs for these!!!!!!!!!!
The identified apex of the simulated jets shows a parabolic motion over time, as shown in Fig. \ref{jet_traj}. The estimated jet trajectories reveal non-ballistic paths of the apex, which is consistent with reported observations \citep{Hansteen2006ApJ, Rouppe2007ApJ660L169R, Pontieu2007PASJ} for chromospheric jets. In the numerical setup, the minimum amplitude for the driver is estimated to be $>20 ~\rm{km ~s^{-1}}$ for the near-photospheric plasma to attain spicular heights (see Table \ref{solar_jet_table}). Also, for most cases, the maximum apex for the simulated jets was found to be proportional to the driver amplitude. In each panel, there is a subgroup reaching lower heights due to reduced driving time. We find that our jets have an asymmetric parabolic path as reported in \cite{Singh2019}. This could be due to a piling of jet material as the jet begins to fall, as the falling matter will interact with raising matter, slowing the jet's descent.
\begin{figure}
\captionsetup[subfigure]{labelformat=empty}
\centering
\subfloat[]{\includegraphics[width=\linewidth]{figures/jet_traj.png}}
\caption{Plots show the effects of multiple parameter combinations on maximum apex height attained by the numerical jets'. Results for driver amplitudes with magnitude $A = 20$ km s$^{-1}$ (top-left), $A = 40$ km s$^{-1}$ (top-right), $A = 60$ km s$^{-1}$ (bottom-left), and $A = 80$ km s$^{-1}$ (bottom-right) are shown with jet-apex following a parabolic trajectory. Driver period ($P$), magnetic field strength ($B$) and driver amplitude ($A$) have units in sec, Gauss and km s$^{-1}$ respectively.}
\label{jet_traj}
\end{figure}
%-------------------------------------------
\subsection{Effects on Jet Apex and Widths}
\label{subsec:jet_apex_widths}
%----------------------- 
To understand the impact of the key parameters investigated, we compared how each parameter affects the maximum apex height (panels to the left) and average cross-section width (CSW) (panels to the right) of the jet over its entire life cycle, as displayed in \fref{parameter_scan_lines}. Varying the magnetic field strength ($B$) has less impact on the heights reached by the jet as all cases have shallow gradients, most likely due to the flow of the jet being field-aligned. However, it must be noted that any deviation between jet flow and magnetic field direction can affect the apex height of the jet; this particular aspect is investigated in Chapter 3. The driver time ($P$) has a nuanced effect, as there is a slight decrease in height for $p=50~\rm{s}$, but peaks between $P=200$ and $300~\rm{s}$. This suggests that to transport significant mass from lower in the atmosphere, an instantaneous pulse will not be sufficient. There needs to be a sustained driver, which past a threshold value, does not impact heights reached, but is important in the behaviour of the CSW variation. Both previously mentioned parameters are grouped by colours and line styles for $B$ and $P$ scans, which correspond to matching initial amplitudes. It is found that the jet height is most sensitive to initial amplitude ($A$) of the parameters studied. Based on these indications, the jet height is modelled with a power-law function given as,  
\begin{equation}
h_{max} = C A^{n},
\end{equation}
by collapsing the data by taking an average at each velocity data point and then fitting an optimal curve using least-squares obtaining values of $C= 10^{-2.21}$ and $n= 1.72$ as shown in the bottom left panel in Fig.~\ref{parameter_scan_lines}. The estimates from the power-law suggest a nonlinear relation between apex height and pulse strength, in contrast to the results reported by \citet{Singh2019}, although both results need to be observationally verified. It must also be noted that \citet{Singh2019} used an instant pressure pulse close to the TR as a driver in their simulation, which might not be directly comparable with our investigation. This could be because in this thesis a larger parameter space is investigated, the jets are initiated at a lower atmospheric height, and the jets studied are faster-moving than studied by \citet{Singh2019}. \np 
%
The panels to the right of Fig.~\ref{parameter_scan_lines} shows the effect of the parameter space on the mean CSW. To obtain the mean CSW, it is measured at every $\rm{Mm}$ the jet reaches and the mean value is calculated over the whole life cycle of the jet. Although this is by no means a perfect method, it does give a useful insight into the global deformation of the jet boundary. Our simulations suggest a strong influence of magnetic field strength in determining the widths of the jet structure, with estimated CSW inversely related to the magnetic field magnitudes. This happens because the jet rises through the stratified atmosphere, it has higher internal pressure (as transporting material with lower atmosphere conditions) compared to the ambient medium. This results in expansion of the jet structure and subsequent distortion of the surrounding field. However, in the case of higher magnetic field magnitudes, the jet experiences strong tension forces that result in greater collimation of the jet structure. These results indicate the possibility of lower cross-sectional widths of jet features located near a strong magnetic field environment than those in quiet Sun regions. Another important aspect that influences the jet width is the lifetime of the driver of the jet. The simulations suggest a linear relationship between the driver periods and the jet CSW. This could be due to the long supply of plasma to the jet by a sustained driver resulting in increased CSW. However, the amplitude of the driver had a minimal effect on the width for the jets aligned with radial magnetic fields, although this might not be the case if the jet direction were misaligned with the background magnetic fields.
\begin{figure}
\captionsetup[subfigure]{labelformat=empty}
\centering
\subfloat[]{\includegraphics[width=\linewidth]{figures/test_combine_image.png}}
\caption{Panels compare the effects of different physical parameters (top to bottom: magnetic field strength, driver period, driver amplitude) on the maximum apex height (left) and mean cross-sectional widths (right) for the simulated jet structure. The shaded-region (bottom left) indicates the 1$\sigma$ error for the power law fit (red line) to the parameter scan. It should be noted that driver period ($P$), magnetic field strength ($B$) and driver amplitude ($A$) have units in sec, Gauss and km s$^{-1}$ respectively. {\color{green}!!need to make readable!!}}
\label{parameter_scan_lines}
\end{figure}
%------------------------------------------------------------------------
\section{Synthetic Jet Morphology}
%------------------------------------------------------------------------------
Based on typical criteria for jets of supersonic flows, as outlined in section~\ref{struc_of_jets}, the structures of the jet are akin to a heavy under-expanded jet \citep{Norman1982, Edgington-Mitchell2014} as the jet density and pressure are greater than the ambient medium. From the parameter scan, we define a ``standard'' jet as the one that is most reflective of a spicular jet. The standard jet reaches a height of approx. $8~\rm{Mm}$ with a velocity of observed spicules, showing clear boundary deformation, and is suitable middle ground in the range of morphology seen in the parameter scan. This is depicted in Fig.~\ref{standard_jet}, with $P=300~\rm{s}$, $B=60~\rm{G}$ and $A=60~\kms$, where snapshots of the density (a-d), temperature (e-h) and numerical Schlieren (i-l) are shown. \np
%
The numerical Schlieren represents a normalized density gradient magnitude and is defined as follows,
\begin{equation}
   S_{ch} = \exp{\left( -c_0 \left[ \frac{|\boldsymbol{\nabla} \rho|-c_1 |\boldsymbol{\nabla} \rho|_{max}}{c_2 |\boldsymbol{\nabla} \rho|_{max}-c_1|\boldsymbol{\nabla} \rho|_{max}} \right] \right)},
\end{equation}
%------------------------
where $c_0=5$, $c_1=0.05$ and $c_2=-0.001$. It is important to note while the numerical Schlieren is a physically defined quantity, its use in our analysis is purely from a qualitative perspective as it returns a synthetic shadowgraph image. Shadowgraph images are created by pointing a single light source at a fluid flow in a dark room. It is possible to see the flow patterns because any changes in temperature, pressure, or density, change the refraction index, thus the bending of light rays will create dark and bright regions in the image. An important facet of our simulation is related to the field-aligned jet motions (Fig.~\ref{standard_jet}) that show distinct kinematic and morphological characteristics during rising and falling phases in columns (a-b) and (c-d), respectively. In the rise phase (left-panels of Fig.~\ref{standard_jet}), the simulated jet shows complex internal substructures and sausage-like deformation of cross-sectional widths in density (a-d) and Schlieren (i-l) data. However, during the fall phase (columns of c-d of fig.~\ref{standard_jet}), the complex/internal beam substructure ceases due to driver switch-off. At this stage, the boundary deformation is no longer symmetric, though the jet-axis has noticeable transverse displacement. Modification in sausage-like wave properties associated with CSW estimates during rising and falling phases of simulated jets were recently reported by \cite{Dover2020ApJ90572D}, as described in more detail in chapter 4. The temperature profile of the numerical jet suggests periodic distortion in the TR layer over the jet lifetime. The TR deforms as the jet penetrates this layer while generating waves, also known as TR quakes \citep{Scullion2011ApJ74314S}. During its entire lifetime, the jet structure remains isothermal/cool without much variation for any combinations of the selected parameters. Without the inclusion of radiative transfer, any significant insight into the potential heating of the solar atmosphere is out of the scope of this research. \np
%------------------------
\begin{figure}
\captionsetup[subfigure]{labelformat=empty}
\centering
\subfloat[]{\includegraphics[width=0.8\linewidth]{figures/sj_jet_rho_Te_sch.png}}
\caption{Panels show snapshots of temporal evolution of the ``standard’’ numerical jet with uniform radial magnetic field in stratified atmosphere with driver period ($P = 300$ sec), magnetic field ($B = 60$ G) and amplitude ($A = 60$ km s$^{-1}$). The density (a-d), temperature (e-h) and Schlieren (i-l) images highlight complex internal substructures of the simulated jet with bright apex, possibly due to high-density concentrations. The temperature (middle) panel indicates the isothermal nature of the jet structure with a colder plasma component from the photospheric layer.}
\label{standard_jet}
\end{figure}
%
%------------------------
%\begin{figure}
%\captionsetup[subfigure]{labelformat=empty}
%\centering
%\subfloat[]{\includegraphics[width=0.8\linewidth]{figures/sch_example.jpeg}}
%\caption{Example of Schlieren pictures from a supersonic flow taken from \cite{Edgington-Mitchell2014}. Schlieren imaging is a technique that can be used to photograph the flows of fluids with varying density. As the environment changes in pressure, temperature or density the refraction index changes. A knife-edge is placed before the sensor that captures the image, causing the refracted rays of light to bend above and below. This is what creates areas of brightness and darkness that reveal the inhomogeneities in the flow.}
%\label{sch_example}
%\end{figure}
%
Figs. \ref{paramter_scan_one} and \ref{paramter_scan_two} showcase the limit cases for the simulated jet with density structure highlighting variations in parameter space in each row for similar time-instance, as shown in fig.~\ref{standard_jet}. The panels (a-d) and (e-h) in Fig~\ref{paramter_scan_one} exhibits the effect of strong ($B = 80~\rm{G}$) and weak ($B = 20~\rm{G}$) magnetic field strengths, respectively. For the strong magnetic field, the jet was found to be more collimated, with higher density mostly concentrated near the apex. In this case, the complex beam structure is less visible, but due to decreased CSW, it increases the number of knots (areas of enhanced density along the central jet axis) present. The sausage-like jet boundary deformation is less prominent when the magnetic field is stronger. In contrast, in the case of the weak magnetic field, the jet is more diffusive as it balloons out in the solar atmosphere, and due to the large CSW, it decreases the number of knots, as only one is observed. This clearly shows that there is a relationship between CSW and the number of knots. These knots are areas of density enhancements which, when the jet ceases to be driven, flow down the jet to create interesting internal dynamics, particularly in the weak magnetic field example (see (e-f) in Fig.~\ref{paramter_scan_one}). Not only are there changes in the jet boundary deformation in the rise and fall phases of the jet, but there are also changes in the jet beam itself, which may be a useful diagnostic for observers when investigating the potential driver for solar jets. \np    
%
Figs. \ref{paramter_scan_one} (i-l) and \ref{paramter_scan_two} (a-d) showcase the effects of high ($A = 80 ~\kms$) and low ($A = 20 ~\kms$) velocity magnitudes on jet morphology. For higher velocity, the simulated jet had an internal substructure similar to the standard jet, but with more knot features at matching time-steps. The jet shows kinking behaviour from j-l, possibly due to initial high velocity in a slender structure which is more susceptible to kink instability. The transversal displacement travels upwards along the jet beam and starts to bunch closer together when the jet reaches its falling phase, which in turn, changes the directions of flow. For both the standard jet and high amplitude, for example at $t = 288 ~\rm{s}$, the internal substructures disappear, indicating a close association between driver time and knots lifetimes. For $A = 20$ km s$^{-1}$ (top panel: \fref{paramter_scan_two}), the simulations suggest that for a lower velocity the jet heights are reduced and there is an absence of any significant substructures. The initial amplitudes show that they are a key factor in the dynamics and morphology of the jet, as they are responsible for the complex internal substructures and the CSW variations. The horizontal dynamics are a particularly important aspect of this research, as numerical simulations of solar jets do not typically report on CSW variation, even though in observations of spicules they are not just vertically dynamic, as solar spicules exhibit complex motions horizontally \citep{Sharma2018ApJ85361S,Antolin2018ApJ85644A}. \np
%
Panels e-f and i-l in Fig.~\ref{paramter_scan_two} highlight the role of driver period for two cases with magnitudes $P = 50$ and $200~\rm{s}$, respectively. In case of shorter driver period ($P = 50 ~\rm{s}$), the jet at $t = 21~\rm{s}$ shows a similar evolutionary trend as for the longer durations ($P = 200$ and $300~\rm{s}$), with a lifetime of around $t = 223~\rm{s}$. For long duration drivers ($P = 200~\rm{s}$ ), jet boundaries are smooth and show no kink-like deformation in the fall-phase. A key difference between the two driver periods is the amount of mass injection in the jet, which has direct implications for the CSW as the less time the driver is given to supply energy, the faster the jet boundary undulates. \np
%--------------------
\begin{figure}
\captionsetup[subfigure]{labelformat=empty}
\centering
\subfloat[]{\includegraphics[width=0.8\linewidth]{figures/den_plot_1.png}}
\caption{Evolution of the synthetic jet with a different combination of magnetic field and driver amplitude parameters, in a stratified solar atmosphere with uniform radial magnetic field. Parameters are varied for the ``standard’’ jet configuration to identify the effects of a particular parameter on jet morphology and kinematics. Panels (top-bottom) show variations with $B = 80$ G (a-d), $B = 20$ G  (e-h) and $A = 80$ km s$^{-1}$ (i-l), respectively.}
\label{paramter_scan_one}
\end{figure}
\begin{figure}
\captionsetup[subfigure]{labelformat=empty}
\centering
\subfloat[]{\includegraphics[width=0.8\linewidth]{figures/den_plot_2.png}}
\caption{Similar to \fref{paramter_scan_one} with panels (top-bottom) highlighting variations with $A = 20$ km s$^{-1}$ (a-d), $P = 200$ (e-h)$,~50$ s (i-l) respectively for ``standard’’ jet configuration.}
\label{paramter_scan_two}
\end{figure}
%---------------------
Overall, the results indicate that changing parameters, whether linked to the driver or the environment through which a jet travels, both have a noticeable impact on the dynamics and morphology of the jet. The parameter scan shows the impact of the morphology of the jet and raises an important question about what is driving these solar jets and at what heights. This series of simulations suggest that if the driver is an instant pulse/short driving time would lead to very thin, low-density jets without complex structures. It also predicts that in regions with a strong magnetic field the jets would be thin, but denser. For each simulation shown, it displays the jet heads being the densest part of the jet. Observational evidence for these bright bulb-like substructures at the jet apex was recently reported in a few studies \citep{depontieu2017, srivastava2018NatAs} for chromospheric jets. \citet{depontieu2017} used a combination of 2.5D radiative MHD simulations and observations for spicule structures, and identified bright leading substructures as ``heating fronts". These authors proposed the formation of these regions as a consequence of upward propagating magnetic waves and/or dissipation of electric currents due to ambipolar diffusion between ions and neutrals. \citet{srivastava2018NatAs} reported ubiquitous observations of chromospheric jets with a leading bright apex, trailed by a dark region in running-difference images. They labelled these jets as ``tadpoles" and suggested the role of pseudo-shocks in the formation of these observed bright substructures.

%------------------------------------------------------------------------------     
\subsection{Jet Beam Structure}
\label{subsec:j_beam_struc}
%------------------------------------------------------------------------------
Our simulations revealed a yet to be observed phenomenon for solar jets, as the synthetic jets contain complex beam substructures. Multiple cases with varied driver amplitudes/periods and magnetic field strength (Figs.~\ref{standard_jet} and \ref{paramter_scan_two}), highlighted criss-cross/knot patterns within the jet beam, along with CSW variations. Complex beam structure seems to be a common feature in the nature in jets, for example with astrophysical \citep{van_Putten_1996ApJ467L57V, DeGouveiaDalPino2005, Hada2013ApJ77570H, Cohen2014ApJ787151C, Hervet2017AnA606A103H} and laboratory jets \citep{Menon2010, Edgington-Mitchell2014, Ono2014}. With current resolution limits, these features would be challenging to observe in solar jets. Two possible mechanisms for the presence of the complex beam structures:
\begin{enumerate}
\item{Knots are the manifestation of non-equilibrium pressure forces prominent within and around the jet structure. As the jet raises it to bring plasma from the lower atmosphere, it thereby has a higher pressure than the ambient medium. Thus, initially, the internal jet pressure results in expansion of the jet which is then counter-balanced by the tension forces of the jet. Eventually, the magnetic pressure dominates the internal plasma pressure, and the jet boundary decreases while resulting in compression of the jet structure. Essentially, there is a ``tug of war" occurring between the pressure forces of the jet and the surrounding magnetic field. This process over height and time appears as CSW deformation of the jet, along with the sites of high density/pressure as knots within the jet structure.}
\item{The knot features could be the sites within the jet structure where internal shock waves \citep{Norman1982} become reflected from the jet boundary. If the jet has supersonic velocities  (synthetic jets have Mach numbers are around $\sim$1-3), it can give rise to a myriad of internal substructures due to high velocities.}
\end{enumerate}
Continuing further in the lens of (2), a schematic overview of these beam (sub)structures is given in Fig. \ref{cartoon_jet_waves}, highlighting two main features, osculating jet boundary and criss-cross pattern formation, that are present in supersonic jets. The boundary deformation arises due to non-equilibrium between the pressure of the jet and the ambient medium. The jet expands from the point of origin as the jet pressure is highest, however, the jet repeatedly overshoots equilibrium points due to the effects of the boundary, communicated to the interior by the sound waves, which are travelling more slowly than the supersonic flow of the jet. Hence, the jet goes through a sequence of expansions and contractions. Not only do supersonic flows give a sausage-like boundary deformation, but the areas of low/high pressure/density are out of phase with maximum/minimum CSW of the jet (see fig. \ref{cartoon_jet_waves}). \np 
%
The knots inside the jet beam are the manifestation of shock waves trapped inside the jet boundaries. An expansion fan (blue dashed lines in Fig.~\ref{cartoon_jet_waves}) forms at the base of the jet due to the difference in pressure between the jet and the ambient atmosphere. This expansion fan causes an outward flow, making the jet enlarge. The Mach lines of the expansion waves reflect off the jet boundary inwards towards the jet centre in the form of compression waves and a compression fan due to pressure continuity (red lines in Fig.~\ref{cartoon_jet_waves}). The compression waves are reflected at a nearly constant angle from the jet boundary, and as this boundary is curved, the Mach lines of the compression waves tend to converge into a conical shock wave before reaching the centre of the jet. This incident shock either goes under a regular reflection or becomes a Mach disk depending on the angle between the incident shock and the central jet axis, for small and large angles respectively (see black lines Fig.~\ref{cartoon_jet_waves}). These shock waves determine the crisscrossing in the jet beam as well as the shape of the knots. As the flow passes through this shock it increases the pressure in the jet. When the reflected shock reaches the jet boundary it forces the boundary outwards, creating an expansion fan and thus allowing the process to be repeated. \np
%
These shock waves explains the formation of stationary knots while the flow is active, along with CSW deformations (which do not have an anti-node), density enhancements and cavities in the jet structure. In addition, this provides a vital clue about the disappearance of the complex beam substructures during the driver’s switch-off phase. In our simulations, the standard jet (Fig.~\ref{standard_jet}) clearly showcase these knots, and their disappearance with the driver switch-off at around $t=288~\rm{s}$. Similar behaviour is evident from \fref{paramter_scan_one} and \fref{paramter_scan_two}. If this substructure can be seen in jet observations then it gives insight into the driving times.
%----------------
\begin{figure}
\captionsetup[subfigure]{labelformat=empty}
\centering
\subfloat[]{\includegraphics[width=0.8\linewidth]{figures/jet_diagram.eps}}
\caption{A cartoon depicting complex internal substructures in a supersonic jet. The cross-sectional width variations due to formation of regions with high and low pressure as a consequence of internal shock waves, are shown. These wave patterns appear as knots in our simulations, highlighting the role of initial velocity of momentum pulse in generating axisymmetric deformation in a jet structure. }
\label{cartoon_jet_waves}
\end{figure}
%------------------------------------------------------------------------------
\section{Summary and Discussion}
\label{sec:c2discussion}
%------------------------------------------------------------------------------
In summary, we have used a simple model to study the effect on jet heights and widths when it is driven by a momentum pulse with enough strength to raise near photospheric material to spicules' heights. This approach is taken to focus on the dynamics of the jets across an extensive parameter scan. The main results of this approach are:  
\begin{itemize}
\item{The visible apex of the jet structure followed a non-ballistic parabolic trajectory with heights comparable with observed spicule motions. A parameter scan suggests a strong influence of driver amplitude in determining the longitudinal dynamics of a jet structure.}

\item{Jets emanating from regions with higher background magnetic field magnitudes are more collimated/dense than those originating in quiet Sun conditions.}

\item{First reports of complex internal substructures are formed during the rising phase of a solar jet structure. These knot patterns are possibly formed due to the reflection of internal shock waves at jet boundaries.}

\item{Temperature maps from our simulations reveal periodic distortions of the transition region due to penetration of jet structure at this layer. However, the jet remained isothermal for its entire life for all combinations of initial parameters. For these simulations, it would be difficult to conclude how much heating these jets can provide to the corona. More physics would need to be added (e.g. radiative transfer and inclusions of neutrals), which is out of the scope of the scientific goals.}

\item{Simulated jets in our study show a bright, bulb-like apex, similar to observed cases of chromospheric jets.}
\end{itemize}
Additionally, the morphology of the synthetics jets is sensitive to the parameters covered in these simulations, i.e. magnetic field strength, driver times and initial amplitude. This may lend credence to linking jet-like events such as dynamics fibrils (AR), mottles (QS) and TI spicules (QS), but apparent differences could partly be attributed to different atmospheric/magnetic environment conditions and differences in parameters of the driver. However, more theoretical and observational evidence is required, for example, running a numerical simulation with the same driver, but changing environmental conditions to match those of observed ARs and QS. \np
%
The synthetic jets show interesting internal dynamics in the beam structures, which need to be confirmed observationally. Two possible mechanisms are proposed; (1) ``tug of war" of jet and ambient pressure, or (2) shock waves caused by the supersonic flow. However, the supersonic flow framework explains the beam structures that are created, the reason that there are areas of density enhancements referred to as knots, cavities along the central jet axis, and the boundary deformation. Knots are seen in many jets from large astrophysical jets \citep{van_Putten_1996ApJ467L57V, DeGouveiaDalPino2005, Hada2013ApJ77570H, Cohen2014ApJ787151C, Hervet2017AnA606A103H} to small laboratory jets \citep{Menon2010, Edgington-Mitchell2014, Ono2014}, hence it makes natural sense that knots are present in solar jets if flow speeds and driver times are sufficient. While this has not yet been observed in small scale solar jets, it is likely due to current resolution limits. In Fig.~\ref{degrid}, the impact of resolution in identifying these features is shown. Observationally the highest spatial resolution is achieved by the Swedish Solar Telescope (SST), at around $70-110~\rm{km}$ \citep{Scharmer2003SPIE4853341S,Berger2003ApJ}, as shown in panels e-f. It may be possible to identify jet substructures in the foreseeable future with telescopes such as Daniel K. Inouye Solar Telescope (DKIST). DKIST can achieve spatial resolution of approximately $14-60~\rm{km}$ \citep{Rast2020arXiv,Rimmele2020SoPh}, which is represented by panels a-d. By achieving higher observational resolution, it will be possible to confirm these beam structures, and also to make significant leaps forward in our understanding of solar jets. \np
%
As highlighted earlier, the question of what drives the spicular jet is not yet resolved. The presence/absence of complex beam structures, if observationally verified, would be a significant step forward in understanding the nature of the driver of jets. If observed, then, this can be used to measure lifetime drivers and be used to identify whether a jet is in a rising or falling phase. If they are not observed, then this points towards a short pulse event (e.g. magnetic recognition) as the driver. These substructures could give a new window through which to investigate this phenomenon.
%----------------
\begin{figure}
\captionsetup[subfigure]{labelformat=empty}
\centering
\subfloat[]{\includegraphics[width=0.8\linewidth]{figures/res_den_plot.png}}
\caption{Using the same time step as panel (b) in Fig.~\ref{standard_jet}, the density grid has been degraded to give a rough estimate of the appearance of the jet over a range of resolutions. The panels a-d give covers the possible resolution range of DKIST and panels e-f show the resolution in ranges of SST.}
\label{degrid}
\end{figure}
%---------------------------------%
%%%%%%%%%%%%%%%%%%%%%%%%%%%%%%%%%%%%%%%%%%%%%%%%%%%%%%%
% STOP COPYING HERE
%%%%%%%%%%%%%%%%%%%%%%%%%%%%%%%%%%%%%%%%%%%%%%%%%%%%%%%

\bibliographystyle{plainnat}
\bibliography{references}

\end{document}
