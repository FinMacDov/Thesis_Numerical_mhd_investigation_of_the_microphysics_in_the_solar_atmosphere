\documentclass[12pt]{ociamthesis}

\usepackage{amssymb}
\usepackage{titlesec}
\usepackage{amsmath}
\usepackage{float}
\usepackage{graphicx}
\usepackage{caption}
\usepackage{subfig}
\usepackage{graphicx}
\usepackage{xcolor}
\usepackage[round]{natbib}
\usepackage[section]{placeins}
\usepackage{mathrsfs}
\usepackage{bm}
\usepackage{stmaryrd}
\usepackage[utf8]{inputenc}
\usepackage{bibentry}
\usepackage{wasysym}
\usepackage{xfrac}

\usepackage{geometry}
 \geometry{
 a4paper,
 left=40mm,
 right=30mm,
 top=30mm,
 bottom=30mm
 }

\definecolor{theblue}{HTML}{0000CD}

% disable this package for printed version
\usepackage[colorlinks=true, linktocpage=true, allcolors=theblue]{hyperref}

\titleformat{\chapter}[display]
  {\bfseries\Large}
  {\filright\MakeUppercase{\chaptertitlename} \Large\thechapter}
  {1ex}
  {}
  [\vspace{1ex} \hrule \vspace{1pt} \hrule]

\newcommand{\adv}{    {\it Adv. Space Res.}} 
\newcommand{\araa}{    {\it Annual Review of Astron and Astrophys.}} 
\newcommand{\annG}{   {\it Ann. Geophys.}} 
\newcommand{\aap}{    {\it Astron. Astrophys.}}
\newcommand{\aaps}{   {\it Astron. Astrophys. Suppl.}}
\newcommand{\aapr}{   {\it Astron. Astrophys. Rev.}}
\newcommand{\ag}{     {\it Ann. Geophys.}}
\newcommand{\aj}{     {\it Astron. J.}} 
\newcommand{\apj}{    {\it Astrophys. J.}}
\newcommand{\apjl}{   {\it Astrophys. J. Lett.}}
\newcommand{\apss}{   {\it Astrophys. Space Sci.}} 
\newcommand{\bain}{   {\it Bulletin of the Astronomical Institutes of the Netherlands.}} 
\newcommand{\cjaa}{   {\it Chin. J. Astron. Astrophys.}} 
\newcommand{\gafd}{   {\it Geophys. Astrophys. Fluid Dyn.}}
\newcommand{\grl}{    {\it Geophys. Res. Lett.}}
\newcommand{\ijga}{   {\it Int. J. Geomagn. Aeron.}}
\newcommand{\jastp}{  {\it J. Atmos. Solar-Terr. Phys.}} 
\newcommand{\jgr}{    {\it J. Geophys. Res.}}
\newcommand{\mnras}{  {\it Mon. Not. Roy. Astron. Soc.}}
\newcommand{\na}{     {\it New Astronomy}}
\newcommand{\nat}{    {\it Nature}}
\newcommand{\pasp}{   {\it Pub. Astron. Soc. Pac.}}
\newcommand{\pasj}{   {\it Pub. Astron. Soc. Japan}}
\newcommand{\pre}{    {\it Phys. Rev. E}}
\newcommand{\solphys}{{\it Solar Phys.}}
\newcommand{\sovast}{ {\it Soviet  Astron.}} 
\newcommand{\ssr}{    {\it Space Sci. Rev.}}
\newcommand{\caa}{    {\it Chinese Astron. Astrohpys.}} 
\newcommand{\apjs}{   {\it Astrophys. J. Suppl.}}

\def\UrlFont{\sf}

\newcommand{\bs}[1]{\boldsymbol{#1}}
\newcommand{\bn}{\boldsymbol{\nabla}}
\newcommand{\rgas}{\mathcal{R}}
\newcommand{\eref}[1]{Eq. \eqref{#1}}
\newcommand{\fref}[1]{Fig. \eqref{#1}}
\newcommand\encircle[1]{%
  \tikz[baseline=(X.base)] 
    \node (X) [draw, shape=circle, inner sep=0] {\strut #1};}
\newcommand{\Alfven}{Alfv\'{e}n } 
\newcommand{\Alfvenic}{Alfv\'{e}nic }
\newcommand{\size}{0.75}
\newcommand\measureISpecification{4ex}% not defined in mwe
\newcommand{\ctab}[1]{\raisebox{\dimexpr \measureISpecification/2 -.748ex}{#1}}% vertically centers numbers
\newcommand{\si}[1]{\;\rm{#1}}
\newcommand{\mfig}[4]{
  \begin{figure}
  \begin{center}
  \includegraphics[width=#1\linewidth]{#2}
  \caption{#3}
  \label{#4}
  \end{center}
  \end{figure}}


\begin{document}

\baselineskip=18pt

\setcounter{secnumdepth}{3}
\setcounter{tocdepth}{3}

\setcounter{chapter}{1}

%%%%%%%%%%%%%%%%%%%%%%%%%%%%%%%%%%%%%%%%%%%%%%%%%%%%%%%
% START COPYING HERE
%%%%%%%%%%%%%%%%%%%%%%%%%%%%%%%%%%%%%%%%%%%%%%%%%%%%%%%

\chapter{The Dynamics of Straight Jets in a Solar Atmosphere}

%------------------------------------------------------------------------------
\section*{Abstract}
%------------------------------------------------------------------------------

\let\thefootnote\relax\footnotetext{

This chapter is based on the following refereed journal article:
\begin{itemize}
\item Barbulescu, M., Erd\'elyi, R. (2018); Magnetoacoustic Waves and the Kelvin-Helmholtz Instability in a Steady Asymmetric Slab. I: The Effects of Varying Density Ratios, \solphys, Volume 293, Issue 6
\end{itemize}
}
Todo
%------------------------------------------------------------------------------
\section{Introduction}
\label{sec:c2intro}
%------------------------------------------------------------------------------

%------------------------------------------------------------------------------
\section{Numerical Recipe}
\label{sec:Numerical}
%------------------------------------------------------------------------------
We present numerical simulations of solar jets in a 2D setting. We use the grid-adaptive MPI-AMRVAC version 2.0 software developed at KU Leuven \citep{Xia_2017,Keppens_2012,Porth_2014}, which solves the MHD equations in the following form (included gravity as an additional source term),
\begin{equation} \label{eq1}
\partial_t \rho + \bn \cdot (\bs{v} \rho) = 0 ,
\end{equation}   
\begin{equation}\label{eq2}
\partial_t (\rho \bs{v})  + \bn \cdot ( \bs{v} \rho \bs{v} - \bs{BB}) + \bn p_{tot} = \rho \bs{g},
\end{equation}
\begin{equation}\label{eq3}
\partial_t e + \bn \cdot (\bs{v} e - \bs{BB}\cdot\bs{v}+\bs{v}p_{tot}) = \rho \bs{g} \cdot \bs{v} ,
\end{equation}
\begin{equation}\label{eq4}
\partial_t \bs{B} + \bn \cdot (\bs{vB}-\bs{Bv} ) = 0 .
\end{equation}
This is the full system of the MHD equations, with the following variables: time ($t$), mass density ($\rho$), plasma velocity $\bs{v}$, magnetic field ($\bs{B}$). The total pressure ($p_{tot}$) and total energy density ($e$) is given respectively by,
\begin{equation}
p_{tot} = p+\frac{\bs{B}^2}{2},
\end{equation}
\begin{equation}
e = \frac{p}{\gamma-1}+ \frac{\rho \bs{v}^2+\bs{B}^2}{2},
\end{equation}
where $\gamma=5/3$ denoted the ratio of specific heats. Solar gravitational acceleration is given by,
\begin{equation}
\bs{g}=-\frac{GM_{\astrosun}}{r^2}\hat{r},
\end{equation}
where $G$ is the gravitational constant and $M_{\astrosun}$ denotes the solar mass. 
%------------------------------------------------------------------------------
\section{Atmospheric Equilibrium}
\label{sec:atmos_equil}
%------------------------------------------------------------------------------
The transition region is located at $2$ Mm where the temperature smoothly connects an $8000$ K photosphere to $1.5$ MK corona as shown in \fref{atoms_profile}. The temperature of the atmosphere as a function of height is mathematically represented as,   
\begin{equation}\label{te_pro}
T_0(y>0) = T_{ch}+\frac{T_{c} - T_{ch}}{2} \left[ \tanh \left( \frac{y-y_{tr}}{w_{tr}} \right)+1 \right],
\end{equation}
where $y$ is the vertical position, $T_0(0)=T_{ch}=8\times10^3 \ \rm{K}$ ($T_{c}=1.8\times10^6 \ \rm{K}$) is the chromospheric (coronal) temperature and $y_{tr}=2$ Mm ($w_{tr}=0.02$ Mm) is the TR height (width). As there is a uniform vertical magnetic, hence $\bs{J} \times \bs{B}=0$ and magnetohydrostatic equilibrium is achieved with,
\begin{equation}
\frac{dp}{dy} = - \rho g.
\end{equation}
Temperature and pressure are related through the ideal gas law,
\begin{equation}
p = \frac{ \rho \rgas T}{M},
\end{equation} 
where $M$ is the mean atomic weight, $T$ is the temperature and $\rgas$ is the universal gas constant. Therefore, one can obtain the $\rho$ and $p$ in the following form,  
\begin{equation}\label{p_pro}
p(y) = p_0 \exp \left( - \int_0^y  \frac{1}{H(y')} dy' \right), 
\end{equation} 
\begin{equation}\label{rho_pro}
\rho(y) = \rho_0 \frac{T_0}{T(y)} \exp \left( \int_0^y \frac{1}{H(y') }dy' \right),
\end{equation}
where $\rho_0$ and $p_0$ are the initial mass density and pressure equilibrium, respectively, at the lower boundary of the computational domain, and the pressure scale heights are given by,
\begin{equation}
H(y) = \frac{\rgas T(y)}{Mg}.
\end{equation}
Using the temperature profile given by \eref{te_pro} and taking $\rho_0 = 2.34e-4\times10^{-4}\;\rm{kg}\;\rm{m^{-3}}$ we construct pressure and mass density as function of height by means of \eref{p_pro} and \eref{rho_pro}.
%------------------------------------------------------------------------------
\section{Numerical Method}
\label{sec:Numerical_Method}
%------------------------------------------------------------------------------
We use the open source MPI-AMRVAC software version 2.0 \cite{Xia_2017} to simulation solar jets in a simple solar atmosphere. We set the domain size to $50$ Mm $\times$ $30$ Mm with a level one resolution of $32$ $\times$ $24$ (giving a physical resolution of $\sim$ $1.56$ Mm $\times$ $1.25$ Mm). This coarse level one resolution was chosen to allow the dissipation of shocks as they travel towards the boundaries. To accurately capture the sharpness of the TR we applied 7 levels of AMR giving a spatial resolution of $\sim$ $12$ km$\times$ $10$ km. We use unidirectional grid stretching in the horizontal direction, where from the origin the grid cells change by a constant factor of $1.1$ from cell to cell. Due to the choice of discretisation scheme for the boundary conditions, we employ ghost cells of 2 grid layers encasing the physical domain in each direction, which have a physical size of $3 \ \rm{Mm}$. The numerical scheme applied for spatial discretisation is HLL (Harten-Lax-van Leer) \cite{hll_1983} and a third order \u{C}ada limiter \citep{CADA20094118} for the time discretisation. We use CFL number of $0.8$ and GLM-MHD method to maintain $\bs{\nabla} \cdot \bs{B}=0$ \citep{DEDNER2002645}. The initial magneto-hydrostatics stratification places the TR at $2$ Mm where the temperature smoothly links $8000$ K chromosphere to a $1.8$ Mm corona (see Fig. 1). For the left and right boundary we utilise a periodic boundary condition. In the  ghost cells of the lower boundary we fix $\rho$, $e$ and $\bs{B}$ to their initial values. In the ghost cells for the upper boundary the values for $\rho$, $e$ are determined by the gravitational stratification and $\bs{B}$ was extrapolated assuming zero normal gradient. For both upper and lower boundary we take an antisymmetric boundary for the velocity components.
%fffffffffffffffffffffff
\mfig{1}{figures/combine_1.png}{Example of grid at $t=0$.}{atoms_profile}
%fffffffffffffffffffffffff
%------------------------------------------------------------------------------      
\subsection{Driver}
\label{subsec:driver}
%------------------------------------------------------------------------------
To drive the jet we us a momentum pulse at the base of computational domain for a specified period of time (ranging from $50-300s$). The jet is launched symmetrically by a driver in the center of the computational domain which varies both spatially and temporally. In the x-direction the jet vertical velocity is Gaussian with the FHWM of $350 \ \rm{km}$ ($j_w$) and as the jet evolves in time it will reach a switch off phase in which it will shut off with a hyperbolic tangent,
\begin{equation}
v_{j}(x) = -\frac{A}{2} \left( \tanh \left( \frac{\pi (t-t_{d})}{t_d}+ \pi \right) +1 \right) \exp \left( - \left(\frac{x-x_0}{\Delta x} \right)^2  \right),
\end{equation}    
where $v_j$ is the velocity of the jet, $A$ is the amplitude of the driver, $t$ is time, $t_{d}$ is the time for $v_j=0$ , $P$ is the period of the driver and $x_0$ is central location of the jet injection. The driver the width of the Gaussian is determined by $\Delta x$ which is the FWHM based of the jet width (see \fref{fig4}),
\begin{equation}
\Delta x = \dfrac{j_w}{4 \sqrt{2 \log{2}}},
\end{equation}  
%fffffffffffffffffffff
%\begin{figure}
%\hspace{-1.5cm}
%\captionsetup[subfigure]{labelformat=empty}
%\subfloat[]{\includegraphics[width=0.6\linewidth]{figures/driver_dt_off.png}} 
%\subfloat[]{\includegraphics[width=0.6\linewidth]{figures/driver_dx.png}} 
%\caption{Example of driver velocity with $A=60$ km s$^{-1}$, $P$ and switch off time of $300$ s (LHS) and Gaussian distribution for $A=60$ km s$^{-1}$ and $j_w=375$ km marked by red points (RHS).}
%\label{fig4}
%\end{figure}
%ffffffffffffffffffffff
\section{Simulation Results}
The way the jet is excited is analogous to jets created in a laboratory, in the sense we artificially created 'nozzle' in which drive our material into the atmosphere. Akin to laboratory jets our jet has high density ratio ($\eta_j >1$) and has Mach numbers ($\sim 1-3$). When describing a jets these parameters are key for defining its morphology and structure: $v_j$, $\eta_j = \frac{\rho_j}{\rho_a}$, $K = \frac{p_{d}}{p_a}$, where $\rho_a(p_a)$ is the ambient density(pressure) and $p_{d}$ is the pressure at the driver. $\eta_j$ measures the density contrast of the jet itself and the ambient medium it propagates through. If $\eta_j>1(\eta_j<1)$ then a jet is labeled to as heavy(light). For $K$ with denotes our jet to ambient pressures ratio determine whether the jet is under-expanded ($K>1$) or over-expanded ($K<1$). In our case we are dealing with the expected dynamics of a heavy under expanded-jet. In a paper by \cite{Norman1982} they describe the structures that occur in HD super sonic jets. The jets described in the paper are generated from a high pressure reservoir contain in the nozzle and internal structure of these jets is depicted in \fref{cartoon_jet_waves}. In the simulations shown in \fref{jet_simulation} we can see areas of enhanced brightness along the central axis of the jet, which we will refer to as knots. This phenomena occurs in laboratory jets \textit{e.g.} \citep{Ono2014,Edgington-Mitchell2014,Menon2010} and in astrophysical jets \textit{e.g.} \cite{Blandford2019,Belan2011,DeGouveiaDalPino2005,Birkinshaw1996}. In our case two possible mechanisms for explaining the knotted structures in the jet simulations: \\
\par (1) The knots could be created by the interaction of the pressure forces occurring inside and outside the jet. If we tracked a small portion of the jet as it evolves upwards it would undergo a series of expansion and contractions. Initially the jet pressure is greater than the surrounding atmosphere, thus as it rises it expands. However, this expansion lowers the jet pressure and as jet diameter increases it bends surrounding magnetic field lines. This increases the magnetic tension until it's greater than the jet pressure, causing the jet boundary to reconverges towards the jet center. As the jet diameter shrinks the jet pressure responds by increasing and becomes the dominant force pushing the jet boundary outwards. Thus, the cycle repeats as the jet tries to reach an equilibrium with its environment. \\
\par (2) When a jet is going supersonic it will have a myriad of internal structures cause by shock waves \citep{Norman1982}. A schematic overview of these structures are given in Fig. \eqref{cartoon_jet_waves}. In the simulation the jet speeds reach roughly Mach $\sim 1-3$ of the local sound speed and the body of the jet itself has plasma beta $\sim 2-10$, therefore the MHD waves do not play the main role in internal structure of the jet and will be similar to dynamics of an HD supersonic jet. The notable features of interest in a supersonic jet is the osculating jet boundary and cross pattern occurring inside the jet. The jet boundary undulates as the gas periodically over expands and collapses in as it tries to reach and equilibrium with the ambient atmospheric pressure. The jet repeatedly overshoots the equilibrium position because the effects of the boundary are communicated to the interior of the jet by sound waves, which are traveling slower than supersonic flow of the jet. This leads the to lowest\textbackslash highest points of pressure in the jet being out of phase with the highest\textbackslash lowest jet diameter (see Fig. \eqref{cartoon_jet_waves}). As the jets pressure is greater than ambient atmospheric pressure the dynamics of the jet are akin to an under-expanded jet as shown in \cite{Norman1982,Edgington-Mitchell2014}. The crisscross shock pattern is created by a series of shock and expansion structures that can create these knots. An expansion fan (blue dashed lines in Fig. \eqref{cartoon_jet_waves}) forms at the base of the due to the difference in pressure between the jet and the ambient atmosphere. This expansion fan causes an outward flow making the jet enlarge. The Mach lines of the expansion waves reflects of the jet boundary inwards towards the jet center in the form of compression waves and a compression fan due pressure continuity (red lines in Fig. \eqref{cartoon_jet_waves}). The compression waves reflected at a near constant angle from the jet boundary and as this boundary is curved the Mach lines of the compressions waves have a tendency to converge into a conical shock wave before reaching the center of the jet. This incident shock either goes under a regular reflection or becomes a Mach disk depending on the angle between the incident shock and the central jet axis, for small and large angles respectively (see black lines Fig. \eqref{cartoon_jet_waves}). As the flow passes through this shock it will increase the pressure in the jet. When the reflected shock reaches the jet boundary it knocks the boundary outwards, creating an expansion fan and thus allows the process to be repeated.
\par A schematic diagram of basic structure of a jet is given by \fref{ssj}. When a HD super sonic jet moves through the ambient medium, a bow shock is formed is formed in front of the advancing head of the jet and at the top of jet itself there is a Mach disk which is usually much stronger than the bow shock \citep{Chakrabarti1988}. The region close to advancing head of the jet is referred as working surface of the jet. Between the bow shock and the jet there are two other distinct regions the cocoon and screen which are separated by a contact discontinuity. The cocoon is the shocked jet medium and screen is the shocked ambient medium. The velocity of the jet head $v_h$ by comparing the momentum fluxes the jet and the external matter at the head of the jet and is estimated by the following,
\begin{equation}\label{v_j_speed}
v_h \approx \dfrac{v_j}{1+ \left( \dfrac{\rho_e}{\rho_j} \right)^{\sfrac{1}{2}}} 
\end{equation}
From \eref{v_j_speed} we can see for light jets ($\eta_j<1$) this means that speed of the head of the jet will be slower than the jet speed itself, this leads to a pile up of material which forms the cocoon. For heavy jets ($\eta_j>1$), the head of the jet will advance with speeds comparable to the jet itself and thus leads to much less back flow. This is why there is very little back flow in these jet The bow shock can be seen early on in the simulation and we see that the temperature increase at the head of the jet.
\par We have developed software to track the max jet heights (!get image example!) as it evolves and to measure the widths of the jet every megameter. The software tracts the jet using the tracer quantity that was set in simulation. A tracer tracks the density evoultion of th simaultion as described in \cite{Porth_2014}, we assign the jet with a vlaue 100 and everthing else as 0. We then set a threshold of $15\%$ of the jet value and convert into binary image (1 belong to jet and 0 not belonging to jet). To find the edges we take a horizontal slice every mega meter and look for changes and peak of the jet by taking the highest index value in the vertical. From this we create a data set which is used to produce Figs. (\ref{max_h_vs_b}-\ref{hight_dt}). We study how the different driver times, magnetic and drive amplitude effect the max height and average widths of the jet to see what effect this has on the jet morphology. From \fref{max_h_vs_b} it shows that strength of the magnetic has minimal impact on the max height reach by the jet. It important to note that the magnetic field may play a significant role in max jet heights if jets direction was not in-line with the magnetic field. As we can see from the color bands which corresponds to max amplitude reached by driver, this suggest that this is a key parameter for the max jet height. In \fref{max_h_vs_a} we conclude that max height is the jet mainly determined by the max amplitude reached by the driver with a parabolic trend. We observe that for our driver which is a momentum pulse we don't not get a linear trend between max jet height and pulse strength as reported in \cite{Singh2019}. \cite{Singh2019} used a a single instantaneous pressure pulse placed $1.8$ Mm. This suggests that height at which the driver is placed, its amplitude and its duration of driving is driven important to jet evolution. For example the longer distance that a pulse has to travel through the chromosphere to the TR the stronger the shocks the grow and the higher the jets will reach \citep{Shibata1982}. In \fref{max_h_vs_a} wee see that the driver time has some impact on max height. It shows that for short driver time of $50$ s there is a reduction in heights reached by the jets and there is no significant different between jet heights for a 200 and 300 s driver. We can see that the lines are banded together with matching line styles which again corresponds to matching velocities of the driver and reinforces that initial velocity is the main factor in determining max jet height if jet direction is inline with the magnetic field direction.
\par In \fref{mean_w_vs_A} we observe that the initial amplitude has little effect on jet widths. We a see pairing between drivers with $200$ and $300$ s, indicated by matching line styles, which is expected based on \fref{mean_w_vs_A}. The only parameter this doesn't apply is for $B=20 G$, this is due to the jet being weakly confined by the magentic feild. In \fref{mean_w_vs_dt} shows a linear trend between driving time, this is because with a longer driver more material will be supplied to the jet. There can also be a pile up in the jet beam as falling material interacts with raising material causing the jet to slighly ballon. We can see that the lines are grouped together based on their magnetic field strength represent by matching colors. This is an expected result as for weaker magnetic fields we get much larger jets widths, this is due to the magnetic field not being strong enough to compress the jet as it raises. Where as where the magnetic feild is stronger we observe that the jet widths are significantly reduced. This interpretation is evidenced by \fref{mean_w_vs_b} where we see larger jet widths for weaker magnetic field. We observe that they are group by there driver times as they have matching line styles. In \fref{hight_dt} we take use a the following paramters $P=300$ s, $B=50$ G and $A=60$ km s$^{-1}$ and run it with a much higher cadence in output. Each line in this \fref{hight_dt} represents a different height in the jet. We clearly the boundary of the jet are osculating over time. We don't see this phenomenon in simulations with a jet driven with a single instant pulse (!double check this claim! such as \citep{kuz2017ApJ,Singh2019}) we don't observe oscillations in there jet simulations, even though clearly in observations solar jets they oscillate (get refs). In Figs. (\ref{A20}-\ref{A80}) we have plotted the trajector of the max height of the jet over time, each Fig. is a displays all jets with the same driving amplitude. We find that our jets have a asymmetric parabolic path as reported in \cite{Singh2019}. We believe this is due to a piling of jet material as the jet begins to falls as falling matter will interact with raising matter. This will slow the jets decent. It is possible the magnetic field could impede the jets decent if the direction of falling material was at an angle with respect the magnetic field. We can see that heights reached are within the limits of what is expected for spicules. 
%fffffffffffffffffffffffffff
\mfig{0.8}{figures/jet_diagram.eps}{schematic overview of the structure of a under-expanded supersonic HD jet.}{cartoon_jet_waves}
%fffffffffffffffffffffffffff
\mfig{0.8}{figures/jet_P300_B60A_60_t_1_10_13_19.png}{Simulation of jet in Stratified atmosphere with uniform magnetic field in vertical direction. This is the ``standard" jet with parameter $P=300$ s, $B=60$ G, $A=60$ km s$^{-1}$. Top, middle and bottom panels show density, temperature and numerical Schlieren respectively.}{jet_simulation}
%fffffffffffffffffffffffffff
\mfig{0.5}{figures/diagram_of_ssj.png}{A schematic diagram giving an overview of the expected structures found in an HD super sonic jet dran in frame of the bow shock. Diagram taken from \cite{Chakrabarti1988}.}{ssj}
%fffffffffffffffffffffffffff
\mfig{1}{figures/jet_P300_B60A_60T_0103.png}{todo.}{j_track_example}
%fffffffffffffffffffffffffff
\mfig{1}{figures/jet_traj.png}{todo.}{pscan_traj}
%fffffffffffffffffffffffffff
\mfig{1}{figures/h_and_width_multi_image.eps}{todo.}{pscan_full}
%fffffffffffffffffffffffffff
\mfig{1}{figures/parameter_scan_1.png}{todo.}{pscan_den_plot}
\mfig{1}{figures/paramter_scan_2.png}{todo.}{pscan_den_plot}
%fffffffffffffffffffffffffff
\mfig{1}{figures/combine_zoom.png}{todo.}{jet_head}

%%fffffffffffffffffffffffffff
%\mfig{0.8}{figures/maxhb.png}{Max height reached for varying magnetic field strength.}{max_h_vs_b}
%%fffffffffffffffffffffffffff
%\mfig{0.8}{figures/maxha.png}{Max height reached for varying velocity field strength.}{max_h_vs_a}
%%fffffffffffffffffffffffffff
%\mfig{0.8}{figures/maxhdt.png}{Max height reached for varying driver time.}{max_h_vs_dt}
%%fffffffffffffffffffffffffff
%\mfig{0.8}{figures/mean_w_vs_A.png}{Mean width against different amplitudes for each jet in parameter scan.}{mean_w_vs_A}
%%fffffffffffffffffffffffffff
%\mfig{0.8}{figures/mean_w_vs_P.png}{Mean width against different driver times field strengths for each jet in parameter scan.}{mean_w_vs_dt}
%%fffffffffffffffffffffffffff
%\mfig{0.8}{figures/mean_w_vs_B.png}{Mean width against different magnetic field strengths for each jet in parameter scan.}{mean_w_vs_b}
%%fffffffffffffffffffffffffff
%\mfig{0.8}{figures/P300_B50_A60.png}{Widths of jet every Mm for P=300s, B=50G and A=50 km s$^-1$.}{hight_dt}
%%fffffffffffffffffffff
%\mfig{0.8}{figures/P300_B100_A20.png}{Plot of the max height evolution over time for $A=20$ km s$^{-1}$. $P$, $B$ and $A$ are scaled in s, G and km s$^{-1}$, respectively.}{A20}
%%fffffffffffffffffffffffffff
%\mfig{0.8}{figures/P300_B100_A40.png}{Same as \fref{A20}, but for $A=40$ km s$^{-1}$.}{A40}
%%fffffffffffffffffffffffffff
%\mfig{0.8}{figures/P300_B100_A60.png}{Same as \fref{A60}, but for $A=60$ km s$^{-1}$.}{A20}
%%fffffffffffffffffffffffffff
%\mfig{0.8}{figures/P300_B100_A80.png}{Same as \fref{A20}, but for $A=80$ km s$^{-1}$.}{A80}
%%fffffffffffffffffffffffffff
%------------------------------------------------------------------------------
\section{Applications}
\label{sec:c2app}
%------------------------------------------------------------------------------


%------------------------------------------------------------------------------
\section{Summary and Discussion}
\label{sec:c2discussion}
%------------------------------------------------------------------------------


%%%%%%%%%%%%%%%%%%%%%%%%%%%%%%%%%%%%%%%%%%%%%%%%%%%%%%%
% STOP COPYING HERE
%%%%%%%%%%%%%%%%%%%%%%%%%%%%%%%%%%%%%%%%%%%%%%%%%%%%%%%

\bibliographystyle{plainnat}
\bibliography{references}

\end{document}
