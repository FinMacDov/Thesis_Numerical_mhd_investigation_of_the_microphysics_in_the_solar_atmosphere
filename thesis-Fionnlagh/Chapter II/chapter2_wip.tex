\documentclass[12pt]{ociamthesis}

\usepackage{amssymb}
\usepackage{titlesec}
\usepackage{amsmath}
\usepackage{float}
\usepackage{graphicx}
\usepackage{caption}
\usepackage{subfig}
\usepackage{graphicx}
\usepackage{xcolor}
\usepackage[round]{natbib}
\usepackage[section]{placeins}
\usepackage{mathrsfs}
\usepackage{bm}
\usepackage{stmaryrd}
\usepackage[utf8]{inputenc}
\usepackage{bibentry}
\usepackage{wasysym}
\usepackage{xfrac}

\usepackage{geometry}
 \geometry{
 a4paper,
 left=40mm,
 right=30mm,
 top=30mm,
 bottom=30mm
 }

\definecolor{theblue}{HTML}{0000CD}

% disable this package for printed version
\usepackage[colorlinks=true, linktocpage=true, allcolors=theblue]{hyperref}

\titleformat{\chapter}[display]
  {\bfseries\Large}
  {\filright\MakeUppercase{\chaptertitlename} \Large\thechapter}
  {1ex}
  {}
  [\vspace{1ex} \hrule \vspace{1pt} \hrule]

\newcommand{\adv}{    {\it Adv. Space Res.}} 
\newcommand{\araa}{    {\it Annual Review of Astron and Astrophys.}} 
\newcommand{\annG}{   {\it Ann. Geophys.}} 
\newcommand{\aap}{    {\it Astron. Astrophys.}}
\newcommand{\aaps}{   {\it Astron. Astrophys. Suppl.}}
\newcommand{\aapr}{   {\it Astron. Astrophys. Rev.}}
\newcommand{\ag}{     {\it Ann. Geophys.}}
\newcommand{\aj}{     {\it Astron. J.}} 
\newcommand{\apj}{    {\it Astrophys. J.}}
\newcommand{\apjl}{   {\it Astrophys. J. Lett.}}
\newcommand{\apss}{   {\it Astrophys. Space Sci.}} 
\newcommand{\bain}{   {\it Bulletin of the Astronomical Institutes of the Netherlands.}} 
\newcommand{\cjaa}{   {\it Chin. J. Astron. Astrophys.}} 
\newcommand{\gafd}{   {\it Geophys. Astrophys. Fluid Dyn.}}
\newcommand{\grl}{    {\it Geophys. Res. Lett.}}
\newcommand{\ijga}{   {\it Int. J. Geomagn. Aeron.}}
\newcommand{\jastp}{  {\it J. Atmos. Solar-Terr. Phys.}} 
\newcommand{\jgr}{    {\it J. Geophys. Res.}}
\newcommand{\mnras}{  {\it Mon. Not. Roy. Astron. Soc.}}
\newcommand{\na}{     {\it New Astronomy}}
\newcommand{\nat}{    {\it Nature}}
\newcommand{\pasp}{   {\it Pub. Astron. Soc. Pac.}}
\newcommand{\pasj}{   {\it Pub. Astron. Soc. Japan}}
\newcommand{\pre}{    {\it Phys. Rev. E}}
\newcommand{\solphys}{{\it Solar Phys.}}
\newcommand{\sovast}{ {\it Soviet  Astron.}} 
\newcommand{\ssr}{    {\it Space Sci. Rev.}}
\newcommand{\caa}{    {\it Chinese Astron. Astrohpys.}} 
\newcommand{\apjs}{   {\it Astrophys. J. Suppl.}}

\def\UrlFont{\sf}

\newcommand{\bs}[1]{\boldsymbol{#1}}
\newcommand{\bn}{\boldsymbol{\nabla}}
\newcommand{\rgas}{\mathcal{R}}
\newcommand{\eref}[1]{Eq. \eqref{#1}}
\newcommand{\fref}[1]{Fig. \eqref{#1}}
\newcommand\encircle[1]{%
  \tikz[baseline=(X.base)] 
    \node (X) [draw, shape=circle, inner sep=0] {\strut #1};}
\newcommand{\Alfven}{Alfv\'{e}n } 
\newcommand{\Alfvenic}{Alfv\'{e}nic }
\newcommand{\size}{0.75}
\newcommand\measureISpecification{4ex}% not defined in mwe
\newcommand{\ctab}[1]{\raisebox{\dimexpr \measureISpecification/2 -.748ex}{#1}}% vertically centers numbers
\newcommand{\si}[1]{\;\rm{#1}}
\newcommand{\mfig}[4]{
  \begin{figure}
  \begin{center}
  \includegraphics[width=#1\linewidth]{#2}
  \caption{#3}
  \label{#4}
  \end{center}
  \end{figure}}
\newcommand{\kms}{~\rm{km ~s^{-1}}}
\newcommand{\kgm}{~\rm{kg ~m^{-3}}}
\newcommand{\np}{\\ \\}
\begin{document}

\baselineskip=18pt

\setcounter{secnumdepth}{3}
\setcounter{tocdepth}{3}

\setcounter{chapter}{1}

%%%%%%%%%%%%%%%%%%%%%%%%%%%%%%%%%%%%%%%%%%%%%%%%%%%%%%%
% START COPYING HERE
%%%%%%%%%%%%%%%%%%%%%%%%%%%%%%%%%%%%%%%%%%%%%%%%%%%%%%%

\chapter{The Dynamics of Filed-aligned Jets in a Solar Atmosphere}
%-------------------------------------------------------------------------------
\let\thefootnote\relax\footnotetext{

This chapter is based on the following refereed journal article:
\begin{itemize}
\item Barbulescu, M., Erd\'elyi, R. (2018); Magnetoacoustic Waves and the Kelvin-Helmholtz Instability in a Steady Asymmetric Slab. I: The Effects of Varying Density Ratios, \solphys, Volume 293, Issue 6
\end{itemize}
}
%------------------------------------------------------------------------------
\section{Introduction}
\label{sec:c2intro}
%------------------------------------------------------------------------------
Despite the numerous numerical simulation, very few papers simulating solar jets include an investigation of a parameter space.
%------------------------------------------------------------------------------      
\section{Parameter Space}
\label{subsec:paramater_space}
%------------------------------------------------------------------------------
The parameter space chosen for investigation is $P=50,~200$ and $300~\rm{s}$ (driver times), $B_y=20,~40,~60,~80,~100~\rm{G}$ (magnetic filed strength) and $A=20,~40,~60$ and $80 \kms$ (initial amplitudes). These values are based around typical values for classical spicules. The driver times based on the $5$ minute oscillation of the p-mode \citep{Leighton1962ApJ135474L}, as they are potential driver for multiple solar features (spicules, mottles  and dynamic fibrils) \citep{Pontieu2004Natur}. The magnetic field strength are in the ranges observed values for spicules which is consistent with spectropolarimetric estimates for limb spicules \citep{centeno2010, suarez2015}. The range of velocities are chosen based on the energy need to lift near photospheric mass ($?\kgm$) and observed speeds seen in spicules which have a range of $?-?\kms$ !Need to get a group of citations!. \np
% may include a summarry table in the intro that is used in presenations, then we could refere to it in text.
%
%fffffffffffffffffffffffffff
\mfig{1}{figures/jet_P300_B60A_60T_0103.png}{Example of jet tracking software. Solid blue dots mark the jet edges and the yellow triangle marks the jets apex.}{j_track_example}
%fffffffffffffffffffffffffff
To understand the jet dynamics we need to quantify important aspects of the jet, e.g. speed, life time, trajectory and jet boundary deformation. This is achieved by temporally tracking the jets width (solid blue dots) and apex (yellow triangle), using jet tracking software (see \fref{j_track_example}). The jet tracking software we developed uses the tracer quantity developed by \cite{Porth_2014}, to locate each data point. The tracer follows the mass flow of jet as it enters the computational domain and is assigned a non-physical value of $100$ (jet value). To separate jet pixels from non-jet pixels we use a threshold of tracer value being greater than $15\%$ of the jet value to convert into binary array where $1$ belongs jet cell and $0$ is assumed to be ambient medium. At specified height intervals (our case every $1 \rm{Mm}$) horizontal slices are taken to identify jet side boundaries and the the apex is located by finding the highest vertical index belonging to the jet.
\subsection{Jet Tracking}
\subsection{Jet Trajectories}
\subsection{Effects on Jet Apex and Widths}
%fffffffffffffffffffffffffff
\mfig{1}{figures/jet_traj.png}{The effect of the parameter space on the jet trajectory. Each panel is grouped together by initial amplitude and from top left to bottom right they are group by $A= 20, ~40, ~60$ and $80 \kms$. Units are omitted from legend for readability, the units for $P,~B$ and $A$ are $\rm{s}~, \rm{G}$ and $\kms$, respectively.}{pscan_traj}
%fffffffffffffffffffffffffff
% !possibley add figs for these!!!!!!!!!!
By temporally tracking (see \fref{pscan_traj}) the apex data we seen that the jets have parabolic trajectories. These trajectories have been compared to expected ballistic trajectories and found to be non-ballistic. Solar jet-like features such as spicules, mottles and dynamic fibrils have been shown to have non-ballistic parabolic trajectories e.g. mottles, dynamic fibrils and TI spicules \citep{ Hansteen2006ApJ,Rouppe2007ApJ660L169R,Pontieu2007PASJ}, which is in agreement with our simulations. In \fref{pscan_traj} exhibits that for driver $<40 \kms$ then we are reaching the limits of spicules heights. In each panel there is a subgroup reaching lower heights due reduced driving time. \np  
%
These jet-like events are predicted to have linear correlation between deceleration and maximum velocity and if the jet is initiated with shock waves the relationship is defined by the following \citep{Heggland2007ApJ6661277H},
\begin{equation}
d = \frac{v_{max}}{P_{w}/2},
\end{equation}
where $d$ is declaration, $v_{max}$ is the maximum velocity during the lifetime of the jet and $P_{wave}$ is the period of the wave. In !fig (todo)! !!1! we show that our jets have a linear relationship, by that found in \citep{Heggland2007ApJ6661277H}, but this is due to the nature in which these jets are driven as it is not a short pulse like event, the parameter $P_{w}=P$, which in our cases is linked to the life time of the jet rather than the period of a wave. While we find that we match observation with a linear correlation, its not with the same gradient of line as theoretically predicted for these jet-like events. \np
%fffffffffffffffffffffffffff
\mfig{1}{figures/test_combine_image.png}{The effects of the parameter space on both the max apex (panels on the left) and mean jet widths (panels on right). Units are omitted from legend for readability, $P,~B$ and $A$ are in $\rm{s},~\rm{G}$ and $\kms$, respectively}{pscan_full}
%fffffffffffffffffffffffffff
To understand the global impact of the key parameters investigated, we compared how each parameter effects the max apex height and average width of the jet over its entire lift cycle. In \fref{pscan_full}, the panels to the left, top to bottom show the influence of changing $B$, $P$ and $A$ on the maximum apex reached during the jets lifetime. Varying the magnetic field strength has little impact on the heights reached by a jet as all lines have shallow gradients. This parameter could potentially have a larger effect of jet heights if the flow of the jet is not aligned with he magnetic field. This particular aspect is investigated in chapter !!(?)!.  The driver time has small effect on the jet heights, there is a slight decrease in height for $p=50$, but then peaks for $P=200~\rm{s}$ and $P=300~\rm{s}$. This suggested to transport mass from lower in the atmosphere through the solar atmosphere then a instantaneous pulse will not be sufficient, there needs to be longer driver time. In both previously mentioned parameters, they are grouped by colors and line styles for $B$ and $P$ scans, which corresponds to matching initial amplitudes. This alludes to initial amplitude being the most important parameter for determining jet heights, as indicated by the last leftmost panel in \fref{pscan_full}. The maximum height reached by a solar jet is mainly determined by the initial Amplitude of the driver and has a parabolic trend. As the magnetic field plays small role in jet heioghts and as the driver time flattens at $200~\rm{s}$, our best chance of observing a jet is with longer driving times, therefore we can model that jet heights with a simple power law of the form, 
\begin{equation}
h_{max} = C v_j^{n},
\end{equation} 
by collapsing the data by taking a average at each velocity data point and then fit an optimal curve using least squares obtaining values of $C= 10^{-2.21}$ and $n= 1.72$ as shown in leftmost lowest panel in \fref{pscan_full} as a solid red line with solid data markers. The light blue shading gives and error range of . Visually and from the power law, both clearly show that for our choice of driver it doesn't have a linear trend between apex jet height and pulse strength, which contradicts results in \cite{Singh2019}. We should take note that in \cite{Singh2019} an instant pressure pulse close to the TR was used as a driver, which may not be directly comparable to our study. \np     
%
On the panels to the right of \fref{pscan_full} shows the effect of the parameter space on the mean widths. The means widths are calculated measure the width of the jet for every $1~\rm{Mm}$ over the entire life time of the jet. This metric is not perfect, but gives useful overview of the impact of these parameters that make physical sense.     
% need to edit ---------------------
The effect of the parameter space on the jet widths are show on the panels on the right hand side \fref{parameter_scan_lines}. To obtain the width data, it is measured at every megameter the jet reaches and the mean value is calculated over the whole life of the jet. This is by no means a perfect method, but does give useful insight into the global deformation of the jet boundary and returns result that make physical sense. For example by varying the magnetic field strength we observe that the weaker the field strength the less collimated the jet is. This is because as the jet rises it has a higher pressure than the ambient medium cause it to expand and bend surrounding field lines. The stronger the magnetic field lines, the greater tension they have and thus greater collimation of the jet. This parameter scan suggests that for regions with strong magnetic field the jet widths should be reduced and this trend should be noticeable in observations {\color{green} Can you think of any study that has reported this or looked into this?}. By increasing the driving time, it increases the jet widths and we observe a linear trend in the data. This is because with a longer driver more material is supplied to the jet, thus increasing the jet widths. The lines are grouped together based on their magnetic field strength represent by matching colors due to collimation from the magnetic field. This parameter scan indicates the amplitude of the driver has minimal effect on jet widths for magnetic field aligned flow. With non-field aligned flow we suspect this parameter would have a greater effect on jet widths, but this needs to be confirmed. \np
%
Based on the results of the parameter scan we define a ``standard'' jet, the one we believe most reflective of classical spicules. This is depicted in \fref{standard_jet}, with $P=300$ s, $B=60$ G and $A=60$ km s$^{-1}$ where the top, middle and bottom panels show changes in the density, temperature and numerical Schlieren shown in terms of normalized density gradient magnitude. The numerical Schlieren is defined as follows,
\begin{equation}
    S_{ch} = \exp{\left( -c_0 \left[ \frac{|\boldsymbol{\nabla} \rho|-c_1 |\boldsymbol{\nabla} \rho|_{max}}{c_2 |\boldsymbol{\nabla} \rho|_{max}-c_1|\boldsymbol{\nabla} \rho|_{max}} \right] \right)}, 
\end{equation}

%------------------------------------------------------------------------------      
\subsection{Jet Head}
\label{subsec:j_head}
%------------------------------------------------------------------------------      
%------------------------------------------------------------------------------      
\subsection{Jet Beam Structure}
\label{subsec:j_beam_struc}
%------------------------------------------------------------------------------      
%------------------------------------------------------------------------------      
\subsection{Jet boundary deformation}
\label{subsec:j_boundary_Def}
%------------------------------------------------------------------------------ 
%------------------------------------------------------------------------------
\section{Summary and Discussion}
\label{sec:c2discussion}
%------------------------------------------------------------------------------
     
%------------------------------------------------------------------------------      
\section{Stuff to go somewhere}
%------------------------------------------------------------------------------      
The way the jet is excited is analogous to jets created in a laboratory, in the sense we artificially created 'nozzle' in which drive our material into the atmosphere. Akin to laboratory jets our jet has high density ratio ($\eta_j >1$) and has Mach numbers ($\sim 1-3$). When describing a jets these parameters are key for defining its morphology and structure: $v_j$, $\eta_j = \frac{\rho_j}{\rho_a}$, $K = \frac{p_{d}}{p_a}$, where $\rho_a(p_a)$ is the ambient density(pressure) and $p_{d}$ is the pressure at the driver. $\eta_j$ measures the density contrast of the jet itself and the ambient medium it propagates through. If $\eta_j>1(\eta_j<1)$ then a jet is labeled to as heavy(light). For $K$ with denotes our jet to ambient pressures ratio determine whether the jet is under-expanded ($K>1$) or over-expanded ($K<1$). In our case we are dealing with the expected dynamics of a heavy under expanded-jet. In a paper by \cite{Norman1982} they describe the structures that occur in HD super sonic jets. The jets described in the paper are generated from a high pressure reservoir contain in the nozzle and internal structure of these jets is depicted in \fref{cartoon_jet_waves}. In the simulations shown in \fref{jet_simulation} we can see areas of enhanced brightness along the central axis of the jet, which we will refer to as knots. This phenomena occurs in laboratory jets \textit{e.g.} \citep{Ono2014,Edgington-Mitchell2014,Menon2010} and in astrophysical jets \textit{e.g.} \cite{Blandford2019,Belan2011,DeGouveiaDalPino2005,Birkinshaw1996}. In our case two possible mechanisms for explaining the knotted structures in the jet simulations: \\
\par (1) The knots could be created by the interaction of the pressure forces occurring inside and outside the jet. If we tracked a small portion of the jet as it evolves upwards it would undergo a series of expansion and contractions. Initially the jet pressure is greater than the surrounding atmosphere, thus as it rises it expands. However, this expansion lowers the jet pressure and as jet diameter increases it bends surrounding magnetic field lines. This increases the magnetic tension until it's greater than the jet pressure, causing the jet boundary to reconverges towards the jet center. As the jet diameter shrinks the jet pressure responds by increasing and becomes the dominant force pushing the jet boundary outwards. Thus, the cycle repeats as the jet tries to reach an equilibrium with its environment. Essentially, there is a ``tug of war" is occurring between the total pressure forces inside and outside the jet.  \\
\par (2) When a jet is going supersonic it will have a myriad of internal structures cause by shock waves \citep{Norman1982}. A schematic overview of these structures are given in Fig. \eqref{cartoon_jet_waves}. In the simulation the jet speeds reach roughly Mach $\sim 1-3$ of the local sound speed and the body of the jet itself has plasma beta $\sim 2-10$, therefore the MHD waves do not play the main role in internal structure of the jet and will be similar to dynamics of an HD supersonic jet. The notable features of interest in a supersonic jet is the osculating jet boundary and cross pattern occurring inside the jet. The jet boundary undulates as the gas periodically over expands and collapses in as it tries to reach and equilibrium with the ambient atmospheric pressure. The jet repeatedly overshoots the equilibrium position because the effects of the boundary are communicated to the interior of the jet by sound waves, which are traveling slower than supersonic flow of the jet. This leads the to lowest\textbackslash highest points of pressure in the jet being out of phase with the highest\textbackslash lowest jet diameter (see Fig. \eqref{cartoon_jet_waves}). As the jets pressure is greater than ambient atmospheric pressure the dynamics of the jet are akin to an under-expanded jet as shown in \cite{Norman1982,Edgington-Mitchell2014}. The crisscross shock pattern is created by a series of shock and expansion structures that can create these ``knots". An expansion fan (blue dashed lines in Fig. \eqref{cartoon_jet_waves}) forms at the base of the due to the difference in pressure between the jet and the ambient atmosphere. This expansion fan causes an outward flow making the jet enlarge. The Mach lines of the expansion waves reflects of the jet boundary inwards towards the jet center in the form of compression waves and a compression fan due pressure continuity (red lines in Fig. \eqref{cartoon_jet_waves}). The compression waves reflected at a near constant angle from the jet boundary and as this boundary is curved the Mach lines of the compressions waves have a tendency to converge into a conical shock wave before reaching the center of the jet. This incident shock either goes under a regular reflection or becomes a Mach disk depending on the angle between the incident shock and the central jet axis, for small and large angles respectively (see black lines Fig. \eqref{cartoon_jet_waves}). As the flow passes through this shock it will increase the pressure in the jet. When the reflected shock reaches the jet boundary it knocks the boundary outwards, creating an expansion fan and thus allows the process to be repeated.
%------ put this elsewhere
\par A schematic diagram of basic structure of a jet is given by \fref{ssj}. When a HD super sonic jet moves through the ambient medium, a bow shock is formed is formed in front of the advancing head of the jet and at the top of jet itself there is a Mach disk which is usually much stronger than the bow shock \citep{Chakrabarti1988}. The region close to advancing head of the jet is referred as working surface of the jet. Between the bow shock and the jet there are two other distinct regions the cocoon and screen which are separated by a contact discontinuity. The cocoon is the shocked jet medium and screen is the shocked ambient medium. The velocity of the jet head $v_h$ by comparing the momentum fluxes the jet and the external matter at the head of the jet and is estimated by the following,
\begin{equation}\label{v_j_speed}
v_h \approx \dfrac{v_j}{1+ \left( \dfrac{\rho_e}{\rho_j} \right)^{\sfrac{1}{2}}} 
\end{equation}
From \eref{v_j_speed} we can see for light jets ($\eta_j<1$) this means that speed of the head of the jet will be slower than the jet speed itself, this leads to a pile up of material which forms the cocoon. For heavy jets ($\eta_j>1$), the head of the jet will advance with speeds comparable to the jet itself and thus leads to much less back flow. This is why there is very little back flow in these jet The bow shock can be seen early on in the simulation and we see that the temperature increase at the head of the jet.
\par We have developed software to track the max jet heights (!get image example!) as it evolves and to measure the widths of the jet every megameter. The software tracts the jet using the tracer quantity that was set in simulation. A tracer tracks the density evoultion of th simaultion as described in \cite{Porth_2014}, we assign the jet with a vlaue 100 and everthing else as 0. We then set a threshold of $15\%$ of the jet value and convert into binary image (1 belong to jet and 0 not belonging to jet). To find the edges we take a horizontal slice every mega meter and look for changes and peak of the jet by taking the highest index value in the vertical. From this we create a data set which is used to produce Figs. (\ref{max_h_vs_b}-\ref{hight_dt}). We study how the different driver times, magnetic and drive amplitude effect the max height and average widths of the jet to see what effect this has on the jet morphology. From \fref{max_h_vs_b} it shows that strength of the magnetic has minimal impact on the max height reach by the jet. It important to note that the magnetic field may play a significant role in max jet heights if jets direction was not in-line with the magnetic field. As we can see from the color bands which corresponds to max amplitude reached by driver, this suggest that this is a key parameter for the max jet height. In \fref{max_h_vs_a} we conclude that max height is the jet mainly determined by the max amplitude reached by the driver with a parabolic trend. We observe that for our driver which is a momentum pulse we don't not get a linear trend between max jet height and pulse strength as reported in \cite{Singh2019}. \cite{Singh2019} used a a single instantaneous pressure pulse placed $1.8$ Mm. This suggests that height at which the driver is placed, its amplitude and its duration of driving is driven important to jet evolution. For example the longer distance that a pulse has to travel through the chromosphere to the TR the stronger the shocks the grow and the higher the jets will reach \citep{Shibata1982}. In \fref{max_h_vs_a} wee see that the driver time has some impact on max height. It shows that for short driver time of $50$ s there is a reduction in heights reached by the jets and there is no significant different between jet heights for a 200 and 300 s driver. We can see that the lines are banded together with matching line styles which again corresponds to matching velocities of the driver and reinforces that initial velocity is the main factor in determining max jet height if jet direction is inline with the magnetic field direction.
%------------------------
\par In \fref{mean_w_vs_A} we observe that the initial amplitude has little effect on jet widths. We a see pairing between drivers with $200$ and $300$ s, indicated by matching line styles, which is expected based on \fref{mean_w_vs_A}. The only parameter this doesn't apply is for $B=20 G$, this is due to the jet being weakly confined by the magentic feild. In \fref{mean_w_vs_dt} shows a linear trend between driving time, this is because with a longer driver more material will be supplied to the jet. There can also be a pile up in the jet beam as falling material interacts with raising material causing the jet to slighly ballon. We can see that the lines are grouped together based on their magnetic field strength represent by matching colors. This is an expected result as for weaker magnetic fields we get much larger jets widths, this is due to the magnetic field not being strong enough to compress the jet as it raises. Where as where the magnetic feild is stronger we observe that the jet widths are significantly reduced. This interpretation is evidenced by \fref{mean_w_vs_b} where we see larger jet widths for weaker magnetic field. We observe that they are group by there driver times as they have matching line styles. In \fref{hight_dt} we take use a the following paramters $P=300$ s, $B=50$ G and $A=60$ km s$^{-1}$ and run it with a much higher cadence in output. Each line in this \fref{hight_dt} represents a different height in the jet. We clearly the boundary of the jet are osculating over time. We don't see this phenomenon in simulations with a jet driven with a single instant pulse (!double check this claim! such as \citep{kuz2017ApJ,Singh2019}) we don't observe oscillations in there jet simulations, even though clearly in observations solar jets they oscillate (get refs). In Figs. (\ref{A20}-\ref{A80}) we have plotted the trajector of the max height of the jet over time, each Fig. is a displays all jets with the same driving amplitude. We find that our jets have a asymmetric parabolic path as reported in \cite{Singh2019}. We believe this is due to a piling of jet material as the jet begins to falls as falling matter will interact with raising matter. This will slow the jets decent. It is possible the magnetic field could impede the jets decent if the direction of falling material was at an angle with respect the magnetic field. We can see that heights reached are within the limits of what is expected for spicules. 
%fffffffffffffffffffffffffff
\mfig{0.8}{figures/jet_P300_B60A_60_t_1_10_13_19.png}{Simulation of jet in Stratified atmosphere with uniform magnetic field in vertical direction. This is the ``standard" jet with parameter $P=300$ s, $B=60$ G, $A=60$ km s$^{-1}$. Top, middle and bottom panels show density, temperature and numerical Schlieren respectively.}{jet_simulation}
%fffffffffffffffffffffffffff
\mfig{1}{figures/parameter_scan_1.png}{Simulation of jet in stratified atmosphere with uniform magnetic field in vertical direction. Examples of parameter scan with variations of one parameter of standard jet and all panels show the density evolution. Top, middle and bottom panels are $B=80,20$ G and $A=80$ km s$^{-1}$, respectively.}{pscan_den_plot1}
\mfig{1}{figures/paramter_scan_2.png}{Continuation of \fref{pscan_den_plot1} with top, middle and bottom panels are $A=20$ km s$^{-1}$, $P=50$ and $200$ s, respectively.}{pscan_den_plot2}
%fffffffffffffffffffffffffff
\mfig{0.8}{figures/jet_diagram.eps}{schematic overview of the structure of a under-expanded supersonic HD jet.}{cartoon_jet_waves}
%fffffffffffffffffffffffffff
\mfig{0.5}{figures/diagram_of_ssj.png}{A schematic diagram giving an overview of the expected structures found in an HD super sonic jet dran in frame of the bow shock. Diagram taken from \cite{Chakrabarti1988}.}{ssj}
%fffffffffffffffffffffffffff
\mfig{1}{figures/combine_zoom.png}{todo.}{jet_head}

%%%%%%%%%%%%%%%%%%%%%%%%%%%%%%%%%%%%%%%%%%%%%%%%%%%%%%%
% STOP COPYING HERE
%%%%%%%%%%%%%%%%%%%%%%%%%%%%%%%%%%%%%%%%%%%%%%%%%%%%%%%

\bibliographystyle{plainnat}
\bibliography{references}

\end{document}
