\documentclass[12pt]{ociamthesis}

\usepackage{amssymb}
\usepackage{titlesec}
\usepackage{amsmath}
\usepackage{float}
\usepackage{graphicx}
\usepackage{caption}
\usepackage{subfig}
\usepackage{graphicx}
\usepackage{xcolor}
\usepackage[round]{natbib}
\usepackage[section]{placeins}
\usepackage{mathrsfs}
\usepackage{bm}
\usepackage{stmaryrd}
\usepackage[utf8]{inputenc}
\usepackage{bibentry}
\usepackage{wasysym}
\usepackage{xfrac}

\usepackage{geometry}
 \geometry{
 a4paper,
 left=40mm,
 right=30mm,
 top=30mm,
 bottom=30mm
 }

\definecolor{theblue}{HTML}{0000CD}

% disable this package for printed version
\usepackage[colorlinks=true, linktocpage=true, allcolors=theblue]{hyperref}

\titleformat{\chapter}[display]
  {\bfseries\Large}
  {\filright\MakeUppercase{\chaptertitlename} \Large\thechapter}
  {1ex}
  {}
  [\vspace{1ex} \hrule \vspace{1pt} \hrule]

\newcommand{\adv}{    {\it Adv. Space Res.}} 
\newcommand{\araa}{    {\it Annual Review of Astron and Astrophys.}} 
\newcommand{\annG}{   {\it Ann. Geophys.}} 
\newcommand{\aap}{    {\it Astron. Astrophys.}}
\newcommand{\aaps}{   {\it Astron. Astrophys. Suppl.}}
\newcommand{\aapr}{   {\it Astron. Astrophys. Rev.}}
\newcommand{\ag}{     {\it Ann. Geophys.}}
\newcommand{\aj}{     {\it Astron. J.}} 
\newcommand{\apj}{    {\it Astrophys. J.}}
\newcommand{\apjl}{   {\it Astrophys. J. Lett.}}
\newcommand{\apss}{   {\it Astrophys. Space Sci.}} 
\newcommand{\bain}{   {\it Bulletin of the Astronomical Institutes of the Netherlands.}} 
\newcommand{\cjaa}{   {\it Chin. J. Astron. Astrophys.}} 
\newcommand{\gafd}{   {\it Geophys. Astrophys. Fluid Dyn.}}
\newcommand{\grl}{    {\it Geophys. Res. Lett.}}
\newcommand{\ijga}{   {\it Int. J. Geomagn. Aeron.}}
\newcommand{\jastp}{  {\it J. Atmos. Solar-Terr. Phys.}} 
\newcommand{\jgr}{    {\it J. Geophys. Res.}}
\newcommand{\mnras}{  {\it Mon. Not. Roy. Astron. Soc.}}
\newcommand{\na}{     {\it New Astronomy}}
\newcommand{\nat}{    {\it Nature}}
\newcommand{\pasp}{   {\it Pub. Astron. Soc. Pac.}}
\newcommand{\pasj}{   {\it Pub. Astron. Soc. Japan}}
\newcommand{\pre}{    {\it Phys. Rev. E}}
\newcommand{\solphys}{{\it Solar Phys.}}
\newcommand{\sovast}{ {\it Soviet  Astron.}} 
\newcommand{\ssr}{    {\it Space Sci. Rev.}}
\newcommand{\caa}{    {\it Chinese Astron. Astrohpys.}} 
\newcommand{\apjs}{   {\it Astrophys. J. Suppl.}}

\def\UrlFont{\sf}

\newcommand{\bs}[1]{\boldsymbol{#1}}
\newcommand{\bn}{\boldsymbol{\nabla}}
\newcommand{\rgas}{\mathcal{R}}
\newcommand{\eref}[1]{Eq. \eqref{#1}}
\newcommand{\fref}[1]{Fig. \eqref{#1}}
\newcommand\encircle[1]{%
  \tikz[baseline=(X.base)] 
    \node (X) [draw, shape=circle, inner sep=0] {\strut #1};}
\newcommand{\Alfven}{Alfv\'{e}n } 
\newcommand{\Alfvenic}{Alfv\'{e}nic }
\newcommand{\size}{0.75}
\newcommand\measureISpecification{4ex}% not defined in mwe
\newcommand{\ctab}[1]{\raisebox{\dimexpr \measureISpecification/2 -.748ex}{#1}}% vertically centers numbers
\newcommand{\si}[1]{\;\rm{#1}}
\newcommand{\mfig}[4]{
  \begin{figure}
  \begin{center}
  \includegraphics[width=#1\linewidth]{#2}
  \caption{#3}
  \label{#4}
  \end{center}
  \end{figure}}
\newcommand{\kms}{~\rm{km ~s^{-1}}}
\newcommand{\kgm}{~\rm{kg ~m^{-3}}}
\newcommand{\np}{\\ \\}
\begin{document}

\baselineskip=18pt

\setcounter{secnumdepth}{3}
\setcounter{tocdepth}{3}

\setcounter{chapter}{1}

%%%%%%%%%%%%%%%%%%%%%%%%%%%%%%%%%%%%%%%%%%%%%%%%%%%%%%%
% START COPYING HERE
%%%%%%%%%%%%%%%%%%%%%%%%%%%%%%%%%%%%%%%%%%%%%%%%%%%%%%%

\chapter{The Dynamics of Filed-aligned Jets in a Solar Atmosphere}
%-------------------------------------------------------------------------------
\let\thefootnote\relax\footnotetext{

This chapter is based on the following refereed journal article:
\begin{itemize}
\item Barbulescu, M., Erd\'elyi, R. (2018); Magnetoacoustic Waves and the Kelvin-Helmholtz Instability in a Steady Asymmetric Slab. I: The Effects of Varying Density Ratios, \solphys, Volume 293, Issue 6
\end{itemize}
}
%------------------------------------------------------------------------------
\section{Introduction}
\label{sec:c2intro}
%------------------------------------------------------------------------------
Understanding the physical mechanisms that maintain the observed thermal profile at the solar transition region (TR), remains an intriguing problem in solar physics. This region separates the plasma pressure dominated lower solar atmosphere (photosphere and chromosphere) from magnetic pressure dominant corona. Observations suggest ubiquitous thin magnetic flux tube (MFT) features  in this region that topologically link the photosphere with the corona, and transport mass, energy and momentum across the TR to balance the radiative losses. Moreover, these MFT structure acts as conduits for magnetohydrodynamic (MHD) wave modes to propagate from lower solar atmosphere to corona and transfer energy between these regions. 

Depending upon the observed wavelengths, physical characteristics (\textit{e.g.}, length, lifetime, velocity), regions (quiet Sun, active region or coronal hole) and/or locations (at-limb or on-disk), MFT structures are further categorized as different features (\textit{e.g.}, spicules, mottles, dynamic fibrils, marcospicules, x-ray jets, EUV jets, coronal jets) that permeate the solar atmosphere \citep[see reviews by:][]{Beckers1968, Beckers1972ARA&A, Tsiropoula2012}. These MFT structures are key to understand the wave and pulse propagation in solar atmosphere with their observed dynamics (transverse-, rotational-, width-perturbations) reflect the presence of confined MHD wave modes (kink, torsional Alfv\'en, sausage/fluting).

In recent years, field-aligned/longitudinal propagation of dominant MHD wave/pulse in MFT structures become a key focus of research to understand the chromospheric heating \citep{Narain1990, Zaqarashvili_2009SSRv, Jess2015}. Understanding of wave/pulse propagation in a dynamic, gravitationally stratified, inhomogeneous environment can provide vital clues about dissipation mechanisms and/or formation of shock-like behavior associated with thin MFT structures. Observations suggest upward flow velocities in range 15-110 kms$^{-1}$  for typical lifetimes of around 50-400 sec for off-limb spicular structures. The visible apex of these observed features attains a maximum height of $\sim$7 Mm and descends back on the surface following a parabolic trajectory \citep{Pereira2012,Pereira2016ApJ82465P}. 

Several theoretical and numerical approaches to gain an insight in the morphology and dynamical properties of jets in the solar atmosphere led to a plethora of models \citep[early reviews by:][]{Sterling_2000SoPh} that included physical mechanisms such as: pressure/velocity pulse originating in middle-upper chromosphere leading to formation of shocks \citep{Shibata1982, Suematsu1982SoPh7599S, Hollweg1982ApJ257345H, Sterling1990ApJ349647S, Heggland2007ApJ6661277H, kuzma2017ApJ84978K}, Alfv\'en waves in a MFT waveguide that nonlinearly couple and generate shocks \citep{Hollweg1982SoPh7535H, Hollweg1992ApJ389731H, Kudoh1999ApJ514493K, Matsumoto2010ApJ7101857M}, magnetic reconnection driven jets \citep{Yokoyama1995Natur37542Y, Yokoyama1996PASJ48353Y, Archontis2005ApJ6351299A, Isobe2008ApJ679L57I, Nishizuka2008ApJ683L83N, Sterling2010ApJ, Gonz2017ApJ, Gonz2018ApJ856176G}, and/or Joule heating due to ion-neutral collisional dampening \citep{Haerendel1992Natur360241H, James2003AA}. More realistic 3D simulations with complex physics mimicking lower solar atmosphere were employed by \citep{Mart2017Sci3561269M, Mart2018ApJ860116M, Mart2020ApJ88995M} using the Bifrost code \citep{Gudiksen2011AA531A154G}. By combining a multitude of physical mechanisms that involved radiative losses, thermal conduction along magnetic field, ion-neutral and non local thermal equilibrium (NLTE) effects, these authors highlighted the potential role of spicular jets in heating at the solar transition region.

Despite the numerous observational, theoretical and numerical studies used to understand the morphology and formation of jets in the lower solar atmosphere, qualitative investigation on parameters that influence the observed dynamical characteristics of these features remains missing. Here, we aim to address the fundamental problem regarding the propagation of a momentum pulse/wave originating at the top of photosphere and propagating through chromosphere and the TR to lower corona in an idealized/stratified solar atmosphere using a simplistic numerical model. The parameter space comprises of driver times, amplitudes and magnetic field strength, which are examined to assess their role in determining the height, width, internal (sub)structures and dominant MHD wave mode.

Despite the numerous numerical simulation, very few papers simulating solar jets include an investigation of a parameter space.
%------------------------------------------------------------------------------      
\section{Parameter Space}
\label{subsec:paramater_space}
%------------------------------------------------------------------------------
Using the numerical setup as described in section \ref{section:numerical_recipe}, we have conducted numerous simulations to quantify the effects the of three key parameters (magnetic field strength, driver period and initial amplitude) on spicular jets journey through the solar atmosphere. The parameter space chosen for investigation is $P=50,~200$ and $300~\rm{s}$ (driver times), $B_y=20,~40,~60,~80,~100~\rm{G}$ (magnetic filed strength) and $A=20,~40,~60$ and $80 \kms$ (initial amplitudes) to to qualitatively estimate their impact on jet morphology (maximum height and width) and kinematics (trajectory, transverse displacement). All these values are based around typical values for classical spicules (see section \ref{subsec:Spicules} and table \ref{solar_jet_table}). The driver times based on the $5$ minute oscillation of the p-mode \citep{Leighton1962ApJ135474L}, as they are potential driver for multiple solar features (spicules, mottles  and dynamic fibrils) \citep{Pontieu2004Natur}. In addtion the driver time is based around the measured lifetime of spicular jets. The magnetic field strength are in the ranges observed values for spicules which is consistent with spectropolarimetric estimates for limb spicules \citep{centeno2010, suarez2015}. The range of velocities are chosen based on the energy need to lift near photospheric mass ($?\kgm$) and observed speeds seen in spicules which have a range of $10-150\kms$ (see table \ref{solar_jet_table}). The magnetic field estimates are consistent with reported magnitudes from spectropolarimetric studies for off-limb spicules \citep{centeno2010, suarez2015}. \np
% may include a summarry table in the intro that is used in presenations, then we could refere to it in text.
%-------------------------
\subsection{Jet Tracking}
\label{subsec:jet_tracking}
%--------------------
To understand the jet dynamics we need to quantify important aspects of the jet, e.g. speed, life time, trajectory and jet boundary deformation. This is achieved by temporally tracking the jets width (solid blue dots) and apex (yellow triangle), using jet tracking software (see \fref{j_track_example}). The jet tracking software developed for this thesis uses the tracer quantity first introduced in MPI-AMRVAC by \cite{Porth_2014}. In essence the tracer tracks the mass flow of jet as it enters the computational domain, it bears no physical meaning, but is useful for tracking the flow of any desired region in a simulation. The injected jet material a high arbitrary value of $100$ (jet value), while ambient environment set to 0 and using a threshold of $15\%$, any snapshot can be split into cells belonging to the jet and ambient medium. This is converted into binary array where $1$ belongs jet cell and $0$ is assumed to be ambient medium, which allows for easy edge detection. The cross-sectional widths are estimated as a function of the distance between the opposite jet cells at edges, taken from a horizontal slice across the simulated jet at specified height(s). By have a high resolution and tracking the jet edges it possible to obtain useful information of the jet boundary such as, the temporal deformation and jet widths. Furthermore, the visible apex of the jet is selected as the highest index of a jet cell in the simulation which allows us to track the jets trajectory, speed, acceleration and deceleration. 
\begin{figure}
\centering
{\includegraphics[width=0.8\linewidth]{figures/jet_P300_B60A_60T_0103.png}} 
\caption{Image show an example of jet tracking software to accurately estimate height and cross-sectional width parameters for the simulated jet structure. The jet-apex is marked with a yellow triangle with blue dots on jet edges provide an estimate of widths for each height, during different evolutionary phases (rise/fall) of the jet.}
\label{jet_tracker}
\end{figure}
%---------------------------
\subsection{Jet Trajectories}
\label{subsec:jet_traj}
%------------------------
% !possibley add figs for these!!!!!!!!!!
The identified apex of the simulated jets shows a parabolic motion over time, as shown in figure \eqref{jet_traj}, with each panel highlighting the dynamics based on the initial velocity of the driver. The estimated jet trajectories reveal non-ballistic path(s) of the apex, which is consistent with reported observations \citep{Hansteen2006ApJ, Rouppe2007ApJ660L169R, Pontieu2007PASJ} for chromospheric jets (\textit{e.g.}, mottles, fibrils, spicules). In the numerical setup, the minimum amplitude for the driver is estimated to be $>20 ~\rm{km ~s^{-1}}$ for the near-photospheric plasma to attain spicular heights (see table \ref{solar_jet_table}). Also for most cases, the maximum apex for the simulated jets were found to be proportional to the driver amplitude. In each panel there is a subgroup reaching lower heights due reduced driving time. \np  
\begin{figure}
\captionsetup[subfigure]{labelformat=empty}
\centering
\subfloat[]{\includegraphics[width=\linewidth]{figures/jet_traj.png}} 
\caption{Plots show the effects of multiple parameter combinations on maximum apex height attained by the numerical jet. Results for driver amplitudes with magnitude $A = 20$ km s$^{-1}$ (top-left), $A = 40$ km s$^{-1}$ (top-right), $A = 60$ km s$^{-1}$ (bottom-left), and $A = 80$ km s$^{-1}$ (bottom-right) are shown with jet-apex following a parabolic trajectory during its lifetime. Driver period ($P$), magnetic field strength ($B$) and driver amplitude ($A$) have units in sec, Gauss and km s$^{-1}$ respectively.}
\label{jet_traj}
\end{figure}
These jet-like events are predicted to have linear correlation between deceleration and maximum velocity and if the jet is initiated with shock waves the relationship is defined by the following, \citep{Heggland2007ApJ6661277H},
\begin{equation}
d = \frac{v_{max}}{P_{w}/2},
\end{equation}
where $d$ is declaration, $v_{max}$ is the maximum velocity during the lifetime of the jet and $P_{wave}$ is the period of the wave. In !fig (todo)! !!1! we show that our jets have a linear relationship, by that found in \citep{Heggland2007ApJ6661277H}, but this is due to the nature in which these jets are driven as it is not a short pulse like event, the parameter $P_{w}=P$, which in our cases is linked to the life time of the jet rather than the period of a wave. While we find that we match observation with a linear correlation, its not with the same gradient of line as theoretically predicted for these jet-like events.
%-------------------------------------------
\subsection{Effects on Jet Apex and Widths}
\label{subsec:jet_apex_widths}
%-----------------------  
To understand the global impact of the key parameters investigated, we compared how each parameter effects the max apex height (panels to the left) and average CSW (panels to the right) of the jet over its entire lift cycle as displayed in \fref{parameter_scan_lines}. Varying the magnetic field strength ($B$) has little impact on the heights reached by a jet as all cases have shallow gradients, most likely due to the flow of the jet being field aligned. However, it must noted that any deviation between jet flow and magnetic field direction can effect the apex height of the jet, this particular aspect is investigated in chapter 3. The driver time ($P$) have a nuanced effect, there is a slight decrease in height for $p=50~\rm{s}$, but peaks between $P=200$ and $300~\rm{s}$. This suggested to transport significant mass from lower in the atmosphere then a instantaneous pulse will not be sufficient, there needs to be longer driver time, but past a certain threshold the driver timing doesn't impact heights reached. In both previously mentioned parameters, they are grouped by colors and line styles for $B$ and $P$ scans, which corresponds to matching initial amplitudes. This alludes to initial amplitude ($A$) being the most important parameter for determining jet heights. Based on these indications, we modeled the jet height with a power law function given as,   
\begin{equation}
h_{max} = C v_j^{n},
\end{equation} 
by collapsing the data by taking a average at each velocity data point and then fit an optimal curve using least squares obtaining values of $C= 10^{-2.21}$ and $n= 1.72$ as shown in lowest left panel in \fref{parameter_scan_lines}. The estimates from power law clearly suggest a nonlinear relation between apex height and pulse strength, in contrast to reported by \citet{Singh2019}. It must however be noted that \citet{Singh2019} used an instant pressure pulse close to the TR as a driver in their simulation, which might not be directly comparable with our investigation. \np  
%
On the panels to the right of \fref{parameter_scan_lines} shows the effect of the parameter space on the mean CSW.To obtain the mean CSW, it is measured at every $\rm{Mm}$ the jet reaches and the mean value is calculated over the whole life cylce of the jet. This is by no means a perfect method, but does give useful insight into the global deformation of the jet boundary, while returning result that make physical sense. Our simulations suggest strong influence of magnetic field strength in determining the width(s) of the jet structure with estimated cross-sectional widths inversely related to the magnetic field magnitudes. This is due to the fact that as the jet rises through the stratified atmosphere, it has higher internal pressure (as transporting material with lower atmosphere conditions) compared to the ambient medium. This results in expansion of the jet structure and subsequent bending/distortion of the surrounding field. However, in case of higher magnetic field magnitudes, the jet experiences strong tension forces that results in greater collimation of the jet structure. These results further indicate the possibility of lower cross-sectional widths of jet features located near strong magnetic field environment than those in quiet Sun regions. Another important aspect that influences the jet width is the lifetime of the driver of the jet. The simulations suggest a linear relation between the driver periods and the jet CSW. This could possibly due to the longer supply of plasma to the jet by a sustained driver resulting in increased CSW. However, amplitude of the driver had a minimal effect on the width for the jets aligned with radial magnetic fields, though this might not be the case with jet direction miss-aligned with the background magnetic fields. \np
\begin{figure}
\captionsetup[subfigure]{labelformat=empty}
\centering
\subfloat[]{\includegraphics[width=\linewidth]{figures/test_combine_image.png}} 
\caption{Panels compare the effects of different physical parameters (top-bottom: magnetic field strength, driver period, driver amplitude) on the maximum apex height (left) and mean cross-sectional widths (right) for the simulated jet structure. The shaded-region (bottom-left) indicates the 1$\sigma$ error for the power law fit (red line) to the parameter scan. It should be noted that driver period ($P$), magnetic field strength ($B$) and driver amplitude ($A$) have units in sec, Gauss and km s$^{-1}$ respectively.}
\label{parameter_scan_lines}
\end{figure}
%------------------------------------------------------------------------
\section{Synthetic Jet Morphology}
%------------------------------------------------------------------------------
% need to edit ---------------------
Based on the results of the parameter scan we define a ``standard'' jet, the one we believe most reflective of classical spicules. This is depicted in \fref{standard_jet}, with $P=300$ s, $B=60$ G and $A=60$ km s$^{-1}$ where the show changes in the density (a-d), temperature (e-h) and numerical Schlieren (i-l), which is typically is representation of normalized density gradient magnitude. The numerical Schlieren is defined as follows,
\begin{equation}
    S_{ch} = \exp{\left( -c_0 \left[ \frac{|\boldsymbol{\nabla} \rho|-c_1 |\boldsymbol{\nabla} \rho|_{max}}{c_2 |\boldsymbol{\nabla} \rho|_{max}-c_1|\boldsymbol{\nabla} \rho|_{max}} \right] \right)}, 
\end{equation}
%------------------------
Based on the results of the parameter scan, we define a ``standard'' jet from our simulations, which has more similar characteristics akin to the classical spicules. The result shown in \fref{standard_jet}, with $P=300$ s, $B=60$ G and $A=60$ km s$^{-1}$ where the top, middle and bottom panels highlight variations in the density, temperature and numerical Schlieren (indicative of normalized density gradient magnitude). The numerical Schlieren is defined as,
\begin{equation}
S_{ch} = \exp{\left( -c_0 \left[ \frac{|\boldsymbol{\nabla} \rho|-c_1 |\boldsymbol{\nabla} \rho|_{max}}{c_2 |\boldsymbol{\nabla} \rho|_{max}-c_1|\boldsymbol{\nabla} \rho|_{max}} \right] \right)}, 
\end{equation}
where $c_0=5$, $c_1=0.05$ and $c_2=-0.001$. It's important to noted that despite the fact that the numerical Schlieren is a physically defined term, its use in our analysis is purely from a qualitative perspective due to their ability to visualize significant variations of a chosen parameter. An important facet of our simulation is related with the field-aligned jet motions (\fref{standard_jet}) that show distinct kinematic and morphological characteristics during rise- and fall-phase. In the rise-phase (left-panels of \fref{standard_jet}), the simulated jet show complex internal substructures and sausage-like deformation of cross-sectional widths in density and Schlieren data. However, during the fall-phase (right-panels of \fref{standard_jet}), the complex/internal beam substructure cease due to driver switch-off. At this stage, the boundary deformation is no longer symmetric, though the jet-axis has noticeable transverse displacement akin to an ideal MHD kink wave mode. Modification in sausage-like wave properties associated with cross-sectional width estimates during rise- and fall-phases of simulated jets were recently reported by \citep{Mackenzie_Dover_2020}.
%------------------------
\begin{figure}
\captionsetup[subfigure]{labelformat=empty}
\centering
\subfloat[]{\includegraphics[width=0.8\linewidth]{figures/sj_jet_rho_Te_sch.png}}
\caption{Panels show snapshots of temporal evolution of the ``standard’’ numerical jet with uniform radial magnetic field in stratified atmosphere with driver period ($P = 300$ sec), magnetic field ($B = 60$ G) and amplitude ($A = 60$ km s$^{-1}$). The density (a-d), temperature (e-h) and Schlieren (i-l) images highlight complex internal substructures (\textit{knots}) of the simulated jet with bright apex, possibly due to high density concentrations. The temperature (middle) panel indicates isothermal nature of the jet structure with colder plasma component from photospheric layer. }
\label{standard_jet}
\end{figure}
%--------------------
\begin{figure}
\captionsetup[subfigure]{labelformat=empty}
\centering
\subfloat[]{\includegraphics[width=0.8\linewidth]{figures/den_plot_1.png}} 
\caption{Evolution of the simulated jet with different combination of magnetic field and driver amplitude parameters, in a stratified solar atmosphere with uniform radial magnetic field. Parameters are varied for the ``standard’’ jet configuration to identify the effects of a particular parameter on jet morphology and kinematics. Panels (top-bottom) show variations with $B = 80$ G (a-d), $B = 20$ G  (e-h) and $A = 80$ km s$^{-1}$ (i-l), respectively.}
\label{paramter_scan_one}
\end{figure}
%---------------------
\begin{figure}
\captionsetup[subfigure]{labelformat=empty}
\centering
\subfloat[]{\includegraphics[width=0.8\linewidth]{figures/den_plot_2.png}} 
\caption{Similar to \fref{paramter_scan_one} with panels (top-bottom) highlighting variations with $A = 20$ km s$^{-1}$ (a-d), $P = 200$ (e-h)$,~50$ s (i-l), respectively for ``standard’’ jet configuration.}
\label{paramter_scan_two}
\end{figure}
%---------------------
The temperature profile of the numerical jet suggest periodic distortion in the TR layer over jet lifetime. The TR deforms as the jet penetrates this layer while generating waves, also known as TR quakes \citep{Scullion2011ApJ74314S}. During its entire lifetime, the jet structure remains isothermal/cool without much variations for any combinations of the selected parameters. Figures \ref{paramter_scan_one} and \ref{paramter_scan_two}, show cases for the simulated jet with density structure highlighting variations in parameter space in each row for similar time-instance as shown in \fref{standard_jet}. The top and middle panels in \fref{paramter_scan_one} exhibits the effect of strong ($B = 80$ G) and weak ($B = 20$ G) magnetic field strengths respectively. For the strong magnetic field, the jet was found to be more collimated with higher density, mostly concentrated near the apex. In this case, there are no visible complex beam structure and no sausage-like jet boundary deformation. However, in case of weak magnetic field, the jet is more diffusive as it balloons out in higher solar atmospheric layers. Here, simulations suggest clear complex beam structure and sausage-like deformation.       

Figures \ref{paramter_scan_one} (bottom panel) and \ref{paramter_scan_two} (top panel), showcase the effects of high ($A = 80$ km s$^{-1}$) and low ($A = 20$ km s$^{-1}$) velocity magnitudes on jet morphology. For higher velocity ($A = 80$ km s$^{-1}$), the simulated jet had internal substructure similar to the standard jet, but with more \textit{criss-cross/knot-like} features at matching time-steps. The jet shows kinking behavior from $t = 223.3$ sec onwards, possibly due to initial high velocity in slender structure which is more susceptible to kink instability. However, at $t = 287.7$ sec the internal substructures disappear with standard jet, indicating a close association between driver time and \textit{knot} lifetimes. For $A = 20$ km s$^{-1}$ (top panel: \fref{paramter_scan_two}), the simulations suggests that for a lower velocity the jet heights are reduced along with absence of any significant substructures. 

The middle and bottom panels of Figure \ref{paramter_scan_two} highlight the role of driver period for two cases with magnitudes $P = 50$ sec and $200$ sec, respectively. In case of shorter driver period ($P = 50$ sec), the jet at $t = 21.5~\rm{sec}$ show similar evolutionary trend as for the longer durations ($P = 200$ and $300~\rm{sec}$), with a lifetime of around $t = 223.3~\rm{sec}$. For long duration drivers ($P = 200$ sec), jet boundaries are smooth and show no kink-like deformation in the fall-phase. A key difference between the two driver periods is the amount of mass injection in the jet which has direct implications for the cross-sectional widths. At the end of the lifetime of the jet ($t = 223.3~\rm{sec}$), the jet has no complex internal substructures, suggesting a stronger role of initial velocities in generating these substructures.
%----------------------------------------------------------------------------      
\subsection{Jet Head}
\label{subsec:j_head}
%------------------------------------------------------------------------------  
\begin{figure}
\centering
{\includegraphics[width=\linewidth]{figures/sch_zoom.png}} 
\caption{Panels show Schlieren images of the jet-apex with complex substructures highlighting regions with potential high and low plasma densities. Presence of higher plasma density and/or \textit{knot}-like patterns can make the apex appear brighter than rest of the jet structure.}
\label{basic setup}
\end{figure} 
%------------------    
%------------------------------------------------------------------------------      
\subsection{Jet Beam Structure}
\label{subsec:j_beam_struc}
%------------------------------------------------------------------------------
Our simulations for chromospheric jet structures revealed a peculiar aspect with complex internal substructures. Multiple cases with varied driver amplitudes/periods and magnetic field strength (Figures. \ref{standard_jet}, \ref{paramter_scan_one}) highlighted \textit{criss-cross/knot}-like patterns within the jet boundary, along with cross-sectional distortion of the feature. Similar behavior appear to be ubiquitous and were reported in previous studies associated with astrophysical \citep{van_Putten_1996ApJ467L57V, DeGouveiaDalPino2005, Hada2013ApJ77570H, Cohen2014ApJ787151C, Hervet2017AnA606A103H} and laboratory jets \citep{Menon2010, Edgington-Mitchell2014, Ono2014}.

Simulation results indicate the presence of these patterns at the sites of higher densities within the jet structure. We refer these sites as \textit{knots} in our investigation and propose two possible mechanisms for their formation:

\begin{enumerate}
\item{\textit{Knots} are manifestation of non-equilibrium pressure forces prominent within and around the jet structure. The internal pressure results in expansion of the jet cross-sectional area which is then counter-balanced by the tension forces of the jet. Eventually, the magnetic force (pressure) dominates the internal plasma pressure and the jet boundary decreases while resulting in compression of the jet structure. These processes over height and time appears as cross-sectional deformation of jet, along with the sites of high density/pressure as \textit{knots} within the jet structure.}

\item{The \textit{knot} features could also be the sites within the jet structure where internal shock waves \citep{Norman1982} gets reflected from the jet boundary. If the jet has supersonic velocities,  (simulated jets Mach numbers are around $\sim$1-3), it can give rise to a myriad of internal substructures due to high velocities.}
\end{enumerate}

A schematic overview of these beam (sub)structures are given in Fig. \ref{cartoon_jet_waves}, highlighting osculating jet boundary and cross pattern occurring inside the jet. The jet boundary undulates as the gas periodically over expands and collapses inwards, as it tries to reach an equilibrium with the ambient atmospheric pressure. Jet structure repeatedly overshoot its equilibrium primarily due to the effects of the boundary, communicated to the interior by the sound waves, which are traveling slower than the supersonic flow of the jet. This leads to the formation of low/high regions of pressure in the jet being out-of-phase with the cross-sectional width (Fig. \ref{cartoon_jet_waves}). 

The mechanism for the formation of these low/high pressure points within the jet structure could be explained in terms of perturbed pressure balance between the jet and its surrounding. Since the jet internal pressure is greater than the ambient atmospheric pressure, the dynamics of the jet become similar to an under-expanded jet \citep{Norman1982, Edgington-Mitchell2014}. As a result, \textit{criss-cross} pattern appear due to a series of internal shock waves and subsequent expanding structures, identified as \textit{knots} in our simulations. These expanding fans or \textit{knots} (blue dashed lines in Fig. \ref{cartoon_jet_waves}) are basically an outward flow formed at the sites of enhanced pressure differences between the jet (internal shocks) and ambient atmosphere.

The Mach lines of these expansion waves reflects inwards at the jet boundary while forming compression waves/fan due to pressure continuity (red lines in Fig. \ref{cartoon_jet_waves}). These compression waves are reflected at a near constant angle at the jet boundary and due to the curvature at the edges, the Mach lines of reflected compression waves tends to converge into a conical shock wave, before arriving at the jet axis. Depending on the angle between the incident shock and jet axis, the shock could either be reflected back or forms a Mach disk for small and large angles respectively (see black lines: Fig. \ref{cartoon_jet_waves}). As the plasma flow passes though this shock region, it will increase the local pressure at the site. This increase in pressure, together with the outward expanding fan allows repetition of the process. 

This explains the formation of stationary \textit{criss-cross} pattern and the \textit{knots}, along with cross-sectional deformations, density enhancements and cavities in the jet structure. Also, this provides a vital clue about the disappearance of these complex substructures during the driver’s switch-off phase. In our simulations, the standard jet (\fref{standard_jet}), clearly showcase these \textit{knot}-like structures and with the driver switch-off at $t=287.7~\rm{s}$, these internal substructures disappears. Similar behavior is evident from \fref{paramter_scan_one} and \fref{paramter_scan_two}, though, the presence of \textit{knots} appears to be sensitive for particular parameters like strong magnetic field, low driving speed and shorter driving times.  
%----------------
\begin{figure}
\captionsetup[subfigure]{labelformat=empty}
\centering
\subfloat[]{\includegraphics[width=0.8\linewidth]{figures/jet_diagram.eps}}
\caption{A cartoon depicting complex internal substructures in a supersonic jet. The cross-sectional width variations due to formation of regions with high and low pressures as a consequence of internal shock waves are shown. These wave patterns appear as \textit{criss-cross/knot}-like in our simulations, highlighting the role of initial velocity of momentum pulse in generating axisymmetric deformation in a jet structure. }
\label{cartoon_jet_waves}
\end{figure}
%----------------
%In the initial rising phase, the boundary deformation is axisymmetric and we get a sausage-like deformation, which can develop kink instability, this can be during both the rising and falling phase of the jet dependent on initial parameters.
% I may not include this
%\par !probably wont include in paper! When describing a jets these parameters are key for defining its morphology and structure: $v_j$, $\eta_j = \frac{\rho_j}{\rho_a}$, $K = \frac{p_{d}}{p_a}$, where $\rho_a(p_a)$ is the ambient density(pressure) and $p_{d}$ is the pressure at the driver. $\eta_j$ measures the density contrast of the jet itself and the ambient medium it propagates through. If $\eta_j>1(\eta_j<1)$ then a jet is labeled to as heavy(light). For $K$ with denotes our jet to ambient pressures ratio determine whether the jet is under-expanded ($K>1$) or over-expanded ($K<1$). In our case we are dealing with the expected dynamics of a heavy under expanded-jet. When a HD super sonic jet moves through the ambient medium, a bow shock is formed is formed in front of the advancing head of the jet and at the top of jet itself there is a Mach disk which is usually much stronger than the bow shock \citep{Chakrabarti1988}. The region close to advancing head of the jet is referred as working surface of the jet. Between the bow shock and the jet there are two other distinct regions the cocoon and screen which are separated by a contact discontinuity. The cocoon is the shocked jet medium and screen is the shocked ambient medium. The velocity of the jet head $v_h$ by comparing the momentum fluxes the jet and the external matter at the head of the jet and is estimated by the following,
%\begin{equation}\label{v_j_speed}
%v_h \approx \dfrac{v_j}{1+ \left( \dfrac{\rho_e}{\rho_j} \right)^{\sfrac{1}{2}}} 
%\end{equation}
%From \eref{v_j_speed} we can see for light jets ($\eta_j<1$) this means that speed of the head of the jet will be slower than the jet speed itself, this leads to a pile up of material which forms the cocoon. For heavy jets ($\eta_j>1$), the head of the jet will advance with speeds comparable to the jet itself and thus leads to much less back flow. This is why there is very little back flow in these jet. The bow shock can be seen early on in the simulation and we see that the temperature increase at the head of the jet (see left most panel in Fig. \eqref{paramter_scan_one}.
% useful links for jets heads: 
% https://books.google.co.uk/books?id=ZdjImbOLi9gC&pg=PA449&lpg=PA449&dq=how+to+calculate+speed+of+the+head+of+the+jet+astrophysics&source=bl&ots=SI82iL7XlT&sig=ACfU3U2q3yT93ZitzgWwFXm0YaPYX_l21w&hl=en&sa=X&ved=2ahUKEwiquezgm5rqAhV1RxUIHRFrDikQ6AEwD3oECAgQAQ#v=onepage&q=how%20to%20calculate%20speed%20of%20the%20head%20of%20the%20jet%20astrophysics&f=false
% https://iopscience.iop.org/article/10.1088/1742-6596/1031/1/012022/pdf
% https://science.sciencemag.org/content/sci/346/6207/325.full.pdf
% http://articles.adsabs.harvard.edu//full/1994ApJ...435..261D/0000263.000.html
% http://articles.adsabs.harvard.edu/pdf/1993ApJ...410..686D
% https://iopscience.iop.org/article/10.1086/311159/pdf
% http://articles.adsabs.harvard.edu/pdf/1996ApJ...467L..57V
%------------------------------------------------------------------------------
\section{Summary and Discussion}
\label{sec:c2discussion}
%------------------------------------------------------------------------------
In summary we have taken simple model to study the effect on jet heights and widths when it is driven by a momentum pulse with enough strength to raise near photopsheric material within spicules heights. This approach is taken to focus on the dynamics of the jets. We find that due to the velocities involved in this model induces the dynamic behavior in both the vertically and horizontally, sausage-like and transverse motions. The horizontal dynamics are particularly an important aspect of this research as numerical simulations of solar jets don't typically report on cross sectional width variation, even though in observations of spicules they are not just vertically dynamic, as Solar spicules exhibit complex motions horizontally e.g. \citep{Sharma2018ApJ85361S,Antolin2018ApJ85644A}. The simulations have showed that jet boundary deformation can change dependent on the life cycle of the jet, which is studied in more detail in \citep{Mackenzie_Dover_2020}. This aspect needs further investigation as it could have implications for tracking solar jets in observations and chromospheric seismology.    

We found that the jets apex heights follow a non-ballistic parabolic path and we recover heights and widths in spicule range. This trend is observed in spicules, mottles, dynamic fibrils and marcospicules. The main parameter in determining the jet height is the amplitude of the driver. As we have obtained a scaling law between jet apex heights and initial velocity amplitude of driver, it makes it possible to estimate the energy involved in the driver based on observed spicule height. This relation needs to be confirmed or refuted by observations, first before drawing further conclusions. The parameter scan shows the impacted the morphology of the jet and raises important question of understanding what are driving these solar jets and at what heights. These series of simulation suggest that if the driver is an instant pulse or short driving time would lead to very thin, low density jets without beam structures. These models predict that in regions with strong magnetic field the jets would be thin, but be much denser and have have little beam structure, if they have sufficient driving time.

These simulation have shown interesting internal dynamics in the beam structures, that need to be confirmed observationally. We propose two possible mechanisms. However the supersonic flow framework explains the beam structures that are created, why there are areas of density enhancements refereed to as called knots, cavities along the central jet axis and the boundary deformation. Knots are seen in many jets from large astrophysical jets \citep{van_Putten_1996ApJ467L57V, DeGouveiaDalPino2005, Hada2013ApJ77570H, Cohen2014ApJ787151C, Hervet2017AnA606A103H} to small laboratory \citep{Menon2010, Edgington-Mitchell2014, Ono2014}, hence it makes natural sense that knots are could present in solar jets, if flow speeds and driver times are sufficient. While this has not yet been observed in small scale solar jets, we believe this is due to current resolution limits. It may be possible to identify jet-substructures in the foreseeable future with studies using data from Daniel K. Inouye Solar Telescope (DKIST) and European Solar Telescope (EST). In the simulations we have seen that the internal structures are directly linked to the driver time. An important outstanding question is what drives solar jets and if we could resolve these fine structures it may be possible to measure the driver times by their appearance and disappearance or if there are no structures present then this would indicate an short pulse-like event\textit{e.g.} magnetic recognition. This would give a new window to investigate this phenomena.

In future studies we plan to investigate what the effects a slight angle of between magnetic field strength and direction of flow would have on the jet widths and heights. As spicules are typically observed with an incline of $~20^{\circ}$ \citep{Tavabi2012JMPh31786T}.  
%------------------------------------------------------------------------------      
\section{Stuff to go somewhere}
%------------------------------------------------------------------------------    
%-----------------------------------------------------------------------------  
The way the jet is excited is analogous to jets created in a laboratory, in the sense we artificially created 'nozzle' in which drive our material into the atmosphere. Akin to laboratory jets our jet has high density ratio ($\eta_j >1$) and has Mach numbers ($\sim 1-3$). When describing a jets these parameters are key for defining its morphology and structure: $v_j$, $\eta_j = \frac{\rho_j}{\rho_a}$, $K = \frac{p_{d}}{p_a}$, where $\rho_a(p_a)$ is the ambient density(pressure) and $p_{d}$ is the pressure at the driver. $\eta_j$ measures the density contrast of the jet itself and the ambient medium it propagates through. If $\eta_j>1(\eta_j<1)$ then a jet is labeled to as heavy(light). For $K$ with denotes our jet to ambient pressures ratio determine whether the jet is under-expanded ($K>1$) or over-expanded ($K<1$). In our case we are dealing with the expected dynamics of a heavy under expanded-jet. In a paper by \cite{Norman1982} they describe the structures that occur in HD super sonic jets. The jets described in the paper are generated from a high pressure reservoir contain in the nozzle and internal structure of these jets is depicted in \fref{cartoon_jet_waves}. In the simulations shown in \fref{jet_simulation} we can see areas of enhanced brightness along the central axis of the jet, which we will refer to as knots. This phenomena occurs in laboratory jets \textit{e.g.} \citep{Ono2014,Edgington-Mitchell2014,Menon2010} and in astrophysical jets \textit{e.g.} \cite{Blandford2019,Belan2011,DeGouveiaDalPino2005,Birkinshaw1996}. In our case two possible mechanisms for explaining the knotted structures in the jet simulations: \\
\par (1) The knots could be created by the interaction of the pressure forces occurring inside and outside the jet. If we tracked a small portion of the jet as it evolves upwards it would undergo a series of expansion and contractions. Initially the jet pressure is greater than the surrounding atmosphere, thus as it rises it expands. However, this expansion lowers the jet pressure and as jet diameter increases it bends surrounding magnetic field lines. This increases the magnetic tension until it's greater than the jet pressure, causing the jet boundary to reconverges towards the jet center. As the jet diameter shrinks the jet pressure responds by increasing and becomes the dominant force pushing the jet boundary outwards. Thus, the cycle repeats as the jet tries to reach an equilibrium with its environment. Essentially, there is a ``tug of war" is occurring between the total pressure forces inside and outside the jet.  \\
\par (2) When a jet is going supersonic it will have a myriad of internal structures cause by shock waves \citep{Norman1982}. A schematic overview of these structures are given in Fig. \eqref{cartoon_jet_waves}. In the simulation the jet speeds reach roughly Mach $\sim 1-3$ of the local sound speed and the body of the jet itself has plasma beta $\sim 2-10$, therefore the MHD waves do not play the main role in internal structure of the jet and will be similar to dynamics of an HD supersonic jet. The notable features of interest in a supersonic jet is the osculating jet boundary and cross pattern occurring inside the jet. The jet boundary undulates as the gas periodically over expands and collapses in as it tries to reach and equilibrium with the ambient atmospheric pressure. The jet repeatedly overshoots the equilibrium position because the effects of the boundary are communicated to the interior of the jet by sound waves, which are traveling slower than supersonic flow of the jet. This leads the to lowest\textbackslash highest points of pressure in the jet being out of phase with the highest\textbackslash lowest jet diameter (see Fig. \eqref{cartoon_jet_waves}). As the jets pressure is greater than ambient atmospheric pressure the dynamics of the jet are akin to an under-expanded jet as shown in \cite{Norman1982,Edgington-Mitchell2014}. The crisscross shock pattern is created by a series of shock and expansion structures that can create these ``knots". An expansion fan (blue dashed lines in Fig. \eqref{cartoon_jet_waves}) forms at the base of the due to the difference in pressure between the jet and the ambient atmosphere. This expansion fan causes an outward flow making the jet enlarge. The Mach lines of the expansion waves reflects of the jet boundary inwards towards the jet center in the form of compression waves and a compression fan due pressure continuity (red lines in Fig. \eqref{cartoon_jet_waves}). The compression waves reflected at a near constant angle from the jet boundary and as this boundary is curved the Mach lines of the compressions waves have a tendency to converge into a conical shock wave before reaching the center of the jet. This incident shock either goes under a regular reflection or becomes a Mach disk depending on the angle between the incident shock and the central jet axis, for small and large angles respectively (see black lines Fig. \eqref{cartoon_jet_waves}). As the flow passes through this shock it will increase the pressure in the jet. When the reflected shock reaches the jet boundary it knocks the boundary outwards, creating an expansion fan and thus allows the process to be repeated.
%------ put this elsewhere
\par A schematic diagram of basic structure of a jet is given by \fref{ssj}. When a HD super sonic jet moves through the ambient medium, a bow shock is formed is formed in front of the advancing head of the jet and at the top of jet itself there is a Mach disk which is usually much stronger than the bow shock \citep{Chakrabarti1988}. The region close to advancing head of the jet is referred as working surface of the jet. Between the bow shock and the jet there are two other distinct regions the cocoon and screen which are separated by a contact discontinuity. The cocoon is the shocked jet medium and screen is the shocked ambient medium. The velocity of the jet head $v_h$ by comparing the momentum fluxes the jet and the external matter at the head of the jet and is estimated by the following,
\begin{equation}\label{v_j_speed}
v_h \approx \dfrac{v_j}{1+ \left( \dfrac{\rho_e}{\rho_j} \right)^{\sfrac{1}{2}}} 
\end{equation}
From \eref{v_j_speed} we can see for light jets ($\eta_j<1$) this means that speed of the head of the jet will be slower than the jet speed itself, this leads to a pile up of material which forms the cocoon. For heavy jets ($\eta_j>1$), the head of the jet will advance with speeds comparable to the jet itself and thus leads to much less back flow. This is why there is very little back flow in these jet The bow shock can be seen early on in the simulation and we see that the temperature increase at the head of the jet.
\par We have developed software to track the max jet heights (!get image example!) as it evolves and to measure the widths of the jet every megameter. The software tracts the jet using the tracer quantity that was set in simulation. A tracer tracks the density evoultion of th simaultion as described in \cite{Porth_2014}, we assign the jet with a vlaue 100 and everthing else as 0. We then set a threshold of $15\%$ of the jet value and convert into binary image (1 belong to jet and 0 not belonging to jet). To find the edges we take a horizontal slice every mega meter and look for changes and peak of the jet by taking the highest index value in the vertical. From this we create a data set which is used to produce Figs. (\ref{max_h_vs_b}-\ref{hight_dt}). We study how the different driver times, magnetic and drive amplitude effect the max height and average widths of the jet to see what effect this has on the jet morphology. From \fref{max_h_vs_b} it shows that strength of the magnetic has minimal impact on the max height reach by the jet. It important to note that the magnetic field may play a significant role in max jet heights if jets direction was not in-line with the magnetic field. As we can see from the color bands which corresponds to max amplitude reached by driver, this suggest that this is a key parameter for the max jet height. In \fref{max_h_vs_a} we conclude that max height is the jet mainly determined by the max amplitude reached by the driver with a parabolic trend. We observe that for our driver which is a momentum pulse we don't not get a linear trend between max jet height and pulse strength as reported in \cite{Singh2019}. \cite{Singh2019} used a a single instantaneous pressure pulse placed $1.8$ Mm. This suggests that height at which the driver is placed, its amplitude and its duration of driving is driven important to jet evolution. For example the longer distance that a pulse has to travel through the chromosphere to the TR the stronger the shocks the grow and the higher the jets will reach \citep{Shibata1982}. In \fref{max_h_vs_a} wee see that the driver time has some impact on max height. It shows that for short driver time of $50$ s there is a reduction in heights reached by the jets and there is no significant different between jet heights for a 200 and 300 s driver. We can see that the lines are banded together with matching line styles which again corresponds to matching velocities of the driver and reinforces that initial velocity is the main factor in determining max jet height if jet direction is inline with the magnetic field direction.
%------------------------
\par In \fref{mean_w_vs_A} we observe that the initial amplitude has little effect on jet widths. We a see pairing between drivers with $200$ and $300$ s, indicated by matching line styles, which is expected based on \fref{mean_w_vs_A}. The only parameter this doesn't apply is for $B=20 G$, this is due to the jet being weakly confined by the magentic feild. In \fref{mean_w_vs_dt} shows a linear trend between driving time, this is because with a longer driver more material will be supplied to the jet. There can also be a pile up in the jet beam as falling material interacts with raising material causing the jet to slighly ballon. We can see that the lines are grouped together based on their magnetic field strength represent by matching colors. This is an expected result as for weaker magnetic fields we get much larger jets widths, this is due to the magnetic field not being strong enough to compress the jet as it raises. Where as where the magnetic feild is stronger we observe that the jet widths are significantly reduced. This interpretation is evidenced by \fref{mean_w_vs_b} where we see larger jet widths for weaker magnetic field. We observe that they are group by there driver times as they have matching line styles. In \fref{hight_dt} we take use a the following paramters $P=300$ s, $B=50$ G and $A=60$ km s$^{-1}$ and run it with a much higher cadence in output. Each line in this \fref{hight_dt} represents a different height in the jet. We clearly the boundary of the jet are osculating over time. We don't see this phenomenon in simulations with a jet driven with a single instant pulse (!double check this claim! such as \citep{kuz2017ApJ,Singh2019}) we don't observe oscillations in there jet simulations, even though clearly in observations solar jets they oscillate (get refs). In Figs. (\ref{A20}-\ref{A80}) we have plotted the trajector of the max height of the jet over time, each Fig. is a displays all jets with the same driving amplitude. We find that our jets have a asymmetric parabolic path as reported in \cite{Singh2019}. We believe this is due to a piling of jet material as the jet begins to falls as falling matter will interact with raising matter. This will slow the jets decent. It is possible the magnetic field could impede the jets decent if the direction of falling material was at an angle with respect the magnetic field. We can see that heights reached are within the limits of what is expected for spicules. 
%%fffffffffffffffffffffffffff
%\mfig{0.8}{figures/jet_P300_B60A_60_t_1_10_13_19.png}{Simulation of jet in Stratified atmosphere with uniform magnetic field in vertical direction. This is the ``standard" jet with parameter $P=300$ s, $B=60$ G, $A=60$ km s$^{-1}$. Top, middle and bottom panels show density, temperature and numerical Schlieren respectively.}{jet_simulation}
%%fffffffffffffffffffffffffff
%\mfig{1}{figures/parameter_scan_1.png}{Simulation of jet in stratified atmosphere with uniform magnetic field in vertical direction. Examples of parameter scan with variations of one parameter of standard jet and all panels show the density evolution. Top, middle and bottom panels are $B=80,20$ G and $A=80$ km s$^{-1}$, respectively.}{pscan_den_plot1}
%\mfig{1}{figures/paramter_scan_2.png}{Continuation of \fref{pscan_den_plot1} with top, middle and bottom panels are $A=20$ km s$^{-1}$, $P=50$ and $200$ s, respectively.}{pscan_den_plot2}
%%fffffffffffffffffffffffffff
%\mfig{0.8}{figures/jet_diagram.eps}{schematic overview of the structure of a under-expanded supersonic HD jet.}{cartoon_jet_waves}
%%fffffffffffffffffffffffffff
%\mfig{0.5}{figures/diagram_of_ssj.png}{A schematic diagram giving an overview of the expected structures found in an HD super sonic jet dran in frame of the bow shock. Diagram taken from \cite{Chakrabarti1988}.}{ssj}
%%fffffffffffffffffffffffffff
%\mfig{1}{figures/combine_zoom.png}{todo.}{jet_head}

%%%%%%%%%%%%%%%%%%%%%%%%%%%%%%%%%%%%%%%%%%%%%%%%%%%%%%%
% STOP COPYING HERE
%%%%%%%%%%%%%%%%%%%%%%%%%%%%%%%%%%%%%%%%%%%%%%%%%%%%%%%

\bibliographystyle{plainnat}
\bibliography{references}

\end{document}
