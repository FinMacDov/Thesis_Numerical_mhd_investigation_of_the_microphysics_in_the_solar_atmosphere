\documentclass[12pt]{ociamthesis}

\usepackage{amssymb}
\usepackage{titlesec}
\usepackage{amsmath}
\usepackage{float}
\usepackage{graphicx}
\usepackage{caption}
\usepackage{subfig}
\usepackage{graphicx}
\usepackage{xcolor}
\usepackage[round]{natbib}
\usepackage[section]{placeins}
\usepackage{mathrsfs}
\usepackage{bm}
\usepackage{stmaryrd}
\usepackage[utf8]{inputenc}

\definecolor{theblue}{HTML}{0000CD}

% disable this package for printed version
\usepackage[colorlinks=true, linktocpage=true, allcolors=theblue]{hyperref}

\titleformat{\chapter}[display]
  {\bfseries\Large}
  {\filright\MakeUppercase{\chaptertitlename} \Large\thechapter}
  {1ex}
  {}
  [\vspace{1ex} \hrule \vspace{1pt} \hrule]

\newcommand{\adv}{    {\it Adv. Space Res.}} 
\newcommand{\annG}{   {\it Ann. Geophys.}} 
\newcommand{\aap}{    {\it Astron. Astrophys.}}
\newcommand{\aaps}{   {\it Astron. Astrophys. Suppl.}}
\newcommand{\aapr}{   {\it Astron. Astrophys. Rev.}}
\newcommand{\ag}{     {\it Ann. Geophys.}}
\newcommand{\aj}{     {\it Astron. J.}} 
\newcommand{\apj}{    {\it Astrophys. J.}}
\newcommand{\apjl}{   {\it Astrophys. J. Lett.}}
\newcommand{\apss}{   {\it Astrophys. Space Sci.}} 
\newcommand{\cjaa}{   {\it Chin. J. Astron. Astrophys.}} 
\newcommand{\gafd}{   {\it Geophys. Astrophys. Fluid Dyn.}}
\newcommand{\grl}{    {\it Geophys. Res. Lett.}}
\newcommand{\ijga}{   {\it Int. J. Geomagn. Aeron.}}
\newcommand{\jastp}{  {\it J. Atmos. Solar-Terr. Phys.}} 
\newcommand{\jgr}{    {\it J. Geophys. Res.}}
\newcommand{\mnras}{  {\it Mon. Not. Roy. Astron. Soc.}}
\newcommand{\nat}{    {\it Nature}}
\newcommand{\pasp}{   {\it Pub. Astron. Soc. Pac.}}
\newcommand{\pasj}{   {\it Pub. Astron. Soc. Japan}}
\newcommand{\pre}{    {\it Phys. Rev. E}}
\newcommand{\solphys}{{\it Solar Phys.}}
\newcommand{\sovast}{ {\it Soviet  Astron.}} 
\newcommand{\ssr}{    {\it Space Sci. Rev.}}
\newcommand{\caa}{{\it Chinese Astron. Astrohpys.}} 
\newcommand{\apjs}{    {\it Astrophys. J. Suppl.}}

\def\UrlFont{\sf}

\begin{document}

\baselineskip=18pt

%%%%%%%%%%%%%%%%%%%%%%%%%%%%%%%%%%%%%%%%%%%%%%%%%%%%%%%
% START COPYING HERE
%%%%%%%%%%%%%%%%%%%%%%%%%%%%%%%%%%%%%%%%%%%%%%%%%%%%%%%

\chapter{Conclusions}

\section{Overview of Thesis}

This Thesis constitutes a study of magnetoacoustic waves and the magnetohydrodynamic Kelvin-Helmholtz instability of two novel equilibrium configurations.
Chapter 1 introduces the background material needed to develop the new theory.
The equations of ideal magnetohydrodynamics are derived from Maxwell's equations and the equations of gas dynamics.
Subsequently, the basic theory of MHD waves is derived from the ideal linear MHD equations, and the Kelvin-Helmholtz instability and its possible applications to solar physics are introduced.

The first of the two novel models is studied in Chapter 2.
The equilibrium configuration being investigated is that of a steady slab of magnetised plasma subject to a steady flow, embedded in a non-magnetic atmosphere, with different background parameters on each side of the slab.
This configuration differs from those in previous models since it considers both a steady flow within the slab, as well as asymmetry of the exterior parameters.
A dispersion relation for magnetoacoustic waves propagating along the slab is derived and its solutions are obtained in both approximate analytical form, and in general numerical form.
Applications to solar physics are discussed in the context of determining the parameters of Kelvin-Helmholtz unstable flanks of coronal mass ejections.

The second model is investigated in Chapter 3.
It involves the study of an interface separating temporally oscillating background plasmas in a magnetic environment.
The analytical study of transient flows in MHD has been rather uncommon due to the difficulty of obtaining results in closed form.
However, utilising the incompressible plasma approximation allowed us to obtain the governing equation of the stability of the boundary in the form of Mathieu's equation.
We studied the general stability of its solutions as well as the nature of the initial value problem for the given form of the parameters.
An applications to the transverse wave induced Kelvin-Helmholtz instability in coronal loops using this model is also presented in this Chapter.

\section{Summary of Results}

\subsection{Chapter 2}

The dispersion relation of waves propagating along a steady slab in an asymmetric environment was obtained in Subsection \ref{subsec:disprel}, and the possible modes of propagation are classified in terms of the characteristic speeds in Subsection \ref{subsec:modes}.
The analytical solutions of the dispersion relation in the incompressible plasma limit and the thin and wide slab approximations are obtained in Section \ref{sec:analytical}.
General numerical solutions are obtained in Section \ref{sec:numerics} and it is found that the modes are consistent with the classification in terms of the characteristic speeds from Subsection \ref{subsec:modes}.
The Kelvin-Helmholtz threshold value for the most unstable mode, the slow surface kink mode, is also obtained in Section \ref{sec:numerics} and it is found that the asymmetry in background density may significantly lower the threshold value for thin slabs.
Finally, we estimate the densities of a Kelvin-Helmholtz unstable flank of a coronal mass ejection in Section \ref{sec:c2app}.
We find that the flank of the CME is at least three orders of magnitude denser than the background corona for the observed parameters of the KHI.

\subsection{Chapter 3}

The governing relation of the stability of a boundary separating temporally oscillating flows is found to be Mathieu's equation in Section \ref{sec:goveq}.
In order to better understand the stability of the boundary separating oscillating flows at an angle, we first study the stability of steady flows at an angle in Subsection \ref{subsec:steady}.
We find that the initial value problem in this case is ill-posed regardless of the magnitude of the flow.
The stability of the transient configuration is studied in Subsection \ref{subsec:oscillating}, and it is found that solutions are always unstable for some value of the wavenumber magnitude.
The initial value problem is found to be ill-posed when the flow magnitude is above a certain threshold and well-posed when it is below this threshold.
The application of the results of this model are applied to the study of the KHI of transverse coronal loop oscillations in Section \ref{sec:loop}.
In Subsection \ref{subsec:sigma} we introduce the concept of $\sigma$-stability and find the analytical condition for coronal loops to be $\sigma$-stable.
Finally, in Subsection \ref{subsec:loop} we introduce a new set of parameters which should generalise the local stability analysis to the three dimensional and cylindrical nature of coronal loops.
We find that coronal loops are always unstable to the KHI, but may become $\sigma$-stable if there is magnetic twist present.

\section{Future Work}

A natural extension to the investigation performed in Chapter 2 would be to introduce parallel magnetic fields of different magnitudes in the exterior regions of the slab.
This would significantly alter the classification of the modes of propagation with respect to the characteristic speeds since there would be two new Alfv\'en speeds.

A different possible direction would be to replace the boundaries of the slab with thin regions of linear density change.
The slab would, therefore, no longer be delimited by tangential discontinuities, but by linear density transition regions.
This would severely complicate the analysis since the linear change in density introduces a singularity in the Alfv\'en speed.

The model in Chapter 3 may be extended by including a linear decrease of the flow from one side of the interface to the other.
Similarly, a linear decrease in density could also be considered, which would let us study the effects of resonant absorption on the instability.
Incorporating both of the aforementioned modifications would yield a model closer to what we believe is the structure of a coronal loop.
However, such a model is bound to be mathematically very complex and there is no assurance that a governing equation can be obtained analytically.

A different avenue of exploration is that of the application of the existing model, with only minor modifications, to other solar and magnetospheric phenomena.
A potential application would be to kink oscillations of solar prominences.
An important differences between solar prominences and coronal loops is that the plasma in the former may not be fully ionized.
This effect would have to be taken into account in the model.


%%%%%%%%%%%%%%%%%%%%%%%%%%%%%%%%%%%%%%%%%%%%%%%%%%%%%%%
% STOP COPYING HERE
%%%%%%%%%%%%%%%%%%%%%%%%%%%%%%%%%%%%%%%%%%%%%%%%%%%%%%%

\bibliographystyle{aasjournal}
\bibliography{references}  

\end{document}
