\documentclass[12pt]{ociamthesis}

\usepackage{amssymb}
\usepackage{titlesec}
\usepackage{amsmath}
\DeclareMathOperator{\arcsec}{arcsec}
\usepackage{float}
\usepackage{graphicx}
\usepackage{caption}
\usepackage{subfig}
\usepackage{xcolor}
\usepackage[section]{placeins}
\usepackage{mathrsfs}
\usepackage{bm}
\usepackage{stmaryrd}
\usepackage{siunitx}
\usepackage{rotating}
\usepackage[utf8]{inputenc}
\usepackage[round]{natbib}
\usepackage{epigraph}

\usepackage{wrapfig}
\usepackage{lscape}
\usepackage{epstopdf}

\usepackage{afterpage}
\usepackage{pdflscape}
\usepackage{xfrac}


\usepackage{geometry}
 \geometry{
 a4paper,
 left=40mm,
 right=30mm,
 top=30mm,
 bottom=30mm
 }

\definecolor{theblue}{HTML}{0000CD}

% disable this package for printed version
\usepackage[colorlinks=true, linktocpage=true, allcolors=theblue]{hyperref}

\titleformat{\chapter}[display]
  {\bfseries\Large}
  {\filright\MakeUppercase{\chaptertitlename} \Large\thechapter}
  {1ex}
  {}
  [\vspace{1ex} \hrule \vspace{1pt} \hrule]

\newcommand{\adv}{    {\it Adv. Space Res.}} 
\newcommand{\araa}{    {\it Annual Review of Astron and Astrophys.}} 
\newcommand{\annG}{   {\it Ann. Geophys.}} 
\newcommand{\aap}{    {\it Astron. Astrophys.}}
\newcommand{\aaps}{   {\it Astron. Astrophys. Suppl.}}
\newcommand{\aapr}{   {\it Astron. Astrophys. Rev.}}
\newcommand{\ag}{     {\it Ann. Geophys.}}
\newcommand{\aj}{     {\it Astron. J.}} 
\newcommand{\apj}{    {\it Astrophys. J.}}
\newcommand{\apjl}{   {\it Astrophys. J. Lett.}}
\newcommand{\apss}{   {\it Astrophys. Space Sci.}} 
\newcommand{\bain}{   {\it Bulletin of the Astronomical Institutes of the Netherlands.}} 
\newcommand{\cjaa}{   {\it Chin. J. Astron. Astrophys.}} 
\newcommand{\gafd}{   {\it Geophys. Astrophys. Fluid Dyn.}}
\newcommand{\grl}{    {\it Geophys. Res. Lett.}}
\newcommand{\ijga}{   {\it Int. J. Geomagn. Aeron.}}
\newcommand{\jastp}{  {\it J. Atmos. Solar-Terr. Phys.}} 
\newcommand{\jgr}{    {\it J. Geophys. Res.}}
\newcommand{\mnras}{  {\it Mon. Not. Roy. Astron. Soc.}}
\newcommand{\na}{     {\it New Astronomy}}
\newcommand{\nat}{    {\it Nature}}
\newcommand{\pasp}{   {\it Pub. Astron. Soc. Pac.}}
\newcommand{\pasj}{   {\it Pub. Astron. Soc. Japan}}
\newcommand{\pre}{    {\it Phys. Rev. E}}
\newcommand{\solphys}{{\it Solar Phys.}}
\newcommand{\sovast}{ {\it Soviet  Astron.}} 
\newcommand{\ssr}{    {\it Space Sci. Rev.}}
\newcommand{\caa}{    {\it Chinese Astron. Astrohpys.}} 
\newcommand{\apjs}{   {\it Astrophys. J. Suppl.}}
\newcommand{\zap}{   {\it Zeitschrift fuer Astrophysik}}

\newcommand{\bs}[1]{\boldsymbol{#1}}
\newcommand{\bn}{\boldsymbol{\nabla}}
\newcommand{\rgas}{\mathcal{R}}
\newcommand{\eref}[1]{Eq. \eqref{#1}}
\newcommand{\fref}[1]{Fig. \eqref{#1}}
\newcommand\encircle[1]{%
  \tikz[baseline=(X.base)] 
    \node (X) [draw, shape=circle, inner sep=0] {\strut #1};}
\newcommand{\Alfven}{Alfv\'{e}n } 
\newcommand{\Alfvenic}{Alfv\'{e}nic }
\newcommand{\size}{0.75}
\newcommand\measureISpecification{4ex}% not defined in mwe
\newcommand{\ctab}[1]{\raisebox{\dimexpr \measureISpecification/2 -.748ex}{#1}}% vertically centers numbers
\newcommand{\mfig}[4]{
  \begin{figure}
  \begin{center}
  \includegraphics[width=#1\linewidth]{#2}
  \caption{#3}
  \label{#4}
  \end{center}
  \end{figure}}
\newcommand{\kms}{~\rm{km ~s^{-1}}}
\newcommand{\kgm}{~\rm{kg ~m^{-3}}}
\newcommand{\np}{\\ \\}
\newcommand{\degs}{^{\circ}}

\setcounter{secnumdepth}{3}
\setcounter{tocdepth}{3}


\begin{document}
% NEED TO KILL THIS FOR FINAL COMPILATION!!!!!!!!!!
%\pagenumbering{gobble}
% ALLOWS TO REMOVE PAGE NUMBERS, WHICH IS BETTER FOR CONVERTING TO WORD DOC

\baselineskip=18pt


%%%%%%%%%%%%%%%%%%%%%%%%%%%%%%%%%%%%%%%%%%%%%%%%%%%%%%%
% START COPYING HERE
%%%%%%%%%%%%%%%%%%%%%%%%%%%%%%%%%%%%%%%%%%%%%%%%%%%%%%%
%------------------------------------------------------------------------------
\chapter{Conclusion and Future Work}
\label{chap:con_and_fut_work}
%-------------------------------
\section{Conclusion}
\label{sec:con}
%-------------------------------
The thesis aimed to investigate the fundamental dynamics and morphology of spicular jets. We have achieved this through a simple model, which uses a momentum pulse to drive the jets to simulate a transfer of mass in an idealised solar atmosphere and with a vertical uniform magnetic field. We have taken full advantage of MPI-AMRVAC mesh options to simulate the synthetic jets' rising and falling dynamics with a high spatial resolution. \np
%
In Chapter~\ref{chap:sj} vertical launching of the jet was investigated. An extensive parameter space was investigated, which determined that spicule dynamics and morphology are sensitive to the parameters related to the driver (lifetimes and initial amplitude) and magnetic environment (magnetic field strength). The vertical and horizontal evolution of spicules is measured with our own automated jet tracking software. By using the jet tracking software we capture the main observable features of spicules with the synthetic jets, as we obtained similar heights, speeds, widths and trajectories. Despite numerous numerical studies on the subject of spicular jets, few have reported on the CSW variations and fine structures of spicular jets. We report on both of these and find complex fine structures in the jet beam of spicules due to shock waves, which also create symmetrical deformation of the jet boundary. These fine structures are yet to be observed, but we show that higher resolution observations with telescopes such as DKIST may be able to identify these fine structures.  \np
%
In Chapter~\ref{chap:tilted_jets} we continued to study with the same model for launching jets, but explored the effects that misalignment between the direction of the driver and the magnetic field has on the dynamics and morphology of the synthetic jet. The tilt of the synthetic jet flows introduces interesting dynamics that were not seen in Chapter~\ref{chap:sj}. The main result of this study, even with small tilts angles of $5\degs$, there is a significant impact on the appearance of the synthetic jets. In this thesis, we have presented one of the few models that obtained both the transverse and horizontal dynamics of the spicule. An interesting result is that the misaligned flows cause whip motions and redistribution of the densities to the jet edges. We tentatively propose that the whip motions are possibly a consequence of a DKI, but further research is required. Our research highlights the importance of properly tracking the central axis of the jets to obtain accurate jet trajectories and measurements of CSWs. Non-field aligned flows give another possibility for producing transverse motion in spicules. Given the complexity of the structure of the atmosphere that spicules propagate in, we conjecture that it is likely that non-field flow will occur, and it is an important aspect to include in any models of spicules. \np
%
In Chapter~\ref{cap:obs} we revisit the field-aligned jets and compare their properties with H$\alpha$ observations captured by SST. CSW variations are present in both observations and synthetic jets. We perform a wavelet analysis to measure any significant wave periods. One of the main results in this thesis is that we are the first to report differences in the jet boundary deformation when the jet is in its rising or falling phase. Under the wave interpretation, there is frequency modulation in the falling phase of all the jets, with shorter periods in the falling phase in comparison to the rising phase. We propose that this is due to the change in direction of the mass flows of the jet with respect to the waves. In this wave-based framework, the results of the synthetic jets could be interpreted as identifying the fundamental mode and its first overtone of the sausage wave. However, I think that the results for the CSW variations of the synthetic jets highlight that we should be cautious with this interpretation. As described in Chapter~\ref{chap:tilted_jets}, we showed that CSW variations are possible due to supersonic flows. The boundary deformation of the synthetic jets shows no obvious sign of having an antinode which is an expected feature for a wave, and this may indicate CSW variations are not due to waves, but rather a consequence of the supersonic flow. This result should be kept in mind for observers reporting variations of spicule widths, as there may be other phenomena that can account for these motions. \np
%
The numerical study presented here has limitations when compared with the spicules actual nature. To have a closer match with spicules', extra physics would need to be included, such as radiative transfer, heat conduction, and deploying a 3D model. These limitations are mostly addressed in the future work section. Despite this, there is value in studying less complex models, as they gives insight into the fundamental nature of spicules and allow us to more easily unpick the phenomena occurring in the simulations. This thesis highlights the importance of identifying what is driving the jets, as it is the most important factor for determining their dynamics and morphology. We are at the beginning of the next generation of solar telescopes, with mirror sizes approximately four times that of SST, such as DKIST, which recently released the highest resolution image of the Sun's surface to date and allowed us to see features as small as $30~\rm{km}$, and EST, with first light planned in $2026$. When these data become available, we can advance our knowledge of the fine structuring of the spicules and pin down their driving mechanism. This would be a significant step forward and allow us to narrow down the appropriate numerical models from which to simulate spicules. 
%-------------------------------
\section{Future Work}
\label{sect:fut_work}
%-------------------------------
Multiple improvements can be undertaken to build upon the existing work in this thesis. The jet tracking code for the tilted examples could be improved further. Its current weakness is that it relies on horizontal slits to track the central axis. It would be better if one autonomously separated the left and right boundary of the jet, which is not a trivial task. Then, for each boundary, measure a set distance to assign data points to match up with its boundary counterpart. Then, taking the midpoints of these pairs would yield a more accurate position of the central jet axis. Steps that need to be undertaken to integrate numerical simulations with observations are as follows; spicules are 3D objects, therefore with a 2D model we may be missing important dynamics, for example, torsional motions \citep{dePontieu2012ApJ752L12D}, and in astrophysical jets, RTI can format the boundary in 3D jet simulations for fast moving jets \citep{Matsumoto772L1M}. A 3D simulation would better capture the dynamical motion of the spicules and may yield interesting results for the behaviour of the boundary deformation. All the simulations in this thesis use a simplified solar atmosphere, and it would be more appropriate to base the atmospheric stratification on semi-empirical data such as VALC or the C7 atmospheric model \citep{Vernazza1981ApJS45635V,Avrett2008ApJS175229A}. More complex physics could be added; radiative transfer and heat conduction are important to include as they can impact the final height of the spicules \citep{Sterling1990ApJ349647S}. These additional physics are key if we wish to study the synthetic jet's contribution to chromospheric/coronal heating. Carrying out a two-fluid model of the simulation could be another avenue to investigate, as neutrals could play an important role in spicule evolution \citep{kuzma2017ApJ84978K}. Another direction to investigate would be to try different magnetic field configurations. A significant progression would be to see how an expanding flux tube would impact the presence of the knots. One major improvement of the model, particularly useful for comparing to observations, would be to use forward modelling to convert the parameter of the simulation into observables. For example, the FoMo code \citep{Doorsselaere2016FrASS34V} is set up to accept MPI-AMRVAC data structures. Forward modelling would allow for a more concrete comparison to observations, and may allow us to identify potential observational signatures for the jet phenomena simulated throughout the thesis.
%
%%%%%%%%%%%%%%%%%%%%%%%%%%%%%%%%%%%%%%%%%%%%%%%%%%%%%%%
% STOP COPYING HERE
%%%%%%%%%%%%%%%%%%%%%%%%%%%%%%%%%%%%%%%%%%%%%%%%%%%%%%%

\bibliographystyle{plainnat}
\bibliography{references}  

\end{document}
