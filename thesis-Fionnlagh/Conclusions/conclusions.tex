\documentclass[12pt]{ociamthesis}

\usepackage{amssymb}
\usepackage{titlesec}
\usepackage{amsmath}
\DeclareMathOperator{\arcsec}{arcsec}
\usepackage{float}
\usepackage{graphicx}
\usepackage{caption}
\usepackage{subfig}
\usepackage{xcolor}
\usepackage[section]{placeins}
\usepackage{mathrsfs}
\usepackage{bm}
\usepackage{stmaryrd}
\usepackage{siunitx}
\usepackage{rotating}
\usepackage[utf8]{inputenc}
\usepackage[round]{natbib}
\usepackage{epigraph}

\usepackage{wrapfig}
\usepackage{lscape}
\usepackage{epstopdf}

\usepackage{afterpage}
\usepackage{pdflscape}
\usepackage{xfrac}


\usepackage{geometry}
 \geometry{
 a4paper,
 left=40mm,
 right=30mm,
 top=30mm,
 bottom=30mm
 }

\definecolor{theblue}{HTML}{0000CD}

% disable this package for printed version
\usepackage[colorlinks=true, linktocpage=true, allcolors=theblue]{hyperref}

\titleformat{\chapter}[display]
  {\bfseries\Large}
  {\filright\MakeUppercase{\chaptertitlename} \Large\thechapter}
  {1ex}
  {}
  [\vspace{1ex} \hrule \vspace{1pt} \hrule]

\newcommand{\adv}{    {\it Adv. Space Res.}} 
\newcommand{\araa}{    {\it Annual Review of Astron and Astrophys.}} 
\newcommand{\annG}{   {\it Ann. Geophys.}} 
\newcommand{\aap}{    {\it Astron. Astrophys.}}
\newcommand{\aaps}{   {\it Astron. Astrophys. Suppl.}}
\newcommand{\aapr}{   {\it Astron. Astrophys. Rev.}}
\newcommand{\ag}{     {\it Ann. Geophys.}}
\newcommand{\aj}{     {\it Astron. J.}} 
\newcommand{\apj}{    {\it Astrophys. J.}}
\newcommand{\apjl}{   {\it Astrophys. J. Lett.}}
\newcommand{\apss}{   {\it Astrophys. Space Sci.}} 
\newcommand{\bain}{   {\it Bulletin of the Astronomical Institutes of the Netherlands.}} 
\newcommand{\cjaa}{   {\it Chin. J. Astron. Astrophys.}} 
\newcommand{\gafd}{   {\it Geophys. Astrophys. Fluid Dyn.}}
\newcommand{\grl}{    {\it Geophys. Res. Lett.}}
\newcommand{\ijga}{   {\it Int. J. Geomagn. Aeron.}}
\newcommand{\jastp}{  {\it J. Atmos. Solar-Terr. Phys.}} 
\newcommand{\jgr}{    {\it J. Geophys. Res.}}
\newcommand{\mnras}{  {\it Mon. Not. Roy. Astron. Soc.}}
\newcommand{\na}{     {\it New Astronomy}}
\newcommand{\nat}{    {\it Nature}}
\newcommand{\pasp}{   {\it Pub. Astron. Soc. Pac.}}
\newcommand{\pasj}{   {\it Pub. Astron. Soc. Japan}}
\newcommand{\pre}{    {\it Phys. Rev. E}}
\newcommand{\solphys}{{\it Solar Phys.}}
\newcommand{\sovast}{ {\it Soviet  Astron.}} 
\newcommand{\ssr}{    {\it Space Sci. Rev.}}
\newcommand{\caa}{    {\it Chinese Astron. Astrohpys.}} 
\newcommand{\apjs}{   {\it Astrophys. J. Suppl.}}
\newcommand{\zap}{   {\it Zeitschrift fuer Astrophysik}}

\newcommand{\bs}[1]{\boldsymbol{#1}}
\newcommand{\bn}{\boldsymbol{\nabla}}
\newcommand{\rgas}{\mathcal{R}}
\newcommand{\eref}[1]{Eq. \eqref{#1}}
\newcommand{\fref}[1]{Fig. \eqref{#1}}
\newcommand\encircle[1]{%
  \tikz[baseline=(X.base)] 
    \node (X) [draw, shape=circle, inner sep=0] {\strut #1};}
\newcommand{\Alfven}{Alfv\'{e}n } 
\newcommand{\Alfvenic}{Alfv\'{e}nic }
\newcommand{\size}{0.75}
\newcommand\measureISpecification{4ex}% not defined in mwe
\newcommand{\ctab}[1]{\raisebox{\dimexpr \measureISpecification/2 -.748ex}{#1}}% vertically centers numbers
\newcommand{\mfig}[4]{
  \begin{figure}
  \begin{center}
  \includegraphics[width=#1\linewidth]{#2}
  \caption{#3}
  \label{#4}
  \end{center}
  \end{figure}}
\newcommand{\kms}{~\rm{km ~s^{-1}}}
\newcommand{\kgm}{~\rm{kg ~m^{-3}}}
\newcommand{\np}{\\ \\}
\newcommand{\degs}{^{\circ}}

\setcounter{secnumdepth}{3}
\setcounter{tocdepth}{3}


\begin{document}

\baselineskip=18pt


%%%%%%%%%%%%%%%%%%%%%%%%%%%%%%%%%%%%%%%%%%%%%%%%%%%%%%%
% START COPYING HERE
%%%%%%%%%%%%%%%%%%%%%%%%%%%%%%%%%%%%%%%%%%%%%%%%%%%%%%%
%------------------------------------------------------------------------------
\chapter{Conclusion and Future Work}
\label{chap:con_and_fut_work}
%-------------------------------
\section{Conclusion}
\label{sec:con}
%-------------------------------
chapter 3 \np
%
chapter 4 \np
%
chapter 5 \np
%
finishing comments
%-------------------------------
\section{Future Work}
\label{sect:fut_work}
%-------------------------------
There are multiple improvements that can undertaken to build upon the existing work in this thesis. Steps that need to undertaken to blur lines between numerical simulation and observations are as follows. Spicules are 3D objects, therefore a 2D model we may be missing import dynamics, for example torsional motions which is thought to occur in spicular jets \citep{dePontieu2012ApJ752L12D}, and in astrophysical jets its possible for RTI to form at boundary in 3D simulations jet for fast moving jets \citep{Matsumoto772L1M}. A 3D simulation would better capture the dynamical motion of the spicules and may yield interesting results for the behavior of the boundary deformation. All the simulations in this thesis use a simplified solar atmosphere, it would be more appropriate to base the atmospheric stratification one semi-empirical data such as VALC or C7 atmospheric model \citep{Vernazza1981ApJS45635V,Avrett2008ApJS175229A}. More complex physics could be added, radiate transfer and heat conduction are important to include as they can impact the final height of the spicules \citep{Sterling1990ApJ349647S}. Carrying out a two fluid model of the simulation could be another avenue to investigate, as neutral could place an important role in spicule dynamics \citep{kuzma2017ApJ84978K}. Another direction to investigate would be to try different magnetic filed configurations, I think it would be very interesting to see how an expanding flux tube would impact the presence of the knots. One major improvement of the model, particularly useful for comparing to observations would be to use forward modeling to convert the parameter of the simulation into observables. For example the FoMo code \citep{Doorsselaere2016FrASS34V}, is set up to accept MPI-AMRVAC data structures. Forward modelling would allow for more concrete comparison to observations and may allows us to pick out potential observational signatures for the jet phenomena simulated throughout the thesis. \np 
%
We are at the beginning of the next generation of solar telescopes with mirror sizes approximately four times that of SST, such as DKIST  which recently release the highest resolution image of Sun's surface to date and allowed us to see features as small as $30~\rm{km}$, and EST with first light planned in $2026$. When data becomes available from these telescope, they should be used to advance our knowledge of the fine structuring of the spicules and hopefully pin down their driving mechanism to allow to narrow down the appropriate numerical model to capture their characteristics.

%%%%%%%%%%%%%%%%%%%%%%%%%%%%%%%%%%%%%%%%%%%%%%%%%%%%%%%
% STOP COPYING HERE
%%%%%%%%%%%%%%%%%%%%%%%%%%%%%%%%%%%%%%%%%%%%%%%%%%%%%%%

\bibliographystyle{plainnat}
\bibliography{references}  

\end{document}
