\documentclass[12pt]{ociamthesis}

\usepackage{amssymb}
\usepackage{titlesec}
\usepackage{amsmath}
\DeclareMathOperator{\arcsec}{arcsec}
\usepackage{float}
\usepackage{graphicx}
\usepackage{caption}
\usepackage{subfig}
\usepackage{xcolor}
\usepackage[section]{placeins}
\usepackage{mathrsfs}
\usepackage{bm}
\usepackage{stmaryrd}
\usepackage{siunitx}
\usepackage{rotating}
\usepackage[utf8]{inputenc}
\usepackage[round]{natbib}
\usepackage{epigraph}

\usepackage{wrapfig}
\usepackage{lscape}
\usepackage{epstopdf}

\usepackage{afterpage}
\usepackage{pdflscape}


\usepackage{geometry}
 \geometry{
 a4paper,
 left=40mm,
 right=30mm,
 top=30mm,
 bottom=30mm
 }

\definecolor{theblue}{HTML}{0000CD}

% disable this package for printed version
\usepackage[colorlinks=true, linktocpage=true, allcolors=theblue]{hyperref}

\titleformat{\chapter}[display]
  {\bfseries\Large}
  {\filright\MakeUppercase{\chaptertitlename} \Large\thechapter}
  {1ex}
  {}
  [\vspace{1ex} \hrule \vspace{1pt} \hrule]

\newcommand{\adv}{    {\it Adv. Space Res.}} 
\newcommand{\araa}{    {\it Annual Review of Astron and Astrophys.}} 
\newcommand{\annG}{   {\it Ann. Geophys.}} 
\newcommand{\aap}{    {\it Astron. Astrophys.}}
\newcommand{\aaps}{   {\it Astron. Astrophys. Suppl.}}
\newcommand{\aapr}{   {\it Astron. Astrophys. Rev.}}
\newcommand{\ag}{     {\it Ann. Geophys.}}
\newcommand{\aj}{     {\it Astron. J.}} 
\newcommand{\apj}{    {\it Astrophys. J.}}
\newcommand{\apjl}{   {\it Astrophys. J. Lett.}}
\newcommand{\apss}{   {\it Astrophys. Space Sci.}} 
\newcommand{\bain}{   {\it Bulletin of the Astronomical Institutes of the Netherlands.}} 
\newcommand{\cjaa}{   {\it Chin. J. Astron. Astrophys.}} 
\newcommand{\gafd}{   {\it Geophys. Astrophys. Fluid Dyn.}}
\newcommand{\grl}{    {\it Geophys. Res. Lett.}}
\newcommand{\ijga}{   {\it Int. J. Geomagn. Aeron.}}
\newcommand{\jastp}{  {\it J. Atmos. Solar-Terr. Phys.}} 
\newcommand{\jgr}{    {\it J. Geophys. Res.}}
\newcommand{\mnras}{  {\it Mon. Not. Roy. Astron. Soc.}}
\newcommand{\na}{     {\it New Astronomy}}
\newcommand{\nat}{    {\it Nature}}
\newcommand{\pasp}{   {\it Pub. Astron. Soc. Pac.}}
\newcommand{\pasj}{   {\it Pub. Astron. Soc. Japan}}
\newcommand{\pre}{    {\it Phys. Rev. E}}
\newcommand{\solphys}{{\it Solar Phys.}}
\newcommand{\sovast}{ {\it Soviet  Astron.}} 
\newcommand{\ssr}{    {\it Space Sci. Rev.}}
\newcommand{\caa}{    {\it Chinese Astron. Astrohpys.}} 
\newcommand{\apjs}{   {\it Astrophys. J. Suppl.}}
\newcommand{\zap}{   {\it Zeitschrift fuer Astrophysik}}

\newcommand{\bs}[1]{\boldsymbol{#1}}
\newcommand{\bn}{\boldsymbol{\nabla}}
\newcommand{\rgas}{\mathcal{R}}
\newcommand{\eref}[1]{Eq. \eqref{#1}}
\newcommand{\fref}[1]{Fig. \eqref{#1}}
\newcommand\encircle[1]{%
  \tikz[baseline=(X.base)] 
    \node (X) [draw, shape=circle, inner sep=0] {\strut #1};}
\newcommand{\Alfven}{Alfv\'{e}n } 
\newcommand{\Alfvenic}{Alfv\'{e}nic }
\newcommand{\size}{0.75}
\newcommand\measureISpecification{4ex}% not defined in mwe
\newcommand{\ctab}[1]{\raisebox{\dimexpr \measureISpecification/2 -.748ex}{#1}}% vertically centers numbers
\newcommand{\mfig}[4]{
  \begin{figure}
  \begin{center}
  \includegraphics[width=#1\linewidth]{#2}
  \caption{#3}
  \label{#4}
  \end{center}
  \end{figure}}
\newcommand{\kms}{~\rm{km ~s^{-1}}}
\newcommand{\kgm}{~\rm{kg ~m^{-3}}}
\newcommand{\np}{\\ \\}
\newcommand{\degs}{^{\circ}}

\begin{document}

\baselineskip=18pt

\setcounter{secnumdepth}{3}
\setcounter{tocdepth}{3}

%%%%%%%%%%%%%%%%%%%%%%%%%%%%%%%%%%%%%%%%%%%%%%%%%%%%%%%
% START COPYING HERE
%%%%%%%%%%%%%%%%%%%%%%%%%%%%%%%%%%%%%%%%%%%%%%%%%%%%%%%
\section*{List of Symbols}
Below is a list of the notation used throughout the text unless stated otherwise: \\ \\
$\rho$ $\rightarrow$ Density.  \\
$p$ $\rightarrow$ Pressure. \\
$\boldsymbol{v} = (v_x, v_y, v_z)$ $\rightarrow$ Velocity.  \\
$\boldsymbol{B} = (B_x,B_y,B_z)$ $\rightarrow$ Magnetic field strength. \\
$\mu_0$ $\rightarrow$ Magnetic permeability. \\
$\boldsymbol{g} = (0,0,-g_z)$ $\rightarrow$ Gravitational acceleration. \\
$\gamma = 5/3$ $\rightarrow$ Ratio of specific heat. \\
$\eta$ $\rightarrow$ Magnetic diffusivity of the plasma. \\
$\widetilde{\mu}$ $\rightarrow$ Mean atomic weight (the average mass per particle in the units of the proton mass).  \\
$R$ $\rightarrow$ Gas constant.\\
$\boldsymbol{j} = (1 / \mu_0) (\nabla \times \boldsymbol{B})$ $\rightarrow$ Current density.  \\
$H(z) = \dfrac{R T(z)}{\widetilde{\mu} g}$ $\rightarrow$ Scale height.  \\
$G$ $\rightarrow$ Gravitational constant. \\
$M_{\odot}$ $\rightarrow$ Solar mass. \\
$R_{\odot}$ $\rightarrow$ Solar radius. \\
$p_{mag} = \dfrac{B^2}{2 \mu_0} $ $\rightarrow$ Magnetic pressure. \\
$P=p_{tot} = p + p_{mag} $ $\rightarrow$ Total pressure. \\
$m = \rho \boldsymbol{v}$ $\rightarrow$ Momentum density. \\
$e$ $\rightarrow$ Total energy density. \\ 
$\boldsymbol{\xi}$ $\rightarrow$ Plasma displacement. \\
$C^2_s = \gamma \dfrac{p}{\rho}$ $\rightarrow$ Sound speed squared. \\ 
$V_A^2=\dfrac{B^2}{\mu_0 \rho}$ $\rightarrow$ \Alfven speed squared. 
$\eta$ $\rightarrow$ magnetic diffusivity of the plasma
\clearpage
\setcounter{page}{1}
%------------------------------------------------------------------------------
\chapter{Introduction}
\label{chap:intro}
%------------------------------------------------------------------------------
%\epigraphfontsize{\small\itshape}
\epigraph{``We're going to explore the outside world someday, right? Far beyond these walls, there's flaming water, land made of ice, and fields of sand spread wide."}{--- \textit{Armin Arlert}, \textup{Hajime Isayama}, Vol. 2  Attack on Titian}
%------------------------------------------------------------------------------
\section{Abstract}
\label{sec:Abstract}
%------------------------------------------------------------------------------
{\color{green} !!to do and this will get moved else where!!}
The motivation for my work is to conduct numerical model to research into the effects of spicules on the transition region, particularly focusing on whether spicules are the drivers of TRQs \citep{Scullion2011}. We are interested in spicules (solar jets) as these jets are ubiquitous ($100,000$  at  any  time \citep{Beckers1968}) on the Sun, thus are a good candidate for coronal heating. These jets could perturb the transition region (a thin region of approximately $100$ km between the chromosphere and corona where the temperature raises from $10^4$ K to $1$-$2$ MK) causing it to oscillate (akin to when a drum is struck), thus possibly being the cause of Transition Region Quakes (TRQ). TRQ are energetic waves in the transition region that evolves in a similar manner to waves on 2D elastic waveguides. When a solar jet rises up from the photosphere to corona it will cause the Transition region to ripple and thus perturb the surrounding magnetic fields. The goal is to analyse the waves and the energy involved in these waves to investigate if they could significantly contribute to coronal heating. \\ \\ The discovery of a link between these TRQ and Rapid Blueshifted Excursions (RBE) identified in CRISP H-alpha data is potentially very far reaching. Work by \cite{Henriques2016} has already found convincing evidence of links between RBEs and coronal transient events, however they did not study whether their coronal features corresponded to coherent waves (i.e. TRQ manifesting as propagating wave fronts). This task is vital for expanding our knowledge of the coupling between the lower solar atmosphere and the transition region. \\ \\ The importance of small-scale jets is well known as they are suggested to contribute to coronal heating and solar wind acceleration. Spicules themselves may be triggered by magnetic reconnection \citep{Pontieu2007PASJ} or waves \citep{Pontieu2004Natur}. Although, it is currently unclear what the decisive physical factors that trigger spicules, reconnection and waves are both promising candidates for their contribution to the energy exchange between the lower cool atmosphere and the corona. We also observe apparent transverse motion within spicules and intermittent Doppler-shifts in coronal loops could be evidence in support of the existence of \Alfven or kink waves in the solar atmosphere.  \\ \\
Observations have demonstrated the ubiquity of vortex motions present in the solar atmosphere. These are usually found at photospheric Magentic Bright points (MBP) groups. Previous MHD simulations have shown such vortices could be responsible for the generation of various types of MHD waves, yet their relationship with other ubiquitous transients-jets (spicules) is unclear.
%------------------------------------------------------------------------------
\section{Overview and Motivation}
\label{sec:overview}
%------------------------------------------------------------------------------
There are many mysterious about the Sun that remain to be unlocked, two relevant to this thesis are: how does the Sun maintain one million degree corona?; what drives and supplies mass to the solar wind? This thesis unfortunately does not provide the answer to these questions, but does increase our understanding of solar spicular jets that are main candidate in accounting for these phenomenon. \np 
%
Our main interest lies in small scale jets such as classical/type I spicules, dynamic fibrils and mottles. We seek to understand what are the dynamics of small jets by deploying simple numerical models. By using simple models we can identify the fundamental nature of these features. Parameter spaces are investigated for key parameters to study how sensitive the dynamics and morphology of spicular jets are to different environments. There have been numerous numerical studies on these jets. However, typically they focus on initiation mechanism and the vertical dynamics (e.g. heights reaches, speeds, trajectory, \textit{etc}). However, these jets are not just vertically dynamic, but are host of complex motion both in the vertical and horizontal directions. In the literature there few studies on the horizontal dynamics of spicules and this may be due to need for very high resolution to study this phenomenon, but with next generation of solar telescopes around the corner (\textit{e.g. DKIST and EST}), more research should come to pass on the cross-sectional evolution of small scale solar jets. This thesis embarks on highlighting the dynamics that they may be observed in these future studies.
%-------------------
%The motivation for my work is to construct a model that accurately simulates waves in the Transition region. We are interested in spicules (solar jets) as these jets are ubiquitous ($100,000$  at  any  time \citep{Beckers1968}) on the Sun, thus are a good candidate for coronal heating. These jets could perturb the transition region (a thin region of approximately $100$ km between the chromosphere and corona where the temperature raises from $10^4$ K to $1$-$2$ MK) causing it to oscillate (akin to when a drum is struck). When a solar jet rises up from the photosphere to corona it will cause the Transition region to ripple and thus perturb the surrounding magnetic fields. The goal is to analyses the waves and the energy involved in these waves to investigate if they could significantly contribute to coronal heating. \np
%
%My research for my PhD is to construct models using numerical magnetohydronamics (MHD) to investigate these dynamics that occur in the solar atmosphere. MHD is combination of the Navier$–$Stokes equation of fluid dynamics and Maxwell equations of electromagnetism. It describes the motion of magnetic field in the presence of electromagnetic fields, which is well suited for describing the dynamics of plasma in the solar atmosphere. One of main problems that have puzzled scientist for over 70 years is the coronal heating problem. This problem refers to observations that the temperature from the core of the Sun to the photosphere decreases as one would expect and the corona which is the outer layer of the Sun's atmosphere (i.e. furthest away from the nuclear reaction in the core) then logically this would be the coldest, however, the corona is 200 times hotter than the photosphere. This contradiction is referred to as the coronal heating problem. One candidate to explain coronal heating is based on magnetically driven waves. We know that these waves carry significant amount of energy, enough to heat and maintain the corona. However, these waves are not easily dissipated into heat energy. The challenge is to find mechanisms in which the energy in these waves are converted into heat energy.  \np
%------------------------------------------------------------------------------
%\section{Structure and Physical Properties of the Sun}
%\label{sec:structure}
%------------------------------------------------------------------------------
\section{The Sun}
\label{sec:Sun}
%------------------------------------------------------------------------------
The Sun is often referred to as a mundane star by the larger astrophysical community as it is a common main sequence star (G-type), particularly when its compared against the zoo of exotic astrophysical objects \textit{e.g.} stars in Binary systems, Neutron stars, Red Giants, White Dwarfs, Cepheids, Wolf-Rayet stars, \textit{etc.}. However, one of the most fascinating aspects of the Sun is its magnetic field. To quote Robert Leighton, ``If the sun had no magnetic field, it would be as uninteresting as most astronomers think it is". The magnetic field makes the Sun a truly a dynamics star as it moulds and shapes its atmospheric environment \textit{e.g.} coronal loops, prominence's, Rosettes, fibrils, Helmet streamers \textit{etc.}, as well store energy which can be released in dramatic fashion \textit{e.g.} coronal mass ejections, spicules, EUV jets \textit{etc.} Due high resolutions observations observers can fantastic images of these dynamics with both ground and space based instruments such as Solar Dynamic Observatory (SDO) \citep{Lemen2012SoPh27517L}, Solar and Heliospheric Observatory (SOHO), Transition Region and Coronal Explorer (TRACE), Hinode \citep{Tsuneta2008SoPh,Suematsu2008SoPh,Ichimoto2008SoPh}, Swedish Solar telescope (SST) \citep{Scharmer2003SPIE}. The combination of the Sun's magnetic field and it being the only star we can fully resolve gives us a marvellous space laboratory at our astronomical doorstep from which the physics of other stars can be understood. An overview of the basic properties of the Sun taken from \cite{priest2014magnetohydrodynamics}:
Age: $4.6 \times 10^9$ years. \\
Mass: $M_{\odot}= 1.99 \times 10^{30}$ kg. \\
Radius: $R_{\odot} = 6.96 \times 10^5$ km. \\
Surface temperature: $5785$ K. \\
Mean density: $1.4 \times 10^3$ kg m$^{-3}$. \\
Mean distance from Earth: $1$ AU = $1.5 \times 10^8$ km. \\
Surface gravity: $g_{0}=274$ m s$^-2$. \\
Equatorial Rotation Period: $26$ days. \\
Composition: $90 \%$ H, $10 \%$ He, $0.1 \%$ other elements.\np
%
The Sun has multiple concentric layers as seen in Fig. \ref{on_model}. The Sun is powered in its core where nuclear reactions consume hydrogen to form helium. From these reactions, energy is released which ultimately leaves the surface as visible light. The next layer which surrounds the core is the radiative zone where energy generated by the nuclear fusion in the core moves outwards as electromagnetic radiation. The next significant region is the convective zone where the method of energy transportation changes from radiative to convective. This occurs because as you move away from the heat source that is the core the temperature drops as described by the second law of thermodynamics. The temperature is eventually sufficiently low for heavier ions (e.g carbon, nitrogen, oxygen, calcium and iron) to retain a collection of their electrons, which increases the opacity. This makes radiation transport less efficient and consequently traps heat. This creates hot rising gas bubbles that when they reach the surface they cool and begin to drop to the bottom of the convection zone, where they are then reheated, thus repeating the process. The next layer of the Sun is the photosphere, which is the observable surface. \np
%
An interesting aspect of stars is that they are all ``ringing" as they are a host of multiple standing waves. Just as Earthquakes have been studied to probe the interior properties of our planet \textit{e.g.} discovering the Earth has a liquid outer core and solid inner core \citep{Lehmann1936}, the principle is the same with stars. When ignoring the effect of rotation and magnetism, these starquakes for the Sun have two main restoring forces, buoyancy (gravity) or pressure \citep{Appourchaux2010AARv18197A}. \np
%
For those pulsation modes which are dominantly restored by buoyancy are referred to as gravity or g-modes. These g-modes are mostly contained to the core but propagate through the radiative zone. Hence, studying these waves would give us crucial information about the structure and dynamics of the deepest parts of the Sun, that we otherwise can't directly measure. These waves are not able to travel into the convective zone because of their restoring force, hence in the convection zone, they are exponentially damped. This makes it very challenging to detect in the Sun and despite claims of their observations \citep{ Garc2007Sci3161591G, Fossat2017AA604A40F, Fossat2018AA612L1F}, they have been met with skepticism \citep{Appourchaux2010AARv18197A, Schunker2018SoPh29395S,Appourchaux2019AA624A106A, Scherrer2019ApJ87742S}. \np
%
For pulsation modes where pressure is the restoring force, these are known as pressure or p-modes. The p-modes on the Sun was first discovered by \cite{Leighton1962ApJ135474L} found undulations on the Sun with periods near 5 minutes, referred to in many studies as the ``5-minute oscillation". The depth at which p-modes penetrate depends on the frequency of the wave, where low frequencies propagate deep into the interior of the Sun and high frequencies are trapped to the surface. There are approximately $10^6$ resonant p-modes on the Sun, with periods ranging from minutes to hours. As they are pressure waves they freely travel through the convection zone. However, when p-modes are propagating through the stellar interior they encounter a temperature gradient and hence a sound speed gradient. The deeper the wave probes the faster the sound speeds is, this means it refracted back to the surface. Then due to a sudden jump in conditions at the surface of the Sun such as density and pressure, this acts as a solid boundary that the wave can't escape and is reflected inwards. However, there are regions (\textit{e.g.} intergranular lanes) where it is thought p-mode wave leakage occurs which propagate upwards into the solar atmosphere and drivers solar features \citep{Suematsu1990LNP367211S, Pontieu2005ApJ624L61D, Heggland2007ApJ6661277H, Pontieu2004Natur}. \np
%%fffffffffffffffff
%https://astroengine.files.wordpress.com/2012/07/thesis06.pdf
%\mfig{0.8}{figures/on.png}{Overview of the layers of the Sun. Source: \url{https://astroengine.files.wordpress.com/2012/07/thesis06.pdf}.}{on_model}
%fffffffffffffffffff
%%fffffffffffffffff
%https://astroengine.files.wordpress.com/2012/07/thesis06.pdf
\mfig{0.8}{figures/image10.png}{Overview of the layers of the Sun. Source: ESA: \url{https://www.esa.int/About_Us/ESAC/Gravity_waves_detected_in_Sun_s_interior_reveal_rapidly_rotating_core}.}{on_model}
%fffffffffffffffffff
%----------------------------------------------
\section{The Solar Atmosphere}
\label{sec:sol_atmos}
%-----------------------------------------------
The Sun's atmosphere is truly a complex and fascinating environment. It broadly can be split into three main regions from the top down, the corona, transition region (TR), and Chromosphere. Two main factors which separate these regions is their measured temperature (see Fig.~\ref{t_profile_sun}) and plasma beta (see Fig.~\ref{beta_profile_sun}).
%T_regoins
\mfig{0.725}{figures/T_regoins}{A plot of the temperature and density from the photosphere to the corona. Plot taken from \cite{Lang_2006ses}.}{t_profile_sun}
%--------------------------------------------------
\subsection{Corona}
\label{ssec:corona}
%--------------------------------------------------
The Corona is the Sun's upper atmosphere that continually extends reaching approximately tens of millions kilometres into space. It follows the Sun's magnetic field lines which eventually feeds into the solar wind. It is possible to observe the corona with the naked eyes during eclipses as seen in fig. \ref{corona_image}. Otherwise to observe the corona a coronagraph is needed, which is a disk-shaped instrument placed on telescopes that can produce an artificial eclipse by blocking the light from the photosphere. The solar corona is the hot tenuous magnetised outer atmosphere of the Sun with an average temperature of $1-2~\rm{MK}$, but can even reach $10~\rm{MK}$. The high temperature of the corona is evidenced by the presence of ions with many electrons removed from the atom. This is evidenced by atoms such as iron which is $9-13$ times ionised in the corona, which indicate temperatures of $1.3~\rm{MK}$ and $2.3~\rm{MK}$, respectively \citep{narayanan2014introduction}. Despite its high temperature, it has a low amount of heat as it is very rarefied, with densities of the order of $10^{-12}~\rm{kg~m^{-3}}$ \citep{priest2014magnetohydrodynamics}. This in turn means that the energy density of the corona is much lower than that of the lower layers of the solar atmosphere, e.g. the photosphere where the temperature is approximately $5000~{K}$. Despite this, the quiet Sun needs to have a constant energy input of $1.1-1.6~\rm{erg~cm^{-2}~s^{-1}}$ \citep{Sakurai2017PJAB9387S} to maintain observed coronal temperatures. \np
%
One startling fact is the extreme coronal temperatures which are higher than lower regions of the atmosphere (approx. $10^4~\rm{K}$) and photosphere (approx $5,000~\rm{K}$). Intuitively, the temperature profile shown in Fig.~\ref{t_profile_sun} doesn't make sense, as the temperature is increasing with distances from the Sun's heat source (the core), breaking the second law of thermodynamics. This is known as the ``coronal heating problem", which was discovered by \cite{Grotrian1939} and \cite{Edl1943}. The coronal heating problem encompasses many open questions such as: Why is the corona hot?; How does it maintain this heat?; Is the corona heated everywhere, or is heat produced in separate, bomb-like events?; Is it heated in multiple different ways? Many theories have been put forward that boil down to either wave based: \textit{e.g.} acoustic shocks and MHD waves \citep{Alfv1947MNRAS107211A, Uchida1974SoPh35451U, Wentzel1974SoPh39129W, Priest1998Natur393545P, Antolin2008IAUS247279A, Escande2019NatSR914274E} or reconnection: \textit{e.g.} nanoflares \citep{Parker1988ApJ330474P, Cargill1993SoPh147263C, Parnell2000ApJ529554P, Klimchuk2001ApJ553440K,  Cargill2004ApJ605911C, Antolin2021NatAs554A}. Due to the highly dynamic nature of the atmosphere in reality both types of heating probably occur \citep{Parnell2012RSPTA3703217P}. \np
%
The heating process can be broken into three main phases: (1) the generation of a carrier of energy (\textit{i.e.} photospheric driving motions); (2) the transport of energy into the solar atmosphere; (3) the dissipation of this energy in various structures of the atmosphere \citep{Wentzel1974SoPh39129W, Robert2004AG45d34E}. For (1-3) there are a plethora of choices, but the main challenge lies in (3). There is agreement in the field that Sun's magnetic field plays a key role \citep{Parnell2012RSPTA3703217P, Arregui2015RSPTA37340261A}, but the exact physical process that transports the energy from the photosphere upwards and dissipates the magnetic energy into heat, currently remains elusive. Another key obstacle is resolving (3), as for wave heating the dampening time of the waves are roughly proportional to the magnetic Reynolds number. The magnetic Reynolds number is approximately  $10^{14}$ under solar conditions as dependent on the typical length scales of plasma flow, which is large on the Sun. Therefore, in the solar environment, the challenge is finding mechanisms that generate small scales length (e.g. resonant absorption, phase mixing, plasma inhomogeneities), otherwise, the wave energy won't be covered into heat quick enough compared to the coronal cooling timescales \citep{Doorsselaere2020SSRv216140V}. For reconnection/nano-flares heating by a series of small-scale reconnection events, lacks definitive observations \citep{Parnell2012RSPTA3703217P}. It is possible that small scale solar jets, such as spicules, play an important role in coronal heating due to their dynamic nature, which is discussed later on.
%fffffffffffffffffffff
\mfig{0.65}{figures/corona_vangorp.png}{Image of the Corona from a total eclipse that occurred on the March of 2006. Source: NASA APOD 26th of July 2009.}{corona_image}
%fffffffffffffffffffff
%--------------------------------------------------
\subsection{Transition region}
\label{ssec:TR}
%--------------------------------------------------
The transition region (TR) is a thin area of approximately $5,000~\rm{km}$ \citep{Athay1981NASSP45085A} that sharply links the ``cool" chromosphere ($10^{5}~\rm{K}$) to the hot corona ($1~\rm{MK}$) at around $2~\rm{Mm}$ above the solar surface \citep{Lang_2006ses}. The TR is dynamically important to the Sun's atmosphere. Due to the sharp change at the TR waves propagating through the chromosphere can suddenly steepen into shocks waves, that propagate both upwards into the corona, as well reflected back into the chromosphere \citep{Pontieu2005ApJ624L61D, Hansteen2007ASPC369193H, Yuan_2016ApJS, Zhenyong2018ApJ85565H}. These shocks wave can lead to jet-like event \citep{Heggland2007ApJ6661277H, kuz2017ApJ, Pontieu2005ApJ624L61D, De_Pontieu2007ApJ}, as well as the oscillation of the TR interface itself, referred to TRQ, that may transfer energy to surrounding magnetic structures \citep{Scullion2011}. A key aspect to note is that all energy that heats the corona and drives the solar wind must make its journey through this sudden change in environment \citep{Mariska1992strbookM}. 
%--------------------------------------------------
\subsection{Chromosphere}
\label{ssec:Chromosphere}
%--------------------------------------------------
The chromosphere roughly spans from above the photosphere up to $2~\rm{Mm}$ \citep{Lang_2006ses}. In ancient times the chromosphere is only faintly seen as a reddy-pink glow around the boundaries of a Solar eclipse. This rosy colour originates from the Balmer series of transitions for hydrogen emission (H$\alpha$). The atmospheric conditions of the chromospheric are sufficient to cause a quantum transition between the $N=3$ and $2$ energy levels of hydrogen. In the modern age, we can study the chromosphere in great detail thanks to excellent ground and space based telescopes (See Fig.~\ref{messy_chromo}). The chromosphere is typically observed in the H$\alpha$, CaII H, and CaII H K lines \citep{Ayres2019sgspbook27A}. Observations in these spectral lines shed light on why the chromosphere can be described as the "Magnetic Complexity Zone" \citep{Ayres2009astro2010S9A}, due to numerous complex dynamics and structures its host as displayed in Fig.~\ref{messy_chromo}. \np  
%
Together the TR and chromosphere make up the so-called interface region, which encapsulates an area of complex plasma and magnetic fields, which transports matter and energy between the photosphere and the corona. To understand the dynamics and topography of the magnetic field lines, it's important to establish whether gas or magnetic pressure is dominant. This is represented by plasma beta ($\beta$ ratio between gas and magnetic pressure) and its value in the atmosphere is shown by the grey shaded region in Fig.~\ref{beta_profile_sun}. In the photosphere where gas pressure is dominant ($\beta>1$) the magnetic field gets dragged by granular flows, being pinched together as they emanate between granular lanes. At the boundary of supergranular lanes, the magnetic field lines form a larger network and the field lines stem out flux tubes (MTF) as shown in Fig.~\ref{fig:chromo_Cart}. These MFT expand further up in the atmosphere due to dropping gas pressure, where magnetic pressure becomes dominant and the field lines become frozen into the plasma \citep{Ayres2009astro2010S9A}. An interesting region in the chromosphere is where $\beta=1$. In this region the sound speed and the \Alfven speed are equal, dividing the solar atmosphere into magnetic and non-magnetic regions \citep{Tsiropoula2012}. This region can lead to the formation of shocks as it allows for the mode-coupling of purely magnetic waves (\Alfven waves) with magnetoacoustic waves \citep{Hollweg1982SoPh7535H, Rosenthal2002ApJ564508R,Bogdan2003ApJ599626B, Cally2008SoPh251251C, Wang2020ApJ891110W} and has a chaotic topology which forms the canopy observed in H$\alpha$ lines (see Fig.~\ref{messy_chromo} and canopy domain in Fig.~\ref{fig:chromo_Cart}). The magnetic field in the Chromosphere is highly twisted and entangled as a consequence of these transitions of the $\beta$. \np
%
In recent years the correctness of the term "coronal heating problem" has been called into question, and even labelled as a ``paradoxical misnomer" by \cite{Aschwanden2007ApJ}. This is because there is no direct evidence of local heating in the coronal, and researchers should shift their focus on solving how heat is transported from the interface region into the upper atmosphere \citep{Aschwanden2007ApJ}. It is not clear how the transport and heating occurs between the interface region and corona. The answer may lie in the multiple jet features such as spicules, mottles, and dynamics fibrils, that are prevalent in the interface region, that energetically advance through the atmosphere \citep{Tsiropoula2012}. In particular, spicules are hypothesised to be a strong candidate for transport mechanisms to heat the atmosphere \citep{Kudoh1999ApJ514493K, Pontieu2007PASJ, Kudoh2008IAUS247195K, Martinez-Sykora2017,Moore2011ApJ731L18M, Pontieu2017ApJ, Samanta2019Sci, Zuo2019AcASn, Bale2019Natur}.    
%As magnetic pressure dominates ($\beta<1$) in the chormopshere means that magnetic field lines that typically are emanating between granular lanes are frozen into the plasma in the chromosphere get dragged by surface flows of the photosphere
%The magnetic structures in the in the chorosphere tends to be small scale magnetic flux tubes (MTF) as shown stemming from the networks that form at boundaries of supergranulation lanes in Fig.~\ref{fig:chromo_Cart}. These supergranular flows moving the magnetic field lines to edges which causes these networks. Another important aspect of the plasma beta is when $\beta=1$, which occurs in the chromosphere. 
%--------------
%--------------
%fffffffffffffffffff
\mfig{0.725}{figures/Selection_067.png}{A model of the plasma $\beta$ (ratio between gas and magnetic pressure) over an active region on the Sun, taken from \citep{Gary2001SoPh20371G}. A high (low) $\beta$ corresponds to gas (magnetic) pressure being the dominant force. The grey shaded region shows plasma beta at different heights.}{beta_profile_sun}
%fffffffffffffffffffff
\mfig{1}{figures/messy_chrom.png}{Two snapshots are taken from H$_\alpha$ observations using SST (Swedish Solar Telescope). Panel to the left (right) is observations of chromosphere above a Sunspot in AR998  (chromospheric filaments). These snapshots were taken from the SST movie gallery and can be found here: \url{https://ttt.astro.su.se/isf/gallery/movies/2008/halpha_10Jun2008_AR998_mu043.mov}, \url{https://ttt.astro.su.se/isf/gallery/movies/2005/halpha_set1_04Oct2005_region_dt5s.mov}}{messy_chromo}
%fffffffffffffffffffff
%fffffffffffffffffffff
\begin{sidewaysfigure}[ht]
    \includegraphics[width=\linewidth]{figures/Selection_066.png}
    \caption{Cartoon representation of the complexity of the lower atmosphere taken from \cite{Wedemeyer2009SSRv144317W}. The solid black lines show the magnetic field lines stemming from the intergranular lanes. A and B highlight the small-scale loop features and D-F shows the condition for wave and magnetic canopy interaction.}
    \label{fig:chromo_Cart}
\end{sidewaysfigure}
%fffffffffffffffffffff
%------------------------------------------------------------------------------
\section{Jets in the Solar atmosphere}
\label{sec:spicule-jets}
%------------------------------------------------------------------------------
The study of jets on the sun is over $150$ years old with Father Angelo Secchi first observed jets in the chromosphere in the 1870s in the Observatory of Roman Collegium and described them as ``burning fields" (see example in Fig.~\ref{de_flammes}). It took nearly 100 years after Secchi discovery of spicules for researchers to realise the sheer variety of jets that exist in the solar atmosphere \citep{Raouafi2016}. This started with modern observations in the 1970s where the discovery of coronal transients in Fe XIV, macrospicules and explosive events \citep{Demastus1973, Bohlin1975ApJ197L133B, Withbroe1976ApJ, Brueckner1980HiA}. This active decade is because of the launching of the space station Skylab and its capability of carrying EUV observations. More types of jets were discovered in the 1990s, due to observations with space based Yohkoh soft x-ray telescopes. \cite{Shibata1992PASJ} and {Strong1992PASJ} discovered solar x-ray jets which are the largest and most energetic of the coronal jets, and even new jets are being discovered recently \citep{Cho2019ApJ884L38C}. \np
%
Due to many excellent observations, it's clear that solar jets are omnipresent at all times on the sun regardless of the phase of the solar cycle and there is a vast variety of types of jets, occurring across a whole range of scales. No more is this true than the chromosphere which is dominated by spicular jets which are thin, small-scaled, short lived jet structures, rapidly evolving with time and height. These spicular jets occur everywhere from quiet sun (QS) \citep{Pontieu2007astroph2081D,Rouppe2007ApJ660L169R,Pereira2012,Pereira2014ApJ}, around active regions (ARs) \citep{Pontieu2007astroph2081D,Pereira2012,Rouppe2013ApJ77656R,Gafeira2017ApJS2296G} and coronal holes (CHs) \citep{Yamauchi2005ApJ629572Y,Moreno2008ApJ673L211M,Pereira2012,Young2015ApJ801124Y}.  These transients events could make significant contributions to the coronal heating and solar wind acceleration which to this day are unanswered questions \citep{Martinez-Sykora2017, Pontieu2017ApJ, Samanta2019Sci, Zuo2019AcASn, Bale2019Natur}. \np
%
%  Since fibrils are the likely on-disk counterpart of spicules (e.g., Tsiropoula et al. 1994) https://ui.adsabs.harvard.edu/abs/1994A&A...290..285T 
%\par !Notes!: Look into the paths the dynamic Fibs take. How do your decelerations times compare to \citep{Hansteen2006ApJ,De_Pontieu2007ApJ}?, mottles \cite{Beckers1968, Tsiropoula1994A&A,Suematsu1995ApJ} and rapid blueshifted excursions (RBE's) (e.g. ). Chromospheric phenomena that are related to spicules are dark mottles (on disk), (dynamic) fibrils, UV and EUV jets, macrospicules and surges.
%
%\mfig{1}{figures/de_flames_combine.png}{Example of early observations of spicules evolution taken by Father Angelo Secchi which they described as flames. Images are takne from    }{de_flammes}
%"Fig. 53 represents a show of flame seen to have tapered at their end; they had the shape shown in fig 54. After ten minutes, putting the eye to the glasses, we saw them transformed into sails (fig 55)"
%
\mfig{0.75}{figures/flammes_alt.png}{Example of early observations of spicules evolution taken by Father Angelo Secchi which they describe as flames that are so small they resemble grass in gardens. Images are taken from \cite{Secchi1877}.}{de_flammes}
% https://books.google.co.uk/books?hl=en&lr=&id=9K0PAwAAQBAJ&oi=fnd&pg=PA98&dq=Secchi,+P.+A.+1877,+Le+Soleil,+Vol.+2+(Paris:+Gauthier-Villars),+Chapter+II&ots=CXaa_h9BhO&sig=mYEulLxrd-uGAY0gS9dEIaTt3OA&redir_esc=y#v=onepage&q&f=false
%These nets do develop, however, and form real flames, the whole of which offers an aspect similar to that of a field in which weeds are burned after the harvest. These flames are sometimes so small and so fine that they resemble the grass which adorns our gardens.
%------------------------------------------------------------------------------
\subsection{Spicules}
\label{subsec:Spicules}
%------------------------------------------------------------------------------
% better here 
Spicules are thin plasma flows that are omnipresent on the surface of the Sun with approximately $2 \times 10^{7}$ Ca II spicules on the Sun at any time \citep{Judge_2010ApJ}. Spicules are best observed in strong chromospheric and transition region (TR) lines such as H$\alpha$, Ca II H \& K, Mg II H \& K, C II and Si IV lines. They were first observed by \cite{Secchi1877} and were given their name by \cite{Roberts1945ApJ} who stated ``I was amazed at the extremely brief lifetimes and the great frequency of occurrence which visual observations of these spicules indicated". The lifetime they reported ranged from approximately 2-11 minutes which fits in the range of current estimates for spicules lifetimes. These short lifetimes along with temporal and spatial (spicules diameters are a few hundred kilometres) resolutions limits made observing individual spicules historically difficult \citep{Sterling_2000SoPh}. Also, the dynamic nature of the spicules' poses a challenge as they are typically mobbed by other spicules shielding one another, they have bi-directional flows, kinking, twisting, and torsion motions. Other effects such as distortions due to the projection effect and spicule inclination add even more complexity as these effects can give an unreal and misleading picture of spicule dimension and behaviour \citep{Porfir2016A}. \np
%
Despite these challenges in observations much effort has been as there is a substantial amount of observational data on spicules available, giving us an understanding of their basic properties (mass density, temperature, velocity and magnetic field) \citep[see reviews:][]{Beckers1968, Beckers1972ARA&A}, possible driving mechanisms \citep[see review:][]{Sterling_2000SoPh}, and the waves and oscillations spicules can host \citep[see review:][]{Zaqarashvili_2009SSRv}. The main interest that lies in spicules is the potential to solve outstanding problems in solar physics, such as providing mass and energy into the solar atmosphere and wind, heating chromosphere/corona, and driving the solar wind \citep{Pontieu2011Sci, Moore2011ApJ731L18M, Henriques2016, Samanta2019Sci} {\color{green} !!cite ahoy!!}. The mass flux taken by the spicule to the corona exceeds that of the solar wind by two orders of magnitude \citep{Thomas1961}, even $1\%$ mass flux of spicules escapes the Sun, this would be sufficient to supply mass to the solar wind \citep{Pneuman1977AA55305P, Pneuman1978SoPh5749P, Tian2014Sci346A315T, Samanta2015ApJ815L16S}. Spicules have been estimated to have an energy flux of around $5\times10^9\rm{erg~cm^{-2}~s^{-1}}$, if even $1\%$ of spicule energy is dumped into the corona then it could easily power the upper atmosphere \citep{Zaqarashvili_2009SSRv}. Another important factor, due to their ubiquity it is imperative that we can accurately model and describe spicules for us to understand the Sun's chromosphere. The combination of improved spatial and temporal resolution achieved with telescopes, such as Hinode, TRACE, IRIS and SST, over the last two decades, and improved computational models have led to a "renaissance" in spicule research \citep{Aschwanden2019ASSL}. A major factor in this renewed interest was the rediscovery that the spicules contain two distinct types, labelled as Type I (TI) and Type II (TII) spicules. \citep{Pontieu2007PASJ} categorisation is based on their lifetimes, speed and trajectory and not to be confused with Type I and Type II spicules originally introduced by \cite{Beckers1968}, who this separates spicules based on significant differences in their line width. \np
%------------------------------------------------------------------------------
\subsubsection{Classical/Type I Spicule}
\label{subsec:TI}
%------------------------------------------------------------------------------
To be clear with our terminology of spicule we will adopt the definition classical spicule as used in \cite{Sterling2010ApJ7221644S}, \cite{Pereira2013ApJ76469P}, and \cite{Sterling2020ApJ893L45S} to refer to limb spicule features observed pre-Hinode i.e. 2006. Classical spicules are observed at the solar limb, typically observed in H$\alpha$ as thin, finger-like features, that rapidly elongate upwards with an average velocity of around $15-40\kms$ \citep{Pontieu2007PASJ} and are grouped together with other spicules, emanating across the boundaries of supergranular cells (see Spicule I stemming from the rightmost network in Fig.~\ref{fig:chromo_Cart}). This group behaviour was first described as “porcupine” and “wheat” field patterns by \cite{Lippincott1957SCoA215L}. Properties of classical spicules are outlined by these early reviews \cite{Beckers1968,Beckers1972ARA&A}, where these thin jet-like structures are reported to reach heights of $6.5-9.5~\rm{Mm}$ during their $5$ minute lifetime and rise with apparent velocities of $25\kms$. These values are in agreement with more recent H$\alpha$ observations of $40$ spicules with lifetimes $7.1\pm2.3~\rm{mins}$, heights of $7.2\pm2~\rm{Mm}$, and velocity of $27\pm18.1\kms$. These spicules are proposed to be initiated by \textit{p}-mode leakage between granular cells \citep{Pontieu2004Natur}. \np
%
The launching of Hinode/SOT (Solar Optical Telescope) in combination with better ground-based telescopes, such as SST, and improved image processing techniques was a game-changer as it allowed for the study of jets in unprecedented detail both spatially and temporally \citep{Aschwanden2010SoPh262235A}. \cite{Pontieu2007PASJ} used observational Ca II H line data collected by Hinode and employed a slit to temporally track the evolution of spicules. They found in general spicules follow a non-ballistic parabolic path or rise up and rapidly fade out in space-time diagrams. They identified that classical spicules can be split into two distinct populations of spicules based on their velocities, lifetimes, and trajectory. The physical properties of TI spicules are akin to that of classic spicules, as they are measured to have are defined as long lived (in comparison to TII) structures with lifetimes of $\sim 180-420~\rm{s}$, with bidirectional up flows of mass (rise from the limb and then fall) following a non-ballistic parabolic path, and have an upward velocity of ranging around $15-40\kms$ \citep{Pontieu2007PASJ}. These values are in agreement with a more recent study by \cite{Pereira2012} who reports average (standard deviations) values of lifetimes as $262~\rm{s}~(80~\rm{s})$, upward velocities of $30\kms~(9\kms)$, and reach heights of $6.02~\rm{Mm}~(1.21~\rm{Mm})$. \np
%
The heights of spicules are typically measured from their footpoints in the photospheric limb up to where the spicule becomes no longer visible. However, measuring the heights of spicules can be tricky as its roots based in the photosphere can be difficult to identify due to their grouping behaviour, it's not obvious whether the footpoint is in front or behind the limb and the top of the spicule doesn't have a clearly defined boundary. Other observational factors add to this difficulty such as exposure time and seeing conditions. Spicules have historically been mostly observed in the H$\alpha$ line which is difficult to trace down due to the opacity of the chromosphere. Despite these difficulties, there is general agreement on reported spicules heights \citep{Tsiropoula2012}. Using data from H$\alpha$ observations \cite{Beckers1972ARA&A,Beckers1968} estimated the average heights between $6.5-9.5~\rm{Mm}$. \cite{Pasachoff2009SoPh26059P} used both H$\alpha$ observations and combined with TRACE data in the $16000~\rm{ \AA}$ channel to obtain heights in the range $4.2-12.2~\rm{Mm}$ with a mean $7.2\pm2~\rm{Mm}$. \cite{Pasachoff2009SoPh26059P} observed heights in TRACE UV found them be $\sim 2.8~\rm{Mm}$ taller than in H$\alpha$. \cite{Pontieu2007PASJ} measure heights in Ca II H using Hinode/SOT which vary from few hundred km to $10~\rm{Mm}$ with most below $5~\rm{Mm}$. \cite{Pereira2012} report maximum heights of $4-8~\rm{Mm}$ using Ca II H data from Hinode/SOT. Numerous studies report on the lifetime of classic spicules lifetime is between $120-720~\rm{s}$ with an average around $300~\rm{s}$ \citep{Roberts1945ApJ,Rush1954AuJPh7230R, Lippincott1957SCoA215L, Alissandrakis1971SoPh2047A, Cook1984AdSpR459C, Georgakilas1999AA341610G}. More recently, \cite{Pasachoff2009SoPh26059P} found lifetimes between $180$ and $720~\rm{s}$ who studied the spicules in H$\alpha$, with a mean value of $426\pm138~\rm{s}$.
% 
In the past spicules widths ($200-1,000~\rm{km}$) been very close to the resolution limits of observations and hence most impacted by observation conditions such as seeing and/or overlapping effects making it challenging to separate into single spicules \citep{Pontieu2007ASPC, Tsiropoula2012}. In addition spicule widths and height are impacted by which line is used for observations. For example, \cite{Pasachoff2009SoPh26059P} used H$\alpha$ data collected with SST and measured widths of $300-1,100~\rm{km}$ with mean diameter of $660~\rm{km}$ and found that the widths are greater by a factor of $1.5$ with ranges of $700-2,500~\rm{km}$ when using measuring with $1600~\rm{\AA}$ TRACE data. However, this discrepancy in widths could be attributed to TRACE resolution being approximately four times lower than SST. In general, for the classical spicule, widths were measured using H$\alpha$ and Ca II H and K lines, spicules are measured having diameters between $400$ and $2,500~\rm{km}$ \cite{Beckers1968, Dunn1960Obs8031D, Beckers1972ARA&A, Lynch1973SoPh3063L}. Modern studies find spicule widths ranging from around $220-420~\rm{km}$ with a mean (standard deviation) of $384~\rm{km}~(81~rm{km})$ \cite{Pereira2012}. \np
%
An important aspect is that spicules are typically observed to be inclined that range from approximately $20^{\circ}$ \cite{Beckers1968} to $37^{\circ}$ \cite{Trujillo2005ApJ619L191T} with an average of $23^{\circ}$. This is in agreement with \cite{Pereira2012} observations who measured inclinations of around $5^{\circ}-25^{\circ}$. Classical Spicules temperatures range from $5,000-15,000$ K and have densities of approximately $3\times10^{-13}$ g cm$^{-3}$ \citep{Sterling_2000SoPh}. \np 
%
As with the classic spicule, TI spicules are hypothesised to driven by \textit{p}-mode leakage between granular cells, that steepen into magnetoacoustic shocks due to decreasing density of the atmosphere as they move upwards through the chromosphere \citep{Pontieu2004Natur, Pontieu2007PASJ, Mart2009ApJ7011569M}. With shock based driver, numerical models have predicted it creates a non-ballistic parabolic trajectory for the jet and predicts a linear correlation between deceleration of the jet and its maximum velocity \citep{Heggland2007ApJ6661277H}. Interestingly this correlation is not only seen in TI spicules \citep{Pereira2012}, but also mottles \citep{Rouppe2007ApJ660L169R}, dynamic fibrils \citep{De_Pontieu2007ApJ} and macrospicules \citep{Loboda2019ApJ871230L}, alluding to all these phenomena being linked.
%------------------------------------------------------------------------------
\subsubsection{Type II Spicule}
\label{subsec:TII}
%------------------------------------------------------------------------------
The categorisation of spicules in TI and TII outlined by \citep{Pontieu2007PASJ}, who used slits to construct space-time diagrams. In these space-time diagrams, they identified spicules that follow a non-ballistic parabolic trajectory (TI) and spicules that follow a linear trajectory as they rise up in Ca II then fade out (TII), as well each type occurring over different time scales. The TII spicule fast with velocities ranging $30-110\kms$, and short lived reaching their apexes of $>5~\rm{Mm}$ in roughly $50-150~\rm{s}$ in comparison to TI spicule. TII spicules form rapidly ($\sim 10$ s), are very thin ($\leq 200~\rm{km}$ wide) and seem to be rapidly heated to at least TR temperatures, sending material through the chromosphere. Due to their high speeds and rapid formation they were thought to be driven by magnetic reconnection \citep{Pontieu2007PASJ}. The existence of TII spicules was called into question by \cite{Zhang2012ApJ} who revisited the same observational data. \cite{Zhang2012ApJ} questioned the use of slits by \cite{Pontieu2007PASJ}, as any spicule that evolves at an angle to the slit will create errors in measurements of spicule heights and lifetimes. \cite{Zhang2012ApJ} demonstrated that using filtergram data may be more appropriate as it better captures the 3D motion of spicules and they found no evidence of TII spicules. However, this was later refuted by \cite{Pereira2012} where again the same data in \cite{Pontieu2007PASJ} was revisited, but they use a semi-automated procedure that individually track spicule evolution and clearly recapture the TI and TII spicules populations. There have been numerous reports claiming observations of TII spicules and the existence of two different spicule types is generally accepted \citep{Rouppe2009ApJ, Rouppe2015ApJ799L3R, Shetye2016AA589A3S, Rutten2019AA632A96R, Yurchyshyn2020ApJ891L21Y, Chintzoglou2021ApJ90682C}. \np   
%
Interestingly in \cite{Pereira2012}, not only did they recover the sub-populations of spicules, but they found that TII spicules were the most populous. This brings into question if TII spicules are the most common why are classical spicules properties most akin to TI spicules? \cite{Pereira2013ApJ76469P} showed if Hinode data is degraded to have a lower spatio-temporal resolution the classical spicule properties are regained. They propose that for low spatio-temporal resolution a rapidly disappearing TII spicule may fade from and be replaced by another spicule, resulting in one longer lived structure, therefore introducing an observational bias for longer lifetimes and introducing errors in velocity measurements. This highlight the importance of spatial-temporal resolution for these small rapidly evolving structures.\np
%
When observing TII spicules in Ca II passband they appear to rise linearly until their maximum length and then dissipate quickly over their whole length. It has been suggested TII spicules are being heated out of the Ca II passband \citep{Pontieu2007PASJ, Pereira2012, Skogsrud2015ApJ806170S, Chintzoglou2018ApJ85773C, Chintzoglou2021ApJ90682C}. In \cite{Pereira2014ApJ} they reported the first spicule observation with IRIS which added more evidence to the existence of TII spicules. They find that the TII spicules show parabolic space-time diagrams in the IRIS and AIA filters. Although this contradicts their earlier classification of TII spicule i.e. linear spicule \citep{Pereira2012}, there are clearly two spicule types as their defining properties (lifetimes, velocities, and heights) fall into two distinct groups. In multi-thermal studies of TII spicules height are reported to have a range of $8-20~\rm{Mm}$ with an average of $12.5~\rm{Mm}$ \citep{Pereira2014ApJ, Skogsrud2015ApJ806170S}. The study of TII spicules is important because they have a larger potential to be able to transfer energy and mass from the photosphere to the interface region and corona. As it has been previously thought that spicules could not contribute to coronal heating due to a lack of a  coronal counterpart \citep{Withbroe1983ApJ}.
%------------------------------------------------------------------------------
\subsection{Mottles}
\label{subsec:mots}
%------------------------------------------------------------------------------
Mottles are rapid changing hair-like short jets observed on disk in the QS regions that are a fundamental chromospheric structure. They are organised in a complex geometric pattern over the solar disk following the boundaries of the chromospheric network and observed on disc in, typically in H$\alpha$ and Ca II lines. Mottles typically cluster together into two groups \citep{Beckers1963ApJ138648B}:
\begin{enumerate}
\item Chains: a small group of mottles that extrude between the boundary of supergranular cells. The mottles are oriented in the same direction and when observed near the limb, they emanate outwards in the same direction forming what \cite{Cragg1963ApJ138303C} called bushes.
\item Rosettes: a larger group of mottles in a circular collection, which is stretching out radially around a common centre, like the stems of a drooping bouquet. They form around the common boundary area of three or more supergranular cells and have a central bright core and are surrounded by both dark and bright mottles \cite{Tsiropoula2012}.    
\end{enumerate}
% this section is going to be thier properties
The combination of (1) and (2) form a chromosphere network. The main properties of mottles are they have heights of roughly $2-10~\rm{Mm}$ and have lifetimes of $2-15$ minutes which is similar spicules \citep{Suematsu1995ApJ}. They seem to be generated several hundred kilometres above the photosphere and undergo real mass motions of $10-30\kms$. Mottles are very dynamic structures that can vary in appearance as they are curved, straight, thin, thick and have transverse motions \citep{De_Pontieu2007ApJ}. Most mottles have an ascending and descending phase that follows a parabolic trajectory. The largest Doppler signal appears at the beginning of the ascending phase and the end of the descending phase. The velocity profiles of mottles are symmetrical around zero. The deceleration seen in mottles is too small to be purely because of solar gravity (i.e. they don't display perfect ballistic flight). \cite{Rouppe2007ApJ660L169R} found that there is a linear correlation between deceleration and maximum velocity of QS dark mottles and this relation has been observed in dynamic fibrils. They propose that these jets are driven with the same mechanism which is by MHD shock waves and this viewpoint is further evidenced by numerical simulations \citep{De_Pontieu2007ApJ, Hansteen2006ApJ}. \np
%Mottles appear both dark and bright, where bright mottles are located at a lower height than dark mottles. They both occur at in the same regions of the solar chromosphere and histrionically it has been undetermined whether these are the same feature just observed at different heights or if they are separate features altogether \citep{Tsiropoula1993A}. It has both been claimed that they are separate features \citep{Alissandrakis1971SoPh2047A} and that they are the same feature where a bright mottle may be the root of the elongated dark mottle \citep{Banos1970SoPh12106B}. Due to higher resolution observation, such as with SST, it is possible to to identify fine structures in these mottles, and it is now thought that these bright mottles are the bright background below dark mottles \citep{Tsiropoula2012}. It's very important to observes these features at high resolution as features (1) and (2) are a complex sea of structures, one would not able resolve individual strands of the mottles. \np
%this section is relating them to limb spicules    
A standing question is whether mottles and spicules and the on-disk/limb counterparts to one another \citep{Tsiropoula1993A}. There is an observational inconsistency between interpreting spicules and mottles as counterparts. One main issue is that the velocity in spicules is much greater than (calculated from proper motions) is much greater than those measured in mottles (derived from spectroscopic observations) \citep{Grossmann1992AA264236G, Christopoulou2001SoPh19961C}. The discrepancy in the velocities is large with up to an order of magnitude difference \citep{Grossmann1973SoPh28319G}. Spicules and mottles both exhibit different velocity distributions, where mottles have symmetrical bi-directional flow, whereas spicules velocity distributions are asymmetric, with the rising phase being most dominant. These differences in the observations can not be attributed to just the angular distributions of spicules and mottles \citep{Grossmann1992AA264236G} and have lead researchers to believe that mottles and spicules are separate features \citep{Christopoulou2001SoPh19961C}. \cite{Christopoulou2001SoPh19961C} propose two possible reasons for this discrepancy, (I) the fact that the values derived from spectroscopic observations represent averages of $1-\ang{;;2}$. Therefore, as the matter moves with greater velocities it is confined to structures below this resolution limit, then the velocity signal is affected by seeing ({\color{green}!!this is from an old paper, may not be valid as we have better resolution!!)}. (II) For high-velocity spicules, they are near vertical, high altitude structures that occur amidst a sea of mottles that have a low velocity that is highly inclined structures. Therefore, due to geometrical effect when observing at the limb spicules will dominate, whereas on disk the mottles would dominate \citep{Grossmann1992AA264236G}. \np
%
When linking spicules and mottles there are two avenues to consider, indirect and direct observational evidence. The indirect evidence that supports this link are: dark mottles (absorbing features in H$\alpha$ and Ca II lines) and spicules have their optimum visibility at the same wavelength in the wings of H$\alpha$ \citep{Tsiropoula1993A}, follow a parabolic trajectory akin to TI spicules \citep{Rouppe2007ApJ660L169R} and they have similar important physical properties such as lifetime and height. The most convincing observational evidence would be to track a spicule or mottle journey as it crosses the solar limb. This is challenging as when tracing the spicules back on disk, one does not know on what side of the limb the feature you are observing is located, and its expected most spicules will have their roots close to the solar limb so that they would not transform into a clearly identified disk structure \citep{Beckers1968}. \cite{Christopoulou2001SoPh19961C} found multiple examples of individual mottles crossing the solar limb and gives more support to the link between mottles and spicules. In light of the observational evidence both indirect and direct, one can reasonably conclude that spicules and mottles are counterparts to one another.
%------------------------------------------------------------------------------
\subsection{Dynamic Fibrils}
\label{subsec:dfibs}
%------------------------------------------------------------------------------
%Bit about fibrils
%Fibrils are the chromospheric equivalence of coronal loops as they appears as curvilinear structure, that follow the local magnetic field and are dynamic structures with flows, oscillations and waves \cite{Aschwanden2019ASSL}. The main difference from coronal loop is the typical temperatures of fibrils are $\gtrsim 5000$ K and these structures are partially ionized gas, whereas coronal loops have temperatures of $\gtrsim 10^6$ K and are a fully ionized plasma. These features are observed in in close proximity of sunspots (penumbral-fibrils) and are closely linked with the low-lying loops that don't display jet-like behavior. \cite{Beckers1968} reported the life times $3-15$ minutes and reach heights $5-9~\rm{Mm}$. \np
% outline dyn fibs 
Dynamic fibrils are thin tube-like, elongated and are highly dynamic and are observed typically in AR \citep{De_Pontieu2007ApJ,Hansteen2006ApJ}. \cite{Foukal1971SoPh1959F,Foukal1971SoPh20298F} proposed that there is a relationship between the thin MFT structures seen near AR and QS environments. They showed that all these features have similar physical parameters, e.g. length, lifetimes, velocities, density and temperatures, for the fibril and spicular phenomena. The dynamic fibrils they observed appeared to be more elongated than spicules and mottles. Due to the strong magnetic field, they are more inclined to the vertical and appear horizontal when observed on the disk. When dynamic fibrils are observed in H$\alpha$ and Ca II wavelengths, these structures show a close resemblance to the mottles in the QS regions, in terms of appearance in clusters as bushes or rosettes. \cite{De_Pontieu2007ApJ} reported on their observations using SST of dynamic fibrils in H$\alpha$, they found average length $1.25~\rm{Mm}$ with a ranges varying from ($0.4-5.2~\rm{Mm}$) with lifetimes of $120-650~\rm{s}$ and average widths of $340~\rm{km}$. These are in agreement with \cite{Morton2012NatCo31315M} and \cite{Gafeira2017ApJS2297G}, who measured widths of $360\pm120~\rm{km}$ and $260~\rm{km}$, respectively. Dynamic fibrils ascend with a typical velocity of $10-30\kms$ and follow a non-ballistic parabolic path as they rise and fall \cite{Beckers1968}. This flight path is expected if it is driven with chromospheric shock waves that occur when convective flows and p-modes leak into the chromosphere \citep{Langangen2008ApJ6731194L,De_Pontieu2007ApJ}. \np
%
Due to more reliable observations of these jet structures, thanks to developments in observational techniques, such as bigger telescopes combined with real-time wavefront corrections by adaptive optics (AO) systems e.g. \citep{Scharmer2003SPIE4853370S,Rimmele2000SPIE4007218R} and postprocessing methods e.g. \citep{van2005SoPh228191V,von1993AA268374V}, it has been possible to study how dynamics fibrils form. Through a combination of observational data from SST and numerical experiment using the Bifrost 3D Radiative, \cite{Hansteen2006ApJ} showed that that jets in AR are a natural consequence of upwardly propagating slow-mode magneto-acoustic shocks, generated by convective flows and p-mode oscillations in the lower photosphere, and leaking upward into the magnetized chromosphere along inclined flux tubes. This viewpoint is also supported in other works e.g. \citep{Heggland2007ApJ6661277H,De_Pontieu2007ApJ,Pontieu2004Natur,Suematsu1990LNP367211S}, and this driving mechanism for dynamics fibrils is akin to mottles and TI spicules. Given that the dynamic fibril physical properties, dynamics, morphology and driver are all similar to mottles and TI spicules, this strongly suggests that these jets are a similar phenomenon, only located in regions of different magnetic activity and that dynamics fibrils are AR counterparts to QS mottles \citep{Rouppe2007ApJ660L169R}.
%------------------------------------------------------------------------------
\subsection{RREs/RBEs}
\label{subsec:rbe}
%------------------------------------------------------------------------------
Rapid red shifted and blue shifted excursions (RRE and RBEs) are short-lived and rapidly moving absorption features in the strong chromospheric H$\alpha$ and Ca II spectral lines. They appear as sudden shifts in the Doppler estimates at the wing-position of the line of the profile. They are located at the edges of rosettes where there are no dominating shocks compared to the network and internetwork. RBEs were first reported by \cite{Langangen2008ApJ} while searching for on disk counterparts of TII spicules. They reported lengths in the range of $0.5-1.5~\rm{Mm}$ with average $1.2~\rm{Mm}$, widths in range of $300-600~\rm{km}$ with average of $500~\rm{km}$. They showed these are short lived features with lifetimes of $45\pm13~\rm{s}$ and have velocities of order $15-20\kms$.\cite{Rouppe2009ApJ} studied RBEs in Ca II and H$\alpha$ data using CRISP instrument at SST. They reported similar properties with lengths of $\sim 3~\rm{Mm}$, lifetimes of $\sim 45~\rm{s}$ and Doppler velocities $\sim 20\kms$. \cite{Sekse2013ApJ76944S,Sekse2013ApJ764164S} investigates RREs in the red wing of the  Ca II and H$\alpha$ lines, finding average length $\sim 3~\rm{Mm}$ with widths of $\sim 250~rm{km}$ and Doppler velocities of $\sim 15-20\kms$, are similar to those found with RBEs.  In a more recent by \cite{Kuridze2015ApJ80226K} they showed that RREs and RBEs have near-identical lifetimes, widths and lengths. They reported that both have lifetimes have a range of $\sim 20-120~\rm{s}$, with a majority living around $40~\rm{s}$, length have a range of $2-9~\rm{Mm}$ with a typical value of $\sim 3~\rm{Mm}$, widths with a range of $200-500~\rm{km}$ with the majority around $250~\rm{km}$ and found velocities estimated in the range of $50-150\kms$, where the upper limit is super-\Alfvenic in the chromosphere. \np
%
Finding an on disk counterpart to TII spicules is important as it gives an alternative path to explore to solve outstanding problems present by these jets. Having a top view of these features means we will have a view that is not affected by line of sight superposition issues that occur at the limb. These RBEs give an insight into the possible mechanism for jet formation. In \cite{Langangen2008ApJ} the length of RBE is shorter than TII spicule, but this could be caused by intrinsic difference in the visibility. The magnitude of the mass motion in RBEs is lower than the apparent motion observed in TII spicules. It is thought the driver for TII spicules is due to magnetic reconnection. If this reconnection is taken place at different heights in the atmosphere and if the amount of energy is similar for each event, then we would expect the density and velocity of the jet to be dependent on height; i.e. (1) lower heights would give high density jet with low velocity and (2) higher heights give low density jet with high velocity. This inverse relationship between density and velocity nicely accounts for the discrepancy between on disk mass motion and the limb apparent motion. For situation (1) this would show enough absorption to be visible on disk, but would be missed on limb observations due to the fibrilar mess at lower heights. For scenario (2) these events on disk would be difficult to observe due to them having low opacity, but on the limb, they rise past the mess of the lower chromosphere and be clearly visible. This means that the difference seen in mass motion could be due to observation biases that are dependent on the locales of the jet. \cite{Langangen2008ApJ} suggests that Ca II RBEs are linked to Hinode Ca II H spicules observed at limb \cite{Pontieu2007PASJ}. This reasoning is based on the matching physical properties such as, lifetimes, location near network, fading, spatial extent and that they only show blueshift which corresponds to upward motion. \cite{Rouppe2009ApJ} adds more fuel to the fire by reporting more similarities of lifetimes, locations, temporal evolution, velocities, acceleration, occurrence rate, between RBEs and TII spicules. In addition, they report that RBEs undergo significant transverse motions ($\sim 8$ km s$^-1$) during their life, similar to what is observed in  TII spicules ($\sim 12$ km s$^{-1}$) \cite{De_Pontieu2007}. Overall these parameters of RBEs that have been reported agree well with what is observed in TII spicules on the limb. \np
%-------------
%In \cite{Langangen2008ApJ} they used the IBIS instrument on DST (Dunn Solar Telescope) to search for the disk counter part of TII spicules and identified RBE's as a potential candidate. Although they found RBEs have have a similar life time, they appeared to have shorter lengths and lower mass motions speeds. This discrepancy in length is reported being due to line-of-sight effects (i.e. caused by the intrinsic difference in visibility between the on disk and limb observations). Further observations of RBE's have been made using the spectral imaging data in Ca II 854.2 nm and H$\alpha$ lines with the CRisp Imaging SpectroPolarimeter at SST which observed Doppler shifts in the range of $20-50$ km s$^{-1}$. With this instrument they observed average life time of $83.9$ s average lifetimes, apparent velocities of order $50$ km s$^{-1}$ and that RBE's experience significant transverse motion of the order $5-10$ km s$^{-1}$ \citep{Rouppe2009ApJ,Sekse2012ApJ}. These parameters agree well with what is observed in TII spicules on the limb. 
%------------------------------------------------------------------------------ 
\subsubsection{Macrospicule}
\label{subsec:Mspic}
%------------------------------------------------------------------------------
Macrospicules as their name suggests are like a spicule, but on a larger scale. Just like with all spicular features their importance of investigation is linked to the potential to help resolve what is the source of the solar wind generation, how the corona is heated and maintained. Macroispicules extend further into the solar atmosphere, they are longer lived than classical spicules, but macrospicules are sparsely seen and are typically visible in TR lines e.g. He II $304~\rm{\AA}$, N IV $765~\rm{\AA}$, and O V $630~\rm{\AA}$, formed at temperatures approximately $8\times10^4$, $1.4\times10^5$ and $2.5\times10^5~\rm{K}$, respectively. Macrospicules have proved a challenge to identify as they are similar to other large jet-like phenomena, due to their properties having a large range. Macrospicules were first defined around 46 years ago when \cite{Bohlin1975ApJ197L133B} described these features at polar coronal holes using SkyLab's EUV slitless spectrograph. \cite{Bohlin1975ApJ197L133B} identified 25 macrospicules with lengths of $5.8-18.125~\rm{Mm}$ with a lifetime of around $8-45$ minutes. They report that macrospicules to be only visible in He II $304~\rm{\AA}$, but not in Ne $VII 465~\rm{\AA}$ (TR line) or Mg IX $368~\rm{\AA}$ (coronal line). \np
%
Due difficulty with identifying macrospicules older studies tended to be limited to case studies or small groups such as \citep[][e.g.]{Moe1975SoPh4065K, Bohlin1975ApJ197L133B, Labonte1979SoPh61283L, Pike1997SoPh175457P, Pike1998SoPh182333P, Parenti2002AA384303P}. With the most recent of these studies, \cite{Parenti2002AA384303P} studied a single macrospicule extend to $60$ Mm reaching a maximum velocity of $\sim 80$ km s$^{-1}$, average falling speed of $26\kms$, they estimated the temperature around $2\times10^5$ K and density of $10^{-10}$ cm$^{-3}$. While these studies still hold value, it doesn't give a strong statistical representation of the behaviour of these features. This has been rectified in more recent studies such as \cite{Bennett2015ApJ808135B}, \cite{Kiss2017ApJ83547K}, and \cite{Loboda2019ApJ871230L} who all leverage at least $2.5-5$ years of data collected by SDO/AIA with a samples of $101$,
$301$ and $330$ of macrospicules, respectively. These studies agreed with \cite{Wang1998ApJ509461W} that macrospicules are seen in different regions of magnetic environments although they are less numerous, with \cite{Loboda2019ApJ871230L} reporting $63.3\%$ found in CH and $36.7\%$ in QS regions. In general they show that macrospicules range in heights from $7- 70~\rm{Mm}$, widths of $3-16~\rm{Mm}$, maximum velocities of $10-150\kms$ and  lifetimes $3-45$ minutes \citep{Bohlin1975ApJ197L133B, Withbroe1976ApJ, Karovska1994ApJ, Parenti2002AA384303P, Bennett2015ApJ808135B, Kiss2017ApJ83547K, Loboda2019ApJ871230L}. \np
%
Their widths and heights put them among the smallest categories of jets observed in the EUV. Their rising velocities and lifetimes make them plausible candidates for being the EUV counterpart of TII spicules. As with much of the jet discussed here, they have been observed to have parabolic paths that are non-ballistic. Curiously, just like with mottles and dynamic fibrils, there is a correlation between the initial velocities and declarations of the jet, which indicates that macrospicules are driven by magnetoacoustic shocks, but unlike mottles and dynamic fibrils ($1-2$ minutes for chromospheric jets), the shock would require a longer period of $10\pm 2$ minutes \cite{Loboda2019ApJ871230L}. This ultimately suggests that their macrospicule have a different set of formation conditions than chromospheric cousins.
%\cite{Bennett2015ApJ808135B} examined data collect over a 2.5 year time span. Of the 101 macrospicules they were located in different regions of the sun with $30.5\%$ in polar CH, $20\%$ at CH boundaries and $49.5\%$ in QS regoins. They give the general properties as maximum length range of $14-60~\rm{Mm}$ with a mean of $28.1~\rm{Mm}$, width with a range of $3.1-16.1~\rm{Mm}$ with a mean of $7.6~\rm{Mm}$, lifetime with a range of $2.7-28$ minutes with a mean of $13.6$ minutes and maximum velocity ranging from $54-105\kms$ with a mean of $109.7\kms$. In this data set they only consider only consider jets 145 Mm, fell back on the solar surface and whose foot points were located exactly on the solar limb. In \cite{Kiss2017ApJ83547K} they look over a larger lifetime of 5.5 years to gather their sample size of 301. They conclude that for macrospicules the lifetime is $16.75\pm4.5$, minutes, they have a width of $6.1\pm4$ Mm, an average velocity of $73.14\pm25.92\kms$ and a length $28.05\pm7.67~\rm{Mm}$. They define Macrospics as ``thin'' jet phenomena shorter than $70$ Mm, with a visible connection to the solar surface and proceeded by brighting at the base. One caveat of both \cite{Kiss2017ApJ83547K,Bennett2015ApJ808135B} is their assumptions that define what a macrospicule is, meaning their database may not be inclusive of all types of a macrospicules, but a valid representation of a subset population. \cite{Loboda2019ApJ871230L} used 5 years worth of SDO/AIA data to obtain a sample of 330 with $63.3\%$ found in CH and $36.7\%$ in QS regions. They find significantly higher proportion of their macrospicules in CH than in \cite{Bennett2015ApJ808135B}. they find typical lengths of $16-32$ Mm, widths of $3-6$ Mm, typical lifetime 15 with a range of $13-18$ minutes, with velocities of $100$ km s$^{-1}$ with a range of initial velocities of $70-140$ km s$^{-1}$. They focus on these jets that undergo parabolic trajectories.
%They argue that there is no H$\alpha$ macrospicule counterpart observed as \cite{Moe1975SoPh4065K}, where unable to find any correlation either by direct visual inspection or by numerical correlation of the measured position of the spicules in the two wavelengths. Due to limitations of the observations at the time, there was much debate as to whether this was actually the case. \cite{Labonte1979SoPh61283L} use the Big Bear solar Observatory to examined  $32$ of the ``limb surges'' with the H$\alpha$ and D$_3$ filters. They found macrospicules exhibited complex structures with ``knots, twists and loops'' with lengths of $8\arcsec-33\arcsec$ and lifetimes of $4-24$ minutes. For the H$\alpha$ macrospicules heights range from $\sim 6-24$ Mm with the majority around $\sim 10-16$ Mm, widths ranging from $0.5-4\times 10^3$ km with a peak around $1-2\times 10^3$ km and lifetimes ranging from $\sim 4-24$ minutes with most around $\sim 8-12$ minutes. This study challenges the criteria (3) of macrospicule set out by \cite{Bohlin1975ApJ197L133B}. \cite{Labonte1979SoPh61283L} identified three categories of macrospicules: (I) similar to filament eruptions, (II) surge-like macrospicules and (III) a flare brightening type. The next step in understanding macrospicule came with the SOHO observations as this allowed astronomers to study solar features with multi-lined/multi-thermal and with higher temporal and spatial resolution, all of which are key ingredients for jet diagnostics. \cite{Pike1997SoPh175457P} used the Coronal Diagnostic Spectrometer on this space base telescope to observe in He I 584 \AA, O V 630 \AA and Mg IX 368 \AA lines and gave a case study of a single macrospicule. They find that macrospicule are visible in He and O V lines, but in the Mg line, this tell us that they reach TR, but not coronal temperatures. These measured the macrospicule length at 31 Mm, width 13.3 Mm and proposed that EUV macrospicules and X-ray jets are a manifestation of the same underlying physical phenomena, which could be the source of the fast solar wind as well. In a follow up study \cite{Pike1998SoPh182333P} used the same instrument to identify the rotational properties of macrospicules. They observed blueshifted and redshifted emission opposites sides to one another of the jet axis above the limb, which suggests rational motion. These rotational veracities increase with height and these macrospicules where categorised as ``solar tornadoes''. \cite{Parenti2002AA384303P} using the same instrument  \np
%%------------------------------------------------------------------------------
%\subsection{May add elsewhere in text}
%%------------------------------------------------------------------------------
%\par Might add to other sections
%\begin{itemize}
%\item There is no tight definition about the size, length and width of chromospheric spicules, but larger structures morph into filaments, prominences, and arch-filament systems \citep{Aschwanden2019ASSL}.
%\item Mottles, dynamics fibrils and limb spicules are all similar in morphology and dynamics, and part of them likely share a similar (possible the same) driving mechanism. 
%\item \citep{Tsiropoula2012}
%\subitem It seems there are 2 main driving processes going on on the sun, jets driven by shock and jets driven by reconnection.
%\item \cite{Sekse2013ApJ764164S}
%\item  spicule properties have proven to be notoriously difficult to measure at the solar limb due to the super-position of many spicules along the line of sight combined with their narrow spatial extent and significant displacement during their short lifetime. The problem of superposition can be over-come by observing their disk counterpart.
%\end{itemize}
%------------------------------------------------------------------------------
\subsection{X-ray and EUV jets}
\label{subsec:euv}
%------------------------------------------------------------------------------
to do
%------------------------------------------------------------------------------
\subsection{Summery of Spicular Jets}
\label{subsec:jet_table}
%------------------------------------------------------------------------------
\afterpage{%
\begin{landscape}% Landscape page
\begin{table*}
\caption{Values based on cited papers throughout Section~\ref{sec:spicule-jets}}
\label{solar_jet_table}
\begin{center}
\begin{tabular}{|l|l|l|l|l|p{1.7cm}|}
\hline
\textbf{Feature} & \multicolumn{1}{c|}{\textbf{Obs.}} & \multicolumn{1}{c|}{\textbf{Life-time (s)}} & \multicolumn{1}{c|}{\textbf{Velocity ($\rm{km \ s^{-1}}$)}} & \multicolumn{1}{c|}{\textbf{Height (Mm)}} & \multicolumn{1}{c|}{\textbf{Driver}} \\ \hline

Spicule TI  & limb &   \multicolumn{1}{c|}{$180-720$} & \multicolumn{1}{c|}{$20-40$} &  \multicolumn{1}{c|}{$4-12$}  & \multicolumn{1}{c|}{shock} \\ \hline

Spicule TII & limb & \multicolumn{1}{c|}{$ 10-150$} & \multicolumn{1}{c|}{$ 50-150$} & \multicolumn{1}{c|}{$5-20$}  & \multicolumn{1}{c|}{reconnection}  \\ \hline

Dynfibs & \multicolumn{1}{c|}{disk} & \multicolumn{1}{c|}{$120-650$} & \multicolumn{1}{c|}{$10-30$} & \multicolumn{1}{c|}{$0.4-5.2$} & \multicolumn{1}{c|}{shock} \\ \hline

Mottles & \multicolumn{1}{c|}{disk} & \multicolumn{1}{c|}{$120-900$} & \multicolumn{1}{c|}{$10-30$} & \multicolumn{1}{c|}{$2-10$} & \multicolumn{1}{c|}{shock} \\ \hline

Macrospicule & limb  & \multicolumn{1}{c|}{$180-2700$} & \multicolumn{1}{c|}{$10-150$}  & \multicolumn{1}{c|}{$7-70$} & \multicolumn{1}{c|}{shock} \\ \hline

RREs/RBEs & \multicolumn{1}{c|}{disk}  & \multicolumn{1}{c|}{$20-120$} & \multicolumn{1}{c|}{$10-50$}  & \multicolumn{1}{c|}{$1-4.5$} & \multicolumn{1}{c|}{reconnection} \\ \hline
\end{tabular}
\end{center}
\end{table*}
\end{landscape}
}
%---------------------------------------------
\section{MHD Equations}
\label{section:MHD_eqs}
%--------------------------------------------
These magnetic fields support waves oscillations which can be studies to in the coronal environment such as coronal loops which allows us to determine properties such as density \citep{Verwichte_2013A_A}, magnetic field strength \citep{Nakariakov_2001} and temperature \citep{De_Moortel_2003SoPh}.
%------------------------------------------------------------------------------
\subsection{may include else where}
%------------------------------------------------------------------------------
%------------------ add else where
{\color{green} !!add else where!! }These jets change in appearance when observed in different spectral lines and in dependence from site they appear leading to whole range of names have to describe the jet-like phenomena, such as dynamic fibrils, mottles and spicules. It possible some of these jets are the same phenomenon, but due to uncertainty have been given different labels. Comparisons of properties and behaviors of jet-like structures which make up the chromosphere show a similarity of characteristics and eludes that some of these structures, possibly all of these structures are related or even are the same phenomena \cite{Porfir2016A}. For example there has been a long standing question where spicules and mottles are the same. Dark mottles are seen on disk, where as the classic spicule is observed at the limb and they have similar properties \cite{Pontieu2007ASPC}. The formation of jets in the solar chromosphere is one of the most important, but also most poorly understood phenomena of the Sun's magnetized outer atmosphere \citep{Hansteen2006ApJ}.

Observations reveal presence of ubiquitous small-scale jet-like structures on the Sun in a variety of magnetic environments e.g. in quiet-Sun (QS) \citep{Pontieu2007astroph2081D,Rouppe2007ApJ660L169R,Pereira2012,Pereira2014ApJ}, around active regions (ARs) \citep{Pontieu2007astroph2081D,Pereira2012,Rouppe2013ApJ77656R,Gafeira2017ApJS2296G} and coronal holes (CHs) \citep{Yamauchi2005ApJ629572Y,Moreno2008ApJ673L211M,Pereira2012,Young2015ApJ801124Y}. Depending upon the observed wavelengths, energetics, physical characteristics (length, lifetime, velocity), regions (QS, AR or CH) or locations (at-limb or on-disk), these structures are further categorized as different features (e.g. spicules, mottles, dynamic fibrils, marcospicules, x-ray jets, EUV jets, coronal jets, \textit{etc.}) that permeate the solar atmosphere. Advancement in instrumentation and observational techniques has further led to the detection of similar features \citep[\textit{e.g.}][]{Gafeira2017, Cho2019ApJ884L38C}. Despite much investigation inton their physical and dynamical characteristics, still many questions remained unanswered regarding their role in mass, energy and momentum across the transition region, to the corona. \np

% it might be worthwhile defining the classical spicule here then branching off to TI TII spicules
They are similar to the traditional limb spicule {\color{green} !!need to define in spciule section!!}

They reach their apex within 1-2 minutes after it initial appearance and then fall back to the low chromosphere with a similar velocity its starting ascent \citep{Tsiropoula2012}. Spicules have always been an important feature to study since their discovery over 150 years ago. They are ubiquitous on the sun and dominate feature in some chromospheric filtergrams e.g. Ca II H \cite{Pereira2016ApJ82465P}. Because of their vast numbers they are a key observational window for the investigations related to the dynamics of the chromosphere and the mass and energy transfer mechanism across the solar the atmosphere. %\np

These features dominate the chromosphere, thus in order to understand the solar atmosphere it is imperative we understand the dynamics of small scale jets.

These jet-like phenomena change in appearance when observed in different spectral lines and in dependence from site they appear leading to whole range of names to describe them, such as spicules, mottles and dynamic fibrils. All these jets share similar properties and behaviors eluding to the possibility they all are related or even are the same phenomena \citep{Porfir2016A}. For example there has been a long standing question whether spicules (observed on limb) and mottles (observed on disk) are the same \citep{Pontieu2007ASPC}.

% Might be better in coclusion
{\color{green} might be better in conclusion} We are at the beginning of the next generation of solar telescopes with $4$ m European Solar Telescope with first light planned in $2026$ \citep{Matthews2016SPIE} and $4$ m Daniel K. Inouye Solar Telescope (DKIST) which released the highest resolution in image of Sun's surface and allowed us to see features as small as $30$ km and will allow the study of spicules in unprecedented detail.

%%%%%%%%%%%%%%%%%%%%%%%%%%%%%%%%%%%%%%%%%%%%%55
%not needed
%\subsubsection{Numerical Techniques}
%Solving differential equations are key part of understanding the physics occurring in nature. Often coupled systems of differential equations can not be solved analytically without making major assumption to simply the equation and thus removing important physics from the original problem. A classic example of a system of equations which isn't solvable analytically is the three body problem where using Newtonian mechanics you consider three masses interacting with one another through gravitational force. The numerical soultions to this problem allowed the revolution in modern space flights, and the launch of the two Voyager probes. By using numerical techniques on the MHD equations we gain an insight into systems which are too complex to obtain analytically.  \\ \\A differential is the gradient of a function over an infinitesimally small range. The numerical approximation takes this range and makes it finite, calculation the differential from an approximation dependent on the selected method. In the sub-sections I will summarise a selection of solvers used in the MPI-AMRVAC code that are applied in this thesis.
%\subsubsection{Finite Difference Method}
%The general expression for a derivative is the following:
%\begin{equation}
%f'(x)=\lim\limits _{h\to0}\frac{f(x+h)-f(x)}{h} , 
%\end{equation}
%the finite difference method (FDM) approximates this equation by taking the step size $h$ as finite. The general form of a finite difference equation is $f(x+b)-f(x+a)$. The two simplest forms are:
%\begin{align}
%\Delta_+ f =  f(x+h)-f(x), \\
%\Delta_- f = f(x-h)-f(x) .
%\end{align}
%These equations can be used to calculate the derivatives by using the following:
%\begin{align}
%\text{Forward Difference: }&f'(x) = \frac{f(x+h)-f(h)}{h} , \\
%\text{Backwards Difference: }&f'(x) = \frac{f(x)-f(x-h)}{h}
%\end{align}
%These equations can be derived from a Taylor expansion of $f(x\pm h)$,
%\begin{align}
%f(x-h) & = f(x)-h f'(x)+\frac{h^{2}f''(x)}{2!}-\frac{h^{3}f'''(x)}{3!}+\frac{h^{4}f^{iv}(x)}{4!}+...\label{eq:TaylorForward}\\
%f(x+h) & = f(x)+h f'(x)+\frac{h^{2}f''(x)}{2!}+\frac{h^{3}f'''(x)}{3!}+\frac{h^{4}f^{iv}(x)}{4!}+...\label{eq:TaylorBackward}
%\end{align}
%The forward (backwards) difference equations is the first-order truncation of the $f(x+h)$ ($f(x+h)$) Taylor series. From this it can be seen that the truncation error of a forward and backward difference approximation is $O(h)$, and that the accuracy is easily improved by increasing the number of terms included. \\ \\
%Another variation of FDM that reduces the error, while maintaining the first-order nature of these solutions is achieved by combining the forward and backward difference into a central difference approximation of the following form: 
%\begin{equation}
%f(x)=\frac{1}{2} \left(f(x + h) + f(x - h)\right).
%\end{equation}
%Preforming a Taylor expansion and taking first order as previous done results in the following: 
%\begin{equation}
%f'(x)=\frac{f(x+h)-f(x-h)}{2h}.\label{eq:First Order CD}
%\end{equation}
%The  error for central difference approximation $O(h^{2})$ as we have combined both the two Taylor series expansions. The accuracy of the solution is important, for it determines how well the computed solution represents the true solution. The most obvious way to increase the accuracy of the solution is to increase the number of terms included from the Taylor expansion of the forward and backward differences. For example the derivation of a fourth-order central difference scheme which can be calculated by starting from Eqs. \eqref{eq:TaylorForward}-\eqref{eq:TaylorBackward} and subtracting the second from the first, to the fourth order gives, in one dimension the following:
%\begin{equation}
%f(x+h)-f(x-h)=2 h f'(x)+\frac{2 h^{3} f'''(x)}{3!}+O(h^{4}).\label{eq:centraldifferencedx}
%\end{equation}
%The next step is to calculate the same subtraction for $2 h$ which can be written as:
%\begin{equation}
%f(x+2 h)-f(x-2 h)=4 h f'(x)+\frac{16 h^{3}f'''(x)}{3!}+O(h^{4}).\label{eq:CentralDifference2dx}
%\end{equation}
%Then subtracting \eqref{eq:CentralDifference2dx} from $8\times$(\eqref{eq:centraldifferencedx}) and rearranging for $f'(x)$ results in:
%\begin{equation}
%f'(x)=\frac{8f(x+h)-8f(x-h)-f(x+2h)+f(x-2h)}{12h}+O(h^{4}),\label{eq:4thOrderCentralDifferenceUniform}
%\end{equation}
%which is the fourth order central difference scheme in one dimension for a uniform spacing of $\pm h$.
%This scheme can be expanded into $n$ dimensions by using the basic property of differentiation $\frac{\partial^{2}u}{\partial x\partial y}=\frac{\partial}{\partial x}\left(\frac{\partial u}{\partial y}\right)=\frac{\partial}{\partial y}\left(\frac{\partial u}{\partial x}\right)$. This scheme provides good accuracy while being computationally efficient.
%\subsubsection{TVLD}
%Need to add other solvers used in AMRVAC.
%\subsection{Atmospheric Model}
%%Need to add images background for models. 
%SP2RC has a method for constructing 3D MHD equilibrium for multiple magnetic flux tubes in a stratified solar atmosphere \citep{Gent_2013p1,Gent_2014p2}. This solar atmospheric model incorporates a wide and realistic range of scales using the combined results of \cite{Vernazza_1981ApJS}  VALIIIC model for the chromosphere and \cite{McWhirter1975} for the lower corona. This model can then used in Sheffield Advance Code \citep{Griffiths2013} which splits the equation in terms of its steady state for the background, then governing equations can be reduced, solving only the evolution for the perturbations. Capturing the true dynamics of these phenomena requires realistic flux tube models and this is the aim of this project. We will study the jet origin, excitation and multiple jet excitation employing a novel 3D MHD code. We will validate results with high-resolution data (e.g. CRISP) to gain insight into the relationship between various solar transients (e.g. jets, MBPs, RBE). By using this combination of numerical simulations and observations of the penetration of jets from the chromosphere, through the transition region into the corona, we will reveal how momentum and energy are transported to the upper atmosphere.
%%%%%%%%%%%%%%%%%%%%%%%%%%%%%%%%%%%%%%%%%%%%%%%%%%%%%%%
% STOP COPYING HERE
%%%%%%%%%%%%%%%%%%%%%%%%%%%%%%%%%%%%%%%%%%%%%%%%%%%%%%%

\bibliographystyle{plainnat}
\bibliography{references}  

\end{document}
