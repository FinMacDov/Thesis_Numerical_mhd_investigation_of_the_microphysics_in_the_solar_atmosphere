\begin{abstract}

This thesis aims to increase our understanding of solar spicular jets. We carry out a series of numerical studies using MPI-AMRVAC to simulate jets in an idealised stratified solar atmosphere with a vertical uniform magnetic field. The jets are initiated using a  momentum pulse. Multiple key parameters are investigated by varying the initial driver (amplitude, period, inclination with respect to magnetic field) and magnetic field conditions to examine the parameter influence over the jet morphology and kinematics. We show the dynamics and morphology of simulated jets are sensitive to the key parameters. \np 
%
The simulated jet captured key observed spicule characteristics including maximum heights, field aligned and non-filed aligned mass motions/trajectories and cross-sectional width deformations. The simulations were able to capture both the horizontal and transverse boundary deformation of the jet, which are under reported in numerical studies. This may be due to needing for very high resolution to study this phenomenon, but with the next generation of solar telescopes around the corner, more research should come to pass on the cross-sectional evolution of small scale solar jets. Furthermore, the simulations highlight the presence of not yet observed internal knots substructures generated by shock waves reflected within the jet structure. We show that these fine structure may not yet be observable, but they could be captured with new telescopes such as Daniel K. Inoyue Solar Telescope (DKIST). If confirmed these substructures could give a new window through which observers could investigate the drivers of spicular jets. This thesis embarks on highlighting the dynamics that may be observable in these future studies.

\end{abstract}