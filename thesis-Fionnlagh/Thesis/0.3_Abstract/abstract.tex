\begin{abstract}
This thesis aims to increase our understanding of solar spicular jets. We carry out a series of numerical studies using MPI-AMRVAC to simulate jets in an idealised stratified solar atmosphere with a vertical uniform magnetic field. The jets are initiated  by driving them with a  momentum pulse. The relevance of multiple key parameters are investigated by varying the initial driver (amplitude, period, inclination with respect to magnetic field) and magnetic field conditions, to examine the parameter influence over the jet morphology and kinematics. We show the dynamics and morphology of simulated jets are sensitive to the key parameters. \np %
The simulated jets captured key observed spicule characteristics including maximum heights, field-aligned and non-field aligned mass motions/trajectories, and cross-sectional width deformations. The simulations mimic both the observed horizontal and transverse boundary deformation of the jet, which are under-reported in numerical studies. This may be due to the need for a very high spatial resolution to study this phenomenon. With the next generation of solar telescopes around the corner, more research will come to pass on the cross-sectional evolution of small-scale solar jets. Furthermore, the series simulations carried out and presented highlight the presence of not yet observed internal knot substructures generated by shock waves reflected within the jet beam. We show that these fine structures may not yet be observable, but they could be identified with new telescopes such as the Daniel K. Inoyue Solar Telescope (DKIST). If confirmed, these sub-structures will enable a new window through which observers could investigate the physics of spicular jets. This thesis embarks on highlighting the dynamics that may be observable with future facilities.
\end{abstract}