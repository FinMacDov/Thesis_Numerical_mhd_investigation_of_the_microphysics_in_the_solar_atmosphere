\chapter{Comparing Synthetic Jets to Observations}
\label{cap:obs}
%-------------------------------------------
\let\thefootnote\relax\footnotetext{
%
This chapter is based on the following refereed journal article:
\begin{itemize}[label={}]
\item Dover, F., Sharma, R.,  Kors\'os, M., Erd\'elyi, R., (2020); Signatures of Cross-sectional Width Modulation in Solar Spicules due to Field-aligned Flows, \apj, Volume 905, Issue 1, \url{https://doi.org/10.3847/1538-4357/abc349}
\end{itemize}
}
%-------------------------------------------
\section{Introduction}
%-------------------------------------------
In this Chapter, the aim is to investigate the cross-sectional width (CSW) behaviour of straight jets and compare them with observations of H$\alpha$ spicules. Observations indicate that spicules generally emanate near the inter-granulation lanes and remain subject to MHD stresses at their footpoint. The associated motions, irrespective of whether turbulent or coherent, become channelled through the spicule structure and are reflected in their observed dynamic behaviour. Broadly, spicule kinematics can be categorized into three major domains: radially transverse, field-aligned, and torsional. These have been extensively examined and interpreted in terms of discrete MHD wave modes \citep[see review:][]{Zaqarashvili_2009SSRv}. \np  
%
Since the discovery of standing kink oscillations that cause transverse displacement in solar structures \citep{Aschwanden1999ApJ520880A, Nakariakov1999Sci285862N, Aschwanden2002SoPh20699A}, MHD waves in solar structures have gained much attention \citep{Cally1985AuJPh38825C, Kudoh1999ApJ514493K, Fujimura2009ApJ7021443F, Zaqarashvili_2009SSRv, Kuridze2012, Jess2012ApJ744L5J, Mooroogen2017AA607A46M, Allcock2019FrASS648A}. This is due to their potential to provide estimates that we can not directly measure in solar structures (e.g. magnetic field strength), and because they are thought to carry sufficient energy to heat the corona \citep{Alfv1947MNRAS107211A, Gordon1983ApJ266373G, Poedts2002ESASP505273P, Srivastava2017NatSR743147S}. The first report of kink waves in spicules was by \cite{Kukhianidze2006AA}, and numerous studies have since followed, with the goal of identification and study of transverse (kink and torsional \Alfven) wave modes \citep{De_Pontieu2007, Ebadi2014ApSS35331E, Pascoe2016AA585L6P, Sharma2017, Tiwari2019ApJ876106T}. However, reports of CSW variations have remained elusive, due to current resolution limits for observations of jet-like chromospheric features. The presence of CSW variations in thin MFT structures, as a possible consequence of $m$ = 0 MHD sausage and/or $m$ = 2 fluting modes, was postulated in earlier theoretical studies \citep{Ziegler1997a, Ziegler1997, Ruderman2010}. Observations that report on CSW variations in spicular structures are limited to on-disk fibrils, where concurrent transverse and CSW were reported as coupled kink and sausage modes by \cite{Jess2012ApJ744L5J} and \cite{Morton2012}. Similar observations for CSW, and intensity oscillations in on-disk slender chromospheric fibrils were reported by \citet{Gafeira2017ApJS2297G} and interpreted as MHD sausage wave modes. Moreover, an ensemble of coupled transverse, CSW and axisymmetric torsional motions were reported in off-limb spicules by \citet{Sharma2018ApJ85361S}, which were interpreted as nonlinear kink modes. \np
% {\bf RE: be a bit careful as in a plasma with neutrals present, this may not be fully the case; worth mentioning it, given the popularity of the ubject these days}
Spicules are essentially jet-like features where mass flows along their magnetic field with velocities in the range of $25-100\kms$  \citep{Beckers1972ARA&A, Sterling_2000SoPh, Pereira2012}. These velocities may even have a dominant effect on the waves that are present in spicular jets. Plasma flows will interact with the oscillatory modes of spicules, e.g. periodic motions along or anti-parallel to the bulk motion will be affected differently. Wave motions in steady waveguides have been investigated in the past in a few theoretical studies for different slab and cylindrical geometries \citep{Narayanan1991,nakariakov1995, terrahomem2003, soler2008}. It was concluded that mass flows can generate a shift in frequencies for confined MHD waves and can influence the wave propagation, even causing resonant flow instabilities, depending upon the direction and strength of mass flows. Though possible effects of mass flow on wave periodicities/propagation are known theoretically, they are yet to be observed in highly localised dynamic waveguides such as solar spicules. \np
%
At multiple points through this thesis, there has been a comparison between the simulations and observations. Interesting features that have arisen in the synthetic jets, particularly the CSW variations and the knots. In this Chapter, we aim to focus on these aspects of the observations by comparing the CSW variations in off-limb spicules with the observed dynamical behaviour with synthetic jets, and highlight observed solar features where knots may be present. \np
%-------------------------------------------
\section{Cross-sectional Width Variation in Synthetic jets}
\label{sec:CSW_syn_jet}
%-------------------------------------------
The synthetic jets studied in the thesis capture some of the fundamental dynamics of a chromospheric jet in a stratified atmosphere. Though the momentum-driven simulated jet lacks a few observed complex physical mechanisms, e.g. radiative losses, ambipolar diffusion, and ion-neutral effects, it still models crucial kinematic behaviour, such as longitudinal/field-aligned plasma motions with CSW variations (Fig.~\ref{jet_P300_B50_A60}) in both rising and falling phases of the jet evolution. \np
%
Using the jet tracking code as outlined in Chapter~\ref{chap:sj} the CSW have been measured for a jet with parameters of $\theta =0\degs$, $P=300~\rm{s}$, $B=50~\rm{G}$, and $A=60\kms$ (see Fig.~\ref{jet_P300_B50_A60}). The parameters of these jets are similar to those of the standard jet, and hence display similar dynamics, morphology, lifetimes, knots, and CSW variations. The jets' widths are temporally tracked at multiple heights, where the blue marks measure the widths and yellow triangle locked onto the apex of the jet. The results of the widths for each megameter the jet reaches in the computational domain are displayed in Fig.~\ref{CSW_m_heights}, which also demonstrates that there are oscillatory patterns at each height, except from at $6~\rm{Mm}$. Oscillatory motion is not seen at $6~\rm{Mm}$ due to lack of time which the synthetic jet spends at this height. \np
%
We decided to focus on slit height at $2~\rm{Mm}$ as it can be compared against observations, and because choosing a lower height gives more temporal data to analyse than higher slits. The life of a jet can be separated into two phases; the rising phase, where the jet is being driven and is propagating upward, and the fall phase where the jet is no longer being driven and apex height is decreasing with time. These regions are separated by using the parabolic trajectory and calculating the point at which the gradient switched from positive to negative, i.e. the apex of the trajectory, which is marked temporally by the solid black vertical line in Fig.~\ref{CSW_2m_height}. One striking feature is seen in Fig.~\ref{CSW_2m_height}; to the best of our knowledge we are the first to discover that there are distinct differences in the behaviour of the boundary deformation for the rising (green line) and falling (cyan line) phases of the jet. During the rising phase, the CSW is larger and with only a few undulations,  whereas in the fall phase the CSW are shorter and faster undulations. The blue and orange lines show the quadratic fit used to detrend the rise and fall phases respectively in Section~\ref{sec:CSW_comp}.
%-----------------------
\begin{figure*}
\includegraphics[width=1.0\textwidth]{figures/jet_P300_B50_A60_with_markers.eps}
\caption{Panels showing the temporal evolution of the simulated spicule density structure at four time-steps with apex marked by a yellow triangle. From $121.08-243.88 ~\rm{s}$ ($366.68-489.48~\rm{s}$)  the rising (falling) phase. Locations of tracers edges are also shown as blue dots, which are used to estimate the variations in CSW during rising and falling phases of jet structure.}
\label{jet_P300_B50_A60} 
\end{figure*}

\begin{figure*}
\includegraphics[width=1.0\textwidth]{figures/P300B50A60.png}
\caption{CSW variations over multiple heights for synthetic jet with $P=300~\rm{s}$, $B=50~\rm{G}$ and $A=60\kms$.} 
\label{CSW_m_heights} 
\end{figure*}

\begin{figure*}
\includegraphics[width=1.0\textwidth]{figures/raw_data_P300B50A60.png}
\caption{CSW variation for slit at $2~\rm{Mm}$. The solid vertical black line separates the rise (green line) and fall (cyan line) phases, located at the time where the jet reaches its apex. The blue and orange lines are a quadratic fit used to detrend the data.}  
\label{CSW_2m_height} 
\end{figure*}
%-------------------------------------------
\section{Observations of Cross-sectional Width Variation in Spicules}
\label{sec:CSW_spicules}
%-------------------------------------------
In Section \ref{sec:c2discussion} we have discussed that SST currently offers one of the highest spatial resolution observations of the chromosphere, making it an ideal telescope to use for tracking the widths of such small features as spicules. SST is a ground-based telescope located in La Palma, which has one of the best seeing conditions in the world, placed $2400~\rm{m}$ above sea level. It has a nearly $1~\rm{m}$ sized aperture and uses active optics to minimise data degradation due to Earth's atmospheric conditions. The data captured in this section uses the CRisp Imaging SpectroPolarimeter (CRISP), which covers a wavelength range of $510$ to $860~\rm{nm}$, with a field of view of approximately $43.5\times43.5~\rm{Mm}$, and a pixel size of roughly $43~\rm{km}$. The spectral line selected for observation is H$\alpha$ $(656.3~\rm{nm})$, giving an angular resolution of approximately $95~\rm{km}$. H$\alpha$ can only form in the light spectrum from regions of the Sun that are cool enough for hydrogen to exist in its atomic form, which are the photosphere and chromosphere. It is part of the Balmer series and forms absorption or emission spectra for hydrogen when the electron jumps from or falls to $N=2$, where $N$ is the electron orbital number. The H$\alpha$ line is commonly used for observing chromospheric structures as it reveals many complex structure in the messy chromosphere \citep{Parmenter1966PASP78250P, von1985AA146192V, Nishikawa1988PASJ40613N, Judge2006ASPC354259J, Leenaarts2007AA473625L, Rutten2008ASPC39754R, Jess2012ApJ744L5J, Pereira2016ApJ82465P, Rutten2017AA598A89R}. \np
% {\bf RE: if u want to be nice to ur colleagues in h23, u may add here some of their papers using h-alpha observations}
The spicules' structures presented in Figs.~\ref{fig:spa},~\ref{fig:spb} and~\ref{fig:spc} were identified in observations carried out on AR NOAA AR11504 at 07:15-07:48 UT, June 21, 2012. The identification of the spicules and data processing of these observations was carried out in collaboration with Dr. Sharma; we will outline the steps taken. The AR was scanned using 31 equally spaced line positions with $86~\rm{m\angstrom}$ steps from $-1.376$ to $+1.29~\rm{\AA}$, relative to the line centre, along with the additional $4$ positions in the far blue wing from $-1.376$ to $+2.064~\rm{\AA}$. Following data acquisition, post-processing of data was carried out using the Multi-Object Multi-Frame Blind Deconvolution \citep[MOMFBD;][]{van2005SoPh228191V} image restoration algorithm. Also, standard procedures available in the image pipeline for CRISP data were implemented \citep{2015}, including differential stretching and removal of dark and flat fielding. \np
%
The processed data spans around $30~\rm{mins}$ and a cadence of $7.7~\rm{s}$. Following the Nyquist criterion, the temporal-resolution of the dataset allowed the detection of MHD wave modes with periodicities over $15.4~\rm{s}$. Only three spicules (SP:A, SP:B, and SP:C) were selected as they needed to be a high-intensity structure with the least possible superimposition by any other surrounding features during their visible lifetime in an observed passband. The identified spicules have an average lifetime of $118\pm 17.88~\rm{s}$ measured from the H$\alpha$ line-centre. It is possible their true lifetime is longer, as spicules tend to change temperature during their evolution and therefore evolve through multiple passbands, which measurements by \cite{Pereira2014ApJ} have shown to have lifetimes in the range of $500-800~\rm{s}$. The observed features have an average length of about $3\pm0.36~\rm{Mm}$ with apices reaching up to an average height of $4.83~\rm{Mm}$. \np
%
The CSW of observed spicule features were estimated using the method adopted from \citet{Sharma2018}. A single Gaussian function with linear background is used to fit the intensity profile across the observed spicules and the FWHM is taken as a measure of CSW of the feature. For the selected spicules, the average FWHM of the spicules measured at $2~\rm{Mm}$ is $144.47~\rm{km}$, which is comparable with other observations for off-limb spicules \citep{Sharma2018}. The spicules are resolved as there is a $3.36$ ($1.52$) pixel (angular-resolution) cover, bearing in mind that the intensity values belonging to the spicule will occupy a space larger than the FWHM. This process is outlined in Fig.~\ref{fig:spa}, where panels a1-a4 shows snapshots of the identified spicule where the cyan line is the vertical slit to track field-aligned plasma motions and horizontal yellow slit is placed at $2~\rm{Mm}$ to track the CSW. Panel (b) displays time-distance data captured with across the vertical slits, which are used to estimate the speed of the spicules and allow for the separation of the rise and fall phases. Panel (c) shows the process of calculating the FWHM of the spicule, where the dark (light) shaded region marks the unperturbed (perturbed) width $W_u$ ($W$) during the spicule lifetime measure by the yellow slit. The unperturbed width is fixed individually for each spicule as the average FWHM calculated over its lifetime, and the perturbed width is the FWHM calculated at each time step of the observations. The temporal variations in CSW are calculated with $\delta W = W(t) - W_{u}$ and were used to estimate the periodicities. \np
%
SP:A in Fig.~\ref{fig:spa} is an off-limb spicule that was observed at $-1.032~\angstrom$ from the H$\alpha$ line core with a total lifetime of around $161.7~\rm{s}$. The feature had a physical length of $3.8~\rm{Mm}$ at an inclination of $21.4\degs$ over the observed limb, with apex height reaching a maximum of $5.3~\rm{Mm}$. The longitudinal motions in the Lagrangian frame were analysed to estimate the velocity and duration of rising and falling phases, using a vertical slit (Fig.~\ref{fig:spa} a1–a4) along the axis of spicule structure. The time-distance analysis of the vertical slit suggests a parabolic profile (Fig. \ref{fig:spa}(b)) of mass motions with an average ascending velocity of $\sim$46.9 km/s. The plasma attains the maximum height in about 81 sec and then falls back to the surface with an average velocity of 40.2 km/s. For the observed lifetime, the average/unperturbed CSW estimate (W$_{u}$) for the spicule structure was around $176~\rm{km}$.
%----------------------------------
\begin{figure*}
\includegraphics[width=1.0\textwidth]{figures/fig_sp_a.png}
\caption{Panels (a1) – (a4) show the temporal evolution of candidate spicule feature (SP:A) in the H$\alpha$ passband at four instances, with positions of vertical (cyan) and horizontal (yellow) slits used for the estimation of field-aligned mass flows and CSW respectively. Panel (b) shows the time-distance plot from the vertical slit on the spicule, highlighting the rise- and fall-phases of field-aligned mass flow. The maximum height attained by the visible plasma is marked with the ‘+’ symbol, along with estimated velocities (46.9 km/s, 40.25 km/s). Bottom panel (c) shows an example of Gaussian fit for intensity magnitudes for horizontal slit location (marked as a yellow line on (a1) – (a4)), with error bars, denoting the standard deviation for intensity values. The vertical black line marks the position of the amplitude of Gaussian fit, while shaded-regions mark average/unperturbed width (W$_{u}$) during spicule lifetime and perturbed/instantaneous width (W). This figure is taken from \cite{Dover2020ApJ90572D} and was produced by Dr. Sharma.}
\label{fig:spa} 
\end{figure*}
%-------------------------------
\section{Comparison of Cross-sectional Width Variation in Synthetic Jets and Spicules}
\label{sec:CSW_comp}
%-------------------------------
The CSW estimates from both observation and synthetic jets indicate an oscillatory pattern during the feature’s lifetime. These were further analysed using wavelet transform to understand the nature of periodicities and associated wave physics. The results obtained from this analysis is viewed in the framework of ideal MHD waves, as this is likely how this phenomenon would be interpreted in observations. To identify any significant periodicities, wavelet power spectra (WPS) and the associated global power spectrum (GPS) are constructed using open-source software developed by \cite{Torrence1998}. The default Morlet wavelet profile was employed with a two-$\sigma$ confidence level on CSW estimates from observations (top panels) and synthetic jet (bottom panels) as shown in Fig.~\ref{WL_1}. The wavelet analysis is applied to both the rising and falling phases, as well the full signal. For the synthetic jets, the rise and fall phases are independently detrended with a quadratic polynomial (see Fig~\ref{CSW_2m_height}). A low order polynomial is chosen to avoid over fitting. In addition, the detrended data are smoothed to reduce noise with a moving average with a window size of 15 and 10 for the rising and falling phases, respectively. \np
%
The peak frequencies were identified in the wavelet analysis as spectral magnitudes in WPS with significance levels over unity and higher, highlighted as black contours in Fig.~\ref{WL_1}. The peaks also had GPS over two-$\sigma$ (95\%) confidence levels (orange dashed-line in Fig.~\ref{WL_1}). The wavelet spectra for observed width variations during spicule lifetime suggest a strong spectral density concentration around $17~\rm{s}$ for $5$ cycles. Interestingly, for the simulated jet structure, the WPS indicates the presence of harmonics concentrated around $65~\rm{s}$ and $32~\rm{s}$ for $8$ and $5$ cycles, respectively. The oscillatory behaviour between the rising and falling phase is displayed in the last two columns, respectively It becomes clear with the wavelet analysis that there are remarkable differences in the periodicities of the falling and rising phases for each example (see periodicities in Table~\ref{table:1} and Figs.~\ref{WL_1} to~\ref{fig:spc}). This modulation of frequency could be tied to the change in direction of flow and its relation to gravity from rising (working against gravity) and falling phases (working with gravity). Wavelet transforms for observational width variations indicate approximately $27~\rm{s}$ periodicity for 3 cycles, while a powerful $70~\rm{s}$ periodicity is present in both GPS and WPS of the simulation data. However, there is a noticeable shift towards lower/higher periodicities/frequencies for two out of three of our identified spicule cases (Table \ref{table:1}) during the fall-phase of the plasma flows in the jet structure. The wavelet spectrum for observed width variations shows dominant periodicity at $16~\rm{s}$ for $5$ cycles, while for simulated data there are two strong periods concentrated around $30~\rm{s}$ and $68~\rm{s}$, each persistent for $5$ and $2.5$ cycles. These results are consistent with previous reports for CSW oscillations in on-disk fibrils \citep{Gafeira2017ApJS2297G}. \np
%
An important aspect of our investigation is the confirmation of the first overtone in cross-sectional width periodicities associated with dynamic waveguides in the solar chromosphere.  Earlier studies were able to only identify higher harmonics in static waveguides (e.g., in a coronal loop), with the presence of fundamental and first overtone \citep{verwichte2004, guo2015}. Recently, the presence of a second overtone was also reported for coronal loop observations by \citet{duckenfield2019}, associated with transverse kink oscillations. It must be noted that wave harmonics can provide vital clues regarding plasma and magnetic field characteristics of the waveguide via solar magneto-seismology applications \citep{andries2005,andries2009}. The presence of overtones in jets provides a key tool for chromospheric magneto-seismology. This phenomenon is not limited to this one observation, as shown in Fig.~\ref{fig:spb} and Fig.~\ref{fig:spc}. For a summary of the results and physical properties of all the selected spicules see Table \ref{table:1}. \np 
\begin{figure*}
\includegraphics[width=1.0\textwidth]{figures/fig_3.png}
\caption{Panels showing the results of spectral analysis of cross-sectional width estimates of observed (top) and simulated (bottom) jet structures. Each panel depicts temporal evolution of overall widths (1.1) and subsequent rise- (2.1) and fall-phases (3.1), along with Wavelet Power Spectra (WPS: 1.2, 2.2, 3.2) and Global Power Spectra (GPS: 1.3, 2.3, 3.3) during each phase of the evolution of the jet. Vertical red line in plots (Obs: 1.1 and Sim 1.1) marks the time when the field-aligned plasma attained the apex height. Further, plots (Obs: 1.1, Sim: 3.2) provide clear indication of a second harmonic of the cross-sectional width deformations in the dynamic spicular waveguide. This figure is taken from \cite{Dover2020ApJ90572D} and produced by Dr. Kors\'os and Dr. Sharma.}
\label{WL_1} 
\end{figure*}
%
\begin{figure*}[h]
\begin{centering}
\includegraphics[width=1.0\textwidth]{figures/fig_4.png}
\caption{Details of spicule SP:B are shown, where panels (a-c) show the same analysis carried out as for Fig.~\ref{fig:spa}, and the bottom 2 panels show the corresponding wavelet analysis as done in Fig.~\ref{WL_1}. This figure is taken from \cite{Dover2020ApJ90572D} and was created by Dr. Sharma.}
\label{fig:spb} 
\end{centering}
\end{figure*}
%
\begin{figure*}
\begin{centering}
\includegraphics[width=1.0\textwidth]{figures/fig_5.png}
\caption{Same as Fig.~\ref{fig:spb}, but for spicule SP:C, where panels (a-c) show the same analysis carried out as for Fig.~\ref{fig:spa} and the bottom 2 panels show the corresponding wavelet analysis, as done in Fig.~\ref{WL_1}. This figure is taken from \cite{Dover2020ApJ90572D} and made by Dr. Sharma.}
\label{fig:spc} 
\end{centering}
\end{figure*}

\afterpage{%
\begin{landscape}% Landscape page
\begin{table*}
\caption{Summary of observed physical characteristics of candidate spicule structures along with estimated periodicities during rising, falling and overall phases. Physical and spectral parameters of the synthetic jet structure are also provided for a comparison between observed and simulated case(s).}  
\label{table:1}      
%{\centering} 
\begin{center}
\begin{tabular}{cccccccc}
\hline
Spicule       & Lifetime & Length & Apex-height & Unperturbed Width & \multicolumn{3}{c}{Periodicity (s)} \\ \cline{6-8} 
              & (sec)    & (Mm)   & (Mm)        & (km)              & Rising phase  & Falling phase  & Overall   \\ \hline
SP:A          & 161.7    & 3.8    & 5.3         & 176.0             & 27.0        & 16.0        & 17.0      \\
SP:B          & 92.4     & 2.2    & 4.8         & 134.6             & 22.5        & 27.4        & 24.7      \\
SP:C          & 100.1    & 2.8    & 4.4         & 122.8             & 41.2        & 31.3        & 48.8      \\ \hline
Simulated jet & 471.0      & 8.0    & 8.0         & 187.5             & 70.0        & 30.0 \& 68.0   & 32.0 \& 65.0 \\ \hline
\end{tabular}
\end{center}
\end{table*}
\end{landscape}
}
%-------------------------------------------
\section{Blobs in Solar Jet Observations}
\label{sec:EUV_jets}
%-------------------------------------------
Observing knot structures is challenging due to the current resolution limits, as stated throughout the thesis. However, larger-scale jets occur on the Sun, and if the flow speed is sufficient then it is possible that patterns of the knots may be observed. In a series of studies on coronal jets blobs structures were observed \citep{Zhang2014AA567A11Z, Zhang2016SoPh291859Z, Chen2015ApJ81571C, Chen2017ApJ84054C}.  \np
%
\cite{Zhang2016SoPh291859Z} show multiple observations of blob structures in solar jets. The observations were captured by Extreme-Ultraviolet Imager (EUVI) in the Sun-Earth Connection Coronal and Heliospheric Investigation (SECCHI) \citep{Howard2008SSRv13667H} instruments of the Solar TErrestrial RElations Observatory (STEREO) \citep{Kaiser2005AdSpR361483K} and SDO, using the AIA instrument \citep{Lemen2012SoPh27517L}. The images in Fig.~\ref{fig_brigt_blobs_2} were observed with STEREO-B in $171~\AA$ filter with a spatial resolution of approximately $2320~\rm{km}$ ($\ang{;;3.2}$) and a cadence of $75~\rm{s}$. Two jets (J1 and J2) originate around a coronal bright point (BP1). These are thought to arise due to magnetic reconnection \citep{Priest1994ApJ427459P, Mandrini1996SoPh168115M, Longcope1998ApJ507433L, Santos2007ASTRA329S}, and the jets flow along closed magnetic loops. J2 is particularly interesting as multiple blobs form along with the jet, which is highlighted by the white arrows in panels (f-g). If at the base of the jet there are mixed magnetic polarities, a current sheet forms between the interface of these regions, and perturbation to this system can lead to magnetic reconnection. If an instability forms at this current sheet during reconnection then this can lead to plasmoids (regions of magnetic islands)\citep{Drake2006Natur443553D}. These plasmoids are proposed by \cite{Zhang2016SoPh291859Z} to explain the blobs in their jet observations. Interestingly, the formation of the blobs by magnetic reconnection has been proposed for chromospheric anemone jets \citep{Singh2012ApJ75933S}, alluding that blobs may occur over a range of different scales of jets \citep{Zhang2016SoPh291859Z}.\np 
%If the current sheet is extremely thin a perturbation to this system can cause it to go unstable and cause magnetic reconnection. This creates magnetic flux which goes onto form plasmoids which creates magnetic island \citep{Drake2006Natur443553D}. These plasmoids are proposed by by \cite{Zhang2016SoPh291859Z} to formed by a tearing-mode instability. The formation of the blobs by magnetic reconnection has been proposed for chromospheric anemone jets also \citep{Singh2012ApJ75933S}.  \np 
%ffffffffffffffffffff
\begin{figure*}[h]
\begin{centering}
\includegraphics[width=1.0\textwidth]{figures/Selection_065.png}
\caption{Observations of solar jets on  2014 September 10 taken from \cite{Zhang2016SoPh291859Z}. Panels (a-h) are running difference images in $171~\AA$ show the occurrence of two jets (J1, J2). The white arrows highlight where the bright blob-like structures sit along the jet beam.}
\label{fig_brigt_blobs_2} 
\end{centering}
\end{figure*}
%ffffffffffffffffffff
%
\cite{Chen2015ApJ81571C, Chen2017ApJ84054C} reported multiple jets observed at the western edge of AR 11513 on the 2nd and 3rd of July 2012. In these jets, blobs are seen in the jet beam (see Fig.~\ref{fig_brigt_blobs_alt}). The space-borne observations were captured with SDO using the Atmospheric Imaging Assembly (AIA) instrument. AIA has a spatial resolution of $1088~\rm{km}$ ($\ang{;;1.5}$) and a temporal resolution of $12~\rm{s}$ for the EUV lines shown in Fig.~\ref{fig_brigt_blobs_alt}. The example jet clearly shows blobs, as highlighted by slices 1 and 2 in panel (b). The conditions under which these form are similar to that outlined by \cite{Zhang2014AA567A11Z} and \cite{Zhang2016SoPh291859Z}, as the brightening at the base of the jet has been shown to contain magnetic fields of opposite polarities \citep{Chen2017ApJ84054C}. \np
%
The suggested mechanism of blob formation in \cite{Zhang2016SoPh291859Z} is plausible, but so too is the mechanism put forward in Chapter~\ref{!!?!!} for blobs/knots to form in the jet beam when the flow is sufficient. In Fig~\ref{fig_brigt_blobs_2} jets J1 and J2 have high apparent speeds of $145\pm15\kms$ and $203\pm\kms$ respectively \citep{Zhang2016SoPh291859Z} and the jet in Fig.~\ref{fig_brigt_blobs_alt} is approximately $200\kms$ \citep{Chen2017ApJ84054C}. Based on the results of the simulations in the thesis it is feasible with these high speeds in a coronal environment for the blob formation to be generated/driven by shocks containing waves trapped in a jet beam. To rule out blobs forming due to shocks, one needs to measure the flow rate of the jet. This is because, as demonstrated in Chapter~\ref{!!?!!}, when the driver for the jet switches off, the knots start to fall through the jet and disappear. Therefore, if there is a continuous flow then a shock-based process may be plausible, or if it is a short, burst-like release of energy that produces flow then it is more likely to be due to magnetic reconnection.
\begin{figure*}[h]
\begin{centering}
\includegraphics[width=1.0\textwidth]{figures/Selection_064.png}
\caption{EUV jet observed in the west of AR 11513 on July 2, 2012. Each panel represents different passband observations denoted in the top left-hand corner in units of Angstroms. The bright blobs in the jet beam are most noticeable in $171~\AA$ where the slices are located. This figure and the analysis it encapsulates was produced by Dr. Chen.}
\label{fig_brigt_blobs_alt} 
\end{centering}
\end{figure*}
%STEREO originally constitute of two nearly identical satellites, reffered to as STEREO-A and STEREO-B, although STEREO-B is no longer in operation since 2014.    
% ($131~\AA$, $171~\AA$, $193~\AA$, $211~\AA$, $304~\AA$ and $335~\AA$)
%\begin{figure*}[h]
%\begin{centering}
%\includegraphics[width=1.0\textwidth]{figures/Selection_063.png}
%\caption{EUV jet observed in AR west of AR 11513 on 2012 July 2. From left to right the columns show shows the $171\AA$ line observation of the jet, emission measure of the jet, DEM-weighted average temperature of the jet. The time stamps are in the top right hand corner of the first column. This figure and the analysis it encapsulates was produced by Dr. Chen.}
%\label{fig_brigt_blobs} 
%\end{centering}
%\end{figure*}
%============================================================
\section{Summary and Discussion}
\label{sec:sum}
%============================================================
This Chapter compares the CSW properties of the synthetic jets with three observations of spicules captured with high-resolution CRISP/SST H$\alpha$ data. For both the simulation and observation, wave behaviour is identified by CSW estimates. Wavelet analysis of the CSW shows the effects of plasma flows on estimated periods. The Chapter discusses the possibilities that blob structures in observations of large solar jets could be knots produced by trapped shock waves in the jet. The main conclusions drawn from the results as follows:
\begin{itemize}
    \item{CSW variations are present in both the observed and synthetic jets, which highlight that CSW variations are a fundamental property of spicule jets. Earlier reports interpreted these variations as a consequence of confined sausage waves in chromospheric jet structures, although similar characteristics are also shown by MHD fluting wave modes. Here, for the first time, an alternate explanation is provided in terms of internal shock waves for cross-sectional deformations of chromospheric waveguides. In Chapter~\ref{!!?!!} we demonstrated that this phenomenon occurs over a whole range of jets with vastly different length scales, e.g. from astrophysical to laboratory jets. Therefore, the CSW variations are an intrinsic physical characteristic of jets with fast flow speeds. However, insufficient attention has been paid to this for solar jets, primarily due to observational constraints.}

    \item{Spectral analysis using wavelet transform suggests noticeable modulation in estimated periodicities for the CSW in both observed and synthetic jet structures. In most cases, the CSW variations have a shorter period in the falling phase than the rising phase. Similar effects were reported in the past for changes in linear MHD wave characteristics due to mass flows in theoretical and prominence studies.}

    \item{For the first time, the first overtone in the estimated periodicity is identified in both observed and simulated dynamic chromospheric waveguides (spicules). This could have important implications for chromospheric magneto-seismology, further enabling the estimation of much needed information for the longitudinal plasma and magnetic field for chromospheric jets.}
    
    \item{Knot formation could be an alternative explanation for blob structures reported in larger solar jets.}
\end{itemize}
%
This Chapter indicates the possible coupling between the longitudinal mass motions and CSW variations in spicule structures. Understanding such behaviour is crucial for the accurate estimation of the overall energy budget for atmospheric heating and associated dissipation mechanisms. Furthermore, simulations suggest the formation of knot substructures within the jet beam, prevalent during the rise-phase of plasma density. These substructures were closely linked with the estimated CSW variations in the jet and could be generated due to shock waves. The simulation results should inform observers to be cautious in attributing CSW variation in fast-flowing jets to sausage waves, as there other mechanisms, (e.g. shock waves occurring in supersonic flows) that can display CSW variations. \np
%
For the larger-scale EUV jets, it should be noted that regardless of formation mechanisms of knots in the jet beam, they can occur over a whole range of scales in jets. It is likely that with the increase in spatial observation on the horizon for spicules, it is only a matter of time before knot structures are observed in spicular jets.