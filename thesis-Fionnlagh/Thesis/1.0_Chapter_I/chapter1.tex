\section*{List of Symbols}
Below is a list of the notation used throughout the text unless stated otherwise: \\ \\
$\rho$ $\rightarrow$ Density.  \\
$p$ $\rightarrow$ Pressure. \\
$\boldsymbol{v} = (v_x, v_y, v_z)$ $\rightarrow$ Velocity.  \\
$\boldsymbol{B} = (B_x,B_y,B_z)$ $\rightarrow$ Magnetic field strength. \\
$\mu_0$ $\rightarrow$ Magnetic permeability. \\
$\boldsymbol{g} = (0,0,-g_z)$ $\rightarrow$ Gravitational acceleration. \\
$\gamma = 5/3$ $\rightarrow$ Ratio of specific heat. \\
$\eta$ $\rightarrow$ Magnetic diffusivity of the plasma. \\
$\widetilde{\mu}$ $\rightarrow$ Mean atomic weight (the average mass per particle in the units of the proton mass).  \\
$\rgas$ $\rightarrow$ Gas constant.\\
$\boldsymbol{j} = (1 / \mu_0) (\nabla \times \boldsymbol{B})$ $\rightarrow$ Current density.  \\
$H(z) = \dfrac{R T(z)}{\widetilde{\mu} g}$ $\rightarrow$ Scale height.  \\
$G$ $\rightarrow$ Gravitational constant. \\
$M_{\odot}$ $\rightarrow$ Solar mass. \\
$R_{\odot}$ $\rightarrow$ Solar radius. \\
$p_{mag} = \dfrac{B^2}{2 \mu_0} $ $\rightarrow$ Magnetic pressure. \\
$p_{tot} = p + p_{mag} $ $\rightarrow$ Total pressure. \\
$m = \rho \boldsymbol{v}$ $\rightarrow$ Momentum density. \\
$e$ $\rightarrow$ Total energy density. \\ 
$C^2_s = \gamma \dfrac{p}{\rho}$ $\rightarrow$ Sound speed squared. \\ 
$V_A^2=\dfrac{B^2}{\mu_0 \rho}$ $\rightarrow$ \Alfven speed squared.  \\
$\beta=\dfrac{p}{p_{mag}}$ $\rightarrow$ Plasma beta
\clearpage
\setcounter{page}{1}
%------------------------------------------------------------------------------
\chapter{Introduction}
\label{chap:intro}
%------------------------------------------------------------------------------
%\epigraphfontsize{\small\itshape}
\epigraph{``We're going to explore the outside world someday, right? Far beyond these walls, there's flaming water, land made of ice, and fields of sand spread wide."}{--- \textit{Armin Arlert}, \textup{Hajime Isayama}, Vol. 2  Attack on Titan}
%----------------------------------------------------------
\section{The Sun}
\label{sec:Sun}
%----------------------------------------------------------
The Sun is often referred to as a mundane star by the larger astrophysical community, as it is a common main sequence star (G-type), particularly when it is compared with the zoo of exotic astrophysical objects such as stars in binary systems, neutron stars, red giants, white Dwarfs, cepheids, Wolf-Rayet stars, among others. However, one of the most fascinating aspects of the Sun is its magnetic field and our ability to resolve the Sun compared to other astrophysical objects. To quote physicist Robert Leighton, ``If the Sun had no magnetic field, it would be as uninteresting as most astronomers think it is". The magnetic field makes the Sun a truly dynamic star as it moulds and shapes its atmosphere environment, for example with coronal loops, prominences, rosettes, fibrils, helmet streamers etc., as well as storing energy which can be released in dramatic fashion, e.g. through coronal mass ejections, spicules, EUV jets. Due to high resolution observations we can obtain fantastic images of these dynamics with both ground and space based instruments such as Solar and Heliospheric Observatory (SOHO) \citep{Domingo1995SSRv7281D}, Transition Region and Coronal Explorer (TRACE) \citep{Tarbell1994ESASP373375T}, Hinode \citep{Tsuneta2008SoPh,Suematsu2008SoPh,Ichimoto2008SoPh}, Solar Dynamic Observatory (SDO) \citep{Lemen2012SoPh27517L}, The Interface Region Imaging Spectrograph (IRIS) \citep{Pontieu2013SPD4403D}, Daniel K. Inouye Solar Telescope (DKIST) \\citep{Rast2020arXiv,Rimmele2020SoPh}, the Swedish Solar Telescope (SST) \citep{Scharmer2003SPIE}, and the Solar Orbiter \citep{sol_orb_2013SPIE8862E0EM}. The combination of the Sun's magnetic field and it being the only star we can fully resolve gives us a marvellous space laboratory on our astronomical doorstep, from which the physics of other stars can be understood. An overview of the basic properties of the Sun taken from \cite{priest2014magnetohydrodynamics}:
\begin{itemize}
    \item Age: $4.6 \times 10^9$ years.
    \item Mass: $M_{\odot}= 1.99 \times 10^{30}$ kg.
    \item Radius: $R_{\odot} = 6.96 \times 10^5$ km.
    \item Surface temperature: $5785$ K.
    \item Mean density: $1.4 \times 10^3$ kg m$^{-3}$.
    \item Mean distance from Earth: $1$ AU = $1.5 \times 10^8$ km.
    \item Surface gravity: $g_{0}=274$ m s$^-2$.
    \item Equatorial Rotation Period: $26$ days.
    \item Composition: $90 \%$ H, $10 \%$ He, $0.1 \%$ other elements.
\end{itemize}
%
The Sun has multiple concentric layers as seen in Fig. \ref{on_model}, and is powered in its core where nuclear reactions consume hydrogen to form helium. From these reactions, energy is released which ultimately leaves the surface as visible light. The next layer which surrounds the core is the radiative zone, where energy generated by the nuclear fusion in the core moves outwards as electromagnetic radiation. The next significant region is the convective zone, where the method of energy transportation changes from radiative to convective. This occurs because with increased distance from the heat source that is the core, the temperature drops, as described by the second law of thermodynamics. The temperature is eventually sufficiently low for heavier ions (e.g. carbon, nitrogen, oxygen, calcium and iron) to retain a collection of their electrons, which increases the opacity. This makes radiation transport less efficient and consequently traps heat, which in turn creates hot rising gas bubbles which cools when they reach the surface and begin to drop to the bottom of the convection zone, where they are then reheated, thus repeating the process. The next layer of the Sun is the photosphere, which is the observable surface. \np
%
An interesting aspect of stars is that they are all ``ringing" as they are a host of multiple standing waves. Just as earthquakes have been studied to probe the interior properties of our planet thereby discovering the Earth has a liquid outer core and solid inner core \citep{Lehmann1936}, the principle is the same with stars. When ignoring the effect of rotation and magnetism, these starquakes for the Sun have two main restoring forces; buoyancy (gravity) and pressure \citep{Appourchaux2010AARv18197A}. \np
%
Those pulsation modes which are dominantly restored by buoyancy are referred to as gravity or \textit{g}-modes. These \textit{g}-modes are mostly constrained to the core but propagate through the radiative zone. Hence, studying these waves could theoretically give us crucial information about the structure and dynamics of the deepest parts of the Sun that we are otherwise unable to directly measure. These waves are damped in the convective zone. This makes \textit{g}-modes very challenging to detect in the Sun. Claims of their observation \citep{Garc2007Sci3161591G, Fossat2017AA604A40F, Fossat2018AA612L1F}, have been met with skepticism \citep{Appourchaux2010AARv18197A, Schunker2018SoPh29395S,Appourchaux2019AA624A106A, Scherrer2019ApJ87742S}. \np
%
Pulsation modes, where pressure is the restoring force, are known as pressure or \textit{p}-modes. The \textit{p}-modes on the Sun was first discovered by \cite{Leighton1962ApJ135474L}, who found undulations on the Sun with periods near 5 minutes, referred to in many studies as the ``5-minute oscillation". The depth at which \textit{p}-modes penetrate depends on the frequency of the wave, where low frequencies propagate deep into the interior of the Sun and high frequencies are trapped closer to the surface. There are approximately $10^6$ resonant \textit{p}-modes on the Sun, with periods ranging from minutes to hours \citep{Demarque1999PNAS965356D}. As they are pressure waves, they freely travel through the convection zone. However, when \textit{p}-modes are propagating through the solar interior they encounter a temperature gradient and hence a sound speed gradient. The deeper the wave probes, the faster the sound speed is, which means it is refracted back to the surface. The sudden jump in conditions at the solar surface acts as a solid boundary that the wave is unable to escape, and it is reflected inwards. However, there are regions (e.g. intergranular lanes) where it is thought \textit{p}-mode wave leakage occurs, which means they are propagated upwards into the solar atmosphere and are drivers of solar features \citep{Suematsu1990LNP367211S, Pontieu2004Natur, Pontieu2005ApJ624L61D, Heggland2007ApJ6661277H, Pontieu2004Natur}.
%%fffffffffffffffff
%https://astroengine.files.wordpress.com/2012/07/thesis06.pdf
%\mfig{0.8}{figures/on.png}{Overview of the layers of the Sun. Source: \url{https://astroengine.files.wordpress.com/2012/07/thesis06.pdf}.}{on_model}
%fffffffffffffffffff
%%fffffffffffffffff
%https://astroengine.files.wordpress.com/2012/07/thesis06.pdf
\mfig{0.8}{figures/image10.png}{Overview of the layers of the Sun. Source: ESA: \url{https://www.esa.int/About_Us/ESAC/Gravity_waves_detected_in_Sun_s_interior_reveal_rapidly_rotating_core}.}{on_model}
%fffffffffffffffffff
%----------------------------------------------
\section{The Solar Atmosphere}
\label{sec:sol_atmos}
%-----------------------------------------------
The Sun's atmosphere is truly a complex and fascinating environment. It can be broadly  split into three main regions from the top down; the corona, transition region (TR), and chromosphere. Two main factors which separate these regions are their measured temperature (see Fig.~\ref{t_profile_sun}) and plasma beta, the ratio of gas and magnetic pressure (see Fig.~\ref{beta_profile_sun}).
%T_regoins
\mfig{0.725}{figures/T_regoins}{A plot of the temperature and density from the photosphere to the corona. Plot taken from \cite{Lang_2006ses}.}{t_profile_sun}
%--------------------------------------------------
\subsection{Corona}
\label{ssec:corona}
%--------------------------------------------------
The corona is the Sun's upper atmosphere which continually extends, reaching approximately tens of millions of kilometres into space. It follows the Sun's open magnetic field lines which eventually they feed into the solar wind. It is possible to observe the corona with the naked eye during eclipses, as seen in Fig. \ref{corona_image}. Otherwise to observe the corona a coronagraph is needed, which is a disk-shaped instrument placed on telescopes that can produce an artificial eclipse by blocking the light from the photosphere. The solar corona has an average temperature of $1-2~\rm{MK}$, but can even reach $10~\rm{MK}$. The high temperature of the corona is evidenced by the presence of ions with many electrons removed from the atom. This is shown by atoms such as iron which is $9-13$ times ionised in the corona, which indicate temperatures of $1.3~\rm{MK}$ and $2.3~\rm{MK}$, respectively \citep{Grotrian1939,Edl1943}. Despite its high temperature, it has a low amount of heat as it is very rarefied, with densities in the order of $10^{-12}~\rm{kg~m^{-3}}$ \citep{priest2014magnetohydrodynamics}. This in turn means that the energy density of the corona is much lower than that of the lower layers of the solar atmosphere, such as the photosphere, where the temperature is approximately $5000~\rm{K}$. Despite this, the quiet Sun needs to have a constant energy input of $1.1-1.6~\rm{erg~cm^{-2}~s^{-1}}$ \citep{Sakurai2017PJAB9387S} to maintain observed coronal temperatures. \np
%
One startling fact is the extreme coronal temperatures, which are higher than lower regions of the atmosphere (approx. $10^4~\rm{K}$) and photosphere (approx. $5,000~\rm{K}$). Intuitively, the temperature profile shown in Fig.~\ref{t_profile_sun} does not make sense, as the temperature is increasing with increased distances from the Sun's heat source (the core), breaking the second law of thermodynamics. This is known as the ``coronal heating problem", which was discovered by \cite{Grotrian1939} and \cite{Edl1943}. The coronal heating problem encompasses many open questions, such as: Why is the corona hot?; How does it maintain this heat?; Is the corona heated everywhere, or is heat produced in separate, localised events?; Is it heated in multiple different ways? Many theories have been put forward that can be reduced down to either wave based; e.g. acoustic shocks and MHD waves \citep{Alfv1947MNRAS107211A, Uchida1974SoPh35451U, Wentzel1974SoPh39129W, Priest1998Natur393545P, Hollweg1982ApJ254806H, Antolin2008IAUS247279A, Escande2019NatSR914274E}, or reconnection; e.g. nanoflares \citep{Parker1988ApJ330474P, Cargill1993SoPh147263C, Parnell2000ApJ529554P, Klimchuk2001ApJ553440K,  Cargill2004ApJ605911C, Antolin2021NatAs554A}. Due to the highly dynamic nature of the atmosphere, in reality both types of heating probably occur \citep{Zirker1993SoPh14843Z,Parnell2012RSPTA3703217P}. \np
%
The heating process can be broken into three main phases: (1) the generation of a carrier of energy (i.e. photospheric driving motions); (2) the transport of energy into the solar atmosphere; (3) the dissipation of this energy in various structures of the atmosphere \citep{Wentzel1974SoPh39129W, Robert2004AG45d34E}. For (1-3) there are a plethora of choices, but the main challenge lies in (3). There is agreement in the field that the Sun's magnetic field plays a key role \citep{Parnell2012RSPTA3703217P, Arregui2015RSPTA37340261A}, but the exact physical process that transports the energy from the photosphere upwards and dissipates the magnetic energy into heat, remains elusive. Another key obstacle is resolving (3), as for wave heating the dampening time of the waves is roughly proportional to the magnetic Reynolds number. The magnetic Reynolds number is approximately  $10^{14}$ under solar conditions, as it is dependent on the typical length scales of plasma flow, which are large on the Sun. Therefore, in the solar environment, the challenge is finding mechanisms that generate small length scales (e.g. resonant absorption, phase mixing, plasma inhomogeneities), otherwise, the wave energy will not be converted into heat quickly enough compared to the coronal cooling timescales \citep{Doorsselaere2020SSRv216140V}. There is a lack of definitive observation of reconnection/nano-flare heating by a series of small-scale reconnection events \citep{Hudson1991SoPh133357H, Parnell2012RSPTA3703217P}. It is possible that small scale solar jets, such as spicules, play an important role in coronal heating due to their ubiquity and dynamic nature, which is discussed later.
%fffffffffffffffffffff
\mfig{0.65}{figures/corona_vangorp.png}{Image of the Corona from a total eclipse that occurred on the 29th of March, 2006. Source: \url{https://apod.nasa.gov/apod/ap090726.html}.}{corona_image}
%fffffffffffffffffffff
%--------------------------------------------------
\subsection{Transition region}
\label{ssec:TR}
%--------------------------------------------------
The transition region (TR) is a thin region of approximately $5,000~\rm{km}$ \citep{Athay1981NASSP45085A} that sharply links the ``cool" chromosphere ($10^{5}~\rm{K}$) to the hot corona ($1~\rm{MK}$) at around $2~\rm{Mm}$ above the solar surface \citep{Lang_2006ses}. The TR is dynamically important to the Sun's atmosphere. Due to the sharp change at the TR, waves propagating through the chromosphere can suddenly steepen into shock waves that propagate both upwards into the corona, as well being reflected back into the chromosphere \citep{Pontieu2004Natur, Hansteen2007ASPC369193H, Yuan_2016ApJS, Zhenyong2018ApJ85565H}. These shock waves can lead to jet-like events \citep{Pontieu2004Natur, De_Pontieu2007ApJ, Heggland2007ApJ6661277H, kuzma2017ApJ84978K}, as well as the oscillation of the TR interface itself, e.g. referred to as transition region quakes (TRQ), that may transfer energy to surrounding magnetic structures \citep{Scullion2011}. A key aspect to note is that all energy that heats the corona and drives the solar wind must make its journey through this sudden change in environment \citep{Mariska1992strbookM}. 
%--------------------------------------------------
\subsection{Chromosphere}
\label{ssec:Chromosphere}
%--------------------------------------------------
The chromosphere spans from above the photosphere up to $2~\rm{Mm}$ \citep{Lang_2006ses}. In ancient times the chromosphere was only faintly seen as a reddy-pink glow around the boundaries of a solar eclipse. This rosy colour originates from the Balmer series of transitions for hydrogen emission (H$\alpha$). The atmospheric conditions of the chromosphere are sufficient to cause a quantum transition between the $N=3$ and $2$ energy levels of hydrogen. In the modern age, we can study the chromosphere in great detail thanks to excellent ground and space based telescopes (See Fig.~\ref{messy_chromo}). The chromosphere is typically observed in the H$\alpha$, CaII H, and CaII H and K lines \citep{Ayres2019sgspbook27A}. Observations in these spectral lines shed light on the reason that the chromosphere can be described as the ``magnetic complexity zone" \citep{Ayres2009astro2010S9A}, due to numerous complex dynamics and structures it hosts as displayed in Fig.~\ref{messy_chromo}. \np  
%
Together the TR and chromosphere make up the so-called interface region, which encapsulates an area of complex plasma and magnetic fields, which transports matter and energy between the photosphere and the corona. To understand the dynamics and topography of the magnetic field lines, it is important to establish whether gas or magnetic pressure is dominant. This is represented by plasma beta ($\beta$), and its value in the atmosphere is shown by the grey shaded region in Fig.~\ref{beta_profile_sun}. In the photosphere where gas pressure is dominant ($\beta>1$) the magnetic field gets dragged by granular flows, being pinched together as they emanate between granular lanes. At the boundary of supergranular lanes, the magnetic field lines form a larger network and the field lines stem out as magnetic flux tubes (MFT) as shown in Fig.~\ref{fig:chromo_Cart}. These MFT expand further up in the atmosphere due to dropping gas pressure, where magnetic pressure becomes dominant and the field lines become frozen into the plasma \citep{Ayres2009astro2010S9A}. An interesting region in the chromosphere is where $\beta=1$. In this region the sound speed and the \Alfven speed are equal, dividing the solar atmosphere into magnetic and non-magnetic regions \citep{Tsiropoula2012}. This region can lead to the formation of shocks, as it allows for the mode-coupling of purely magnetic waves (\Alfven waves) with magnetoacoustic waves \citep{Hollweg1982SoPh7535H, Rosenthal2002ApJ564508R,Bogdan2003ApJ599626B, Cally2008SoPh251251C, Wang2020ApJ891110W} and has a chaotic topology which forms the canopy observed in H$\alpha$ lines (see Fig.~\ref{messy_chromo} and canopy domain in Fig.~\ref{fig:chromo_Cart}). The magnetic field in the chromosphere is highly twisted and entangled as a consequence of these transitions of the $\beta$. \np
%
In recent years the correctness of the phrase ''coronal heating problem" has been called into question, and even labelled as a ``paradoxical misnomer" by \cite{Aschwanden2007ApJ}. This is because there is no direct evidence of local heating in the corona, and researchers should shift their focus towards solving the ``chromospheric heating problem" \citep{Aschwanden2007ApJ}. It is not clear how the transport and heating occurs between the interface region and corona. The answer may lie in the multiple jet features such as spicules, mottles, and dynamics fibrils, that are prevalent in the interface region, that energetically advance through the atmosphere \citep{Tsiropoula2012}. In particular, spicules are hypothesised to be a strong candidate for transport mechanisms to heat the atmosphere \citep{Kudoh1999ApJ514493K, Pontieu2007PASJ, Kudoh2008IAUS247195K, Mart2017Sci3561269M,Moore2011ApJ731L18M, Pontieu2017ApJ, Samanta2019Sci, Zuo2019AcASn, Bale2019Natur}.    
%--------------
%fffffffffffffffffff
\mfig{0.725}{figures/Selection_067.png}{A model of the plasma $\beta$ (ratio between gas and magnetic pressure) over an active region on the Sun, taken from \cite{Gary2001SoPh20371G}. A high (low) $\beta$ corresponds to gas (magnetic) pressure being the dominant force. The grey shaded region shows plasma beta at different heights.}{beta_profile_sun}
%fffffffffffffffffffff
\mfig{1}{figures/messy_chrom.png}{Two snapshots are taken from H$\alpha$ observations using the SST (Swedish Solar Telescope). The panel to the left (right) shows observations of chromosphere above a sunspot in AR998  (chromospheric filaments). These snapshots were taken from the SST movie gallery and can be found here: \url{https://ttt.astro.su.se/isf/gallery/movies/2008/halpha_10Jun2008_AR998_mu043.mov}, \url{https://ttt.astro.su.se/isf/gallery/movies/2005/halpha_set1_04Oct2005_region_dt5s.mov}}{messy_chromo}
%fffffffffffffffffffff
%fffffffffffffffffffff
\begin{sidewaysfigure}[ht]
    \includegraphics[width=\linewidth]{figures/Selection_066.png}
    \caption{Cartoon representation of the complexity of the lower atmosphere taken from \cite{Wedemeyer2009SSRv144317W}. The solid black lines show the magnetic field lines stemming from the intergranular lanes. A and B highlight the small-scale loop features and D-F shows the condition for wave and magnetic canopy interaction.}
    \label{fig:chromo_Cart}
\end{sidewaysfigure}
%fffffffffffffffffffff
%------------------------------------------------------------------------------
\section{Jets in the Solar Atmosphere}
\label{sec:spicule-jets}
%------------------------------------------------------------------------------
The study of jets on the Sun is over $150$ years old, starting when Father Angelo Secchi first observed jets in the chromosphere in the 1870s in the Observatory of the Roman Collegium, and described them as ``burning fields" (see example in Fig.~\ref{de_flammes}). It took nearly 100 years after Secchi's discovery of spicules for researchers to realise the sheer variety of jets that exist in the solar atmosphere \citep{Raouafi2016}. This started with modern observations in the 1970s, with the discovery of coronal transients in Fe XIV, macrospicules and explosive events \citep{Demastus1973, Bohlin1975ApJ197L133B, Withbroe1976ApJ, Brueckner1980HiA}. This active decade resulted from the launching of the space station Skylab and its capability of carrying out EUV observations. More types of jets were discovered in the 1990s, due to observations with space based Yohkoh soft x-ray telescopes. \cite{Shibata1992PASJ} and \cite{Strong1992PASJ} discovered solar x-ray jets which are the largest and most energetic of the coronal jets, and new jets have even been discovered recently \citep{Cho2019ApJ884L38C}. \np
%
As a result of many excellent observations, it is clear that solar jets are omnipresent at all times on the Sun regardless of the phase of the solar cycle and there is a vast variety of types of jets, occurring across a whole range of scales. Nowhere is this more true than the chromosphere, which is dominated by spicular jets, which are thin, small-scaled, short lived jet structures, rapidly evolving with time and height. These spicular jets occur everywhere from quiet Sun (QS) \citep{Pontieu2007astroph2081D,Rouppe2007ApJ660L169R,Pereira2012,Pereira2014ApJ}, around active regions (ARs) \citep{Pontieu2007astroph2081D,Pereira2012,Rouppe2013ApJ77656R,Gafeira2017ApJS2296G}, and coronal holes (CHs) \citep{Yamauchi2005ApJ629572Y,Moreno2008ApJ673L211M,Pereira2012,Young2015ApJ801124Y}. These transient events could make significant contributions to the coronal heating and solar wind acceleration, which to this day are partially unexplained \citep{Mart2017Sci3561269M, Pontieu2017ApJ, Samanta2019Sci, Zuo2019AcASn, Bale2019Natur}. \np
%
\mfig{0.75}{figures/flammes_alt.png}{Example of early observations of the evolution of spicules taken by Father Angelo Secchi which he describe as flames that are so small they resemble grass in gardens. Images are taken from \cite{Secchi1877}.}{de_flammes}
%------------------------------------------------------------------------------
\subsection{Spicules}
\label{subsec:Spicules}
%------------------------------------------------------------------------------
Spicules are thin plasma flows that are omnipresent on the surface of the Sun, and there are approximately $2 \times 10^{7}$ Ca II spicules on the Sun at any time \citep{Judge_2010ApJ}. Spicules are best observed in strong chromospheric and transition region (TR) lines such as H$\alpha$, Ca II H \& K, Mg II H \& K, C II and Si IV lines. They were first observed by \cite{Secchi1877} and were named by \cite{Roberts1945ApJ}, who stated ``I was amazed at the extremely brief lifetimes and the great frequency of occurrence which visual observations of these spicules indicated". The lifetime he reported ranged from approximately 2-11 minutes which fits in the range of current estimates for spicules lifetimes. These short lifetimes, along with temporal and spatial (spicules diameters are a few hundred kilometres) resolution limits, historically made observing individual spicules difficult \citep{Sterling_2000SoPh}. Also, the dynamic nature of the spicules poses a challenge as they are typically mobbed by other spicules shielding one another, they have bi-directional flows, kinking, twisting, and torsional motions. Other effects such as distortions due to the projection effect and spicule inclination add even more complexity, as these effects can give a misleading picture of spicule dimension and behaviour \citep{Porfir2016A}. \np
%
Despite these challenges in observations, there has been much effort as there is a substantial amount of observational data on spicules available, giving us an understanding of their basic properties (mass density, temperature, velocity and magnetic field) \citep[see reviews:][]{Beckers1968, Beckers1972ARA&A}, possible driving mechanisms \citep[see review:][]{Sterling_2000SoPh}, and the waves and oscillations spicules can host \citep[see review:][]{Zaqarashvili_2009SSRv}. The main interest in spicules lies in the potential to solve outstanding problems in solar physics, such as providing mass and energy into the solar atmosphere and wind, heating the chromosphere/corona, and driving the solar wind \citep{Pontieu2011Sci, Moore2011ApJ731L18M, Henriques2016, Samanta2019Sci}. The mass flux taken by the spicule to the corona exceeds that of the solar wind by two orders of magnitude \citep{Thomas1961}, so if even $1\%$ of the spicule mass flux escapes the Sun, this would be sufficient to supply the solar wind mass \citep{Pneuman1977AA55305P, Pneuman1978SoPh5749P, Tian2014Sci346A315T, Samanta2015ApJ815L16S}. Spicules have been estimated to have an energy flux of around $5\times10^9~\rm{erg~cm^{-2}~s^{-1}}$, therefore if even $1\%$ of spicule energy is dissipated in the corona then it could power the upper atmosphere \citep{Zaqarashvili_2009SSRv}. Another important factor, due to the ubiquity of spicules, is that it is imperative that we can accurately model and describe spicules for us to understand the Sun's chromosphere. The combination of improved spatial and temporal resolution achieved with telescopes, such as Hinode, TRACE, IRIS and SST, over the last two decades, and improved computational models, have led to a ``renaissance" in spicule research \citep{Aschwanden2019ASSL}. A major factor in this renewed interest was the rediscovery that the spicules contain two distinct types, labelled as Type I (TI) and Type II (TII) spicules \citep{Pontieu2007PASJ}. This categorisation is based on their lifetimes, speed, and trajectory, and is not to be confused with the concept of Type I and Type II spicules originally introduced by \cite{Beckers1968}, who separates spicules based on significant differences in their line width. \np
%------------------------------------------------------------------------------
\subsubsection{Classical/Type I Spicule}
\label{subsec:TI}
%------------------------------------------------------------------------------
To be clear with our terminology of ``spicule" we will adopt the definition ``classical spicule" as used in \cite{Sterling2010ApJ7221644S}, \cite{Pereira2013ApJ76469P}, and \cite{Sterling2020ApJ893L45S} to refer to limb spicule features observed pre-Hinode i.e. 2006. Classical spicules are observed at the solar limb, typically observed in H$\alpha$ as thin, finger-like features, that rapidly elongate upwards with an average velocity of around $15-40\kms$ \citep{Pontieu2007PASJ} and are grouped together with other spicules, emanating across the boundaries of supergranular cells (see Spicule I stemming from the rightmost network in Fig.~\ref{fig:chromo_Cart}). This group behaviour was first described as ``porcupine” and ``wheat field" patterns by \cite{Lippincott1957SCoA215L}. Properties of classical spicules are outlined by these early reviews \cite{Beckers1968,Beckers1972ARA&A}, where these thin jet-like structures are reported to reach heights of $6.5-9.5~\rm{Mm}$ during their $5$ minute lifetime, and rise with apparent velocities of $25\kms$. These values are in agreement with more recent H$\alpha$ observations of $40$ spicules with lifetimes $7.1\pm2.3~\rm{mins}$, heights of $7.2\pm2~\rm{Mm}$, and velocities of $27\pm18.1\kms$. These spicules are proposed to be initiated by \textit{p}-mode leakage between granular cells \citep{Pontieu2004Natur}. \np
%
The launching of Hinode/SOT (Solar Optical Telescope) in combination with better ground-based telescopes, such as SST, and improved image processing techniques, was a game-changer as it allowed for the study of jets in unprecedented detail both spatially and temporally \citep{Aschwanden2010SoPh262235A}. \cite{Pontieu2007PASJ} used observational Ca II H line data collected by Hinode and employed a slit to temporally track the evolution of spicules. They found that, in general, spicules follow a non-ballistic parabolic path or rise up and rapidly fade out in space-time diagrams. They identified that classical spicules can be split into two distinct populations of spicules based on their velocities, lifetimes, and trajectories. The physical properties of TI spicules are akin to those of classical spicules, as they are defined as long lived (in comparison to TII) structures with lifetimes of $\sim 180-420~\rm{s}$, with bidirectional up flows of mass (which rise from the limb and then fall) following a non-ballistic parabolic path, and have an upward velocity of ranging around $15-40\kms$ \citep{Pontieu2007PASJ}. These values are in agreement with a more recent study by \cite{Pereira2012} who report average (standard deviations) values of lifetimes as $262~\rm{s}~(80~\rm{s})$, upward velocities of $30\kms~(9\kms)$, and reach heights of $6.02~\rm{Mm}~(1.21~\rm{Mm})$. \np
%
The heights of spicules are typically measured from their footpoints in the photospheric limb, up to the point at which the spicule becomes no longer visible. However, measuring the heights of spicules can be tricky as photospheric roots can be difficult to identify due to their grouping behaviour; it is not obvious whether the footpoints are in front or behind the limb, and the top of the spicule does not have a clearly defined boundary. Other observational factors add to this difficulty, such as exposure time and seeing conditions. Spicules have historically been mostly observed in the H$\alpha$ line, which is difficult to trace down due to the opacity of the chromosphere. Despite these difficulties, there is general agreement on the reported heights of spicules \citep{Tsiropoula2012}. Using data from H$\alpha$ observations, \cite{Beckers1968, Beckers1972ARA&A} estimated the average heights between $6.5-9.5~\rm{Mm}$. \cite{Pasachoff2009SoPh26059P} used both H$\alpha$ observations and TRACE data in the $16,000~\rm{ \AA}$ channel, to obtain heights in the range $4.2-12.2~\rm{Mm}$ with a mean of $7.2\pm2~\rm{Mm}$. \cite{Pasachoff2009SoPh26059P} observed heights in TRACE UV and found them to be $\sim 2.8~\rm{Mm}$ taller than in H$\alpha$. \cite{Pontieu2007PASJ} measure heights in Ca II H using Hinode/SOT which vary from a few hundred km to $10~\rm{Mm}$, with most below $5~\rm{Mm}$. \cite{Pereira2012} report maximum heights of $4-8~\rm{Mm}$ using Ca II H data from Hinode/SOT. Numerous studies report that the lifetime of classical spicules is between $120-720~\rm{s}$ with an average around $300~\rm{s}$ \citep{Roberts1945ApJ,Rush1954AuJPh7230R, Lippincott1957SCoA215L, Alissandrakis1971SoPh2047A, Cook1984AdSpR459C, Georgakilas1999AA341610G}. More recently, \cite{Pasachoff2009SoPh26059P}, who studied the spicules in H$\alpha$, found lifetimes between $180$ and $720~\rm{s}$ with a mean value of $426\pm138~\rm{s}$.
% 
In the past spicules widths ($200-1,000~\rm{km}$) have been very close to the resolution limits of observations and hence most impacted by observation conditions such as seeing and/or overlapping effects, making it challenging to separate into single spicules \citep{Pontieu2007ASPC, Tsiropoula2012}. In addition, spicule widths and heights are impacted by the line which is used for observations. For example, \cite{Pasachoff2009SoPh26059P} used H$\alpha$ data collected with SST and measured widths of $300-1,100~\rm{km}$, with a mean diameter of $660~\rm{km}$, and found that the widths are greater by a factor of $1.5$ with ranges of $700-2,500~\rm{km}$ when measuring with $1600~\rm{\AA}$ TRACE. However, this discrepancy in widths could be attributed to TRACE resolution being approximately four times lower than SST. In general, for the classical spicule, widths were measured using H$\alpha$ and Ca II H and K lines, and spicules are measured as having diameters between $400$ and $2,500~\rm{km}$ \citep{Dunn1960Obs8031D, Beckers1968, Beckers1972ARA&A, Lynch1973SoPh3063L}. Modern studies find spicule widths ranging from around $220-420~\rm{km}$ with a mean (standard deviation) of $384~\rm{km}~(81~\rm{km})$ \cite{Pereira2012}. \np
%
An important aspect is that spicules are typically observed to be inclined, at an angle that ranges from approximately $20^{\circ}$ \citep{Beckers1968} to $37^{\circ}$ \citep{Trujillo2005ApJ619L191T}, with an average of $23^{\circ}$. This is in agreement with the observations of \cite{Pereira2012}, who measured inclinations of around $5^{\circ}-25^{\circ}$. Classical spicules' temperatures range from $5,000-15,000$ K and they have densities of approximately $3\times10^{-13}$ g cm$^{-3}$ \citep{Sterling_2000SoPh}. \np 
%
As with the classic spicule, TI spicules are hypothesised to be driven by \textit{p}-mode leakage between granular cells, that steepen into magnetoacoustic shocks due to decreasing density of the atmosphere as they move upwards through the chromosphere \citep{Pontieu2004Natur, Pontieu2007PASJ, Mart2009ApJ7011569M}. Numerical models have predicted that shock-based drivers create non-ballistic trajectories and predict a linear correlation between deceleration of the jet and its maximum velocity \citep{Heggland2007ApJ6661277H}. Interestingly, this correlation is not only seen in TI spicules \citep{Pereira2012}, but also mottles \citep{Rouppe2007ApJ660L169R}, dynamic fibrils \citep{De_Pontieu2007ApJ}, and macrospicules \citep{Loboda2019ApJ871230L}, alluding to the likelihood that all these phenomena are linked to common driving mechanisms.
%------------------------------------------------------------------------------
\subsubsection{Type II Spicules}
\label{subsec:TII}
%------------------------------------------------------------------------------
The categorisation of spicules into TI and TII was outlined by \cite{Pontieu2007PASJ}, who used slits to construct distance-time diagrams. In these distance-time diagrams, they identified spicules that follow a non-ballistic parabolic trajectory (TI) and spicules that follow a linear trajectory as they rise up in Ca II then fade out (TII). In addition, each type was found to occur over different time scales. The TII spicules are fast, with velocities ranging between $30-110\kms$, and are short lived, reaching their apexes of $>5~\rm{Mm}$ in roughly $50-150~\rm{s}$ in contrast to TI spicules. TII spicules form rapidly ($\sim 10$ s), are very thin ($\leq 200~\rm{km}$ wide) and seem to be rapidly heated to at least TR temperatures, sending material through the chromosphere. Due to their high speeds and rapid formation they were thought to be driven by magnetic reconnection \citep{Pontieu2007PASJ}. The existence of TII spicules was called into question by \cite{Zhang2012ApJ} who revisited the same observational data. \cite{Zhang2012ApJ} questioned the use of slits by \cite{Pontieu2007PASJ}, as any spicule that evolves at an angle to the slit will create errors in measurements of spicule heights and lifetimes. \cite{Zhang2012ApJ} demonstrated that using filtergram data may be more appropriate as it better captures the 3D motion of spicules, and they found no evidence of TII spicules. However, this was later refuted by \cite{Pereira2012} where again the same data used by \cite{Pontieu2007PASJ} were revisited, but they use a semi-automated procedure that individually tracks spicule evolution and clearly recaptures the TI and TII spicule populations. There have been numerous reports claiming observations of TII spicules, and the existence of two different spicule types is generally accepted \citep{Rouppe2009ApJ, Rouppe2015ApJ799L3R, Shetye2016AA589A3S, Rutten2019AA632A96R, Yurchyshyn2020ApJ891L21Y, Chintzoglou2021ApJ90682C}. \np   
%
Interestingly, \cite{Pereira2012} not only recovered the sub-populations of spicules, but they also found that TII spicules were the most populous. This raises the question, if TII spicules are the most common then why are classical spicules' properties most akin to TI spicules? \cite{Pereira2013ApJ76469P} showed that if Hinode data is degraded to have a lower spatio-temporal resolution, the classical spicule properties are regained. They propose that for low spatio-temporal resolution a rapidly disappearing TII spicule may fade out and be replaced by another spicule, resulting in one longer lived structure, therefore introducing an observational bias for longer lifetimes and introducing errors in velocity measurements. This highlights the importance of spatial-temporal resolution for these small and rapidly evolving structures.\np
%
When observing TII spicules in Ca II they appear to rise linearly until their maximum length and then dissipate quickly over their whole length. It has been suggested that TII spicules are being heated out of the Ca II passband \citep{Pontieu2007PASJ, Pereira2012, Skogsrud2015ApJ806170S, Chintzoglou2018ApJ85773C, Chintzoglou2021ApJ90682C}. \cite{Pereira2014ApJ} reported the first spicule observation with IRIS, which added more evidence for the existence of TII spicules. They find that the TII spicules show parabolic space-time diagrams in the IRIS and AIA filters. Although this contradicts their earlier classification of TII spicule i.e. linear spicule \citep{Pereira2012}, there are clearly two spicule types as their defining properties (lifetimes, velocities, and heights) fall into two distinct groups. In multi-thermal studies of TII spicules, heights are reported to have a range of $8-20~\rm{Mm}$ with an average of $12.5~\rm{Mm}$ \citep{Pereira2014ApJ, Skogsrud2015ApJ806170S}. The study of TII spicules is important because they have a larger potential to be able to transfer energy and mass from the photosphere to the interface region and corona. It has been previously thought that spicules could not contribute to coronal heating due to a lack of a  coronal counterpart \citep{Withbroe1983ApJ}.
%------------------------------------------------------------------------------
\subsection{Mottles}
\label{subsec:mots}
%------------------------------------------------------------------------------
Mottles are rapidly changing hair-like short jets observed on disk in the QS regions that are an omnipresent chromospheric structure. They are organised in a complex geometric pattern over the solar disk following the boundaries of the chromospheric network and observed on disk, typically in H$\alpha$ and Ca II lines. Mottles are typically categorised into two groups \citep{Beckers1963ApJ138648B}:
\begin{enumerate}
\item Chains: a small group of mottles that extrude between the boundary of supergranular cells. The mottles are oriented in the same direction and when observed near the limb, they emanate outwards in the same direction, forming what \cite{Cragg1963ApJ138303C} called bushes.
\item Rosettes: a larger group of mottles in a circular collection, which is stretching out radially around a common centre, like the stems of a drooping bouquet. They form around the common boundary area of three or more supergranular cells and have a central bright core and are surrounded by both dark and bright mottles \citep{Tsiropoula2012}.    
\end{enumerate}
% this section is going to be thier properties
The combination of (1) and (2) form a chromosphere network. The main properties of mottles are that they have heights of roughly $2-10~\rm{Mm}$ and have lifetimes of $2-15$ minutes, which is similar to spicules \citep{Suematsu1995ApJ}. They seem to be generated several hundred kilometres above the photosphere and undergo real mass motions of $10-30\kms$. Mottles are very dynamic structures that can vary in appearance as they are curved, straight, thin, thick and have transverse motions \citep{De_Pontieu2007ApJ}. Most mottles have an ascending and descending phase that follows a parabolic trajectory. The largest Doppler signal appears at the beginning of the ascending phase and the end of the descending phase. The velocity profiles of mottles are symmetrical around zero. The deceleration seen in mottles is too small to be purely the result of solar gravity (i.e. they don't display perfect ballistic flight). \cite{Rouppe2007ApJ660L169R} found that there is a linear correlation between deceleration and maximum velocity of QS dark mottles and this relation has been observed in dynamic fibrils. They propose that these jets are driven with the same mechanism, which is by MHD shock waves, and this viewpoint is further evidenced by numerical simulations \citep{De_Pontieu2007ApJ, Hansteen2006ApJ}. \np
% 
An open question is whether mottles and spicules are the on-disk/limb counterparts to one another \citep{Tsiropoula1993A}. There is an observational inconsistency between interpreting spicules and mottles as counterparts. One main issue is that the velocities of spicules (calculated from proper motions) is much greater than those measured in mottles (derived from spectroscopic observations) \citep{Grossmann1992AA264236G, Christopoulou2001SoPh19961C}. The discrepancy in the velocities is large, with up to an order of magnitude difference \citep{Grossmann1973SoPh28319G}. Spicules and mottles both exhibit different velocity distributions; mottles have symmetrical bi-directional flow whereas spicules' velocity distributions are asymmetric, with the rising phase being most dominant. These differences in the observations can not be attributed merely to the angular distributions of spicules and mottles \citep{Grossmann1992AA264236G} and have led researchers to believe that mottles and spicules are separate features \citep{Christopoulou2001SoPh19961C}. \cite{Christopoulou2001SoPh19961C} propose two possible reasons for this discrepancy, (I) the fact that the values derived from spectroscopic observations represent averages of $1-\ang{;;2}$. Therefore, as the matter moves with greater velocities it is confined to structures below this resolution limit, then the velocity signal is affected by seeing. (II) For high-velocity spicules, they are near vertical, high altitude structures that occur amidst a sea of mottles that have a low velocity and are highly inclined structures. Therefore, due to the geometrical effect when observing at the limb, spicules will dominate, whereas on disk the mottles would dominate \citep{Grossmann1992AA264236G}. \np
%
When linking spicules and mottles there are two avenues to consider, indirect and direct observational evidence. The indirect evidence that supports this link are: dark mottles (absorbing features in H$\alpha$ and Ca II lines) and spicules have their optimum visibility at the same wavelength in the wings of H$\alpha$ \citep{Tsiropoula1993A}, follow a parabolic trajectory akin to TI spicules \citep{Rouppe2007ApJ660L169R} and have similar important physical properties such as lifetime and height. The most convincing observational evidence would be to track a spicule or mottle journey as it crosses the solar limb. This is challenging as when tracing the spicules back on disk, one does not know on what side of the limb the observed feature is located, and it's expected that most spicules will be rooted close to the solar limb so that they would not transform into a clearly identified disk structure \citep{Beckers1968}. \cite{Christopoulou2001SoPh19961C} found multiple examples of individual mottles crossing the solar limb and gives more support to the link between mottles and spicules. In light of the observational evidence both indirect and direct, one can reasonably conclude that spicules and mottles are counterparts to one another.
%------------------------------------------------------------------------------
\subsection{Dynamic Fibrils}
\label{subsec:dfibs}
%------------------------------------------------------------------------------
Dynamic fibrils are thin, tube-like, elongated, are highly dynamic, and are observed typically in AR \citep{De_Pontieu2007ApJ,Hansteen2006ApJ}. \cite{Foukal1971SoPh1959F,Foukal1971SoPh20298F} proposed that there is a relationship between the thin MFT structures seen near AR and QS environments. They showed that all these features have similar physical parameters, e.g. length, lifetimes, velocities, density and temperatures, for the fibril and spicular phenomena. The dynamic fibrils they observed appeared to be more elongated than spicules and mottles. Due to the strong magnetic field, they are more inclined to the vertical and appear horizontal when observed on the disk. When dynamic fibrils are observed in H$\alpha$ and Ca II wavelengths, these structures show a close resemblance to the mottles in the QS regions, in terms of appearance in clusters as bushes or rosettes. \cite{De_Pontieu2007ApJ} reported on their observations using SST of dynamic fibrils in H$\alpha$; they found average length $1.25~\rm{Mm}$ with ranges varying from ($0.4-5.2~\rm{Mm}$), with lifetimes of $120-650~\rm{s}$ and average widths of $340~\rm{km}$. These are in agreement with \cite{Morton2012NatCo31315M} and \cite{Gafeira2017ApJS2297G}, who measured widths of $360\pm120~\rm{km}$ and $260~\rm{km}$, respectively. Dynamic fibrils ascend with a typical velocity of $10-30\kms$ and follow a non-ballistic parabolic path as they rise and fall \cite{Beckers1968}. This flight path is expected if it is driven with chromospheric shock waves that occur when convective flows and \textit{p}-modes leak into the chromosphere \citep{Langangen2008ApJ6731194L,De_Pontieu2007ApJ}. \np
%
Due to more reliable observations of these jet structures, thanks to developments in observational techniques, such as bigger telescopes combined with real-time wavefront corrections by adaptive optics (AO) systems, e.g. \cite{Scharmer2003SPIE4853370S} and \cite{Rimmele2000SPIE4007218R}, and postprocessing methods, e.g. \cite{van2005SoPh228191V} and \cite{von1993AA268374V}, it has been possible to study how dynamic fibrils form. Through a combination of observational data from SST and numerical experiments using the Bifrost 3D radiative code, \cite{Hansteen2006ApJ} showed that jets in AR are a natural consequence of upwardly propagating slow-mode magneto-acoustic shocks. These shocks are generated by convective flows and \textit{p}-mode oscillations in the lower photosphere, and leaking upward into the magnetized chromosphere along inclined flux tubes. This viewpoint is also supported in other works \citep{Heggland2007ApJ6661277H,De_Pontieu2007ApJ,Pontieu2004Natur,Suematsu1990LNP367211S}, and this driving mechanism for dynamics fibrils is akin to mottles and TI spicules. Given that the physical properties, dynamics, morphology and driver of dynamic fibrils are all similar to mottles and TI spicules, this strongly suggests that these jets are a similar phenomenon, only located in regions of different magnetic activity, and that dynamics fibrils are AR counterparts to QS mottles \citep{Rouppe2007ApJ660L169R}.
%------------------------------------------------------------------------------
\subsection{RREs/RBEs}
\label{subsec:rbe}
%------------------------------------------------------------------------------
Rapid red-shifted and blue-shifted excursions (RRE and RBEs) are short-lived and rapidly moving absorption features in the strong chromospheric H$\alpha$ and Ca II spectral lines. They appear as sudden shifts in the Doppler estimates at the wing-position of the line of the profile. They are located at the edges of rosettes where there are no dominating shocks compared to the network and internetwork. RBEs were first reported by \cite{Langangen2008ApJ} while searching for on disk counterparts of TII spicules. They reported lengths in the range of $0.5-1.5~\rm{Mm}$ with average $1.2~\rm{Mm}$, and widths in range of $300-600~\rm{km}$ with average of $500~\rm{km}$. They showed these are short lived features with lifetimes of $45\pm13~\rm{s}$ and velocities in the order of $15-20\kms$. \cite{Rouppe2009ApJ} studied RBEs in Ca II and H$\alpha$ data using the CRISP instrument at SST. They reported similar properties with lengths of $\sim 3~\rm{Mm}$, lifetimes of $\sim 45~\rm{s}$ and Doppler velocities $\sim 20\kms$. \cite{Sekse2013ApJ76944S,Sekse2013ApJ764164S} investigated RREs in the red wing of the  Ca II and H$\alpha$ lines, finding average length $\sim 3~\rm{Mm}$, with widths of $\sim 250~\rm{km}$ and Doppler velocities of $\sim 15-20\kms$, which are similar to those found with RBEs.  More recently, \cite{Kuridze2015ApJ80226K} showed that RREs and RBEs have near-identical lifetimes, widths and lengths. They reported that both have lifetimes have a range of $\sim 20-120~\rm{s}$, with a majority living around $40~\rm{s}$, lengths are in the range of $2-9~\rm{Mm}$ with a typical value of $\sim 3~\rm{Mm}$, widths have a range of $200-500~\rm{km}$ with the majority around $250~\rm{km}$, and the velocities found were estimated to be in the range of $50-150\kms$, where the upper limit is super-\Alfvenic in the chromosphere. \np
%
Finding an on disk counterpart to TII spicules is important as it gives an alternative path to explore to solve outstanding problems presented by these jets. Having a top view of these features means that we will have a view that is not affected by line of sight superposition issues that occur at the limb. These RBEs give an insight into the possible mechanism for jet formation. In \cite{Langangen2008ApJ} the length of RBE is shorter than TII spicules, but this could be caused by an intrinsic difference in the visibility. The magnitude of the mass motion in RBEs is lower than the apparent motion observed in TII spicules. It is thought the driver for TII spicules is magnetic reconnection. If this reconnection is taking place at different heights in the atmosphere, and if the amount of energy is similar for each event, then we would expect the density and velocity of the jet to be dependent on height; i.e. (1) lower heights would give high density jet with low velocity, and (2) higher heights give a low density jet with high velocity. This inverse relationship between density and velocity nicely accounts for the discrepancy between on disk mass motion and the limb apparent motion. For situation (1) this would show enough absorption to be visible on disk, but would be missed on limb observations due to the fibrilar mess at lower heights. For scenario (2) these events on disk would be difficult to observe due to their low opacity, but on the limb, they rise past the mess of the lower chromosphere and can be clearly visible. This means that the difference seen in mass motion could be due to observation biases that are dependent on the locales of the jet. \cite{Langangen2008ApJ} suggest that Ca II RBEs are linked to Hinode Ca II H spicules observed at limb \cite{Pontieu2007PASJ}. This reasoning is based on the matching physical properties, such as lifetimes, location near network, fading, spatial extent, and the fact that they only show blueshift which corresponds to upward motion. \cite{Rouppe2009ApJ} add more fuel to the fire by reporting more similarities of lifetimes, locations, temporal evolution, velocities, acceleration, and occurrence rate, between RBEs and TII spicules. In addition, they report that RBEs undergo significant transverse motions ($\sim 8\kms$) during their life, similar to those observed in TII spicules ($\sim 12\kms$) \citep{De_Pontieu2007}. Overall these reported parameters of RBEs agree well with what is observed in TII spicules on the limb. \np
%------------------------------------------------------------------------------ 
\subsubsection{Macrospicules}
\label{subsec:Mspic}
%------------------------------------------------------------------------------
Macrospicules, as their name suggests, are like a spicule but on a larger scale. As with all spicular features their investigation is important because it is linked to the potential to help resolve the nature of the source of the solar wind generation, and how the corona is heated and maintained. Macrospicules extend further into the solar atmosphere and they are longer lived than classical spicules, but macrospicules are sparsely seen and are typically visible in TR lines e.g. He II $304~\rm{\AA}$, N IV $765~\rm{\AA}$, and O V $630~\rm{\AA}$, formed at temperatures approximately $8\times10^4$, $1.4\times10^5$ and $2.5\times10^5~\rm{K}$, respectively. Macrospicules have proved a challenge to identify as they are similar to other large jet-like phenomena, due to their properties having a large range. Macrospicules were first defined around 46 years ago when \cite{Bohlin1975ApJ197L133B} described these features at polar coronal holes using SkyLab's EUV slitless spectrograph. \cite{Bohlin1975ApJ197L133B} identified 25 macrospicules with lengths of $5.8-18.1~\rm{Mm}$ with a lifetime of around $8-45$ minutes. They reported that macrospicules were only visible in He II $304~\rm{\AA}$, but not in Ne $VII 465~\rm{\AA}$ (TR line) or Mg IX $368~\rm{\AA}$ (coronal line). \np
%
Due to the difficulty of identifying macrospicules, older studies tended to be limited to case studies or small groups, such as \cite{Moe1975SoPh4065K}, \cite{Bohlin1975ApJ197L133B}, \cite{Labonte1979SoPh61283L}, \cite{Pike1997SoPh175457P,Pike1998SoPh182333P}, and \cite{Parenti2002AA384303P}. As the most recent of these studies, \cite{Parenti2002AA384303P} studied a single macrospicule which extended to $60$ Mm, reaching a maximum velocity of $\sim 80$ km s$^{-1}$, average falling speed of $26\kms$, estimated temperature of around $2\times10^5~\rm{K}$, and density of $10^{-10}$ cm$^{-3}$. While these studies still hold value, they do not give a strong statistical representation of the behaviour of these features. This has been rectified in more recent studies such as \cite{Bennett2015ApJ808135B}, \cite{Kiss2017ApJ83547K}, and \cite{Loboda2019ApJ871230L}, who all leverage within the range of $2.5-5$ years of data collected by SDO/AIA with samples of $101$,
$301$ and $330$ macrospicules, respectively. These studies agreed with \cite{Wang1998ApJ509461W} that macrospicules are seen in different regions of magnetic environments although they are less numerous, with \cite{Loboda2019ApJ871230L} reporting $63.3\%$ found in CH and $36.7\%$ in QS regions. In general they show that macrospicules range in height from $7- 70~\rm{Mm}$, in width from $3-16~\rm{Mm}$, and have maximum velocities of $10-150\kms$ and  lifetimes $3-45$ minutes \citep{Bohlin1975ApJ197L133B, Withbroe1976ApJ, Karovska1994ApJ, Parenti2002AA384303P, Bennett2015ApJ808135B, Kiss2017ApJ83547K, Loboda2019ApJ871230L}. \np
%
Their widths and heights put them among the smallest categories of jets observed in the EUV. Their rising velocities and lifetimes make them plausible candidates for being the EUV counterpart of TII spicules. As with many of the jets discussed here, they have been observed to have parabolic paths that are non-ballistic. Curiously, just as was found with mottles and dynamic fibrils, there is a correlation between the initial velocities and decelerations of the jets, which indicates that macrospicules are driven by magnetoacoustic shocks, but unlike mottles and dynamic fibrils ($1-2$ minutes for chromospheric jets), the shock would require a longer period of $10\pm 2$ minutes \citep{Loboda2019ApJ871230L}. This ultimately suggests that macrospicules have a different set of formation conditions than their chromospheric cousins.
%------------------------------------------------------------------------------
\subsection{Summary of Spicular Jets}
\label{subsec:jet_table}
%------------------------------------------------------------------------------
We have collected together the basic properties of spicular jets stated through this sections in Table~\ref{solar_jet_table}. This is to serve as quick reference when comparing the simulated jets with the observational properties.  
%\afterpage{%
\begin{landscape}% Landscape page
\begin{table*}
\caption{Values based on cited papers throughout Section~\ref{sec:spicule-jets}}
\label{solar_jet_table}
\begin{center}
\begin{tabular}{|l|l|l|l|l|p{1.7cm}|}
\hline
\textbf{Feature} & \multicolumn{1}{c|}{\textbf{Obs.}} & \multicolumn{1}{c|}{\textbf{Life-time (s)}} & \multicolumn{1}{c|}{\textbf{Velocity ($\rm{km \ s^{-1}}$)}} & \multicolumn{1}{c|}{\textbf{Height (Mm)}} & \multicolumn{1}{c|}{\textbf{Driver}} \\ \hline

Spicule TI  & limb &   \multicolumn{1}{c|}{$180-720$} & \multicolumn{1}{c|}{$20-40$} &  \multicolumn{1}{c|}{$4-12$}  & \multicolumn{1}{c|}{shock} \\ \hline

Spicule TII & limb & \multicolumn{1}{c|}{$ 10-150$} & \multicolumn{1}{c|}{$ 50-150$} & \multicolumn{1}{c|}{$5-20$}  & \multicolumn{1}{c|}{reconnection}  \\ \hline

Dynamic fibrils & \multicolumn{1}{c|}{disk} & \multicolumn{1}{c|}{$120-650$} & \multicolumn{1}{c|}{$10-30$} & \multicolumn{1}{c|}{$0.4-5.2$} & \multicolumn{1}{c|}{shock} \\ \hline

Mottles & \multicolumn{1}{c|}{disk} & \multicolumn{1}{c|}{$120-900$} & \multicolumn{1}{c|}{$10-30$} & \multicolumn{1}{c|}{$2-10$} & \multicolumn{1}{c|}{shock} \\ \hline

Macrospicule & limb  & \multicolumn{1}{c|}{$180-2700$} & \multicolumn{1}{c|}{$10-150$}  & \multicolumn{1}{c|}{$7-70$} & \multicolumn{1}{c|}{shock} \\ \hline

RREs/RBEs & \multicolumn{1}{c|}{disk}  & \multicolumn{1}{c|}{$20-120$} & \multicolumn{1}{c|}{$10-50$}  & \multicolumn{1}{c|}{$1-4.5$} & \multicolumn{1}{c|}{reconnection} \\ \hline
\end{tabular}
\end{center}
\end{table*}
\end{landscape}
%}
%---------------------------------------------
\section{Magnetohydrodynamic Equations}
\label{section:MHD_eqs}
%--------------------------------------------
One useful method for modelling plasma flows (i.e. a magnetic fluid) is by using a combination of the Navier–Stokes equations of fluid dynamics with Maxwell's equations of electromagnetism, which are the foundation of the magnetohydrodynamic (MHD) equations. An interesting property of the MHD equations is they are scale-independent, hence they are equally suitable for describing laboratory plasma for example in tokamaks ($\sim 20~\rm{m}$), and for astrophysical plasmas such as accretion disk of an active galactic nucleus ($\sim 10^{21}~\rm{m}$) \citep{goedbloed2004principles}. MHD theory holds if the typical time (length) scale is greater than the ion gyroperiod (gyroradius) and mean free path time (length), and the plasma velocities are not relativistic \citep{priest2014magnetohydrodynamics}. The MHD equations shown will be ideal form. This is when we assume that the plasma is perfectly conducting, non-resistive, non-viscous, no radiation and we also utilise the ideal gas law ($p = \rho \rgas T/\Tilde{\mu}$), where $p$ is the gas pressure, $\rho$ is the density, $\rgas=8.3\times 10^{3}~\rm{J~K^{-1}~kg^{-1}}$ is the gas constant, and $\Tilde{\mu}=1$ is the mean atomic weight for a neutral plasma.
% ideal gas assumptions
% 1. no intermolecular force
% 2. No Volume
% 3. Collisions perfectly elastic 
%---------------------------------------------
\subsection{Continuity Equation}
\label{section:cont_eq}
%--------------------------------------------
The mass continuity equation that physically represents that matter cannot be created or destroyed is given by,
\begin{equation}\label{eq86}
\frac{\partial \rho}{\partial t} = - \bn \cdot (\rho \boldsymbol{v}),
\end{equation}
where $\bs{v}=(v_x,v_y,v_z)$ is the plasma velocity and $t$ is time. 
%---------------------------------------------
\subsection{Momentum Equation}
\label{section:cont_eq}
%--------------------------------------------
The momentum equation describes the balance between acceleration and pressure/tension forces, as well as any other forces acting in the system, such as gravity, as shown by,
\begin{equation}\label{eq87}
\rho \frac{d \boldsymbol{v}}{dt} = - \nabla p + \frac{1}{\mu_0} (\bn \times \boldsymbol{B}) \times \boldsymbol{B} + \rho \boldsymbol{g},
\end{equation}
where $\bs{B}=(B_x,B_y,B_z)$ is the magnetic field strength, $\mu_0=4 \pi \times 10^{-7}~\rm{Hm^{-1}}$ is the magnetic permeability of a vacuum, and $\bs{g} = (0,0,g_z)$ is the gravitational acceleration. The Lorentz force is a perpendicular force to the magnetic field lines that is a combination of two forces that are created by the magnetic field,
\begin{equation}\label{lor_for} 
    \frac{1}{\mu_0}(\bs{\nabla}\times \bs{B})\times \bs{B} = \frac{1}{\mu_0}( \bs{B} \cdot \bs{\nabla}) \bs{B} - \nabla\left(\frac{ \bs{B}^2}{2\mu_0}\right).
\end{equation}
The two forces on the right hand side of \eref{lor_for} are the magnetic tension and magnetic pressure. The magnetic tension acts to straighten out curved magnetic fields lines, and the magnetic pressure acts to push magnetic fields apart.    %---------------------------------------------
\subsection{Energy Equation}
\label{section:cont_eq}
%--------------------------------------------
The energy equation is given by, 
\begin{equation}\label{eq88}
\frac{D}{D t} \left( \frac{p}{\rho^{\gamma}} \right) = 0,
\end{equation}
where $\gamma=5/3$ is the ratio of specific heats and $D/Dt = \partial/\partial t+\bn \cdot$ is the material derivative, which gives the rate of change when the frame of reference is moving with the fluid flow.
%---------------------------------------------
\subsection{Induction Equation}
\label{section:cont_eq}
%--------------------------------------------
The induction equation describes the temporal evolution of the magnetic field, with respect to the advection and dissipation of the magnetic field,
\begin{equation}\label{eq89}
\frac{\partial \boldsymbol{B}}{\partial t} = \bn \times (\boldsymbol{v} \times \boldsymbol{B}) - \frac{1}{\mu_0} \nabla \times (\eta \bn \times \boldsymbol{B}).
\end{equation}
where $\eta= \mu_0 \sigma_c$ is the magnetic diffusivity and $\sigma_c$ is the electrical conductivity. The coupled behaviour of the magnetic field and plasma is determined by the ratio of the two terms on the right hand side, defined as the magnetic Renolds number,
\begin{equation}
    R_m = \frac{\bn \times (\boldsymbol{v} \times \boldsymbol{B})}{\eta \nabla \bs{B}}=\frac{l_0 V_0}{\eta},
\end{equation}
where $l_0$ ($v_0$) are typical length (velocity) scales. For structures on the Sun, the typical length scales tend to be large in comparison to the magnetic diffusivity, which results in a large magnetic Reynolds number. This has important implications on coronal heating, as it impedes the dissipation of waves in the corona. For plasma in the corona and in some parts of the chromosphere, the ideal limit is applicable and we can assume that we have a perfectly conducting plasma ($R_m\to \infty$) which modifies the induction equation to
\begin{equation}\label{perf_induct}
\frac{\partial \boldsymbol{B}}{\partial t} = \bn \times (\boldsymbol{v} \times \boldsymbol{B}).
\end{equation}
This large $R_m$ has important implications for the motion of the plasma and the magnetic field lines. This is because Alfv\'{e}n's Frozen Flux Theorem shows that the magnetic field moves with the plasma \citep{priest2014magnetohydrodynamics}. The magnetic flux through the surface S bounded by simple connected curve C is given by,
\begin{equation}
    \psi = \iint_s \bs{B} \cdot d\bs{S}.
\end{equation}
The magnetic flux can change in two ways, it can vary temporally with $S$ held fixed or can change by the advection of the plasma bounded by $C$. These changes are mathematically represented by,
\begin{equation}
    \frac{d \psi}{ dt} = \iint \frac{\partial \bs{B}}{ \partial t} \cdot d\bs{S} + \oint_c  \bs{B} \cdot \bs{v} \times d\bs{l},
\end{equation}
where $d\bs{l}$ is a line element parallel to curve $C$. By using Stokes theorem, one obtains
\begin{equation}
    \frac{d}{dt} \iint_S \bs{B} \cdot d \bs{S} = \iint_S \left( \frac{\partial \boldsymbol{B}}{\partial t} - \bn \times (\boldsymbol{v} \times \boldsymbol{B})  \right) \cdot d\bs{S} = 0, 
\end{equation}
which is zero due to \eqref{perf_induct}. This implies that no matter what the plasma does, it is moving around with the same number of field lines going through the surface which encompass the same number of plasma particles. The magnetic field lines are frozen into the plasma, therefore if the plasma moves the magnetic field lines have to move with it, e.g. if the plasma expands so too will the field lines.
%---------------------------------------------
\subsection{Solenoidal Constraint}
\label{section:cont_eq}
%--------------------------------------------
The solenoidal constraint implies that there are no magnetic monopoles or sinks of the magnetic field,
\begin{equation}\label{eq90}
\nabla \cdot \boldsymbol{B} = 0.
\end{equation}
%---------------------------------------------
\subsection{Overview of MHD Waves}
\label{section:mhd_Waves}
%--------------------------------------------
Spicules are thought to be able to act as waveguides in the solar atmosphere, and the dynamics that we encounter later in this thesis may have wave-based origins. To understand the basic wave behaviour of an ideal plasma, we first consider a homogeneous plasma with no gravity, with a uniform vertical magnetic field $\bs{B}=(0,0,B_0)$, in pressure equilibrium, with no initial flow. To study the waves in this medium, we first linearise the ideal MHD equations to study small perturbations (subscript $1$) from the plasma equilibrium (subscript $0$),
\begin{equation} \label{eq91}
\boldsymbol{B} = \boldsymbol{B}_0 + \boldsymbol{B}_1 (\boldsymbol{r},t) , \ \ \boldsymbol{v}_1 = \boldsymbol{v}_1 (\boldsymbol{r}, t) , \ \ \rho = \rho_0 + \rho_1 ( \boldsymbol{r},t) , \ \ p = p_0 + p_1 ( \boldsymbol{r}, t) ,
\end{equation}
where $\bs{r} = (x,y,z)$ is the position vector. Applying these perturbations to the ideal MHD equations and neglecting terms for quadratic order and higher, the perturbations yield,
\begin{equation}\label{mhd_lin1}
\frac{\partial \rho_1}{\partial t} = \rho_0 ( \bn \cdot \boldsymbol{v}_1),
\end{equation}
\begin{equation}
\rho_0 \frac{\partial \boldsymbol{v}_1}{\partial t}  = - \nabla p_1 + \frac{1}{\mu_0} (\nabla \times \boldsymbol{B}_1) \times \boldsymbol{B}_0,
\end{equation}
\begin{equation}
\frac{\partial p_1}{\partial t} = c^2_{s0}  \frac{\partial \rho_1}{ \partial t},
\end{equation}
\begin{equation}
\frac{\partial \boldsymbol{B}_1}{\partial t} = \nabla \times (\boldsymbol{v}_1 \times \boldsymbol{B}_0),
\end{equation}
\begin{equation}\label{mhd_lin2}
\nabla \cdot \boldsymbol{B}_0 = \nabla \cdot \boldsymbol{B}_1 = 0 ,
\end{equation}
where $c_{s0} = \sqrt{\gamma p_0/ \rho_0}$ is the equilibrium sound speed. A general wave equation can be obtained by combining the linearised ideal MHD equations, namely,
\begin{equation}\label{wave_eq}
    \frac{\partial^2 \bs{v}_1}{\partial t^2} = c_{s0}^2 \bn (\bn \cdot \bs{v}) + \frac{1}{\mu \rho_0} (\bn \times (\bn \times (\bs{v}\times\bs{B_0}))) \times \bs{B_0}.
\end{equation}
As we are looking for a wave-like solution, we assume a solution will have the form,
\begin{equation}\label{wave_plane_sol}
    \bs{f}(\bs{r},t) = \hat{f} e^{i( \bs{k}\cdot \bs{r} - \omega t)},
\end{equation}
where $f$ is a placeholder of any perturbed quantity (in this case $\bs{v}$), $\hat{f}$ is the amplitude of each perturbation, $\bs{k}=(k_x,0,k_z)$ is the wave vector ($k_y=0$ for simplicity), and $\omega$ is the angular frequency. Applying \eref{wave_plane_sol} to the wave equation \eref{wave_eq}, one can obtain a dispersion relation,
\begin{equation}\label{disp_rel}
 (\omega^2-k^2_z V_{A0}^2)(\omega^4-\omega^2k^2(c_{s0}+V_{A0})+k^2 k^2_z c_{s0}^2 V_{A0}^2) = 0,
\end{equation}
where $V_{A0}=B_0/\sqrt{\mu \rho_0}$ is the equilibrium \Alfven speed. This dispersion relation contains solutions for three distinct wave modes. The first solution is obtained from the first term, $\omega=\pm k_z V_{A0}$, which is the forward and backward propagating \Alfven waves. \Alfven waves are purely transverse magnetic waves that propagate along the magnetic field lines. The second and third set of solutions are,
\begin{equation}\label{MA_waves}
    \omega^2 = \frac{1}{2} k^2{c_{s0}^2+V_{A0}^2} \left( 1 \pm \sqrt{1-4c^2_T \frac{ k^2_z}{k^2}}  \right),
\end{equation}
where, 
\begin{equation}
  c_{T0} = \frac{c_{s0}V_{A0}}{\sqrt{c_{s0}^2+V_{A0}^2}}, 
\end{equation}
is the tube speed. If we do not consider any special cases then each solution for \eqref{MA_waves} gives the forward and backwards fast (taking the $+$ of the last term) and slow (taking the $-$ of the last term) magnetoacoustic (MA) waves. Hence, these wave modes are a combination of magnetic and sound waves, and therefore the restoring forces are a combination of both pressure and Lorentz force. The mixed properties of these waves illustrate that the wave behaviour is heavily affected by $\beta$. If we consider a special case of $\beta \gg 1$, and can hence omit any magnetic terms, i.e. $V_{A0}=0$, then one obtains the forward and backward propagating sound waves from \eref{MA_waves}, $\omega = \pm k c_{s0}$. The sound waves are only possible if the fluid is compressible and they are longitudinal waves. If the environment is uniform in temperature then they will propagate isotropically. While these systems of equations give a description of a homogeneous medium, these waves cause different behaviours when confined inside a waveguide.
%---------------------------------------------
\subsection{MHD waves in a Slab}
\label{section:Tan_I}
%--------------------------------------------
The simulation carried out in this thesis is 2D. Therefore the synthetic spicular structures initiated may host boundary deformations that are akin to those seen in slab models. A slab model has regions of different plasma properties sitting next to each other in a Cartesian geometry. There have been numerous studies on the wave behaviour of slabs \citep{Roberts1981SoPh6939R, Edwin1982SoPh76239E, Roberts1990IAUS142159R, Goossens1992SoPh138233G, Goossens2009AA503213G, Murawski2015AA577A126M, Allcock2017SoPh29235A, zsamberger2021ApJ906122Z}. We will focus on an early model given by \cite{Roberts1981SoPh6939R} and briefly outline the main steps in their derivation. \cite{Roberts1981SoPh6939R} introduce a magnetic slab setting in the same state as outlined earlier in this Section~\ref{section:mhd_Waves}, bar a change in the magnetic field configuration,
%-------------------------------
\begin{equation}
    B_0(x) = 
    \begin{cases}
      B_i, & \text{if } |x| \leq x_0, \\
      0, & \text{if } |x| > x_0,
    \end{cases}
\end{equation}
where $\pm x_0$ is the location of the boundaries of the magnetic slab and $i$ ($e$) denote the parameters internal (external) to the magnetic slab. In this setup, by combining Eqs.~\eqref{mhd_lin1} - \eqref{mhd_lin2} and assuming a plane wave solution ($f(x)=\hat{f} e^{i(kz-\omega t)}$), one can obtain a wave equation of the form,
\begin{equation}\label{eva_trapp}
    \frac{d^2 \hat{v}_x}{dx^2}-m^2_{i,e} \hat{v}_x = 0,
\end{equation}
and,
\begin{equation}
    m_{i,e}^2 = \frac{(k_z^2 c_{si,se}^2-\omega^2)(k_z^2 V_{Ai,Ae}^2-\omega^2)}{(c_{si,se}^2+V_{Ai,Ae}^2)(k_z^2 c_{Ti,Te}^2-\omega^2)}.
\end{equation}
The solutions of \eref{eva_trapp} have a different functional form depending on the sign of $m^2_{i,e}$. Solutions with $m^2_{i,e}>0$ are trapped by the waveguide and are known as trapped modes. Whereas, solutions with $m^2_{i,e}<0$ leak some wave energy laterally and these are known as leaky modes. As we are not focusing on searching for solutions for the leaky waves, we discuss the trapped waves, which have a solution of the form,
\begin{equation}
        B_0(x) = 
    \begin{cases}
      C_{10} e^{m_e (x+x_0)}, &  \text{if } x < -x_0, \\
      C_{00} \cosh{(m_i x)} + C_{01} \sinh{(m_i x)}, & \text{if } |x| < x_0, \\
      C_{11} e^{-m_e(x-x_0)}, & \text{if } x > x_0,
    \end{cases}
\end{equation}
where $C_{jl}$ for $j=0,1$ and $l=0,1$ are constants. By applying the boundary conditions at the interface to the system of MHD equations one can arrive at the dispersion relation,
\begin{equation}\label{disp_saus_kink}
    D_k(\omega)D_s(\omega)  = 0,
\end{equation}
where,
\begin{equation}\label{kw_eq}
    D_k(\omega) = \rho_em_i(\omega_\textrm{Ae}^2 - \omega^2)\coth{(m_ix_0)} + \rho_im_e(\omega_\textrm{Ai}^2 - \omega^2).
\end{equation}
\begin{equation}\label{sw_eq}
    D_s(\omega) = \rho_em_i(\omega_\textrm{Ae}^2 - \omega^2)\tanh{(m_ix_0)} + \rho_im_e(\omega_\textrm{Ai}^2 - \omega^2),
\end{equation}
The solutions for \eref{disp_saus_kink} lead to two separate waves modes, known as the kink wave (from $D_k =0$) and sausage wave (from $D_s =0$). The trigonometric functions in Eqs.~\eqref{kw_eq} and \eqref{sw_eq} are a key component that determines whether the boundaries of the slab move in (kink wave) or out (sausage wave) of phase, as shown in Fig.~\ref{KW_SW}. \np
%
There are numerous reports of all these MHD waves in the solar atmosphere \citep{Tomczyk2007Sci3171192T, Tomczyk2009ApJ6971384T, jess_alfven_2009, Morton2012NatCo31315M}. By studying these waves one can provide diagnostics of the plasma properties that we otherwise can find challenging to measure directly, such as density \citep{Verwichte_2013A_A}, magnetic field strength \citep{Nakariakov_2001}, and temperature \citep{De_Moortel_2003SoPh}. This is known as solar magneto-seismology. Of particular interest to this thesis is MHD waves potential to deform the boundary of spicules and that these wave modes have been reported to occur in spicules \citep{Kukhianidze2006AA, Okamoto2011ApJ736L24O, Jess2012ApJ744L5J, Verth2016GMS216431V, Sharma2018ApJ85361S}.
%fffffffffffffffffffffff
\mfig{0.75}{figures/KW_SW.png}{Example of the effect of the sausage (left) and kink (right) waves on MFT. This image is extracted from \cite{Morton2012NatCo31315M}.}{KW_SW}
%fffffffffffffffffffffff
%---------------------------------------------
\section{Outline of Thesis}
\label{section:outline}
The thesis aims to investigate the fundamental dynamics of spicular jets by applying simple models. Each chapter will contain the following:
\begin{itemize}
    \item {\bf{Chapter 2}}: An overview is given on the current state of numerical models of spicules. We outline the software we use and detail our numerical setup, which is applied throughout the thesis.
    \item {\bf{Chapter 3}}: We study synthetic jets in a simplified solar atmosphere. We launch vertical jets with a momentum pulse. By changing multiple parameters linked to the driver (length and initial velocity amplitude) and the magnetic field strength, we investigate how the jet height, cross-sectional width variations, and morphology are impacted by the parameter space.
    \item {\bf{Chapter 4}}: In this chapter, we add increasing degrees of tilt from the vertical to see how this impacts the jet dynamics and morphology. A focus parameter space is used to investigate, and an improved jet tracking software is deployed to measure the heights and cross-sectional width variations.     
    \item {\bf{Chapter 5}}: The cross-sectional width variations identified in Chapter 3 and 4 are studied further by comparison with observations. We postulate the cross-sectional width variations in synthetic jets and observations as standing waves, and apply a wavelet analysis. We also make a comparison of the knots observed in the jet beams of the synthetic jets with blobs reported in observations of larger scale solar jets.   
    \item {\bf{Chapter 6}}: A summary of the main results and outline the future steps one could take to build upon the research described in the thesis.  
\end{itemize}
%--------------------------------------------