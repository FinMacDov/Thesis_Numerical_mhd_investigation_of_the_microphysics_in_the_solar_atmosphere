
\chapter{The Maximum Growth Rate}
\label{sec:appendix}

As we have already stated before, the characteristic exponent, $\mu$, is determined by the equation
%
\begin{equation}
\label{eq:A1}
\cosh(\pi \mu) = \bar\eta(\pi),
\end{equation}
%
where $\bar\eta(\tau)$ is the solution to the initial value problem to Equation~\eqref{eq:mathieu} with
%
\begin{equation}
\label{eq:A2}
\bar\eta = 1, \quad \frac{\mathrm{d} \bar\eta}{\mathrm{d} t} = 0 \quad 
\mbox{at} \;\; \tau = 0,
\end{equation}
%
\citep{Abramowitz1965}.
When a perturbation is unstable, its growth rate is given by $\gamma = \Re(\mu)$.
In the context of the $\sigma$-stability analysis, we assume that the growth time of the instability is much larger than the oscillation period.
In terms of dimensionless quantities, this condition is written as $\gamma \ll 1$.
The numerical investigation shows that this condition is only satisfied for all values of $q$ when $K \gg 1$.
In accordance with this, we introduce the small parameter $\epsilon = 1 / K$\/.
Figure~\ref{fig:stability_diagram} shows that $a$ is close to $j^2$ on parts of the line $a = Kq$ corresponding to unstable perturbations when $K \gg 1$, where $j = 1,2,\dots$.
We obtain $a = j^2$ taking $q = j^2\epsilon$\/, which implies that $q = {\cal O}(\epsilon)$.

First we study the case with $j = 1$. Using the expansion valid for small $q$ \citep{Abramowitz1965},
%
\begin{equation}
\label{eq:A3}
a_1(q) = 1 + q + \mathcal{O}(q^2), \quad b_1(q) = 1 - q + \mathcal{O}(q^2),
\end{equation}
%
we obtain that the line $a = Kq$ in Figure~\ref{fig:stability_diagram} intersects the curves $a = b_1(q)$ and $a = a_1(q)$ at $q \approx \epsilon - {\cal O}(\epsilon^2)$ and $q \approx \epsilon + {\cal O}(\epsilon^2)$, respectively.
Then $q = \epsilon + \bar q\epsilon^2$ on the part of the curve $a = Kq$ between the intersection points, where $\bar q$ is a free parameter varying from approximately $-1$ to approximately 1.
It follows that $q = \epsilon + \bar q \epsilon^2$ on the line $a = K q$ between the intersection points, where $\bar q$ is a free parameter.
The equation of the curve $a = Kq$ is now rewritten as $a = 1 + \bar q\epsilon$\/, and Equation \eqref{eq:mathieu} becomes
%
\begin{equation}
\frac{\mathrm{d}^2 \eta}{\mathrm{d}\tau^2} + [1 + \bar q\epsilon
- 2(\epsilon + \bar q\epsilon^2)(\cos(2 \tau)] \eta = 0.
\label{eq:A4} 
\end{equation}
%
To calculate the increment we need to find the solution $\bar\eta(\tau)$ to this equation satisfying the initial conditions Equation~\eqref{eq:A2}.
To do this we use the regular perturbation method with
%
\begin{equation}
\bar\eta = \bar\eta^{(0)} + \bar\eta^{(1)} + \bar\eta^{(2)} + \dots.
\label{eq:A5} 
\end{equation}
%
Substituting Equation~\eqref{eq:A5} into Equations~\eqref{eq:A2} and \eqref{eq:A4}, and collecting the terms of the order of unity, we obtain
%
\begin{equation}
\frac{\mathrm{d}^2\bar\eta^{(0)}}{\mathrm{d}\tau^2} + \bar\eta^{(0)} = 0,
\label{eq:A6} 
\end{equation}
%
and the associated initial conditions
%
\begin{equation}
\bar\eta^{(0)} = 1, \quad \frac{\mathrm{d} \bar\eta^{(0)}}{\mathrm{d} \tau} = 0
\quad \mbox{at} \;\; \tau = 0.
\label{eq:A7} 
\end{equation}
%
The solution to this initial value problem is
% 
\begin{equation}
\bar\eta^{(0)} = \cos\tau.
\label{eq:A8}
\end{equation}
%
Collecting term of the order of $\epsilon$ yields
%
\begin{equation}
\frac{\mathrm{d}^2\bar\eta^{(1)}}{\mathrm{d}\tau^2} + \bar\eta^{(1)} = 
   [2\cos(2\tau) - \bar q]\cos\tau ,
\label{eq:A9} 
\end{equation}
% 
\begin{equation}
\bar\eta^{(1)} = 0, \quad \frac{\mathrm{d}\bar\eta^{(1)}}{\mathrm{d}\tau} = 0 
   \quad \mbox{at} \;\; \tau = 0.
\label{eq:A10}
\end{equation}
%
After straightforward calculation we obtain
% 
\begin{equation}
\bar\eta^{(1)} = \frac{1 - \bar q}2\tau\sin\tau - \frac18\cos(3\tau) + \frac18\cos\tau.
\label{eq:A11}
\end{equation}
%
Finally we collect terms of the order of $\epsilon^2$ to obtain
%
\begin{equation}
\frac{\mathrm{d}^2\bar\eta^{(2)}}{\mathrm{d}\tau^2} + \bar\eta^{(2)} = 
   [2\cos(2\tau) - \bar q]\eta_1^{(1)} + 2\bar q\cos(2\tau)\cos\tau,
\label{eq:A12} 
\end{equation}
% 
\begin{equation}
\bar\eta^{(2)} = 0, \quad \frac{\mathrm{d}\bar\eta^{(2)}}{\mathrm{d}\tau} = 0 
   \quad \mbox{at} \;\; \tau = 0.
\label{eq:A13}
\end{equation}
%
The solution to this initial value problem is given by
% 
\begin{eqnarray}
\bar\eta^{(2)} &=& \frac{1 - \bar q^2}8\tau^2\cos\tau + 
   \frac{2\bar q^2 + 7\bar q -2}{16}\tau\sin\tau - 
   \frac{1 - \bar q}{16}\tau\sin(3\tau) \nonumber\\ 
&+& \frac{\cos(5\tau)}{192} - \frac{2 + 3\bar q}{32}\cos(3\tau) + 
   \frac{11 + 18\bar q}{192}\cos\tau.
\label{eq:A14}
\end{eqnarray}
%
Using Equations~\eqref{eq:A8}, \eqref{eq:A11}, and \eqref{eq:A14} we obtain
% 
\begin{equation}
\bar\eta(\pi) = -1 - \frac{1 - \bar q^2}8\pi^2\epsilon^2 + {\cal O}(\epsilon^3).
\label{eq:A15}
\end{equation}
%
It follows from this equation that 
% 
\begin{equation}
\mu = i \pm \frac\epsilon2\sqrt{1 - \bar q^2} + {\cal O}(\epsilon^2).
\label{eq:A16}
\end{equation}
%
This result implies that
% 
\begin{equation}
\gamma = \frac\epsilon2\sqrt{1 - \bar q^2} + {\cal O}(\epsilon^2), \quad
   \gamma_m = \frac\epsilon2 ,
\label{eq:A17}
\end{equation}
%
where $\gamma_m$ is the maximum value of the instability increment when the point $(a,q)$ is on the part of line $a = Kq$ that is between the curves $a = b_1(q)$ and $a = a_1(q)$.

Now we consider the part of line $a = Kq$ that is between the curves $a = b_j(q)$ and $a = a_j(q)$, $j = 2,3,\dots$ For $q \ll 1$ we have $b_1(q) = n^2 + {\cal O}(q^2)$ and $a_1(q) = n^2 + {\cal O}(q^2)$ \citep{Abramowitz1965}. Since $K = \epsilon^{-1}$\/, it follows that $q = n^2\epsilon(1 + \bar q\epsilon^2)$ and $a = n^2(1 + \bar q\epsilon^2)$, where $\bar q$ is again a free parameter. Substituting these expressions in Eq.~(\ref{eq:A1}) we transform it to
%
\begin{equation}
\frac{\mathrm{d}^2\eta}{\mathrm{d}\tau^2} + j^2[1 + \bar q\epsilon^2 - 
   2(\epsilon + \bar q\epsilon^3)(\cos(2 \tau)] \eta = 0.
\label{eq:A18} 
\end{equation}
%
Then we again look for the solution in the form of the expansion given by Eq.~(\ref{eq:A5}). Substituting this expansion in Equations \eqref{eq:mathieu} and \eqref{eq:A2}, and collecting terms of the order of unity we obtain
%
\begin{equation}
\frac{\mathrm{d}^2\bar\eta^{(0)}}{\mathrm{d}\tau^2} + j^2\bar\eta^{(0)} = 0,
\label{eq:A19} 
\end{equation}
% 
\begin{equation}
\bar\eta^{(0)} = 1, \quad \frac{\mathrm{d}\bar\eta^{(0)}}{\mathrm{d}\tau} = 0 
   \quad \mbox{at} \;\; \tau = 0.
\label{eq:A20}
\end{equation}
%
The solution to this initial value problem is
% 
\begin{equation}
\bar\eta^{(0)} = \cos(j\tau).
\label{eq:A21}
\end{equation}
%
Collecting terms of the order of $\epsilon$ yields
%
\begin{equation}
\frac{\mathrm{d}^2\bar\eta^{(1)}}{\mathrm{d}\tau^2} + j^2\bar\eta^{(1)} = 
   2j^2\cos(2\tau)\cos(j\tau) ,
\label{eq:A22} 
\end{equation}
% 
\begin{equation}
\bar\eta^{(1)} = 0, \quad \frac{\mathrm{d}\bar\eta^{(1)}}{\mathrm{d}\tau} = 0 
   \quad \mbox{at} \;\; \tau = 0.
\label{eq:A23}
\end{equation}
%
After straightforward calculation we obtain
% 
\begin{equation}
\bar\eta^{(1)} = 1 - \frac13\cos(4\tau) - \frac23\cos(2\tau)
\label{eq:A24}
\end{equation}
%
for $j = 2$, and 
% 
\begin{equation}
\bar\eta^{(1)} = \frac{j^2\cos[(j-2)\tau]}{4(j - 1)} - 
   \frac{j^2\cos[(j+2)\tau]}{4(j + 1)} - \frac{n^2\cos(j\tau)}{2(j^2 - 1)}
\label{eq:A25}
\end{equation}
%
for $j > 2$. Collecting terms of the order of $\epsilon^2$ we obtain
%
\begin{equation}
\frac{\mathrm{d}^2\bar\eta^{(2)}}{\mathrm{d}\tau^2} + \bar\eta^{(2)} = 
   2j^2\bar\eta^{(1)}\cos(2\tau) - j^2\bar q\cos(j\tau),
\label{eq:A26} 
\end{equation}
% 
\begin{equation}
\bar\eta^{(2)} = 0, \quad \frac{\mathrm{d}\bar\eta^{(2)}}{\mathrm{d}\tau} = 0 
   \quad \mbox{at} \;\; \tau = 0.
\label{eq:A27}
\end{equation}
%
The solution to this initial value problem is given by
% 
\begin{equation}
\bar\eta^{(2)} = \left(\frac53 - \bar q\right)\tau\sin(2\tau) + 
   \frac{\cos(6\tau)}{24} + \frac29\cos(4\tau) + 
   \frac{29}{72}\cos(2\tau) - \frac23
\label{eq:A28}
\end{equation}
%
for $j = 2$, and by
% 
\begin{eqnarray}
\bar\eta^{(2)} &=& \frac j4\left(\frac{j^2}{j^2 - 1}\right)\tau\sin(j\tau) + 
   \frac{j^4\cos[(j+4)\tau]}{32(j+1)(j+2)} + 
   \frac{j^4\cos[(j+2)\tau]}{8(j+1)(j^2 - 1)}\nonumber\\ 
&-& \frac{j^4(j^4 - 3j^2 + 16)\cos(j\tau)}{16(j^2 - 1)^2(j^2 - 4)} -
   \frac{j^4\cos[(j-2)\tau]}{8(j-1)(j^2 - 1)} + \frac{j^4\cos[(j+4)\tau]}{32(j-1)(j-2)}
\label{eq:A29}
\end{eqnarray}
% 
for $j > 2$. Using Eqs.~(\ref{eq:A21}), (\ref{eq:A24}), (\ref{eq:A25}), (\ref{eq:A28}), and (\ref{eq:A29}), we obtain
% 
\begin{equation}
\bar\eta(\pi) = (-1)^n + {\cal O}(\epsilon^3).
\label{eq:A30}
\end{equation}
%
It follows from this equation that $\mu = {\cal O}(\epsilon^3)$ for even $j$ and $\mu = i + {\cal O}(\epsilon^3)$ for odd $j$\/, and thus $\gamma = {\cal O}(\epsilon^{3/2})$, that is $\gamma \ll \gamma_m$\/.
Hence, $\gamma_m = 1/2K$ is the maximum value of the instability increment with respect to $q$ when $K = \epsilon^{-1}$\/. 