\chapter{Signatures of Flow-modulated Sausage Wave Modes in Solar Spicules}
\section{Introduction}
Spicules are thin, grass-like features that are routinely observed in the lower solar atmosphere as dark (on-disk) or bright (at limb) structures, primarily in cool/chromospheric spectral lines (e.g., H$\alpha$, Ca \textsc{ii} H \& K, He \textsc{i} D3). These localised features effectively trace the chromospheric magnetic field concentrations in spite of the likely role of the neutrals at spicular temperatures and are a focus of many studies for their potential role in balancing the radiative losses and heating at the chromosphere-corona interface \citep{Tsiropoula2012}. Observations suggest that spicules generally emanate near the (inter)granulation lanes and remain subject to magnetohydrodynamic stresses (MHD) at their footpoint. The associated motions, irrespective of whether turbulent or coherent, become channelized through the spicule structure and are well reflected in their observed dynamic behavior. Broadly, spicule kinematics can be categorized in three major domains, viz., radially transverse, field-aligned and torsional, which have been extensively examined and interpreted in terms of discrete MHD wave modes \citep[see review:][]{Zaqarashvili2009}. 

Ubiquitous transverse displacements in spicule features were interpreted as the $m$ = 1, kink MHD wave mode by previous studies, e.g.  \citep{Kukhianidze2006, DePontieu2007, Ebadi2014, Sharma2017}.  Similar dynamic behavior is also found prevalent in on-disk spicule-counterparts, commonly known as Rapid Blue/Red Shifted Excursions \citep[RBE,RREs:][]{RouppevanderVoort2009}. Field-aligned motions are primarily associated with mass flows, and common to many jet-like phenomenon, were reported in mottles \citep{Loughhead1974}, on-disk RBE/RREs \citep{Sekse2013} and off-limb spicules \citep{Pereira2012}. Axisymmetric torsional (rotationally transverse) motions in spicules were claimed to be observed in terms of $m$ = 0 torsional Alfv\'en wave mode by \citet{DePontieu2012}, however, it must be noted that the observations used to make this particular mode identification could also be due to perturbations in ambient chromospheric environment by spicule motions \citep{Goossens2014, Sharma2017}. 

Though much attention was given to the identification and study of transverse (kink and torsional Alfv\'en) wave modes, the observations of cross-sectional width variations remained elusive, largely due to pertinent difficulties with observations of jet-like chromospheric features. The presence of cross-sectional width variations in thin magnetic flux tube structures, as a possible consequence of $m$ = 0 MHD sausage and/or $m$ = 2 fluting modes, were postulated in earlier theoretical studies \citep{Ziegler1997a, Ziegler1997, Ruderman2010}. A few observations of periodic cross-sectional width variations were limited to on-disk fibril structures. Concurrent transverse and cross-sectional width variations were reported as coupled kink and sausage wave modes by \citet{Jess2012} and \citet{Morton2012}. Similar observations for cross-sectional width and intensity oscillations in on-disk slender chromospheric fibrils were reported by \citet{Gafeira2017} and interpreted as MHD sausage wave modes. Moreover, an ensemble of coupled transverse, cross-sectional width and axisymmetric torsional motions were reported in off-limb spicules by \citet{Sharma2018}, which were interpreted as nonlinear kink wave mode.

It must, however, be noted that spicules are essentially jet-like features where mass flows along their magnetic field are predominant with bulk velocities of the range of 25-100 km/s  \citep{Beckers1972ARAA1073B,Sterling2000SoPh19679S,Pereira2012}. Such bulk velocities may even have a dominant effect on the waves that are present in spicular structures. Plasma flows will interact with the oscillatory modes of spicules, e.g. periodic motions along or anti-parallel to the bulk motion will be affected differently. Wave motions in steady waveguides have been investigated in the past in a few theoretical studies \citep{Narayanan1991,nakariakov1995, terrahomem2003, soler2008} for different (slab/cylindrical) geometries. It was concluded that mass flows can generate shift in frequencies for confined MHD waves and can influence the wave propagation even causing resonant flow instabilities, depending upon the direction and strength of mass flows. Though, possible effects of mass flows on wave periodicities/propagation were known theoretically,  it still requires observational verification in highly localised dynamic waveguides such as solar spicules.

Here, we report the first observational evidence of MHD sausage wave modes in off-limb spicule structures, identified as periodic variations in estimated cross-sectional widths. The observed dynamics were compared with MHD numerical simulations that mimic the jet evolution in a stratified atmosphere. Spectral analysis of cross-sectional width deformations in the rise and fall phases of both the observed and simulated spicule features are examined to better understand the possible role of field-aligned mass flows in modulating the observed periodicities of confined sausage wave modes.  

\begin{figure*}
\includegraphics[width=1.0\textwidth]{figures/fig_sp_a.png}
\caption{Panels (a1) – (a4) show the temporal evolution of candidate spicule feature (SP:A) in H$\alpha$ passband at four instances, with positions of vertical (cyan) and horizontal (yellow) slits used for the estimation of field-aligned mass flows and cross-sectional width respectively. Panel (b) shows the time-distance plot from vertical slit on the spicule, highlighting the rise- and fall-phases of field-aligned mass flows. The maximum height attained by the visible plasma is marked with ‘+’ symbol, along with estimated velocities (+46.9 km/s, 40.25 km/s). Bottom panel (c) shows an example of Gaussian fit for intensity magnitudes for horizontal slit location (marked as yellow line on (a1) – (a4)), with error bars, estimated as the standard deviation for intensity values. Vertical black line marks the position of the amplitude of Gaussian fit, while shaded-regions mark average/unperturbed width (W$_{u}$) during spicule lifetime and perturbed/instantaneous width (W).}
\label{fig:1} 
\end{figure*}
\begin{table*}
\caption{Summary of observed physical characteristics of candidate spicule structures along with estimated periodicities during rise-, fall- and overall-phases. Physical and spectral parameters of simulated jet structure are also provided for a comparison between observed and simulated case(s). }             
\label{table:1}      
%{\centering} 
\begin{center}
\begin{tabular}{cccccccc}
\hline
Spicule       & Lifetime & Length & Apex-height & Unperturbed Width & \multicolumn{3}{c}{Periodicity (sec)} \\ \cline{6-8} 
              & (sec)    & (Mm)   & (Mm)        & (km)              & Rise-phase  & Fall-phase  & Overall   \\ \hline
SP:A          & 161.7    & 3.8    & 5.3         & 176.0             & 27.0        & 16.0        & 17.0      \\
SP:B          & 92.4     & 2.2    & 4.8         & 134.6             & 22.5        & 27.4        & 24.7      \\
SP:C          & 100.1    & 2.8    & 4.4         & 122.8             & 41.2        & 31.3        & 48.8      \\ \hline
Simulated jet & 471      & 8.0    & 8.0         & 350.0             & 70.0        & 30.0+68.0   & 32.0+65.0 \\ \hline
\end{tabular}
\end{center}
\end{table*}
%{\centering} 
\section{Data and Analysis}

\subsection{Observational Data}

Data used in our investigation were taken by using the CRisp Imaging SpectroPolarimeter (CRISP) at the Swedish 1-m Solar Telescope \citep[SST:][]{Scharmer2008} on La Palma. The H$\alpha$ imaging-spectroscopic data of 07:15-07:48 UT, June 21, 2012, targeted an active region (AR) NOAA AR11504 with two sunspots at the limb position (heliocentric coordinates w.r.t the disk center, hereby denoted as: $\Theta$ = 893, $\Phi$ = -250). The AR was scanned using 31 equally spaced line positions with 86 m\AA, steps from -1.376 to +1.29 \AA, relative to the line center, along with the additional 4 positions in the far blue wing from -1.376 to 2.064 \AA. Following data acquisition, post-processing of data was carried out using the Multi-Object Multi Frame Blind Deconvolution \citep [MOMFBD;][]{2005} image restoration algorithm. Also, standard procedures available in the image pipeline for CRISP data \citep{2015} including differential stretching and removal of dark- and flat-fielding were implemented. 

The final science-grade data of $\sim$30 min duration had a pixel size of 0.059$\arcsec$ ($\sim$43 km), angular-resolution of 0.13$\arcsec$ ($\sim$95 km), and the cadence of 7.7 sec. Following the the Nyquist criterion, the temporal-resolution of the dataset allowed the detection of MHD wave modes with periodicities over 15.4 sec. Candidate spicule(s) for our study (Table \ref{table:1}) were identified as high-intensity structures with least possible superimposition by any other surrounding features during their visible lifetime in an observed passband. The identified spicule structures have an average lifetime of 118 s in a particular line-scan position from the H$\alpha$ line-center. It must be noted that spicules tend to evolve in multiple passbands during their much longer (500-800 s) lifetimes \citep{Pereira2014}. Further, the observed features have an average length of $\sim3\pm$.36 Mm with apex reaching up to an average height of 4.83 Mm. An example of an identified spicule is given in Figure \ref{fig:1}(a1)-(a4)), which is presented in detail for field-aligned plasma motions (Fig.\ref{fig:1}(b)) and its associated role in modulating the cross-sectional widths (Fig.\ref{fig:1}(c)). \\

\subsection{Wavelet Analysis} 
The cross-sectional width of observed spicule features were estimated using the method adopted from \citet{Sharma2018}. A single Gaussian function with linear background is used to fit intensity profile across the observed spicules, with estimated FWHM is taken as a measure of cross-sectional width of the feature. For the selected cases in our study, the average estimated FWHM corresponds to 144.4 km, at a height of $\sim$2 Mm above the visible limb.


The longitudinal and cross-sectional dynamics of numerical jet are examined by employing a tracer quantity that tracks the mass injection and associated motions \citep{Porth_2014}. A threshold on the tracer quantity filters out the simulated jet feature from the surrounding region and further helps with the identification of jet boundaries. A horizontal slice on filtered structure facilitate the estimation of width for the particular height and to track the variations over time (see the blue circular markers on  Figure~\ref{jet_P300_B50_A60}). The jet's apex is identified by selecting a single data point that belongs to the jet, with the highest vertical value (see yellow triangular markers on  Figure~\ref{jet_P300_B50_A60}). This latter allows for the temporal tracking of the numerical solar jet apex heights and the identification of its rising and falling phases. To perform wavelet analysis, the rise and falling phases are independently detrended with a quadratic polynomial. The detrended data is smoothed to reduce noise by using a moving average with a window size of 15 and 10 for the rising and falling phases, respectively.       

Cross-sectional width estimates from both observed and simulated spicule structures were further analyzed using wavelet transform, to understand the nature of periodicities and associated wave physics. To identify any significant periodicities, we constructed wavelet power spectra (WPS) and the associated global power spectrum (GPS) using a software developed by \cite{Torrence1998}. The default Morlet wavelet profile was employed with a two-$\sigma$ confidence level on cross-sectional width estimates from observations and simulations. 

\begin{figure*}
\includegraphics[width=1.0\textwidth]{figures/jet_P300_B50_A60_with_markers.eps}
\caption{Panels showing the temporal evolution of the simulated spicule density structure at four time-steps with apex marked by a yellow triangle. From $121.08-243.88 ~\rm{s}$ ($366.68-489.48~\rm{s}$)  the rising (falling)-phase Location of tracers edges are also shown as blue dots, which are used to estimate the variations in cross-sectional width during rise- and fall-phases of jet structure.}
\label{jet_P300_B50_A60} 
\end{figure*}

\section{Comparing Simulations with Observations}
To understand the role of field-aligned mass flows on MHD sausage wave modes, the periodicities in cross-sectional width estimates for an typical observed spicule (Figs.\ref{fig:1}) were compared with a simulated jet structure. The sample candidate off-limb spicule (SP:A) was observed at -1.032 $\angstrom$ from the H$\alpha$ line core with a total lifetime of $\sim$161.7 sec in the observed passband. The feature had a physical length of 3.8 Mm at an inclination of $\sim$21.4$^\circ$ over the observed limb, with apex-height reaching up to 5.3 Mm. Longitudinal motions in the Lagrangian frame were analyzed to estimate the velocity and duration of rise- and fall-phases, using a vertical slit (Fig. \ref{fig:1}(a1) – (a4)) along the axis of spicule structure. Time-distance analysis of the vertical slit suggests a parabolic profile (Fig. \ref{fig:1}(b)) of mass motions with an average ascending velocity of $\sim$46.9 km/s. The plasma attains the maximum height in about 81 sec and then falls back to the surface with an average velocity of 40.2 km/s. 

Cross-sectional widths were estimated at a height of $\sim$2 Mm above the visible limb location. For the observed lifetime, the average/unperturbed cross-sectional width estimate (W$_{u}$) for the spicule structure was around $\sim$176 km, which is comparable with other reports for off-limb spicules \citep{Sharma2018}. The temporal variations in cross-sectional width ($\delta$W = W - W$_{u}$) were used to estimate the periodicities in spectral domain and further comparison with simulations. The numerical setup for our study is chosen to highlight the fundamental dynamics of a chromospheric jet in a stratified atmosphere. Though the momentum-driven simulated jet lacks complex physical mechanisms, like e.g. radiative losses, ambipolar diffusion, ion-neutral effects, etc., it still mimics the crucial kinematic behavior, such as, longitudinal/field-aligned plasma motions with cross-sectional width variations (Fig. \ref{jet_P300_B50_A60}) in both rise- and fall-phase of the jet evolution. 

The density structure of the simulated jet also shows complex internal substructures within the jet-beam, prominent during the rise-phase. Formation of these \textit{knot}-like substructures appears to be concurrent with the structural deformations in the jet, and results due to high velocities at the jet footpoint. The \textit{knot}-like features are a common ingredient to many astrophysical jets \citep{Norman1982}, and can arise due to a myriad of internal shock waves creating \textit{crisscross/knot} pattern within the jet structure. The generation of these features has been routinely demonstrated in laboratory jets \citep{Menon2010ShWav20175M,Edgington_Mitchell2014,Ono2014}. However, due to the limited spatial resolution of the current dataset, this theoretically predicted fine structure inside the spicule jet cannot be verified observationally.

Cross-sectional width estimates from both observations and numerical simulations indicate an oscillatory pattern during the feature’s lifetime (Fig. \ref{fig:3}, Sim:1.1, Obs:1.1). These periodic variations were further analyzed to estimate the dominant frequencies in the oscillations during the rise- and fall-phase of the jet evolution. Peak frequencies were identified in the wavelet analysis as spectral-magnitudes in WPS with significance levels over unity and higher, highlighted as black contours in Figures \ref{fig:3}. Furthermore, the peaks also had GPS over two-$\sigma$ (95\%) confidence levels (orange dashed-line in Fig. \ref{fig:3}). Wavelet spectra for observed width variations during spicule lifetime, suggest a strong spectral density concentration around $\sim$17 sec for 5 cycles. Interestingly, for the simulated jet structure, the WPS indicates the presence of harmonics, concentrated around $\sim$65 sec and $\sim$32 sec for 8 and 5 cycles, respectively. 

\begin{figure*}
\includegraphics[width=1.0\textwidth]{figures/fig_3.png}
\caption{Panels showing the results of spectral analysis of cross-sectional width estimates of observed (top) and simulated (bottom) jet structures. Each panel depicts for temporal evolution of overall widths (1.1) and subsequent rise- (2.1) and fall-phases (3.1), along with Wavelet Power Spectra (WPS: 1.2, 2.2, 3.2) and Global Power Spectra (GPS: 1.3, 2.3, 3.3) during each phase of the evolution of the jet. Vertical red-line in plots (Obs: 1.1 and Sim 1.1) marks the time when the field-aligned plasma attained the apex height. Further, plots (Obs: 1.1, Sim: 3.2) provide clear indication of a second harmonic of the sausage waves confined in the dynamic spicular waveguide.}
\label{fig:3} 
\end{figure*}

The effects of mass motions on the spicule becomes clear with the wavelet spectrum for both observed and simulated width estimates during their rise and fall intervals. The cross-sectional width oscillations show remarkable differences in their periodicities estimated when plasma is ascending along the magnetic field, against the gravity (rise-phase), and later tracing the field lines under the influence of gravity (fall-phase). Wavelet transforms for observational width variations indicate $\sim$27 sec periodicity for 3 cycles while a powerful 70 sec periodicity is present in both GPS and WPS of the simulation data. However, there is a noticeable shift towards lower/higher periodicities/frequencies, for two out of three of our identified spicule cases (Table \ref{table:1}), during the fall-phase of the plasma flows in the jet structure. Wavelet spectrum for observed width variations show dominant periodicity at 16 sec for 5 cycles while for simulated data, there are two strong periods, concentrated around $\sim$30 sec and $\sim$68 sec, each persistent for 5 and 2.5 cycles. These results are consistent with previous reports for cross-sectional width oscillations in on-disk fibrils \citep{Gafeira2017}. 

An important aspect of our investigation is the confirmation of a first overtone in cross-sectional width periodicities associated with dynamic waveguides (spicules) in the solar chromosphere.  Earlier studies were able to only identify higher harmonics in static waveguides (e.g., in a coronal loop), with the presence of fundamental and first overtone \citep{verwichte2004, guo2015}. Recently, the presence of a second overtone was also reported for coronal loop observations by \citet{duckenfield2019}, associated with transverse kink oscillations. It must be noted that wave harmonics can provide vital clues regarding plasma and magnetic field characteristics of the waveguide via solar magneto-seismology applications \citep{andries2005,andries2009}. The presence of overtones in jets provides a key tool for chromospheric magneto-seismology with MHD sausage wave modes. 
%============================================================
\section{Summary and Discussion}
\label{sec:sum}
%============================================================
In this Letter, the dynamic response of field-aligned mass flows on confined MHD wave modes within chromospheric jets is investigated. High-resolution CRISP/SST H$\alpha$ data were used to identify wave patterns in off-limb spicules, complemented with analysis of a 2D numerical simulation for jets in a stratified solar atmosphere. Cross-sectional width estimates for both observed and simulated jets indicate periodic oscillatory patterns suggesting MHD sausage wave mode. Further, wavelet transform of time series cross-sectional widths show affects of plasma flows on estimated periods. Major conclusions of our investigation are as follows.
%%
\begin{enumerate}[topsep=1ex]
\item{Cross-sectional width oscillations in observed and simulated jets highlight the ubiquitous presence of MHD sausage wave modes in solar chromospheric localised features. Earlier reports identified these waves in on-disk thin flux tube structures, however, for the first time, are also observed in off-limb spicule structures. It must be noted that this intrinsic/dynamic wave characteristic is well established in astrophysical/laboratory jets, however, lacked attention as received by other MHD wave modes (kink, torisonal Alfv\'en) in chromospheric structures, primarily due to observational/instrumental constraints.}

\item{Spectral analysis reveals modulation in periodicities of confined (sausage) wave modes during the rise- and fall-phase of field-aligned plasma. Observed/simulated periodicities indicate a shift towards lower magnitudes in the fall-phase when plasma motions are under the influence of gravity.}

\item{For the first time, the first overtone in the estimated wave periodicity is identified in both observed and simulated dynamic chromospheric waveguide (spicule). This could have important implications for chromospheric magneto-seismology, further enabling the estimation of much needed longitudinal plasma and magnetic field information for chromospheric jets.}
\end{enumerate}
%%
Our study indicates the possible coupling between the longitudinal mass motions and confined wave modes in spicule structures. Understanding of such behavior is crucial for the accurate estimation of overall energy budget and associated dissipation mechanisms. Furthermore, simulations suggests the formation of \textit{knot}-like substructures within the jet beam, prevalent during the rise-phase of plasma density. These substructures were closely linked with the estimated cross-sectional width variations in the jet and could possibly be generated due to superimposition of shock waves. Identification of signatures of mode coupling and jet-substructures is foreseen in future studies using forthcoming Daniel K. Inouye Solar Telescope (DKIST) and European Solar Telescope (EST).


\begin{figure*}[h]
\begin{centering}
\includegraphics[width=1.0\textwidth]{figures/fig_4.png}
\caption{Spicule SP:B, with similar details as described in caption of Figures \ref{fig:1} and \ref{fig:3}}
\label{fig:spb} 
\end{centering}
\end{figure*}

\begin{figure*}
\begin{centering}
\includegraphics[width=1.0\textwidth]{figures/fig_5.png}
\caption{Spicule SP:C, with similar details as described in caption of Figures \ref{fig:1} and \ref{fig:3}}
\label{fig:spc} 
\end{centering}
\end{figure*}