\chapter{The Effects of Non-field Aligned Flow on Spicular-jets}
\label{chap:tilted_jets}
%==============================================================================
\section{Introduction}
\label{sec:c3intro}
%==============================================================================
A basic question one can ask is, do the flows of plasma in the chromosphere trace the magnetic field lines? Often it is assumed that plasma follows the magnetic field lines in the solar atmosphere. The typical length scale for flows on the Sun tend to be large and the magnetic diffusivity is small, which for ideal MHD means that there is a high magnetic Reynolds number (ratio between advection and diffusion). For a perfectly conducting plasma, the magnetic field lines are frozen into the plasma. This means that the plasma flow is channelled along the field lines, but the perpendicular motion of the flow leads to either the magnetic field lines pushing the plasma or being dragged with the plasma. This is the frozen-in condition that is valid for most plasma features in the coronal environment. However, one can question whether this applies to the chromosphere, where gas pressure can play a more important role than in the corona. \cite{delaCruzRodr2011AA527L8D} investigated whether solar chromospheric fibrils or spicular features trace the magnetic fields. In most cases, these dynamic features trace the field lines, but there were a significant number of examples where there was a misalignment between the plasma structure and the magnetic field lines (approx. $40\%$ of a sample of 32). This misalignment in extreme cases can be larger than $45\degs$. The misalignment of magnetic field and spicular jets was studied further numerically by \cite{Mart2016ApJ831L1M} with a 2.5D radiative MHD simulation, which included ion-neutral interaction effects. The results indicate that the majority of the simulated spicules are misaligned with respect to the magnetic field lines. They record misalignment angles up $25\degs$, and in a more extreme case $40\degs$. Even recent simulations of coronal loops have highlighted that misaligned flows could account for observations of dense plasma falling back to the solar surface \citep{Petralia2018AA609A18P}. \np
%
Another aspect to consider is that spicules are typically inclined from the vertical. Some recent studies measure an average incline of $23.4\degs$ with a range of $0-55\degs$ \citep{Pasachoff2009SoPh26059P}. \cite{Tavabi2012JMPh31786T} reported that spicule inclination from the vertical is affected by the regions in which they manifest, e.g. $<20\degs$ in polar regions, approximately $10\degs$ for CHs, and in ARs they can range from $\pm 60\degs$. All these results are in agreement with older studies on the average inclination of spicules, which report inclinations ranging between $19-35 \degs$ \citep{Beckers1968, Mosher1977SoPh53375M, Heristchi1992SoPh14221H, Tsiropoula2012}. \np
%
The aim of this chapter is to investigate what effect misalignment between jet flow and the magnetic field have on the dynamics and morphology of spicules. The steps taken follow a similar pattern to that of Chapter~\ref{chap:sj}, where a parameter scan is undertaken and, using the jet tracking code, the jets' apex and CSW are measured, and density evolution is investigated. 
%=============================================================
\section{Jet Tracking}
\label{sec:tjt}
%=============================================================
%fffffffffffffffff
\mfig{0.5}{figures/slit_size_example.png}{Example of widths being increased artificially due to a tilt in flow, where the width is shown by the black line across the orange rectangle. (a) is the width of a straight jet, (b) is the width taken at the same height, but with the tilted flow, and (c) displays the width measures with (a) [(b)] top [bottom] black line.}{slit_example}
%fffffffffffffffffff
To keep the results comparable over a range of tilted jets, extra steps have been introduced into the jet tracking software. To accurately measure the width of the jet we can no longer take horizontal slits (as done in Section~\ref{subsec:jet_tracking}) as this could artificially alter the width measurements. For example, imagine a rectangle with a slit across it in a fixed position that finishes on the opposite ends (see (a-b) Fig.~\ref{slit_example}); if one rotates this shape as shown in panel (b), the slit would change in size, hence changing the width measurement of the rectangle as displayed in panel (c). In this case, a slit needs to be placed perpendicular to the central axis, of the rectangle to correct this tilt factor. \np
%
Modifications have been made to the jet tracking software to account for this by tracking the central axis (see yellow stars in Fig.~\ref{imporved_j_track_example}). The yellow stars are the midpoints of horizontal slits taken at $0.1~\rm{Mm}$ intervals. To keep the position of the slits consistent, they are placed at every $1~\rm{Mm}$ based on the jet's length, rather than the height (see the green solid lines in Fig.~\ref{imporved_j_track_example}). The slit is placed perpendicular to the angle of the central jet axis determined by the angle between the data point at each megameter of jet length ($j_{ln}$) and its upper neighbour ($j_{ln+1}$). The edges are identified by first converting the tracer into a binary image as outlined in Section~\ref{subsec:jet_tracking}. A region of $1~\rm{Mm}$ by $0.75~\rm{Mm}$, with $j_{ln}$ as the centre point is selected. The values along the slit are interpolated from the grid and then a gradient is taken to identify the edges. If only one edge is found then the search box is increased approximately by $60\times 50~\rm{km}$ (see Fig.~\ref{search_box_j_track_example}) and the process is repeated until 2 edges are found. The solid blue dots in  Fig.~\ref{imporved_j_track_example} show the method of horizontal slits at every $1~\rm{Mm}$ of height as outlined in Section~\ref{subsec:jet_tracking}. The analysis carried out in this chapter shows the measurements of both methods of taking slits. They are referred to as traced slits when correcting for tilt angles, and horizontal slits to match the earlier method.
%fffffffffffffffff
\mfig{0.35}{figures/jet_P300_B60A_60T_0039.png}{Example of extended jet tracking software for $\theta=10\degs$. Solid blue dots mark the jet edges of the original method and the red solid dot marks the jet's apex. The green solid line corrects for the tilt and represents the data sampled to find the edges (solid red squares). The yellow stars give the central axis of the jet. Animation is available: \url{https://drive.google.com/file/d/1wio9np70eYhTbXQeHIwHztQZJjmNCb1g/view?usp=sharing}.}{imporved_j_track_example}
%fffffffffffffffffff

%=============================================================
\section{Parameter scans}
\label{sec:pscansII}
%=============================================================
Based on the results in Chapter~\ref{chap:sj}, two focused parameter spaces (P1 and P2) are carried out to reduce computational cost. The main goal of P1 is to test the trends observed in Chapter~\ref{chap:sj}, and to determine whether they retain, despite the changes in the flow inclination. The range of values used are: lifetimes with $P=300~\rm{s}$, this parameter was found to be less important in determining jet heights over the investigated range, therefore we stick with the standard jet value; initial amplitudes range from $A=20,~40,~60~\kms$, the velocity upper limit matches the standard jet and as $80~\kms$ is near the upper end of observed spicule velocity, it has been omitted; magnetic field strength covers the same range $B=20,~40,~60,~80~\rm{G}$; tilt angles $T=0,5,15^{\circ}$, this range is to show a small tilt and a value closer to typical tilt angles. The purpose of P2 is to study how the trajectories, CSWs and morphology are affected by the tilting angle of the standard jet ($P=300~\rm{s}, ~B=60~\rm{G}, ~A=60~\kms$). A range of tilts $T=0-60^{\circ}$, in increments of $5^{\circ}$ is investigated. Let us summarise both parameters scans:
\begin{mylist}
\item  Lifetimes $P=300~\rm{s}$, initial amplitudes $A=20,~40,~60~\kms$, magnetic field strengths $B=20,~40,~60,~80~\rm{G}$ and tilt angles $\sigma=0,5,15^{\circ}$
\itemp Lifetimes $P=300~\rm{s}$, initial amplitude $A=60~\kms$, magnetic field strength $B=60~\rm{G}$, and range of tilt angles $\theta=0, ~5, ~10, ~15, ~20, ~25, ~30, ~35,$ $~40, 45, ~50, ~55, ~60^{\circ}$.
\end{mylist}
%
%-----------------------------------
\subsection{P1: Parameter Scan 1}
\label{subsec:pscansII_I}
%-------------------------
%fffffffffffffffff
\mfig{0.65}{figures/example_of_tilt_jet_code.png}{Example of a binary image used for the search area (coloured box) used in the jet tracking. Markers have the same representation as Fig.~\ref{imporved_j_track_example}.}{search_box_j_track_example}
%fffffffffffffffffff
The results of P1 with a horizontal slit are displayed in Fig. ~\ref{p_scan_t_apex}. For each panel, the colour (line style) corresponds to the tilt angle (initial amplitude for top panels or magnetic strength for bottom panels). From the $\theta=0^{\circ}$ data it is clear that in each panel the general trends and groupings are retained. Overall, introducing a tilt into the flow has a subtle effect on jet heights and mean widths. From the relation between maximum heights and magnetic field, the greater the initial amplitude, the more notable the effect the magnetic strength has on reducing apex heights. This trend is seen in the relation between initial amplitude and maximum height, where again, the greater the tilt angle and initial velocity, the more prominent the reduction in height, as the range of values fans out around $A>40~\kms$. For both mean width panels, there is not a clear trend on how the tilt affects this parameter. \np
%
This process is repeated, but using the maximum length in place of the jet apex and using the traced slits for calculating the mean widths. For the maximum length there is a similar trend to that seen in Fig.~\ref{p_scan_t_apex}, where the magnetic field does not have a significant impact on jet lengths (except for $P=300~\rm{s},~A=60~\kms,~T=15^{\circ}$ data), and there is a positive correlation between maximum length and initial amplitude. In the bottom left panel, it is shown that the maximum length varies more for $A>40~\kms$ for increasing tilt angles. In general, it appears that increasing the tilt reduces the length of the jets, because the jet is intersecting the magnetic field, hence the magnetic field provides a resistance to the jet's forward momentum. An exception is in the case of $P=300,~A=20,~T=15$, where the length is increased, which might be due to the combination of a higher velocity jet, with a weaker magnetic field. In this setup the transverse motion would be larger, thus increasing the length, but maintaining a shorter height. The traced slits shows the jets increasingly collimate with the stronger magnetic field (top right panel), and that there is a greater mean width with larger initial amplitudes. The main difference between the traced and horizontal slits is that the traced slits return slightly larger jet widths.
%fffffffffffff
\mfig{0.8}{figures/horizontal_slit_pscan_fixing.png}{Focused parameter scans demonstrate the effect of tilt over a variety of jet configurations. Panels on the left (right) are based on the maximum apex (mean width) of the jet.}{p_scan_t_apex}
%ffffffffffffffff
%
%ffffffffffffffff
\mfig{1}{figures/traced_slit_pscan_fixing.png}{Focused parameter scans shows the effect of tilt over a variety of jet configurations. Panels on the left (right) are based on the maximum length (mean width) of the jet.}{p_scan_t_len}
%ffffffffffffffff
%-----------------------------------
\subsection{P2: Parameter Scan 2}
\label{subsec:pscansII_II}
%-------------------------
For P2, the effect of the tilt on the CSW measurements and the difference in widths measured using both slits are investigated (see Fig.~\ref{width_measure}). This parameter scan gives a clear indication that increasing tilt angle increases the jet CSWs. At $45^{\circ}$ this trend stops, which happens because as the tilt angle increases, the more nebulous the jet becomes. This occurs particularly in the falling phase, when the jet no longer falls as one clearly defined column, thus making it challenging to identify jet boundaries (see (c),(g), and (k) in Fig.~\ref{tj_morph_4}). Both slits display the same trend, but the traced slit (red solid line) consistently measures higher jet widths, which is in agreement with P1. The analogous trends for both slits originate from the method of angle selection for the traced slits. This is due to the tracking of the central axis of the jet being determined by horizontal slits with small intervals. In combination with a small sample of data points being used to determine the tilt angle, this means that the width measurements may not deviate much from the original slit method. It is possible that a different method of identifying the central axis and/or more data points used for angle determination would modify the trend for the traced slits. Overall, even the less sophisticated method currently used is an improvement on the horizontal slit method and gives an accurate measure of the mentioned jet parameters. However, future improvements are outlined in Section~\ref{sect:fut_work}. \np
%
In Fig.~\ref{p_scan_t_len} the same parameter scan is investigated, but using a traced slit to account for the tilting of the jet. The top left panel shows the effect of maximum length against the magnetic field strength. The magnetic field strength, in most cases, does not affect the maximum length of the jet. For $P=300,~A=60,~T=15$, increasing magnetic field reduces the synthetic jet's length. For the panel in the bottom left, with $A>40 \kms$ the tilt begins to reduce the maximum length of the jet, but it shows that the greater the initial velocity the longer the jet length. The panels on the right show the effects of the mean width against the change in the magnetic field strength, and initial velocity, top and bottom respectively. The top right panel shows that with increasing magnetic field straight there is more collimation of the jet. From the bottom left panel, as the initial amplitude increases the greater the jet CSW. The trends that we have seen in Chapter~\ref{chap:sj} are maintained when the jet is launched at an angle.  \np
%ffffffffffffffff
\mfig{1}{figures/mean_w_vs_tilt.png}{Effect of tilt on the mean width measurement, comparing traced slit (red solid line with data markers) and horizontal slit (blue dashed line with data markers).}{width_measure}
%ffffffffffffffff
%=============================================================
\section{Jet Trajectory}
\label{sec:j_traj_t}
%=============================================================
The effect of tilt on the jet is studied using P2. The apex height and length of the jet are tracked and displayed in the left panel of Fig.~\ref{tilt_effect_traj}. There is a clear trend, in that both the maximum height (solid black marked line) and length (dashed red marked line) are decreased with a tilting angle. The maximum length is larger than height, and the difference widens above $20^{\circ}$. The increased difference between height and length can be attributed to larger transversal displacement with increasing tilt. Both trajectories of the jet (right-hand panels) display a decrease in maximum height/length and a slight decrease in jet lifetimes of about $150~\rm{s}$, over the range of tilt angles. The results indicate that once tilted up to $50^{\circ}$, the trajectories deviate from what is typically seen in spicular jets. A similar pattern is seen when the jet length is tracked, but the trajectory is less smooth. This latter is because the jet is undergoing transverse motion and is constantly changing in length. These results indicate that both the magnetic field and horizontal perturbations could alter the trajectory of the jet. \np 
%
An interesting aspect of the results is that introducing a small tilt in the flow has produced spicules with differing heights and lifetimes. This means that, a spicule generated with similar energy can appear differently due to non-field aligned flows, which with more misalignment causes shorter heights and lifetimes. This latter finding could explain why spicules appear and disappear over a variety of heights. However, unlike the situation with TII spicules, their appearance would be similar to their higher propagating counterparts. It is only in an extreme case of tilt angles (approx. $50^{\circ}$) that the trajectory significantly deviates from a parabolic flight.
%
%ffffffffffffffff
\mfig{1}{figures/combine_L_h_comp.png}{Plots shows the effect of non-field aligned flow on apex (top panels), length (bottom panels) and trajectories (rightmost panels).}{tilt_effect_traj}
%ffffffffffffffff
%------------------------------------------------------------------------------
\subsection{Tilted Jet Morphology}
\label{subsec:steady}
%------------------------------------------------------------------------------
Figs.~\ref{tj_morph_1} to \ref{tj_morph_4} visualise the density evolution for the standard jet at various tilt angles. The first row of Fig.~\ref{tj_morph_1} serves as a reference to the $\theta=0\degs$ jets described in Chapter~\ref{chap:sj}. Note that in Figs.~\ref{tj_morph_1} to~\ref{tj_morph_3} the timesteps selected are the same as in the simulation snapshots in Chapter~\ref{chap:sj}, this is purposefully chosen for comparison. \np
%
An interesting aspect of the simulation is that even for small tilting angles of $5-10^{\circ}$, the morphology of the jet is significantly changed, particularly in the falling phase shown by columns containing panels (c-d). At about $t=253~\rm{s}$ for $5^{\circ}$ and $10^{\circ}$, the jet experiences an instability, causing a kinking motion to travel through the jet beam, giving a whip effect (see panel (g) for aftermath). This whip motion causes significant mixing to occur in the jet beam (see panels (g) and (k) in Fig.~\ref{tj_morph_1}). This is also observed for higher degrees of tilt, but does not cause significant mixing in the jet beam. Another phenomenon observed in jets with higher degrees of tilt ($>20^{\circ}$) is finger-like structures occurring in the down-flow of the jet, both in the jet beam and its right-hand side boundary. Both these dynamics are likely produced by instabilities. In both cases, to determine precisely which instabilities cause these dynamics is not a trivial task and requires further investigation. Despite this, from their appearance and with the data available, it is possible to narrow down to three potential candidates:
\begin{enumerate}
    \item Rayleigh-Taylor instability (RTI) is the instability of an interface between two fluids of differing densities, where the heavy fluid is on top. If this system is subjected to gravity, then any slight perturbation leads the denser fluid to fall through the lighter fluid, forming finger-like structures as shown in Fig.~\ref{RT_example}.
    \item Kelvin-Helmholtz instability (KHI) occurs if there is a velocity shear in a single continuous fluid. The shearing between these two fluids, once above a critical threshold, causes the interface to form vertices; see Fig.~\ref{KHI_example}.    
    \item Dynamic kink instability (DKI) occurs when plasma flows on a curved trajectory. There are two opposing forces acting on the jet; the centripetal force which is trying to destabilise the flow, and the Lorentz force which is acting to stabilise the flow, as shown in Fig.~\ref{DKI_example}. If the flow is super-\Alfvenic then the centripetal force can outweigh the Lorentz force, enhancing the transverse displacement. This instability occurs in HD but is called a kink instability (KI) (not to be confused with MHD KI), which like the DKI is caused by centripetal force in curved flows \citep{Drazin2002ihsbookD}. Note that this instability, is different from an MHD kink instability which arises due to the Lorentz force enhancing kink motion.
\end{enumerate}
From the list of potential instabilities, the two candidates for the whip effect are KHI or DKI. KHI is observed in many instances in nature, e.g. from clouds, waves in the ocean, and Jupiter's eddies \textit{etc.}  They are present in numerous solar features: coronal mass ejections \citep{Foullon2011ApJ729L8F, Foullon2013ApJ767170F}, coronal loops \citep{Barbulescu2019ApJ870108B}, prominences \citep{Berger2010ApJ7161288B, Ryutova2010SoPh26775R}, solar jets \citep{Filippov2015MNRAS4511117F,Li2018NatSR88136L}, and spicules \citep{Kuridze2016ApJ830133K, Antolin2018ApJ85644A}. As there are significant velocity differences between the jet and ambient medium, the interface between these two regions can become KH unstable. This can lead to deformation of the jet boundary and if shearing is strong enough, it can destabilise the jet. However, \cite{Chandrasekhar1961hhsbookC} has shown that a magnetic field aligned with the flow will impede the development of KHI, which is the case of the simulation showcased. Another aspect to consider is that KHI would be localised to the jet boundary, and would not propagate through the jet beam as seen in the simulations. Due to these factors, it is unlikely that KHI is responsible for the whip effect occurring in the simulations. \np 
%------------------------------------
A strong candidate to explain the whip effect is the DKI. \cite{Zaqarashvili2020ApJ893L46Z} proposed a model showing DKI could cause transverse motions in spicules. If a spicule is travelling at an angle (in their case the curve was introduced by considering a vertically expanding flux tube) and the flow is super-\Alfvenic, then DKI can enhance transverse motions. DKI can explain the presence of the whiplash motion going through the jet. More investigation is needed to understand why at shallow angles of $5-10\degs$, mixing occurs in the falling phase of the jet beam. One possibility is that above $15^{\circ}$ horizontal motion is sufficient to dampen the instability due to the magnetic field. The magnetic field is acting as a brake to the jet flow, i.e. the more tilt, the more impeded the jet will be thereby resulting in smaller up-flows and a reduction in the effect of the DKI. Interestingly, the whip motion can be seen in Fig.~\ref{paramter_scan_one} for $A=80 \kms$, which does not fully fit in with the DKI, because the flow is not on a curved trajectory. \np 
% RTI in proms review paper: https://link.springer.com/article/10.1007/s41614-017-0013-2
Two possible instabilities to account for finger-like structures in the falling phase of the jet for tilts between $20-45\degs$ are RTI and KHI. RTI is hypothesised to form at prominence boundaries \citep{Berger2008ApJ676L89B,Berger2010ApJ7161288B,Hillier2012ApJ746120H,Berger2017ApJ85060B}, but for spicules, there is no strong observational evidence for RTI occurring. Numerical simulations of astrophysical jets have shown that RTI instability can occur at the jet boundary, where the finger-like structures are seen in the perpendicular cross-section with regard to the central jet axis \citep{Toma2017MNRAS4721253T,Matsumoto2017MNRAS4721421M}, and laboratory plasma jets have shown evidence of RTIs \citep{Zhai2016PhPl23c2121Z}. It may be that with a 3D version of the synthetic jet simulation, RTI would form at the boundaries of the jet. The lack of observations of RTI in spicules, and the fact that RTIs would be oriented in the direction of the magnetic field, makes it unlikely that these are responsible for the finger-like structures seen in the simulations. It is more likely to be KHI, as the finger-like patterns form in the falling phase of the jet where there will be an interaction of up and down flowing material, causing increased shearing. The finger-like structures appear near regions of high density, which are located near the edges of the greatest transverse displacement. With higher resolution or less diffusive numerical schemes the KHI could appear as vortices at these locations. \np
%
The tilt has an important impact on the structuring of the beam. The main noticeable difference is the appearance of knots, as seen in columns containing them (b) in Figs.~\ref{tj_morph_1} to~\ref{tj_morph_4}. For tilt $<15\degs$ the knots are denser and deformed, but for higher degrees of the tilt ($> 10^{\circ}$) the knots are no longer easily identified. This would mean only slight horizontal disturbance to the jets would make it even more challenging to identify knots if present in observations. The tilt causes the densest parts of the jet to change from being contained to the head and knots, as seen in Chapter~\ref{chap:sj}, to the edges where the most transversal displacement is. The greater the tilt, the more the edges of transversal displacement become the densest part of the jet, and the rest of the jet beam reduces in density. The column containing panel (d) reaffirms the earlier result that the jet has a lower apex and reduces the jet lifetimes for increasing tilt.
%
%ffffffffffffffffffff
\begin{figure}
\captionsetup[subfigure]{labelformat=empty}
\centering
\subfloat[]{\includegraphics[width=0.8\linewidth]{figures/RT_example.png}}
\caption{Example of the formation of an RT instability at various time steps. These are results from a simulation shown in \cite{Liang2019PhFl31k2104L}. The red-coloured fluid represents a denser fluid than its blue counterpart and gravity is directed downwards leading to the formation of RTI.}
\label{RT_example}
\end{figure}
%ffffffffffffffffffff
%%ffffffffffffffffffff
%\mfig{1}{figures/tj_den_plot_1.png}{Example of the temporal evolution of different tilts angles for jets from $0,~10,~20^{\circ}$ from top to bottom, respectively.}{tj_morph}
%%ffffffffffffffffffff
\begin{figure}
\captionsetup[subfigure]{labelformat=empty}
\centering
\subfloat[]{\includegraphics[width=\linewidth]{figures/KHI_example.png}}
\caption{Example of the formation of a KHI taken from \cite{Barbulescu2018SoPh29386B}. Stage (a) shows the interface between two fluids where $U_0$ and $U_1$ are arbitrary flow speeds before they are subject to a perturbation shown in stage (b). In stage (c) the perturbation is enhanced by the flows creating non-linear wave steepening. This leads to stage (d) where a vortex forms, mixing both fluids.}
\label{KHI_example}
\end{figure}
%ffffffffffffffffffff
%ffffffffffffffffffff
\begin{figure}
\captionsetup[subfigure]{labelformat=empty}
\centering
\subfloat[]{\includegraphics[width=0.8\linewidth]{figures/DKI.jpg}}
\caption{Cartoon of dynamic kink instability taken from \cite{Zaqarashvili2020ApJ893L46Z}. The blue (red) lines represent the magnetic field (jet). The blue (red) arrow shows the direction for the Lorentz (centripetal) force.}
\label{DKI_example}
\end{figure}
%ffffffffffffffffffff
%ffffffffffffffffffff
\begin{figure}
\captionsetup[subfigure]{labelformat=empty}
\centering
\subfloat[]{\includegraphics[width=\linewidth]{figures/tj_den_plot_0_5_10.png}}
\caption{Example of the temporal evolution of different tilt angles for jets from $0,~5,~10^{\circ}$ from top to bottom. Animation available where cases a, e, i are in the top row: \url{https://drive.google.com/file/d/1HbVzpsZc2der4BR0ItXUUd4mxXm-nVSD/view?usp=sharing} }
\label{tj_morph_1}
\end{figure}
%ffffffffffffffffffff
\begin{figure}
\captionsetup[subfigure]{labelformat=empty}
\centering
\subfloat[]{\includegraphics[width=\linewidth]{figures/tj_den_plot_15_20_25.png}}
\caption{Example of the temporal evolution of different tilt angles for jets from $15,~20,~25^{\circ}$ from top to bottom. Animation available where cases a, e, i are in the bottom row: \url{https://drive.google.com/file/d/1HbVzpsZc2der4BR0ItXUUd4mxXm-nVSD/view?usp=sharing}}
\label{tj_morph_2}
\end{figure}
%ffffffffffffffffffff
\begin{figure}
\captionsetup[subfigure]{labelformat=empty}
\centering
\subfloat[]{\includegraphics[width=\linewidth]{figures/tj_den_plot_30_35_40.png}}
\caption{Example of the temporal evolution of different tilt angles for jets from $30,~35,~40^{\circ}$ from top to bottom. Animation available where cases a, e, i are in the top row: \url{https://drive.google.com/file/d/1xW6KnMehodIazDxL7j4y2uXM8j2v6nBq/view?usp=sharing}}
\label{tj_morph_3}
\end{figure}
%ffffffffffffffffffff
\begin{figure}
\captionsetup[subfigure]{labelformat=empty}
\centering
\subfloat[]{\includegraphics[width=\linewidth]{figures/tj_den_plot_45_50_55.png}}
\caption{Example of the temporal evolution of different tilt angles for jets from $45,~50,~55^{\circ}$ from top to bottom. Animation available where cases a, e, i are in the bottom row: \url{https://drive.google.com/file/d/1xW6KnMehodIazDxL7j4y2uXM8j2v6nBq/view?usp=sharing}}
\label{tj_morph_4}
\end{figure}
%ffffffffffffffffffff
%------------------------------------------------------------------------------
\subsection{Effect of Tilt on Cross-sectional width variation}
\label{subsec:oscillating}
%------------------------------------------------------------------------------
By introducing tilt into the jets, they not only undergo CSW variations but also transverse motions. These are two key dynamical ingredients that need to be captured in spicular models. Spicules are not just vertically dynamic plasma sticks, they display complex motion all through the body of the jet \citep{Sharma2018ApJ85361S}. \np
%
Using the horizontal slits time-distance plots were created at $1-3~\rm{Mm}$ heights (see Figs.~\ref{td_plot_1Mm} to \ref{td_plot_3Mm}), over the range of tilts. From Fig.~\ref{td_plot_1Mm} for $\theta=0^{\circ}$ there are clear sausage-like motions in the CSW variations. It shows that there is a cavity inside the jet, which is due to the process that creates the knots. Even with the smallest tilting angle ($\theta=5\degs$) investigated, it makes substantial changes to the CSW variations of the jet beam. This appears to be the only example where the combination of both CSW variations and transversal displacement can be cleanly observed at this height, as with increasing tilt the sausage-like CSW variations become less noticeable. The main dynamics shift from symmetrically CSW variations, to transversal displacement that reaches up to approximately $1~\rm{Mm}$ in width. For a tilt of $\theta>25\degs$ the most horizontally displaced regions have much denser edges. These could be regions where the material is building up due to the magnetic field redirecting flow as it is perturbed. With increasing amounts of horizontal velocity, the more the jet travels into the magnetic field, hence the jet material collects at turning points creating denser regions at the jet boundary. For the tilted jets, the large scale transverse motions in the simulations are due to the restoring forces of the ambient magnetic field, which are trying to re-establish an equilibrium. Once the tilt passes $\theta=40\degs$, there is less of a clear jet structure. Similar behaviour is seen at $2~\rm{Mm}$ and $3~\rm{Mm}$ (Fig.~\ref{td_plot_2Mm}), where there appears to typically be one-two global sways of the jet during its lifetime. This shows that the wave-like motion propagates throughout the jet. \np
%
The whip motion has been observed in chromospheric jets \citep{Liu2009ApJ707L37L}. In \cite{Liu2009ApJ707L37L}, the observed jet undergoes a transverse disturbance with a whip motion. This jet has length scales (approx. $43.5~\rm{Mm}$), lifetimes (approx. $>1\rm{hr}$), observed speed (approx. $430~\kms$), and proposed origin is magnetic recognition. The whip motion is proposed to be due to the jet being composed of helical threads undergoing untwisting spins, outlined by \cite{Shibata1985PASJ3731S, Shibata1986SoPh103299S} and \cite{Canfield1996ApJ4641016C}. Although the phenomena has very different origins and scale, the overall motion time-distance plots, present in \cite{Liu2009ApJ707L37L}, appear similar to time-distance plots presented in Figs.~\ref{td_plot_1Mm} to~\ref{td_plot_3Mm} that are proposed to be produced by DKI. This whip motion has been shown in simulations of reconnection jets, where material from the chromosphere is ejected by a sling-shot effect due to reconnection and produces a whip motion \citep{Yokoyama1996PASJ48353Y, Kotani2020PASJ7275K}. We do not disagree with these outlined mechanisms, but DKI could be playing a role in the transverse motions. DKI could form as in this reconnection jet scenario there is both curvature and horizontal perturbation to the magnetic field, in addition to a fast flowing jet. \np
%fffffffffffffffffffff
\mfig{1}{figures/td_plot_1Mm.png}{Time distant plot with horizontal slit at $1~\rm{Mm}$ for tilt values $0-55^{\circ}$.}{td_plot_1Mm}
%fffffffffffffffffffff
%
\mfig{1}{figures/td_plot_2Mm.png}{Same as fig.~\ref{td_plot_1Mm}, but at $2~\rm{Mm}$.}{td_plot_2Mm}
\mfig{1}{figures/td_plot_3Mm.png}{Same as fig.~\ref{td_plot_1Mm}, but at $3~\rm{Mm}$.}{td_plot_3Mm}
%fffffffffffffffffffff
%============================================================
\section{Summary and Discussion}
\label{sec:sum}
%============================================================
In summary, the model used in Chapter~\ref{chap:sj} has been generalised to produce non field-aligned flows to study its effect on jet heights, widths and density evolution. A simple approach is taken to capture the basic dynamics of the jet across two focused parameter scans. The main results obtained from this Chapter are:   
\begin{itemize}
\item Over the range of the parameter space studied, the tilt led to a reduction in apex height and length of the jet.
\item Knots may be more challenging to observe in solar jets than first proposed in this thesis. Even with small amounts of tilt the clear structure seen in straight jets becomes deformed and more blended into the jet beam. At high inclination angles, it would be unlikely that knots would be observed. 
\item The whip motion in the jet is probably due to DKI, but more analysis of the simulations is required to confirm this. The assumptions and physical set up (e.g. different form of stratification of atmosphere, geometry, and magnetic field configuration) that are used to derive the dispersion relation in \cite{Zaqarashvili2020ApJ893L46Z} are not directly applicable to the synthetic jet. A new dispersion relation would need to be derived and the wave frequency inside the jet would need to be calculated to determine the point at which the onset of DKI occurs. 
%{\bf RE: ??? While can use stability criteria that jets are DKI if $v_j(y)>1.25 V_{Ae}$}, where $V_{Ae}$ is the external \Alfven speed, the assumptions made for this calculation are not fully applicable to the synthetic jets {\color{green}(e.g. does not account for a stratified medium, simulations are in a slab geometric, \textit{etc.})}.
\item By introducing tilt into the jets, they not only undergo CSW variations but also transverse motions. These are two key dynamical ingredients that need to be captured in spicular models. For small tilting angles ($<10\degs$) there is a combination of both CSW and transverse displacement, but for larger tilts the transverse motion dominates.
\end{itemize}
An important result of these simulations is that even a slight misalignment between flow and magnetic field produces noticeable transversal displacement and has major effects on the morphology. The first evidence of transverse motion in spicules was identified by \cite{Pasachoff1968SoPh5131P}. The mechanism by which transverse motion arises remains a subject of debate. Several mechanisms have been suggested, including overshooting of convective motions in the photosphere, granular buffeting, rebound shocks, global $p$-mode oscillations, and reconnection near foot point \citep{Roberts1979SoPh6123R, Sterling1988ApJ327950S, Vranjes2008A_A478553V, Jess2012ApJ744L5J, Ebadi2014ApSS35331E}. The interpretation of the bulk motion of spicules has typically been through the framework of MHD waves. Many studies report the transverse motion of spicules are due to MHD kink waves \citep{Kukhianidze2006AA, De_Pontieu2007, Jess2012ApJ744L5J, Ebadi2014ApSS35331E, Tavabi2015AA573A4T, Jafarzadeh2017ApJS2299J}. The increased interest is due to the spicules' potential to transport the energy contained in these waves throughout the lower atmosphere \citep{De_Pontieu2007, He2009ApJ705L217H, Morton2012NatCo31315M, Jess2012ApJ744L5J}.  These transverse motions are clearly an important ingredient in the dynamics of spicules and potentially in maintaining the hot solar atmosphere. Non-field aligned flow gives another mechanism to produce transverse motions. This simple mechanism is likely to occur due to the conditions of the chromosphere (where the flow will not always be aligned with the magnetic field), and the complexity of the Sun’s magnetic field. Regardless of the tilt angle, the synthetic jets undergo $1-2$ sways of transverse motion, which are larger with increasing tilt angles. Even with simple models of jets, this highlights the complexity of the dynamics. Understanding how a common component of tilt affects the appearance of the jet is a key part of interpreting the dynamics of spicules, and this should be a factor included in any model of spicules.