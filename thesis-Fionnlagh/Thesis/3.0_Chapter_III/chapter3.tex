
\chapter{An Analytical Model of the Kelvin-Helmholtz Instability of Transverse Coronal Loop Oscillations}

%==============================================================================
\section*{Abstract}
%==============================================================================

\let\thefootnote\relax\footnotetext{

This chapter is based on the following accepted manuscript:
\begin{itemize}
\item Barbulescu, M., Ruderman, M.S., Van Doorsselaere, T., Erd\'elyi, R.; An Analytical Model of the Kelvin-Helmholtz Instability of Transverse Coronal Loop Oscillations, \apj, accepted.
\end{itemize}
}

Recent numerical simulations have demonstrated that transverse coronal loop oscillations are susceptible to the Kelvin-Helmholtz (KH) instability due to the counter-streaming motions at the loop boundary.
We present the first analytical model of this phenomenon.
The region at the loop boundary where the shearing motions are the greatest is treated as a plane interface separating time-periodic counter-streaming flows.
In order to consider a twisted tube, the magnetic field at one side of the interface is inclined.
We show that the evolution of the displacement at the interface is governed by Mathieu's equation and we use this equation to study the stability of the interface.
We prove that the interface is always unstable, and that, under certain conditions, the magnetic shear may reduce the instability growth rate.
The result, that the magnetic shear cannot stabilise the interface, explains the numerically found fact that the magnetic twist does not prevent the onset of the KH instability at the boundary of an oscillating magnetic tube.
We also make use of the notion of the loop $\sigma$-stability.
We say that a transversally oscillating loop is $\sigma$-stable if the KH instability growth time is larger than the damping time of the kink oscillation.
We show that even relatively weakly twisted loops are $\sigma$-stable.

%==============================================================================
\section{Introduction}
\label{sec:c3intro}
%==============================================================================

Transverse oscillations of coronal loops have been a subject of extensive study since their original observation on 14 July 1998 by the \textit{Transition Region and Coronal Explorer} (TRACE) \citep{Aschwanden1999, Nakariakov1999}. For a review of the theory of these oscillations see \cite{Ruderman2009}.

In particular, the damping mechanism of transverse loop oscillations has received much attention \citep[e.g.][]{Ruderman2002, Goossens2002, TVD2004, Dymova2006, Williamson2014}, with the caveat that many studies have relied on the assumption that the oscillations are in the linear regime.
The nonlinear damping of transverse coronal loop oscillations has also been studied, both analytically \citep{Ruderman2010b, Ruderman2014, Ruderman2017}, as well as numerically \citep[e.g.][]{Terradas2004, Magyar2016a}.
The numerical studies revealed important effects, such as that of the ponderomotive force, and the presence of the Kelvin-Helmholtz instability (KHI) at the loop boundaries.
More recently, \cite{Goddard2016} carried out a statistical study of observations of the damping of coronal loop kink oscillations.

\cite{Terradas2008} suggested that a kink oscillation may render a flux tube unstable due to the shear motions at the boundaries.
The authors found that, for a smooth transition layer, the instability developed rapidly where the difference between the internal and external flow amplitudes was the greatest.
However, increasing the thickness of the transitional layer significantly decreased the growth rate of the instability.
It is worth noting that the KHI in smooth transition layers via other mechanisms (e.g. phase mixing, resonant absorption) had also received attention previously \citep[see, for example,][]{Heyvaerts1983, Ofman1994, Poedts1997}.
For a recent review on modelling the KHI see, e.g. \cite{Zhelyazkov2015}.

The topic of the transverse wave induced Kelvin-Helmholtz (TWIKH) instability was subsequently investigated by \cite{Antolin2014}, who suggested that this phenomenon may be responsible for the fine strand-like structure observed in some coronal loops.
The TWIKH instability has since been studied by \cite{Antolin2016, Magyar2016a, Magyar2016b, Karampelas2017, Howson2017a, Howson2017b, Karampelas2018}, who considered aspects of the instability onset, growth rate and observational properties.

\begin{figure*}[t]
\centering
\includegraphics[width=0.9\textwidth]{figures/tube_1}
\caption{Sketch of a straight magnetic flux tube with stationary footpoints undergoing transverse (kink) motion. The panel on the right represents the velocity field in a cross section of the tube, at half the length of the tube. The greatest shearing occurs between the vectors coloured in red, highlighted by the dashed boxes.}
\label{fig:tube1}
\end{figure*}

The configuration of the equilibrium magnetic field is an important aspect of TWIKH instabilities.
It was suggested by \cite{Terradas2008} that a twisted magnetic field may suppress the instability.
The effect of twist on the stability of transverse loop transverse oscillations was studied numerically by \cite{Howson2017b} who investigated the energetics of the instability of a magnetically twisted coronal loop and found that its evolution is affected by the strength of the azimuthal component of the magnetic field.
The authors also found that, when magnetic twist is present, the KHI leads to greater Ohmic dissipation as a result of the production of larger currents.
Furthermore, \cite{Terradas2018} studied the evolution of the instability and found that the  magnetic twist increases the instability growth time.

Numerical simulations have provided some insight into the development of the KHI, but have not thoroughly established what the conditions are needed for its onset.
In this Chapter, we find these requirements analytically by modelling the boundary of the flux tube where the shearing is greatest as a single interface separating regions of different densities and magnetic fields, and performing a local stability analysis.
We emulate the effect of the transverse oscillation by subjecting each region to temporally periodic counter-streaming flows.

The Chapter is organised as follows: in Section~\ref{sec:goveq}, we introduce a Cartesian model of the boundary of a twisted flux tube, and derive the governing equation for the displacement.
The stability of the flow is analysed in Section~\ref{sec:stab}, followed by applications to transverse coronal loop oscillations in Section~\ref{sec:loop}.
Section~\ref{sec:sum} contains the summary of the obtained results and our conclusions.

%=============================================================
\section{The Governing Equation}
\label{sec:goveq}
%=============================================================

It is well established that a magnetic flux tube undergoing transverse oscillation is prone to the Kelvin-Helmholtz instability due to the shearing motions at the boundaries \citep{Terradas2008}.
Considering only the fundamental mode of oscillation, we wish to obtain the TWIKH instability criterion. We start by considering a magnetically twisted flux tube of length $L$.
The amplitude of a fundamental transverse oscillation is greatest at the half-length of the tube, $L/2$, so that is where the shearing is the greatest.
We consider a plane $\Pi$ orthogonal to the tube axis and crossing it at its half-length. The intersection of this plane with the tube boundary is a circle.
We introduce the angle $\varphi$ in the plane $\Pi$, measured from the direction of the oscillation velocity in the counter-clockwise direction.
Then, the shear velocity at the tube boundary takes its maximum at $\varphi = \pi/2$ and $\varphi = 3\pi/2$\/, i.e. at the two points where it is parallel to the oscillation velocity (see Figure~\ref{fig:tube1}).

In order to study the effect of the shearing motions around this region, we model it as a single interface separating temporally periodic counter-streaming flows.
We introduce the Cartesian coordinate system $x$, $y$, $z$ with the $x$\/-axis parallel to the direction of the polarisation of the kink oscillation, and the $z$\/-axis parallel to the tube axis.
The interior and exterior of the tube are represented by the regions $y < 0$ and $y > 0$, respectively.
The equilibrium quantities in these regions are denoted by the subscripts $i$ and $e$\/, respectively.

\begin{figure}[t]
\centering
\subfloat[A twisted magnetic flux tube embedded in a straight magnetic field.]
{\includegraphics[width=0.48\textwidth]{figures/tube_2}}
\hfill
\subfloat[The magnetic fields and flows at the interface.]
{\includegraphics[width=0.48\textwidth]{figures/interface.pdf}}
\caption{Sketch of a twisted magnetic tube, (a), and a diagram of the flows on each side of the boundary during transverse oscillation (b).}
\label{fig:tube_interface}
\end{figure}

We assume that the equilibrium magnetic field is in the $xz$\/-plane.
Since we wish to obtain the stability criteria both for straight and twisted tubes, we assume that the equilibrium magnetic field is parallel to the $z$-axis in the region $y > 0$, and makes an angle $\theta$ with respect to the $z$-axis in the region $y < 0$.
Here, $\theta$ corresponds to the degree of twist (Figure \ref{fig:tube_interface}a), which should be small since highly twisted magnetic flux tubes are prone to other types of instabilities, such as the kink instability, with which we are not concerned in the present study \citep[e.g.][]{Shafranov1958, Kruskal1958, Hood1979}.
In the case of a non-twisted flux tube, $\theta = 0$.

In the present model, the background flows are similar to the velocity field at the boundary of a cylindrical flux tube undergoing a transverse oscillation.
In transverse oscillations of coronal loops, the displacement of the flux tube's boundary is almost perpendicular to the background magnetic field in the low-beta plasma approximation \citep[see, e.g.][]{Ruderman2007}, therefore, we consider unperturbed magnetic fields and flow velocities, in Cartesian coordinates, of the form
%
\begin{align}
\begin{split}
\label{eq:c3equilibrium}
& \mathbf{B_i} = (B_i \sin \theta, 0, B_i \cos \theta),
\\
& \mathbf{B_e} = (0, 0, B_e),
\\
& \mathbf{U_i} = (U \cos(\Omega t) \cos \theta, 0, - U \cos(\Omega t) \sin \theta),
\\
& \mathbf{U_e} = (- U \cos(\Omega t), 0, 0),
\end{split}
\end{align}
%
as illustrated in Figure \ref{fig:tube_interface}b.
Here, the period of the oscillatory flow, $2\pi / \Omega$, corresponds to the period of oscillation of the flux tube.

It is worth noting that the problem of oscillatory counter-streaming flows has been previously studied by, e.g. \cite{Kelly1965} and \cite{Roberts1973}.
Our model is an improvement since we do not only consider parallel flows.
Furthermore, our model differs from that of \cite{Roberts1973} since we consider magnetic fields perpendicular to the flows on each side of the interface.

We study the dynamics of the outlined problem in the framework of linear ideal MHD.
Since, for transverse loop oscillations, the effects of compressibility are not significant \citep{Ruderman2009}, we may use the approximation of incompressible plasma in order to simplify the analysis.
The set of governing equations is, thus, Equations \eqref{eq:mominclin} -- \eqref{eq:divvlin}, which we rewrite here as
% 
\begin{align}
\label{eq:c3mom}
\frac{\mathrm{D} \mathbf{v}}{\mathrm{D} t}
& = - \frac{1}{\rho_{i, e}} \nabla p_T
+ \frac{1}{\mu_0 \rho_{i, e}}( \mathbf{B_{i, e}} \cdot \nabla )\mathbf{b},
\\
\label{eq:c3ind}
\frac{\mathrm{D} \mathbf{b}}{\mathrm{D} t}
& = ( \mathbf{B_{i, e}} \cdot \nabla ) \mathbf{v},
\\
\nabla \cdot \mathbf{v} & = 0,
\\
\nabla \cdot \mathbf{b} & = 0,
\end{align}
%
where $\mathbf v, \mathbf b$ and $p_T$ are the perturbations of the velocity, magnetic field, and total pressure (magnetic plus plasma), $\rho_{i,e}$ are the background internal and external densities, and $\mu_0$ is the magnetic permeability of free space.
Taking into account Equation \eqref{eq:c3equilibrium}, the material derivative, $\mathrm{D}/\mathrm{D} t$, may be written as
%
\begin{equation}
\dfrac{\mathrm{D}}{\mathrm{D} t}
= \begin{cases}
\dfrac{\partial}{\partial t}
+ U \cos(\Omega t) \cos \theta \dfrac{\partial}{\partial x}
- U \cos(\Omega t) \sin \theta \dfrac{\partial}{\partial z}, & \quad y < 0,
\\[0.3cm]
\dfrac{\partial}{\partial t}
- U \cos(\Omega t) \cos \theta \dfrac{\partial}{\partial x}, & \quad y > 0.
\end{cases}
\end{equation}
%
Applying the material derivative to Equation \eqref{eq:c3mom} yields
%
\begin{equation}
\label{eq:c3mom2}
\frac{\mathrm{D}^2 \mathbf{v}}{\mathrm{D} t^2}
= - \frac{1}{\rho_{i, e}} \frac{\mathrm{D}}{\mathrm{D} t} \nabla p_T
+ \frac{1}{\mu_0 \rho_{i, e}}( \mathbf{B_{i, e}} \cdot \nabla) \frac{\mathrm{D} \mathbf{b}}{\mathrm{D} t},
\end{equation}
%
since the differential operators $\mathbf{B_{i, e}} \cdot \nabla$ and $\mathrm{D}/\mathrm{D} t$ are independent of $\mathbf x$ and, thus, commute.
Introducing Equation \eqref{eq:c3ind} into \eqref{eq:c3mom2}, one obtains
%
\begin{equation}
\label{eq:c3mom3}
\frac{\mathrm{D}^2 \mathbf{v}}{\mathrm{D} t^2}
- \frac{1}{\mu_0 \rho_{i, e}}( \mathbf{B_{i, e}} \cdot \nabla )^2 \mathbf{v}
= - \frac{1}{\rho_{i, e}} \frac{\mathrm{D}}{\mathrm{D} t} \nabla p_T.
\end{equation}
%
We now introduce the Lagrangian displacement $\bm \xi = \bm \xi(\mathbf{x}, t)$, which is related to the velocity perturbation by $\mathbf v (\mathbf{x}, t) = \mathrm{D} \bm \xi / \mathrm{D} t$, as introduced in Equation \eqref{eq:xi}.
Equation \eqref{eq:c3mom3} becomes
%
\begin{equation}
\label{eq:c3displacement0}
\frac{\mathrm{D}^3 \bm \xi}{\mathrm{D} t^3}
- \frac{1}{\mu_0\rho_{i, e}} ( \mathbf{B_{i, e}} \cdot \nabla )^2
\frac{\mathrm{D} \bm \xi}{\mathrm{D} t}
= - \frac{1}{\rho_{i, e}} \frac{\mathrm{D}}{\mathrm{D} t} \nabla p_T,
\end{equation}
%
which may be integrated to obtain
%
\begin{equation}
\label{eq:c3displacement}
\frac{\mathrm{D}^2 \bm \xi}{\mathrm{D} t^2}
- \frac{1}{\mu_0\rho_{i, e}} ( \mathbf{B_{i, e}} \cdot \nabla )^2 \bm \xi
= - \frac{1}{\rho_{i, e}} \nabla p_T.
\end{equation}
%
The arbitrary functions of integration which should have been introduced when obtaining Equation \eqref{eq:c3displacement} may safely be omitted since we are only concerned with solutions at $y=0$, and taking into account that the displacement and total pressure must be continuous at the interface.

Taking into account Equation \eqref{eq:divxi} and the fact that the divergence operator commutes with all other differential operators in Equation, we apply $\nabla \cdot$ on Equation \eqref{eq:c3displacement}.
The result is Laplace's equation for the total pressure
%
\begin{equation}
\label{eq:pt1}
\nabla^2 p_T = 0.
\end{equation}
%
We Fourier-decompose the total pressure and write it in the form $p_T = \hat p_T \exp[i (k_x x + k_z z)]$, such that Equation \eqref{eq:pt1} becomes
%
\begin{equation}
\label{eq:pt2}
\left( \frac{\mathrm{d}^2}{\mathrm{d} y^2} - k_x^2 - k_z^2 \right) \hat p_T = 0.
\end{equation}
%
The solution to Equation \eqref{eq:pt2}, which satisfies the condition that the total pressure is continuous at $y = 0$, is
%
\begin{equation}
\label{eq:ptsol}
\hat p_T(y) = p_0\left\{\begin{array}{cc} \mathrm{e}^{ky}, & y \leq 0, \\
\mathrm{e}^{-ky}, & y \geq 0 , \end{array}\right.
\end{equation}
%
where $p_0$ is an arbitrary constant, $\mathbf{k} = (k_x, 0, k_z)$ is the wave vector, and $k = \sqrt{k_x^2 + k_z^2}$.

Introducing $\bm \xi = \hat{\bm \xi} \exp[i (k_x x + k_z z)]$, and using Equation \eqref{eq:ptsol}, the $y$-component of Equation \eqref{eq:c3displacement} may be written as
%
\begin{align}
\begin{split}
\label{eq:xi1}
\bigg( \frac{\partial}{\partial t}
& + i k_x U \cos(\Omega t) \cos\theta
- i k_z U \cos(\Omega t) \sin\theta \bigg)^2 \hat \xi_y
\\
& + v_{A i}^2 \left( k_x \sin\theta
+ k_z \cos\theta \right)^2 \hat \xi_y
= - p_0 \frac{k}{\rho_i} \mathrm{e}^{k y},
\end{split}
\end{align}
%
for $y \leq 0$, and
%
\begin{equation}
\label{eq:xi2}
\left( \frac{\partial}{\partial t}
- i k_x U \cos(\Omega t) \right)^2 \hat \xi_y
+ v_{A e}^2 k_z^2 \hat \xi_y
= p_0 \frac{k}{\rho_e} \mathrm{e}^{- k y},
\end{equation}
%
for $y \geq 0$.
Here, $v_{Ai, e}^2 = B_{i, e}^2 / \mu_0\rho_{i, e}$ are the Alfv\'en speeds on either side of the interface.
Considering Equations \eqref{eq:xi1} and \eqref{eq:xi2} at the boundary (i.e. at $y=0$) allows us to eliminate $p_0$ and obtain a single equation for $\hat \xi_y (t)$,
%
\begin{align}
\begin{split}
\label{eq:xi3}
\rho_i \bigg( \frac{\mathrm{d}}{\mathrm{d} t}
& + i k_x U \cos(\Omega t) \cos\theta
- i k_z U \cos(\Omega t) \sin\theta \bigg)^2 \hat \xi_y
\\
& + \rho_e \bigg( \frac{\mathrm{d}}{\mathrm{d} t}
- i k_x U \cos(\Omega t) \bigg)^2 \hat \xi_y
\\[0.1cm]
& + \big[ \rho_i v_{A i}^2 \big( k_x \sin\theta
+ k_z \cos\theta \big)^2
+ \rho_e v_{A e}^2 k_z^2 \big] \hat \xi_y
= 0.
\end{split}
\end{align}
%
Equation \eqref{eq:xi3} may be rearranged such that we obtain the governing equation for the displacement at the boundary,
%
\begin{align}
\begin{split}
\label{eq:c3goveq}
\bigg\{ \frac{\mathrm{d}^2}{\mathrm{d} t^2}
+ 2 i A & \cos(\Omega t) \frac{\mathrm{d}}{\mathrm{d} t}
- i \Omega A \sin(\Omega t)
- B \cos^2(\Omega t) 
+ C \bigg\} \hat \xi_y
= 0,
\\[0.1cm]
A
& = \frac{U \big[ \rho_i ( k_x \cos\theta - k_z \sin\theta ) 
- \rho_e k_x \big]}{\rho_i + \rho_e},
\\[0.1cm]
B
& = \frac{U^2 \big[ \rho_i \left( k_x \cos\theta
- k_z \sin\theta \right)^2
+ \rho_e k_x^2 \big]}{\rho_i + \rho_e},
\\[0.1cm]
C
& = \frac{\rho_i v_{A i}^2 \left( k_x \sin\theta
+ k_z \cos\theta \right)^2
+ \rho_e v_{A e}^2 k_z^2}{\rho_i + \rho_e}.
\end{split}
\end{align}
%
In order to be able to study the stability of Equation \eqref{eq:c3goveq}, we write $\hat \xi_y$ as
%
\begin{equation}
\label{eq:g0}
\hat \xi_y (t) = g(t) \eta(t),
\end{equation}
%
which may be introduced into Equation \eqref{eq:c3goveq}, such that it becomes
%such that the governing equation may be simplified.
%After introducing Equation \eqref{eq:g0} into Equation \eqref{eq:c3goveq}, the latter equation becomes
%
\begin{align}
\begin{split}
\label{eq:g1}
\frac{\mathrm{d}^2 g}{\mathrm{d} t^2} \eta
+ & 2 \frac{\mathrm{d} g}{\mathrm{d} t} \frac{\mathrm{d} \eta}{\mathrm{d} t}
+ g \frac{\mathrm{d}^2 \eta}{\mathrm{d} t^2}
+ 2 i A \cos(\Omega t) ( \frac{\mathrm{d} g}{\mathrm{d} t} \eta
+ g \frac{\mathrm{d} \eta}{\mathrm{d} t})
\\[0.1cm]
- & i \Omega A \sin(\Omega t) g \eta 
- B \cos^2(\Omega t) g \eta
+ C g \eta
= 0.
\end{split}
\end{align}
%
In order to simplify Equation \eqref{eq:g1}, we must find an appropriate condition for $g$ such that the first order derivatives of $\eta$ vanish.
From Equation \eqref{eq:g1}, we deduce that for there to be no first order derivatives of $\eta$ in Equation \eqref{eq:c3goveq}, we require
%
\begin{equation}
\label{eq:g2}
\frac{\mathrm{d} g}{\mathrm{d} t} \frac{\mathrm{d} \eta}{\mathrm{d} t}
+ i A \cos(\Omega t) g \frac{\mathrm{d} \eta}{\mathrm{d} t}
= 0.
\end{equation}
%
The solution of Equation \eqref{eq:g1} is
%
\begin{equation}
\label{eq:g3}
g(t) = \exp\left\{- \frac{i A}{\Omega} \sin(\Omega t)\right\}.
\end{equation}
Note that $|g(t)| = 1$, meaning that it does not affect our local stability analysis.
%
Introducing Equation \eqref{eq:g3} into Equation \eqref{eq:g1} and simplifying the extra terms, we obtain
\begin{equation}
\frac{\mathrm{d}^2 \eta}{\mathrm{d} t^2} + [(A^2 - B)\cos^2(\Omega t) + C] \eta = 0,
\end{equation}
%
which may be rewritten as
%
\begin{equation}
\label{eq:mathieu}
\frac{\mathrm{d}^2 \eta}{\mathrm{d} \tau^2}
+ [a - 2 q \cos(2 \tau)] \eta = 0,
\end{equation}
%
where $\tau = \Omega t$, and
\begin{align}
\begin{split}
\label{eq:aq0}
a & = \frac{1}{2} (A^2 - B) + C,
\\[0.1cm]
q & = \frac{1}{4} (B - A^2).
\end{split}
\end{align}
%

It is now convenient to rewrite the wave vector in terms of its magnitude, $k$, and the angle between the wave vector and the $x$-axis, $\phi$.
Thus,
%
\begin{equation}
\label{eq:kphi}
k_x = k\cos\phi, \quad k_z = k\sin\phi.
\end{equation}
%
Considering Equations \eqref{eq:c3goveq}, \eqref{eq:kphi} and the angle sum formulae, Equation \eqref{eq:aq0} becomes
%
\begin{align}
\begin{split}
\label{eq:aq}
q & = \frac{r \kappa^2 M_A^2[\cos(\theta + \phi) + \cos\phi]^2}{4 (1 + r)^2},
\\
\alpha & = \frac{\kappa^2[\sin^2(\theta + \phi) + r \bar v_A^2 \sin^2\phi]}{1 + r}, 
\\
a & = \alpha - 2 q,
\end{split}
\end{align}
%
where $\tau = \Omega t$\/, $r = \rho_e / \rho_i$ is the density ratio, $M_A = U / v_{Ai}$ is the Alfv\'en Mach number,  $\bar v_A = v_{Ae} / v_{Ai}$ is the ratio of Alfv\'en speeds, and $\kappa = k v_{Ai} / \Omega$ is the dimensionless wavenumber.
From Equation \eqref{eq:aq}, it is straightforward to see that $q$ and $a$ are invariant with respect to the substitution $\phi + \pi \to \phi$.
This enables us to only consider values of $\phi$ in the interval $[-\pi/2, \pi/2]$ when studying the stability of Equation \eqref{eq:mathieu}.

It is important to note that, since $|g(t)| = 1$, the variable substitution does not affect the stability analysis.
Hence, unstable perturbations of the boundary correspond to unstable solutions of Equation \eqref{eq:mathieu}.
Equation \eqref{eq:mathieu} is known as Mathieu's equation \citep{McLachlan1946}.
Mathieu's equation also arises in other MHD problems, namely, it describes the amplification of MHD waves by periodic external forcing \citep[e.g.][]{Zaqarashvili2000,Zaqarashvili2002,Zaqarashvili2005}, and the Rayleigh-Taylor instability of a magnetic interface in the presence of oscillating gravity \citep{Ruderman2018}.

%=============================================================
\section{Investigation of Stability}
\label{sec:stab}
%=============================================================

In this section, we use Equation \eqref{eq:mathieu} to study the stability of the tangential discontinuity with an oscillating shear velocity.
For comparison, we first briefly outline the well-known results related to the stability of a tangential discontinuity separating steady flows.
To the best of our knowledge, these results were first obtained by \cite{Syrovatskii1957} \citep[see also][]{Chandrasekhar1961}.
\vspace*{3mm}

%------------------------------------------------------------------------------
\subsection{Stability of Steady Flows}
\label{subsec:steady}
%------------------------------------------------------------------------------

Before analysing the fully time dependent governing Equation \eqref{eq:mathieu}, we return to Equation \eqref{eq:c3goveq} and set $\Omega = 0$, in order to perform the analysis of the configuration in the presence of steady flows.
Equation \eqref{eq:c3goveq} becomes
%
\begin{equation}
\label{eq:goveqsteady}
\bigg\{ \frac{\mathrm{d}^2}{\mathrm{d} t^2}
+ 2 i A \frac{\mathrm{d}}{\mathrm{d} t}
- B + C \bigg\} \hat \xi_y
= 0.
\end{equation}
%
Since the coefficients in Equation \eqref{eq:goveqsteady} are independent of $t$, we can look for the solution to this equation proportional to $\mathrm{e}^{-i \omega t}$, where $\omega$ is the angular frequency of the perturbation.
We obtain the dispersion relation 
%
\begin{align}
\begin{split}
\label{eq:c3disprel}
(1 + r) \bar c_{ph}^2
& - 2 M_A [ \cos(\theta + \phi) - r \cos\phi ] \bar c_{ph}
+ M_A^2 [ \cos^2(\theta + \phi) + r \cos^2\phi ]
\\[0.1cm]
& - \sin^2(\theta + \phi)
- r \bar v_{A}^2 \sin^2\phi
= 0,
\end{split}
\end{align}
%
where $\bar c_{ph}^2 = \omega^2 / k^2 v_{A i}^2$ is the non-dimensionalised phase speed.

We note that if the roots to Equation \eqref{eq:c3disprel} are real, then $\hat \xi_y (t)$ is oscillatory and the system is neutrally stable.
However, if complex conjugate roots exist, one of the roots has a positive imaginary part, meaning that $|\mathrm{e}^{-i \omega t}| \to \infty$ as $t \to \infty$\/, and the equilibrium configuration is unstable.
Equation \eqref{eq:c3disprel} has complex roots when its discriminant,
%
\begin{align}
\begin{split}
\label{eq:discriminant}
\Delta
& = 4 M_A^2 \big[ \cos(\theta + \phi) - r \cos\phi \big]^2
\\
& - 4 (1 + r) \big\{ M_A^2 [ \cos^2(\theta + \phi) + r \cos^2\phi ]
- \sin^2(\theta + \phi)
- r \bar v_{A}^2 \sin^2\phi
\big\}
\end{split}
\end{align}
%
is negative, which occurs when $M_A > M_{A0}$, where 
%
\begin{equation}
\label{eq:M_A0}
M_{A0}^2 = \frac{(1 + r) [ \sin^2(\theta + \phi) + r \bar v_{A}^2 \sin^2\phi ] }
{r [ \cos(\theta + \phi) + \cos\phi ]^2}.
\end{equation}
%
Considering \eqref{eq:discriminant}, the solutions of Equation \eqref{eq:c3disprel} may be written as
%
\begin{equation}
\label{eq:c3disprelsol}
\omega = \frac{k v_A}{2 (1 + r)}
\bigg[ 2 M_A \big(\cos(\theta + \phi) - r \cos\phi \big) \pm \sqrt{\Delta} \bigg].
\end{equation}
%

Equation \eqref{eq:M_A0} is singular for some specific values of $\phi$ and $\theta$ which are obtained by solving the equation 
\begin{equation}
\label{eq:phising}
\cos\theta - \tan\phi \sin\theta + 1 = 0.
\end{equation}
By inspection, we immediately find that
\begin{equation}
\label{eq:thetasing1}
\theta=(2n+1)\pi,
\end{equation}
satisfies Equation \eqref{eq:phising}, where $n$ is an integer, and $\phi$ is arbitrary.
This is the case when the flows on either side of the interface are parallel and the system cannot be unstable since there is no velocity shear.
Solving Equation \eqref{eq:phising} explicitly, we find a second solution
\begin{equation}
\label{eq:phisingsol}
\phi = \arctan\left( \frac{1 + \cos\theta}{\sin\theta} \right),
\end{equation}
and by using the half angle formula, we find that
\[
\frac{1 + \cos\theta}{\sin\theta} = \frac{1 - \cos\left((2n+1)\pi - \theta\right)}{\sin\left((2n+1)\pi - \theta\right)} = \tan\left(\frac{(2n+1)\pi - \theta}{2} \right).
\]
Equation \eqref{eq:phisingsol} may, thus, be written as
\begin{equation}
\label{eq:thetasing2}
\theta = (2n+1)\pi - 2 \phi.
\end{equation}
Equations \eqref{eq:thetasing1} and \eqref{eq:thetasing2} define the lines in the $\phi\theta$-plane where solutions to Equation \eqref{eq:c3disprel} are stable regardless of $M_A$.

The minimum value of $M_{A0}$ in terms of $\theta$ may be obtained by differentiating Equation \eqref{eq:M_A0} with respect to $\phi$ and solving the resulting equation,
\begin{equation}
\label{eq:phimineq}
\tan\phi(\cos\theta + r \bar v_A^2) + \sin\theta = 0.
\end{equation}
Equation \eqref{eq:phimineq} has the solution
%
\begin{equation}
\label{eq:phimin}
\phi = \phi_0 \equiv -\arctan \left(\frac{ \sin\theta}{\cos\theta + r \bar v_{A}^2}\right).
\end{equation}
%
Substituting Equation~\eqref{eq:phimin} into Equation \eqref{eq:M_A0}, yields the minimum value of $M_{A0}$,
%
\begin{equation}
\label{eq:M_A0min}
\min \{ M_{A0}^2 \} = \frac{\bar v_{A}^2 (1 + r)\tan^2(\theta/2)}{1 + r \bar v_{A}^2} .
\end{equation}
%
It follows that the system is stable for any value of $M_A$ below $\min\{M_{A0}\}$, while there are always unstable perturbations when $M_A > \min\{M_{A0}\}$.
The value of $\min \{ M_{A0}^2 \}$ is plotted with respect to $\theta$ in Figure \ref{fig:M_A0_min}.
Although for applications to transverse loop oscillations we only consider $\theta \ll 1$, here we included a wider range of values for completeness.

\begin{figure}[t]
\centering
\includegraphics[width=0.9\textwidth]{figures/MA0_min}
\caption{The minimum value of $M_{A0}$ with respect to $\theta$, for $\bar v_A^2 = r^{-1} = 3$.}
\label{fig:M_A0_min}
\end{figure}

\begin{figure}[t]
\centering
\includegraphics[width=0.9\textwidth]{figures/MA0}
\caption{Contour plot of $M_{A0}$ with respect to $\phi$ and $\theta$, for $\bar v_A^2 = r^{-1} = 3$.}
\label{fig:M_A0}
\end{figure}

The value of $M_{A 0}$ is illustrated as a contour plot with respect to $\phi$ and $\theta$ in Figure \ref{fig:M_A0}.
Both the singular values obtained in Equations \eqref{eq:thetasing1} and \eqref{eq:thetasing2} are found in this figure.
Furthermore, for the values of $\bar v_A = \sqrt{3}$ and $r = 1/3$, which were considered here, Equation \eqref{eq:phimin} reduces to $\phi_0 = (2 n + 1) \pi -\theta/2$.

One final remark must be made regarding the stability of solutions of Equation \eqref{eq:c3disprel}.
Since $\omega$ is proportional to $k$, as shown in Equation \eqref{eq:c3disprelsol}, it follows that the instability growth rate is also proportional to $k$.
This implies that the growth rate tends to infinity as $k \to \infty$.
Since the growth rate of the instability is unbounded, we say that the initial value problem describing the evolution of the boundary is \emph{ill-posed}.
If the growth rate were bounded as $k \to \infty$, we would have said that the initial value problem were \emph{well-posed}.
In the case of the stability of Equation \eqref{eq:mathieu} for the oscillatory flows defined in Equation \eqref{eq:c3equilibrium}, the question of whether the initial value problem is well- or ill-posed is discussed in Subsection \ref{subsec:ivp}.

%------------------------------------------------------------------------------
\subsection{Stability of Oscillating Flows}
\label{subsec:oscillating}
%------------------------------------------------------------------------------

We now use Equation~\eqref{eq:mathieu} to study the stability for arbitrary values of the equilibrium quantities.
Floquet's theorem states that Equation~\eqref{eq:mathieu} has a solution of the form
%
\[
\eta_+(\tau) = \mathrm{e}^{\mu \tau} P(a, q,\tau) ,
\]
%
where $\mu = \mu(a, q)$ is the characteristic exponent, and $P(a, q,\tau)$ is a periodic function in $\tau$, with period $\pi$ \citep[see, e.g.,][]{McLachlan1946, Abramowitz1965}.
Since Equation~\eqref{eq:mathieu} is invariant with respect to the substitution $-\tau \to \tau$ it follows that $\eta_-(\tau) = \mathrm{e}^{-\mu \tau} P(a, q,-\tau)$ is also a solution to this equation.
Then, the general solution to Equation~\eqref{eq:mathieu} is the linear combination of $\eta_+(\tau)$ and $\eta_-(\tau)$ unless $i\mu$ is an integer number.

The parameter $\mu$ determines the nature of solutions to Mathieu's equation. 
We may always assume that $\Re(\mu) > 0$, unless $\mu$ is purely imaginary, where $\Re$ indicates the real part of a quantity. Since we may write 
\[
\mathrm e^{\mu \tau} = \exp(\Re(\mu) \Omega t)\exp(i\Im(\mu) \Omega t),
\]
where $\Im$ indicates the imaginary part of a quantity, it follows that purely imaginary values of $\mu$ correspond to neutrally stable solutions, while real and complex values correspond to unstable solutions.
Hence, $\Re(\mu) > 0$ corresponds to an unstable perturbation.
Unfortunately, $\mu$ cannot be easily computed analytically, and, for this reason, we perform a numerical analysis to gain further insight.

\begin{figure}[!ht]
\centering
\subfloat[]
{\includegraphics[width=0.95\textwidth]{figures/stability_diagram}}
\\
\subfloat[]
{\includegraphics[width=0.85\textwidth]{figures/growth_diagram}}
\caption{The stability diagram for solutions to Mathieu's equation (a).
Solutions are stable/unstable for $(q, a)$ in the white/hatched region.
%The curves $a = a_j(q)$ and $a = b_j(q)$ are shown by solid and dotted lines, respectively. The blue, green, and red straight lines correspond to $K \approx 4$, $K \approx - 0.2$, and $K = -2$, respectively.
In (b), the real part of $\mu$ is plotted for $q > 0$.}
\label{fig:stability_diagram}
\end{figure}

Following \cite{McLachlan1946}, we plot the stability diagram of Equation~\eqref{eq:mathieu} in the $qa$\/-plane (Figure~\ref{fig:stability_diagram}a).
In accordance with the definition of $q$ in Equation \eqref{eq:aq}, we only consider $q > 0$.
The white and hatched regions correspond to purely imaginary and real/complex values of $\mu$, respectively, and thus, to stable and unstable solutions to Equation~\eqref{eq:mathieu}.
The contours bounding the regions are defined by the condition that $i\mu$ is an integer number, so that Equation~\eqref{eq:mathieu} has either $\pi$ or $2\pi$\/-periodic solutions when the point $(q,a)$ is on one of these contours.
These contours are called the characteristic curves, and are defined by the equations $a = a_j(q)$ and $a = b_j(q)$.
These functions satisfy the inequalities $a_j < b_{j+1} < a_{j+1}$\/, where $j = 0,1,2,\dots$.
The curves $a_j(q)$ and $b_j(q)$ are shown by solid and dotted lines, respectively, in Figure~\ref{fig:stability_diagram}a. The asymptotic behaviour of $a_j(q)$ and $b_{j+1}(q)$ for large $q$ is given by $a_j(q) \sim b_{j+1}(q) \sim -2q$ \citep{Abramowitz1965}.

Complementary to the above, Figure~\ref{fig:stability_diagram}b shows the values of the characteristic exponent $\mu$. Purely imaginary solutions are plotted in white, and are separated from real/complex solutions by the characteristic curves, while the real part of $\mu$ is plotted in contours in the unstable regions.

\begin{figure}[t]
\centering
\includegraphics[width=0.85\textwidth]{figures/stability_diagram_lines}
\caption{The stability diagram for solutions to Mathieu's equation, for three possible values of $K$.
The curves $a = a_j(q)$ and $a = b_j(q)$ are shown by solid and dotted lines, respectively, as in Figure \ref{fig:stability_diagram}a.
The blue, green, and red straight lines correspond to $K \approx 4$, $K \approx - 0.2$, and $K = -2$, respectively.}
\label{fig:stability_diagram_lines}
\end{figure}

The coefficients in Equation~\eqref{eq:mathieu} depend on six dimensionless parameters.
Four of these parameters, $r$, $\theta$, $M_A$, and $\bar v_A$, are only dependent on the equilibrium quantities, while the other two, $\kappa$, and $\phi$, are related to particular perturbations, and are thus arbitrary.
Hence, we must study the behaviour of solutions to Equation \eqref{eq:mathieu} for all possible values of these two parameters.
When $\phi$ is fixed and $\kappa$ varies from 0 to $\infty$, Equations \eqref{eq:aq} describe a straight line in the $qa$-plane.
The equation of this line may be written as 
%
\begin{equation}
\label{eq:aqK}
a = Kq, \quad K = \frac{4 M_{A 0}^2}{M_A^2} - 2.
\end{equation}
%
From Equations \eqref{eq:M_A0} and \eqref{eq:aqK}, we note that $K > -2$ for any $\theta \neq 0$ and any values of the other parameters.
Considering the asymptotic behaviours of the characteristic curves, it follows that the line $a = K q$ always intersects all curves $a = a_j(q)$ and $a = b_{j+1}(q)$, for $j = 0,1,\dots$.
Hence, there always exist some values of $\kappa$ and $\phi$ for which perturbations are unstable, regardless of the values of the other parameters.
This implies that the tangential discontinuity separating oscillating flows is unstable for any value of $M_A$, which is qualitatively different from the discontinuity separating steady flows considered in Subsection \ref{subsec:steady}.
In the case of no magnetic shearing, i.e. when $\theta = 0$, perturbations with $\phi = \phi_0 = 0$ and any $\kappa$ are unstable since the line $a = K q$ will always be under the curve $a_0(q)$.
This is illustrated by the red line in Figure \ref{fig:stability_diagram_lines}.
The green and blue lines in Figure \ref{fig:stability_diagram_lines} correspond to $\theta = 0.5^\circ$ and $\theta = 1^\circ$, respectively, and $\phi = \phi_0$.
The straight lines in Figure \ref{fig:stability_diagram_lines} are further discussed in Subsection \ref{subsec:sigma}.

%------------------------------------------------------------------------------
\subsection{The Initial Value Problem}
\label{subsec:ivp}
%------------------------------------------------------------------------------

In Subsection \ref{subsec:oscillating}, it was demonstrated that all solutions of Equation \eqref{eq:mathieu}, with $q$ and $a$ satisfying Equation \eqref{eq:aq}, are unstable for arbitrary $k$.
Recall from the definition in Subsection \ref{subsec:steady} that the \emph{initial value problem} for Equation \eqref{eq:mathieu} is said to be \emph{well-posed} if the growth rate of the instability is bounded as $k \to \infty$, and \emph{ill-posed} if the growth rate is unbounded as $k \to \infty$.
The unbounded growth problem as $k \to \infty$ consequence of neglected dissipation, which scales like $k^2$.

In Subsection \ref{subsec:steady} it was shown that, if the flows on each side of the boundary are steady, the configuration is unstable for $M_A > \min\{M_{A0}\}$ and stable for $M_A < \min\{M_{A0}\}$.
Furthermore, the initial value problem for $M_A > \min\{M_{A0}\}$ was shown to be ill-posed.
In the current Subsection, we prove that the initial value problem for Equation \eqref{eq:mathieu}, with $q$ and $a$ satisfying Equation \eqref{eq:aq}, is ill-posed for $M_A > \min\{M_{A0}\}$, and well-posed for $M_A < \min\{M_{A0}\}$.

In order to prove that the initial value problem is ill-posed for $M_A > \min\{M_{A0}\}$, we use the comparison theorem for second order linear ordinary differential equations, also called the Sturm-Picone comparison theorem \citep[e.g.][]{Coddington1955}.
This theorem states that, given two equations of the form
%
\begin{equation}
\label{eq:theoremeq}
\frac{\mathrm{d}^2 f}{\mathrm{d} t^2} + g_{1, 2}(t) f = 0,
\end{equation}
%
where $g_{1,2}(t)$ are piecewise continuous functions on an interval $[t_0, t_1]$, with $g_1 \geq g_2$, it may be shown that $f_1 \geq f_2$, where $f_{1, 2}(t)$ are the solutions of the two Equations \eqref{eq:theoremeq}.
This theorem is only valid if the two Equations \eqref{eq:theoremeq} have identical initial conditions.

Consider $M_A > M_{A0}$, such that $K < 2$, as may be seen from Equation \eqref{eq:aqK}.
The scaled variables $\tilde{a} = \kappa^{-2} a$\/, $\tilde{q} = \kappa^{-2} q$, and $\tilde{\tau} = \kappa\tau$ are introduced, and Equation~\eqref{eq:mathieu} is rewritten as
%
\begin{equation}
\label{eq17}
\frac{\mathrm{d}^2 \eta}{\mathrm{d}\tilde{\tau}^2}
+ [\tilde{a} - 
2\tilde{q}\cos(2\tilde{\tau}/\kappa)] \eta = 0. 
\end{equation}
%
It is important to note that $\tilde{a}$ and $\tilde{q}$ are independent of $\kappa$, and also that $\tilde{a} = K\tilde{q}$.
The aim, now, is to find some Equation of the form 
%
\begin{equation}
\label{eq:eqncomparison}
\frac{\mathrm{d}^2 \eta}{\mathrm{d}\tilde{\tau}^2} + g \eta = 0. 
\end{equation}
%
such that
%
\begin{equation}
\label{eq:ivp1}
\tilde{a} - 2\tilde{q}\cos(2\tilde{\tau}/\kappa) \geq g,
\end{equation}
%
on the interval $\tilde{\tau} \in [0,\tilde{\tau}_0]$, where $\tilde{\tau}_0$ is to be determined.

Let $g = - 4 h^2 \tilde q$.
For $\tilde \tau = 0$, the inequality
%
\begin{equation}
\label{eq:ivp2}
2 - K > 4 h^2,
\end{equation}
%
should be satisfied, which means that we may write
%
\begin{equation}
\label{eq:ivp3}
h = \frac{1}{2}\sqrt{1 - K/2}.
\end{equation}
%
For $\tilde \tau = \tilde \tau_0$, the equality
%
\begin{equation}
\label{eq:ivp4}
\tilde{a} - 2\tilde{q}\cos(2\tilde{\tau}/\kappa) = - 4h^2\tilde{q},
\end{equation}
%
should be satisfied.
Equation \eqref{eq:ivp4} may be rearranged as
%
\begin{equation}
\label{eq:ivp5}
\tilde \tau_0 = \frac{1}{2} \kappa \arccos(1/2 + K/4) = \kappa \arcsin h
\end{equation}
%
From Equations \eqref{eq:ivp2}, \eqref{eq:ivp3}, \eqref{eq:ivp4}, and \eqref{eq:ivp5} it follows that 
%
\begin{equation}
\label{eq:ivp6}
\tilde{a} - 2\tilde{q}\cos(2\tilde{\tau}/\kappa) \geq - 4h^2\tilde{q},
\end{equation}
%
for $\tilde{\tau} \in [0,\tilde{\tau}_0]$, where $\tilde \tau_0 = \kappa \arcsin h$.
Both the left-hand and right-hand sides of Equation \eqref{eq:ivp6} are continuous on the interval $[0,\tilde{\tau}_0]$.

In order to be able to use the comparison theorem, the equation
%
\begin{equation}
\label{eq:ivp7}
\frac{\mathrm{d}^2 \eta}{\mathrm{d}\tilde{\tau}^2} - 4h^2\tilde{q}\eta = 0, 
\end{equation}
%
is considered.
A solution to Equation \eqref{eq:ivp7} is
%
\begin{equation}
\label{eq:eta1}
\eta_1 = \eta_0\exp(2h\tilde{q}^{1/2}\tilde\tau) = 
\eta_0\exp\big(\tau\sqrt{q(1 - K/2)}\big),
\end{equation}
%
where $\eta_0$ is an arbitrary constant.
From Equation \eqref{eq:eta1} and the definition of $q$ in Equation \eqref{eq:aq}, it follows that $\eta_1$ is unbounded as $\kappa \to \infty$.
The solution in Equation \eqref{eq:eta1} satisfies the initial conditions
%
\begin{equation}
\label{eq:etaics}
\eta_1 = \eta_0, \quad \frac{\mathrm{d}\eta_1}{\mathrm{d}\tilde{\tau}} =
2h\eta_0\tilde{q}^{1/2} \quad \mbox{at} \;\; \tilde{\tau} = 0.
\end{equation}
%
We also consider a solution $\eta_2$ to Equation~(\ref{eq17}) satisfying the same initial conditions.
Then, it follows from Equation \eqref{eq:ivp6} and the comparison theorem that $\eta_2 \geq \eta_1$ for $\tilde{\tau} \in [0,\tilde{\tau}_0]$.
The initial conditions, Equation \eqref{eq:etaics}, may be rewritten for $\eta_2$ as
%
\begin{equation}
\label{eq:eta2}
\eta_2 = \eta_0, \quad \frac{\mathrm{d}\eta_2}{\mathrm{d}\tau} =
2h\eta_0\kappa^{-1} q^{1/2} \quad \mbox{at} \;\; \tilde{\tau} = 0.
\end{equation}
%
Considering Equation \eqref{eq:eta2}, it is straightforward that $\eta_2$ is bounded at $\tau = 0$ for $\kappa \in (0,\infty)$.
From the definition of $\mathrm{d}\eta_2/\mathrm{d}\tau$ in Equation \eqref{eq:eta2} and the definition of $q$ in Equation \eqref{eq:aq}, it follows that $\mathrm{d}\eta_2/\mathrm{d}\tau$ is also bounded at $\tau = 0$ for $\kappa \in (0,\infty)$.
Then, it follows from the inequality $\eta_2 \geq \eta_1$ and Equation \eqref{eq:eta1} that, for any $\tau \in (0,\arcsin h)$, there is such a solution to Equation~(\ref{eq17}) that it is bounded together with its first derivative at $\tau = 0$ for any value of $\kappa$\/, but it is unbounded at $\tau = \tau_0$ as $\kappa \to \infty$.
Hence, the instability growth rate is unbounded and the initial value problem describing the evolution of the perturbed discontinuity is ill-posed when $M_A > \min \{ M_{A0} \}$.

Now, we assume that $M_A < M_{A0}(\phi)$, so that, in accordance with Equation \eqref{eq:aqK}, $K > 2$ and $a > 2q$.
We calculate the instability growth rate for $\kappa \gg 1$.
Let $\bar\eta(\tau)$ be the solution to Equation~\eqref{eq:mathieu}, satisfying the initial conditions
%
\begin{equation}
\label{eq23}
\bar\eta = 1, \quad \frac{\mathrm{d}\bar\eta}{\mathrm{d}\tau} = 0 \quad
\mbox{at} \quad \tau = 0. 
\end{equation}
%
Then, the characteristic exponent is defined by the equation \citep{Abramowitz1965}
%
\begin{equation}
\label{eq24}
\cosh(\pi\mu) = \bar\eta(\pi).
\end{equation}
%
We use the WKB method and look for a solution to Equation~\eqref{eq:mathieu} in the form  $\eta_+ = \mathrm{e}^{\kappa\Theta}$\/. Substituting this expression into Equation~\eqref{eq:mathieu} we obtain
%
\begin{equation}
\label{eq25}
\kappa^{-1}\frac{\mathrm{d}^2\Theta}{\mathrm{d}\tau^2} + 
\left(\frac{\mathrm{d}\Theta}{\mathrm{d}\tau}\right)^2 +
\tilde{a} - 2\tilde{q}\cos(2\tau) = 0. 
\end{equation}
%
From Equation \eqref{eq23}, it follows that the initial conditions for Equation \eqref{eq25} are
%
\begin{equation}
\Theta = 0, \quad \frac{\mathrm{d} \Theta}{\mathrm{d}\tau} = 0 \quad
\mbox{at} \quad \tau = 0.
\end{equation}
%
We look for the solution of Equation \eqref{eq25} as the perturbation expansion
 %
\begin{equation}
\label{eq26}
\Theta = \Theta_1 + \kappa^{-1}\Theta_2 + \dots 
\end{equation}
%
Substituting this expansion into Equation~\eqref{eq25} and collecting terms of the order of unity we obtain
%
\begin{equation}
\label{eq27}
\left(\frac{\mathrm{d}\Theta_1}{\mathrm{d}\tau}\right)^2 =
2\tilde{q}\cos(2\tau) - \tilde{a}. 
\end{equation}
%
The solution to this equation satisfying the condition $\Theta_1 = 0$ at $\tau = 0$ is
%
\begin{equation}
\label{eq28}
\Theta_1 (\tau) = i\int_0^\tau\sqrt{\tilde{a} - 2\tilde{q}\cos(2\tau')}\,\mathrm{d}\tau',
\end{equation}
%
where we chose the plus sign of the square root.
It is clear that the function being integrated in Equation \eqref{eq28} is even, meaning that its integral over the interval $(0, x)$ is odd.
Therefore, $\Theta_1 (\tau)$ is an odd function.

In the next order approximation we collect terms of the order of $\kappa^{-1}$ in Equation~\eqref{eq25} to obtain
%
\begin{equation}
\label{eq29}
\frac{\mathrm{d}^2\Theta_1}{\mathrm{d}\tau^2} + 
\frac{\mathrm{d}\Theta_1}{\mathrm{d}\tau}
\frac{\mathrm{d}\Theta_2}{\mathrm{d}\tau} = 0. 
\end{equation}
%
Using Equation~\eqref{eq28} we find that the solution to this equation satisfying the condition $\Theta_2 = 0$ at $\tau = 0$ is
%
\begin{equation}
\label{eq30}
\Theta_2 (\tau) = -\frac{1}{2} \ln \bigg( \frac{\tilde{a} - 2\tilde{q}\cos(2\tau)}{\tilde{a} - 2\tilde{q}} \bigg).
\end{equation}
%
Since the function inside the natural logarithm in Equation \eqref{eq30} is even and the logarithm does not affect the parity, it follows that $\Theta_2(\tau)$ is an even function.

Recall that $\eta_-(\tau) = \eta_+(-\tau)$ is also a solution to Equation~\eqref{eq:mathieu}. Then, since $\Theta_1(\tau)$ is an odd function and $\Theta_2(\tau)$ is an even function, it follows that 
%
\begin{equation}
\label{eq31}
\bar\eta = \frac{\eta_+ + \eta_-}2 = 
\mathrm{e}^{\Theta_2}\cos(\kappa\Theta_1) + {\cal O}\big(\kappa^{-1}\big).
\end{equation}
% 
Introducing the notation $\chi = \Theta_1(\pi)$ and $\gamma = \Theta_2(\pi)$ we transform Equation \eqref{eq:mathieu} to
%
\begin{equation}
\label{eq32}
\cosh(\pi\mu) = \mathrm{e}^\gamma\cos(\kappa\chi).
\end{equation}
%
When the absolute value of the right-hand side of this equation does not exceed unity the two values of $\mu$ satisfying this equation are purely imaginary and the corresponding wave mode is neutrally stable. When the absolute value of the right-hand side is larger than unity one of the two values of $\mu$ satisfying this equation has positive real part and the corresponding wave mode grows exponentially.
However, we can observe that the right-hand side of Equation~\eqref{eq32} is bounded for any $\kappa$.
This implies that the real part of $\mu$ is also bounded, and the same is true for the growth rate.
We made this conclusion for a particular value of $\phi$ and $M_A < M_{A0}(\phi)$.
If we now assume that $M_A < \min \{ M_{A0} \}$, then the growth rate of any wave mode is bounded.
This means that the initial value problem describing the evolution of the discontinuity is well-posed when $M_A < \min \{ M_{A0} \}$.
From Equation \eqref{eq:M_A0min} we see that this condition may be written in the approximate form as
%
\begin{equation}
\label{eq33}
M_A < \frac{\bar v_A\theta}2\sqrt{\frac{1 + r}{1 + r \bar v_A^2}}, 
\end{equation}
%
since, typically, $\theta \ll 1$.

%============================================================
\section{Application to Transverse Coronal Loop Oscillations}
\label{sec:loop}
%============================================================

The aim of this section is twofold.
First, we further elaborate the analysis of Section~\ref{sec:stab} by considering the $\sigma$-stability of Equation \eqref{eq:mathieu}.
Afterwards, we apply some of the results obtained in Subsections~\ref{subsec:oscillating} and \ref{subsec:ivp} to the stability of coronal loop oscillations.

%------------------------------------------------------------------------------
\subsection{The \texorpdfstring{$\sigma$}{TEXT}-stability}
\label{subsec:sigma}
%------------------------------------------------------------------------------

We, now, use the concept of $\sigma$-stability, first introduced by \cite{Goedbloed1974} and \cite{Sakanaka1974}.
This concept is used in studies of thermonuclear plasma confinement where it is necessary that perturbation amplitudes remain sufficiently small on some relevant time scale.
An equilibrium is $\sigma$-stable if the amplitudes of unstable perturbations grow at most like $\exp(\sigma t)$.

We apply the concept of $\sigma$-stability to the analysis of the KH instability induced by transverse oscillations of solar coronal loops.
We say that a transverse coronal loop oscillation is $\sigma$-stable if the growth time of the KH instability exceeds the damping time due to resonant absorption.
Let $t_D = \alpha P$ be the damping time, where $P = 2 \pi / \Omega$ is the oscillation period, and $\alpha$ varies from 1 to 5 \citep[see, e.g.,][]{Goddard2016}.
It follows from our definition that $\sigma = 1/ \Omega t_D$, or
%
\begin{equation}
\label{eq34}
\sigma = \frac{1}{2\pi\alpha}.
\end{equation}
%
When $\alpha$ varies from 1 to 5, $\sigma$ decreases from approximately 0.16 to 0.03.
We see that, in any case, the interface cannot be $\sigma$\/-stable if the maximum growth rate exceeds 0.16, which implies that if the interface is $\sigma$\/-stable then the increment is much less than unity.
It is shown in Appendix~\ref{sec:appendix} that, in this case, the maximum growth rate for fixed $\phi$ is approximately equal to $1/2K$.
Then, the maximum growth rate for all values of $\phi$ is $1/2K_m$\/, where $K_m = \min_\phi K$.
Hence, the $\sigma$\/-stability condition reads
%
\begin{equation}
\label{eq35}
K_m \geq \frac1{2\sigma}, \quad K_m = \frac{4\min\{M_{A 0}^2\}}{M_A^2} - 2.
\end{equation}
%
To estimate $K_m$ we take as typical values $r = 1/3$ and $\bar v_A^2 = 3$.
Then, using Equations~\eqref{eq:M_A0min} and \eqref{eq35}, and taking into account the fact that, typically, $\theta \ll 1$, we reduce the $\sigma$-stability criterion to
%
\begin{equation}
\label{eq36}
\theta \geq \frac{M_A}{2} \sqrt{4 + \frac1\sigma}.
\end{equation}
%
The typical displacement of a kink-oscillating coronal loop is of the order of the loop radius.
Then, the ratio of the velocity to $v_{Ai}$ is of the order of the loop radius and length.
Hence, the typical value is $M_A = 0.01$.
Now, it follows from Equation~\eqref{eq36} that the interface is $\sigma$\/-stable if $\theta \gtrsim 1^\circ$ for $\alpha = 1$, and $\sigma$\/-stable if $\theta \gtrsim 2^\circ$ for $\alpha = 5$.
Even the maximum value $\theta = 2^\circ$ corresponds to only about a half-turn of magnetic field lines from one loop footpoint to the other.
Hence, the loop boundary is $\sigma$\/-stable for a very moderate magnetic twist.

\begin{figure}[t]
\centering
\includegraphics[width=0.9\columnwidth]{figures/tilt_exponent.png}
 \caption{The growth rate of the instability, $\mu$, plotted with respect to $q$. The red, green, and blue lines correspond to the lines in Figure \ref{fig:stability_diagram}a}
 \label{fig:mu}
\end{figure}

In Figure \ref{fig:mu}, we present the values of $\mu$ associated with the three straight lines in Figure \ref{fig:stability_diagram}.
We assumed that $r = 1/3$, $\bar v_A^2 = 3$, $M_A = 0.01$, and $\phi = \phi_0$ so that $K = K_m$\/.
For $\theta = 0$, $\mu$ is a monotonically increasing function of $\kappa$, and perturbations with any $q$ are unstable.
The green curve corresponds to $\theta = 0.5^\circ$, and is unbounded as $\kappa \to \infty$ since $\min M_{A0} \approx 0.0062 < M_A$.
Finally, the blue curve, which corresponds to $\theta = 1^\circ$, is bounded for $\kappa \in (0,\infty)$ since $\min M_{A0} \approx 0.0123 > M_A$.
The equation of the dashed line is $\mu = 0.16$, and we see that the loop with $\theta = 1^\circ$ is $\sigma$-stable for $\sigma$ defined in Equation~\eqref{eq34} with $\alpha = 1$.

We note that if a magnetic loop is $\sigma$-stable, then the initial value problem describing the evolution of its boundary perturbation is well-posed.
However, the converse is not always true.
The initial value problem is well-posed if the growth rate is bounded, but it may still be very large.
On the other hand, a magnetic loop is $\sigma$-stable when the maximum growth rate is below a definite and, usually, sufficiently small number.

%--------------------------------------------------------------
\subsection{Coronal Loop Parameters}
\label{subsec:loop}
%--------------------------------------------------------------

\begin{figure}[!ht]
\centering
\subfloat[$n = 1$]
{\includegraphics[width=0.85\textwidth]{figures/loop_tilt_stability}}
\\
\subfloat[$n = 4$]
{\includegraphics[width=0.85\textwidth]{figures/loop_tilt_stability_n}}
\caption{
The dependence of the solutions on $m$ in the $qa$-plane, for $M_A = 0.01$, $r = 1/3$, 
$\bar{v}_A^2 = 3$, $n=1$ (top) and $n=4$ (bottom).
The red, green and blue dots correspond to increasing degrees of twist.
}
\label{fig:stability_loop}
\end{figure}

\begin{figure}[!ht]
\centering
\subfloat[$\theta=0^\circ$]
{\includegraphics[width=0.9\textwidth]{figures/loop_tilt_growth_theta=0.png}}
\\
\subfloat[$\theta=0.5^\circ$]
{\includegraphics[width=0.9\textwidth]{figures/loop_tilt_growth_theta=05.png}}
\caption{
The dependence of the growth rate on $m$ for $M_A = 0.01$, $r = 1/3$, 
$\bar{v}_A^2 = 3$, $\theta=0^\circ$ (top) and $\theta=0.5^\circ$ (bottom).
The red and blue dots correspond to $n=1$ and $n=4$, respectively.
}
\label{fig:growth_loop}
\end{figure}

The model that we outlined in the previous sections can be only applied for the local analysis of the stability of the boundary of an oscillating magnetic tube.
In this analysis, we can consider oscillations with the characteristic scale in the azimuthal direction that is much smaller than the tube radius $R$\/, and   the characteristic scale in the axial direction that is much smaller than the tube length $L$\/.
Hence, we take 
%
\begin{equation}
\label{eq37}
k_x = \frac mR, \quad k_z = \frac{\pi n}L,
\end{equation}
%
where $m$ and $n$ are sufficiently large integer numbers. Using Equations~\eqref{eq:kphi} and \eqref{eq37} we obtain 
%
\begin{equation}
\label{eq38}
k^2 = \frac{m^2}{R^2} +  \frac{\pi^2 n^2}{L^2}, \quad 
\tan\phi = \frac{\pi nR}{mL} . 
\end{equation}
%
We assume that $n \lesssim |m|$. Since in coronal magnetic loops $R \ll L$, it follows that we may use the approximate expressions
%
\begin{equation}
\label{eq39}
k \approx \frac{|m|}R , \quad \phi \approx \frac{\pi nR}{mL} . 
\end{equation}
%
Throughout this section we assume that $\bar v_A^2 = r^{-1}$. This assumption holds if the magnitudes of the interior and exterior magnetic fields are equal, which is typically true for coronal loops. We also assume that $\theta \ll 1$.
Then, we obtain the approximate expressions
%
\begin{equation}
\label{eq40}
M_{A 0}^2 = \frac{1 + r}{4r}\bigg[\left(\theta + \frac{\pi nR}{mL}\right)^2 +
\frac{\pi^2 n^2 R^2}{m^2 L^2}\bigg] ,
\end{equation}
%
\begin{equation}
\label{eq41}
\min \{ M_{A0}^2 \} = \frac{(1 + r)\theta^2}{8r} .
\end{equation}
%
The condition $M_A <  \min \{ M_{A0}^2 \}$ gives 
%
\begin{equation}
\label{eq42}
\theta > M_A\sqrt{ \frac{8r}{1 + r}} .
\end{equation}
%
If we take $r = 1/3$, the right-hand side of this inequality is approximately equal to $M_A$, that is it is of the order of 0.01.
Hence, the inequality~\eqref{eq42} can be satisfied even for quite moderated twist.
If the inequality is satisfied, then the IVP describing the evolution of the tube boundary is well-posed and the growth rate of perturbations is bounded.

In Figures \ref{fig:stability_loop} and \ref{fig:growth_loop}, we show the dependence of the growth rate on $m$ for $n = 1$ (left) and $n=4$ (right), $M_A = 0.01$, $r = 1/3$, $\bar{v}_A^2 = 3$, $R/L = 200$, and $\theta = 0^\circ$ (red), $\theta = 0.5^\circ$ (green) and $\theta = 1^\circ$ (blue).
We note that, obviously, $n = 1$ does not satisfy the condition that $n$ is large, so we considered $n = 1$ only for comparison.
While, for $n=1$, the points in the $qa$-plane corresponding to $\theta = 0^\circ$ are virtually unchanged as compared to the line in Figure \ref{fig:stability_diagram_lines}, for $n=4$ they are shifted upwards considerably.
This is also the case for $\theta = 0.5^\circ$.
We see that for $n=1$ there are some modes which are unstable in the range selected, for $n=4$ there are no such modes.
There may be unstable modes for $\theta = 0.5^\circ$ and $n=4$, but only for very large $m$.
In terms of the IVP, for $\theta = 1^\circ$, corresponding to a well-posed solution, no value of $m$ corresponds to an unstable solution in the $qa$-plane.
In general, well-posed solutions seem to be unstable only for very large $m$.
These results are significant since they suggest that very localised longitudinal perturbations of the flux tube are generally more stable.

%============================================================
\section{Summary and Discussion}
\label{sec:sum}
%============================================================

In this work, we performed the first analytical study of the transverse wave induced Kelvin-Helmholtz instability of solar coronal loops. We modelled the region on the loop boundary where the shear flows are the greatest as a tangential discontinuity separating time-periodic counter-streaming flows.
To model the magnetic twist in coronal loops we assumed that the equilibrium magnetic fields on either side of the discontinuity are not parallel.
The flow velocities at the two sides of the discontinuity have opposite directions and equal magnitudes oscillating harmonically.
For the sake of mathematical simplicity, we assumed that the plasma on both sides of the interface is incompressible.
Using the linearised set of ideal MHD equations, we derived the governing equation describing the evolution of the shape of the tangential discontinuity, known as Mathieu's equation.

We employed Mathieu's equation to study the stability of the discontinuity.
For comparison, we first presented the results of the stability analysis in the case of steady flows, which we obtained by setting the flow oscillation frequency to zero.
In this case, the stability of the discontinuity is determined by the Alfv\'en Mach number, which is defined as the ratio of the background velocity magnitude to the Alfv\'en speed at one side of the interface.
The discontinuity is unstable when the Alfv\'en Mach number exceeds a critical value, and the instability growth rate is proportional to the wavenumber, and thus unbounded.
This implies that the initial value problem describing the evolution of the perturbed discontinuity is ill-posed.
We note that the critical Alfv\'en number is zero when there is no magnetic shear. 

In contrast to the interface separating steady flows, the tilted magnetic field cannot stabilise the discontinuity if the flows oscillate. A similar result was obtained by \cite{Roberts1973} in the case of MHD tangential discontinuity with the magnetic field having the same direction at both sides and the flow velocity parallel to the magnetic field.

Even though the interface is always unstable, the critical Alfv\'en Mach number still plays an important role in the stability properties.
We showed that the growth rate of the instability is unbounded when the Alfv\'en Mach number exceeds the instability threshold, and thus the initial value problem is ill-posed.
Hence, in this case the stability properties are qualitatively the same as in the case of steady flows.
On the other hand, when the Alfv\'en Mach number is below its critical value, the instability increment is bounded, and the initial value problem is well-posed.

In Section \ref{subsec:sigma}, we applied the concept of $\sigma$-stability to kink oscillating coronal loops, which states that the loop is $\sigma$-stable if the growth time of the instability exceeds the resonant damping time of the transverse oscillation.
We obtained the criterion for the $\sigma$-stability and showed that, for parameters typical for transverse coronal loop oscillations, even moderate magnetic twist makes the loop boundary $\sigma$-stable.

In Section \ref{subsec:loop}, we used our model to perform a local stability analysis of the sections of the loop boundary where the amplitudes of the shear flows are the greatest (see Figures \ref{fig:tube1} and \ref{fig:tube_interface}).
The local analysis is only valid for perturbations with the azimuthal wavelength much smaller than the radius of the loop cross-section $R$, and the axial wavelength much smaller than the loop length $L$\/.
In accordance with these latter assumptions, we took $k_x = m/R$ and $k_z = \pi n/L$, where $k_x$ is the component of the wave vector in the azimuthal direction, and $k_z$ is the component of the wave vector in the axial direction, and $|m|$ and $n$ are positive integer numbers.
We note that, while $n$ is positive, $m$ can be either positive or negative.
We found that the nature of solutions is changed by this new definition of the parameters.
While, previously, all solutions were unstable regardless of the background parameters, the discretisation of the parameter space has introduced the possibility that unstable solutions exist only for sufficiently large values of $|m|$.

Finally, we note that our study does not include the effects of strong shear induced by resonant absorption, which may be significant in the generation of the KHI, as suggested by \cite{Howson2017b} and \cite{Terradas2018}.
